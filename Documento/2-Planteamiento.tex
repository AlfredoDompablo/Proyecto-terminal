\section{Planteamiento del Problema}

Los métodos tradicionales, como los muestreos manuales y los análisis de laboratorio, son limitados para evaluar de forma continua y precisa la salud de los ríos. Estas técnicas, aunque útiles para obtener mediciones puntuales, no permiten monitorear de forma continua la variación de parámetros de calidad del agua o la acumulación de residuos sólidos flotantes. Esta deficiencia compromete la capacidad de detectar oportunamente problemas ambientales que afectan el flujo natural de los ríos y la calidad del agua, intensificando la liberación de sustancias tóxicas por la descomposición de residuos y poniendo en riesgo la biodiversidad y las comunidades que dependen de estos cuerpos de agua \cite{monitoreo2023}.

Aunque México ha tenido avances significativos en materia de transparencia de datos sobre el agua, en gran parte gracias a colectivos e individuos comprometidos, persiste un reto fundamental: traducir la información técnica disponible en plataformas, bases de datos y registros en formatos comprensibles para el público general. Tal como se señala en el artículo “Calidad del agua en México, un reto vital”, publicado por la revista \textit{Este País}, es necesario transformar dichos datos en mensajes claros que cualquier persona pueda interpretar fácilmente y entender sus implicaciones para la salud y el medio ambiente \cite{estepais2024}.

Ante este desafío, es esencial desarrollar un sistema integral que combine monitoreo automatizado con herramientas accesibles para el procesamiento e interpretación de los datos. Este sistema debe ser capaz de capturar información precisa sobre la calidad del agua y los residuos sólidos flotantes, presentándola de manera clara y útil tanto para las autoridades encargadas de la gestión hídrica como para las comunidades cercanas a los cuerpos de agua. De esta forma, se busca facilitar la toma de decisiones informadas y mejorar la capacidad de respuesta ante riesgos ambientales\cite{monitoreo2023}.

Actualmente se están explorando soluciones basadas en redes de sensores inalámbricos (WSN, por sus siglas en inglés) e inteligencia artificial, estas tecnologías aún enfrentan desafíos significativos. Entre los principales desafíos técnicos destacan: (i) la precisión de los sensores en ambientes con alta turbidez; (ii) la capacidad de los sistemas de visión artificial para identificar residuos sólidos bajo condiciones variables de iluminación y visibilidad; y (iii) el manejo eficiente de los grandes volúmenes de datos generados \cite{monitoreo2023}.

Los sensores superan las limitaciones de los métodos manuales al ofrecer mediciones continuas y representativas de las condiciones del agua. Realizan mediciones precisas que minimizan los errores típicos de la recolección, transporte y almacenamiento de muestras, como los cambios en los parámetros causados por el contacto con el aire, la agitación o los retrasos en el análisis. Además, eliminan la necesidad de conservar muestras y permiten un monitoreo constante de los parámetros de calidad del agua, garantizando la recolección continua de datos sin interrupciones.

La visión artificial proporciona una solución automatizada mediante cámaras y algoritmos avanzados para la detección de residuos sólidos flotantes. Estos sistemas procesan imágenes, detectando patrones y acumulaciones de desechos que podrían pasar inadvertidos con métodos tradicionales. Dicha capacidad resulta particularmente valiosa en cuerpos de agua lóticos, donde la presencia de materiales flotantes puede variar rápidamente. Asimismo, permiten el registro y almacenamiento de imágenes para análisis posteriores, facilitando la construcción de series históricas y el estudio de patrones de contaminación.

Este reto trasciende la generación de información, abarcando críticamente su accesibilidad. Por consiguiente, los sistemas de monitoreo deben garantizar no solo la precisión en la recolección de datos, sino también su presentación mediante interfaces visuales, intuitivas y contextualizadas que permitan a autoridades y ciudadanía tomar decisiones basadas en evidencia.

En este contexto, surge la siguiente pregunta de investigación:

\textbf{¿Cómo desarrollar un sistema automatizado de monitoreo, basado en una red inalámbrica de sensores y visión artificial, que permita medir parámetros clave de calidad del agua y detectar residuos sólidos en cuerpos de agua lóticos afectados por actividades industriales y agrícolas?}
