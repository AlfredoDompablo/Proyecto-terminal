
\section{Cronograma}
A continuación, se presenta el cronograma propuesto para las actividades correspondientes a la fase de Proyecto Terminal 1. En este cronograma se detallan las actividades clave, los responsables de su ejecución y los objetivos específicos de cada una.

\begin{table}[H]
    \centering
    \renewcommand{\arraystretch}{1.5}
    \begin{tabular}{ |p{1.5cm}|p{5cm}|p{5.5cm}|p{2.5cm}| }

        \hline
        \textbf{N° Actividad} & \textbf{Nombre de la Actividad} & \textbf{Objetivo} & \textbf{Responsable} \\
        \hline
        1 & Investigación preliminar sobre tecnologías de monitoreo & Revisar tecnologías de sensores y sistemas de visión artificial existentes para detectar basura flotante y monitorear la calidad del agua & Alumno 1, Alumno 2 \\\hline
        2 & Definición de objetivos y alcance del proyecto & Establecer los objetivos específicos y generales del sistema, delimitando las áreas de aplicación y funcionalidad & Alumno 1, Alumno 2 \\\hline
        3 & Estudio de las variables de calidad del agua & Analizar las variables clave como pH, oxigenación, turbidez, temperatura y conductividad que serán monitoreadas & Alumno 1 \\\hline
        4 & Selección de microcontroladores y cámaras & Definir qué tipo de microcontroladores y cámaras se adecuan mejor a los requisitos del proyecto, considerando consumo de energía y rendimiento & Alumno 2 \\\hline
        5 & Estudio de protocolos de comunicación para la red inalámbrica & Investigar los protocolos de comunicación más adecuados (LoRa, ZigBee, WiFi) para la transmisión de datos entre los nodos sensores y el servidor & Alumno 1 \\\hline
        6 & Diseño de la arquitectura del sistema & Definir la topología de la red, la disposición de los nodos y la estructura general del sistema para asegurar eficiencia y cobertura & Alumno 1, Alumno 2 \\\hline
        7 & Diseño preliminar del nodo sensor & Definir los componentes clave del nodo (sensores, alimentación, módulo de comunicación) y diseñar la estructura física que los protegerá & Alumno 2 \\\hline
        8 & Selección de sensores para calidad del agua & Elegir los sensores adecuados para medir pH, oxigenación, turbidez y conductividad en base a su precisión, tamaño y durabilidad & Alumno 1 \\\hline

    \end{tabular}
    \caption{Cronograma de actividades del proyecto - Parte 1}
    \label{tab:cronograma_proyecto_parte1}
\end{table}
\begin{table}[H]
    \centering
    \renewcommand{\arraystretch}{1.5}
    \begin{tabular}{ |p{1.5cm}|p{5cm}|p{5.5cm}|p{2.5cm}| }

        \hline
        \textbf{N° Actividad} & \textbf{Nombre de la Actividad} & \textbf{Objetivo} & \textbf{Responsable} \\
        \hline

        9 & Diseño del sistema de visión artificial & Establecer los algoritmos de procesamiento de imágenes para la clasificación de basura flotante y definir los tipos de residuos a identificar & Alumno 2 \\\hline
        10 & Análisis de opciones de bases de datos & Investigar y seleccionar el sistema de bases de datos que mejor se adecue a las necesidades del proyecto (almacenamiento, escalabilidad) & Alumno 1 \\\hline
        11 & Definición del servidor y almacenamiento & Establecer las características del servidor donde se almacenarán los datos recolectados, considerando seguridad y eficiencia & Alumno 1, Alumno 2 \\\hline
        12 & Diseño preliminar de la aplicación web & Crear el diseño inicial de la interfaz web para la visualización de los datos en forma de gráficas e imágenes recolectadas por los sensores y cámaras & Alumno 1, Alumno 2 \\\hline
        13 & Revisión y corrección del diseño final del sistema & Revisar y ajustar el diseño de todos los módulos del sistema antes de pasar a la etapa de implementación & Alumno 1, Alumno 2 \\\hline

        14 & Implementación del nodo sensor & Ensamblar los componentes del nodo y configurar los sensores para medir parámetros clave de calidad del agua & Alumno 1 \\\hline
        15 & Desarrollo de algoritmos para la visión artificial & Programar los algoritmos de procesamiento de imágenes para la detección y clasificación de basura flotante & Alumno 2 \\\hline

    \end{tabular}
    \caption{Cronograma de actividades del proyecto - Parte 2}
    \label{tab:cronograma_proyecto_parte2}
\end{table}
\begin{table}[H]
    \centering
    \renewcommand{\arraystretch}{1.5}
    \begin{tabular}{ |p{1.5cm}|p{5cm}|p{5.5cm}|p{2.5cm}| }

        \hline
        \textbf{N° Actividad} & \textbf{Nombre de la Actividad} & \textbf{Objetivo} & \textbf{Responsable} \\
        \hline

        16 & Integración de los nodos en la red inalámbrica & Configurar la red inalámbrica para permitir la comunicación entre los nodos sensores y el servidor & Alumno 1 \\\hline
        17 & Implementación del servidor y base de datos & Desplegar el servidor y configurar la base de datos para almacenar los datos de los sensores y la visión artificial & Alumno 1 \\\hline
        18 & Programación de la transmisión de datos & Desarrollar la lógica que permitirá transmitir los datos recolectados a una central de procesamiento en intervalos regulares & Alumno 2 \\\hline
        19 & Desarrollo de la aplicación web & Programar y diseñar la interfaz web para visualizar los datos y las imágenes de los residuos flotantes detectados & Alumno 1, Alumno 2 \\\hline
        20 & Pruebas de transmisión de datos & Verificar la transmisión inalámbrica de los datos recolectados por los nodos y medir la integridad de la información transmitida & Alumno 1 \\\hline
        21 & Pruebas de la precisión de los sensores & Realizar pruebas controladas para verificar la precisión de los sensores y ajustar los parámetros en base a los resultados obtenidos & Alumno 1 \\\hline
        22 & Pruebas del sistema de visión artificial & Probar el sistema de visión artificial para detectar basura flotante y ajustar los algoritmos si es necesario & Alumno 2 \\\hline

    \end{tabular}
    \caption{Cronograma de actividades del proyecto - Parte 3}
    \label{tab:cronograma_proyecto_parte3}
\end{table}
\begin{table}[H]
    \centering
    \renewcommand{\arraystretch}{1.5}
    \begin{tabular}{ |p{1.5cm}|p{5cm}|p{5.5cm}|p{2.5cm}| }

        \hline
        \textbf{N° Actividad} & \textbf{Nombre de la Actividad} & \textbf{Objetivo} & \textbf{Responsable} \\
        \hline

        23 & Integración de módulos & Unificar todos los componentes del sistema en una solución completa que permita la interacción fluida entre sensores, cámaras y la plataforma web & Alumno 1, Alumno 2 \\\hline
        24 & Pruebas finales del sistema completo & Realizar pruebas en un entorno controlado para verificar el funcionamiento del sistema en condiciones similares a las del entorno real & Alumno 1, Alumno 2 \\\hline
        25 & Documentación final del proyecto & Redactar el informe final del proyecto con los resultados obtenidos en las pruebas y las soluciones implementadas & Alumno 1, Alumno 2 \\\hline

    \end{tabular}
    \caption{Cronograma de actividades del proyecto - Parte 4}
    \label{tab:cronograma_proyecto_parte4}
\end{table}
