\section{Cronograma}
A continuación, se presenta el cronograma propuesto para las actividades correspondientes a la fase de Proyecto Terminal 1. En este cronograma se detallan las actividades clave, los responsables de su ejecución y los objetivos específicos de cada una.
\begin{table}[H]
    \centering
    \renewcommand{\arraystretch}{1.5}
    \begin{tabular}{ |p{3cm}|p{8cm}| }
        \hline
        \textbf{Alumno 1} & Dompablo Celaya Oscar Alfredo \\\hline
        \textbf{Alumno 2} & López Ramírez Itzel \\\hline
    \end{tabular}
    \caption{Nombres de los responsables}
    \label{tab:Nombre de los responsables}
\end{table}

\begin{table}[H]
    \centering
    \renewcommand{\arraystretch}{1.5}
    \begin{tabular}{ |p{1.5cm}|p{5cm}|p{5.5cm}|p{2.5cm}| }

        \hline
        \textbf{N° Actividad} & \textbf{Nombre de la Actividad} & \textbf{Objetivo} & \textbf{Responsable} \\
        \hline
        1 & Investigación sobre tecnologías de monitoreo y visión artificial & Identificar soluciones actuales y evaluar su aplicabilidad para medir la calidad del agua y detectar residuos flotantes & Alumno 1, Alumno 2 \\\hline
        2 & Análisis de redes inalámbricas de sensores & Comprensión detallada del funcionamiento y configuración de redes inalámbricas de sensores para su aplicación en monitoreo ambiental & Alumno 1, Alumno 2 \\\hline
        3 & Estudio sobre el procesamiento de imágenes & Adquisición de conocimientos sobre técnicas de procesamiento de imágenes, con énfasis en su aplicación para la detección de residuos sólidos mediante visión artificial & Alumno 1, Alumno 2 \\\hline
        4 & Análisis de algoritmos de visión artificial & Evaluar algoritmos para clasificar residuos flotantes & Alumno 2 \\\hline
        5 & Definición de objetivos y alcance del proyecto & Establecer Requerimientos funcionales y específicos del sistema & Alumno 1, Alumno 2 \\\hline
        6 & Estudio de variables de calidad del agua & Investigar parámetros clave como pH, oxigenación, turbidez, temperatura y conductividad & Alumno 1 \\\hline
        7 & Selección preliminar de sensores & Determinar sensores más adecuados para las variables de calidad del agua & Alumno 1, Alumno 2 \\\hline
        8 & Investigación de protocolos de comunicación & Analizar protocolos como LoRa y ZigBee para la red inalámbrica & Alumno 1 \\\hline
        9 & Estudio de microcontroladores y cámaras & Evaluar opciones técnicas para seleccionar componentes adecuados & Alumno 2 \\\hline
        10 & Diseño preliminar de arquitectura del sistema & Definir la topología de red y la interacción de módulos & Alumno 1, Alumno 2 \\\hline

    \end{tabular}
    \caption{Cronograma de actividades del proyecto - Parte 1}
    \label{tab:cronograma_proyecto_parte1}
\end{table}
\begin{table}[H]
    \centering
    \renewcommand{\arraystretch}{1.5}
    \begin{tabular}{ |p{1.5cm}|p{5cm}|p{5.5cm}|p{2.5cm}| }

        \hline
        \textbf{N° Actividad} & \textbf{Nombre de la Actividad} & \textbf{Objetivo} & \textbf{Responsable} \\
        \hline

       11 & Caracterización de sensores & Probar las especificaciones técnicas de los sensores seleccionados & Alumno 1 \\\hline
        12 & Diseño conceptual del sistema de visión artificial & Establecer el enfoque inicial para la detección de residuos flotantes & Alumno 2 \\\hline
        13 & Selección de infraestructura de almacenamiento & Determinar servidores y bases de datos adecuados para el proyecto & Alumno 1 \\\hline
        14 & Diseño de una base de datos  & Diseñar una base de datos estructurada para el almacenamiento de los datos & Alumno 1, Alumno 2\\\hline
        15 & Diseño preliminar de la aplicación web & Prototipar una interfaz web para visualización de datos & Alumno 1, Alumno 2 \\\hline
        16 & Desarrollo de la lógica de transmisión de datos & Planificar la integración de datos de sensores en el servidor & Alumno 2 \\\hline
        17 & Simulación de la red inalámbrica & Validar la comunicación de los nodos en un entorno controlado & Alumno 1 \\\hline
        18 & Desarrollo de algoritmos de visión artificial & Implementar algoritmos iniciales para detección y clasificación de residuos & Alumno 2 \\\hline
        19 & Diseño del nodo sensor & Crear el diseño conceptual del nodo con sensores, comunicación y alimentación & Alumno 1 \\\hline
        20 & Diseño del módulo de visión artificial & Definir la integración de cámaras y algoritmos para procesar imágenes & Alumno 2 \\\hline
        21 & Diseño del servidor y base de datos & Configurar el almacenamiento y procesamiento centralizado de datos & Alumno 1 \\\hline

    \end{tabular}
    \caption{Cronograma de actividades del proyecto - Parte 2}
    \label{tab:cronograma_proyecto_parte2}
\end{table}
\begin{table}[H]
    \centering
    \renewcommand{\arraystretch}{1.5}
    \begin{tabular}{ |p{1.5cm}|p{5cm}|p{5.5cm}|p{2.5cm}| }

        \hline
        \textbf{N° Actividad} & \textbf{Nombre de la Actividad} & \textbf{Objetivo} & \textbf{Responsable} \\
        \hline
        22 & Redacción del documento técnico intermedio & Documentar avances y ajustar planificación para la siguiente fase & Alumno 1, Alumno 2 \\\hline
        23 & Validación inicial de componentes & Probar individualmente sensores y cámaras para asegurar su funcionalidad & Alumno 1, Alumno 2 \\\hline
        24 & Desarrollo del prototipo de red inalámbrica & Implementar la comunicación básica entre nodos y servidor & Alumno 1 \\\hline
        25 & Evaluación de precisión en sensores & Realizar pruebas para ajustar parámetros según resultados obtenidos & Alumno 1 \\\hline
        26 & Pruebas de visión artificial & Verificar la capacidad de clasificación de residuos en imágenes controladas & Alumno 2 \\\hline
        27 & Integración parcial de módulos & Unificar nodos sensores, visión artificial y servidor para pruebas iniciales & Alumno 1, Alumno 2 \\\hline

    \end{tabular}
    \caption{Cronograma de actividades del proyecto - Parte 3}
    \label{tab:cronograma_proyecto_parte3}
\end{table}
