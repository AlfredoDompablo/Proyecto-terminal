\section{Objetivo General}
\label{sec:objetivos}
Diseñar un sistema automatizado basado en una red inalámbrica de sensores distribuidos y visión artificial para medir parámetros de calidad del agua y detectar residuos sólidos en cuerpos de agua que se mueven siempre en una misma dirección como ríos, riachuelos y arroyos. 

%Este sistema permitirá la transmisión periódica de los datos a una central, donde serán procesados para facilitar la toma de decisiones informadas por parte de las autoridades competentes, contribuyendo a una gestión más eficiente y precisa de los recursos hídricos.%

\subsection{Objetivos específicos}
\begin{itemize}
 %   \item Diseñar la arquitectura del sistema de monitoreo automatizado. 
    \item Diseñar e implementar una red inalámbrica de sensores equipada con nodos sensores capaces de medir parámetros clave de la calidad del agua, como pH, oxigenación, temperatura, turbidez y conductividad. 
    %Cada nodo sensor contará con una estructura resistente que proteja los sensores y un módulo de visión artificial para detectar basura flotante. 
    %\item Diseñar e implementar una red inalámbrica de sensores para medir parámetros de la calidad del agua, como pH, oxigenación y conductividad así como detectar basura flotante.
    \item Implementar un sistema de visión artificial para analizar imágenes capturadas por las cámaras y detectar residuos flotantes en el agua.
    \item Diseñar nodo concentrador para recolectar y gestionar la información de los sensores, y transmitirla al servidor de manera estructurada.
   % \item Implementar el servidor para almacenar y gestionar los datos recopilados, asegurando su organización y confiabilidad.
    %\item Diseñar una base de datos estructurada para almacenar y organizar los datos provenientes de los nodos sensores, garantizando su accesibilidad y disponibilidad para su posterior análisis.
    \item Diseñar e implementar un servidor con una base de datos estructurada para almacenar, gestionar y organizar los datos recopilados por los nodos sensores, buscando facilitar su confiabilidad, accesibilidad y disponibilidad para su posterior análisis.
    
   % \item Desarrollar una página web para presentar los datos recolectados de forma clara y comprensible, facilitando el acceso del público a la información sobre la calidad del agua y la detección de residuos sólidos.
    \item Desarrollar una página web para presentar los datos recolectados de forma clara y comprensible, facilitando el acceso del público a la información sobre la calidad del agua y la detección de residuos sólidos, la cual será alojada en un servidor local o en la nube.

\end{itemize}
