%===============Lista======================
\begin{itemize}
    \item Primer elemento
    \item Segundo elemento
    \item Tercer elemento
\end{itemize}

\begin{enumerate}
    \item Primer elemento
    \item Segundo elemento
    \item Tercer elemento
\end{enumerate}

\begin{description}
    \item[Elemento 1] Descripción del primer elemento
    \item[Elemento 2] Descripción del segundo elemento
    \item[Elemento 3] Descripción del tercer elemento
\end{description}


%======================= Introducir codigo ===================
\begin{listing}[H]
\begin{minted}{c}
#include <stdio.h>
int main() {
   printf("Hello, World!"); /*printf() outputs the quoted string*/
   return 0;
}
\end{minted}
\caption{Hello World in C}
\label{listing:2}
\end{listing}


%======================= Introducir Imagenes ===================
\begin{figure}[H]
    \centering
    \includegraphics[width=0.25\textwidth]{Imagen.jpg}
    \caption{Información de la imagen}
    \label{fig:mesh1}
\end{figure}

%======================= URL ===================
\href{http://www.overleaf.com}{Texto  cualquiera}
\url{http://www.overleaf.com}

%======================= Ecuaciones ===================
\begin{equation*}
    G = 4.3\times10^{-6}\units{\frac{Km^2Kpc}{s^2\msol}} \hspace{0.5cm} L = 2^{3.5} = 11.3 \units{\lsol}
\end{equation*}

%======================= Agregar Tablas =======================
\begin{table}[H]
    \centering
    \begin{tabular}{ |p{3cm}|p{3cm}|p{3cm}|  }
        \hline
        \multicolumn{3}{|c|}{Nombre de la tabla} \\%numero de columnas
        \hline
        Nombre Columna 1 & Nombre Columna 2 & Nombre Columna 3 \\
        \hline
        Afghanistan & AF &AFG \\
        Aland Islands & AX   & ALA \\
        Albania &AL & ALB \\
        Algeria    &DZ & DZA \\
        American Samoa & AS & ASM \\
        Andorra & AD & AND   \\
        Angola & AO & AGO \\
    \hline
    \end{tabular}
    \caption{Caption}
    \label{tab:my_label}
\end{table}


\begin{table}[H]
    \centering
    \begin{tabular}{|l||*{5}{c|}}\hline
        \backslashbox[48mm]{Room}{Date}%tamanio del slach
        &\makebox[3em]{5/31}&\makebox[3em]{6/1}&\makebox[3em]{6/2}
        &\makebox[3em]{6/3}&\makebox[3em]{6/4}\\\hline\hline
        Meeting Room &&&&&\\\hline
        Auditorium &&&&&\\\hline
        Seminar Room &&&&&\\\hline
    \end{tabular}
    \caption{Caption}
    \label{tab:my_label}
\end{table}