\chapter{Introducción}

\section{Introducción}

El agua es un recurso esencial para la vida en el planeta, y su calidad tiene un impacto directo en la salud de los ecosistemas, la biodiversidad y las actividades humanas. A nivel mundial, la contaminación hídrica ha alcanzado niveles críticos, comprometiendo el bienestar de millones de personas y el equilibrio de los ecosistemas acuáticos. En México, la situación es particularmente alarmante: menos del 25\% de las aguas residuales descargadas en cuerpos de agua reciben tratamiento adecuado, según el Fondo para la Comunicación y la Educación Ambiental, agravando significativamente el deterioro de la calidad del agua \cite{agua2006}.

%*************************************Eliminado
%Entre los ejemplos más representativos de esta crisis se encuentra el río Atoyac, uno de los cuerpos de agua más contaminados del país. Desde la década de 1980, este río ha sido severamente impactado por descargas industriales y agrícolas, lo que ha provocado la acumulación de residuos tóxicos, metales pesados y aguas residuales no tratadas. Estas condiciones han incrementado notablemente las enfermedades en las comunidades cercanas, incluyendo casos de cáncer y enfermedades renales \cite{hernandez2021}.

%De manera similar, el río Sonora fue escenario de uno de los desastres ambientales más graves en la historia reciente de México. En 2014, un fallo en el sistema de represas de la mina Buenavista del Cobre, propiedad de Grupo México, liberó 40,000 metros cúbicos de ácido sulfúrico y sulfato de cobre al río. Este derrame contaminó el agua, el suelo y el aire con metales pesados como arsénico, plomo y mercurio, superando los límites establecidos por las normas nacionales e internacionales. A pesar de los esfuerzos de remediación, los efectos persisten, afectando a más de 22,000 personas y deteriorando los medios de vida de las comunidades dependientes del río \cite{diagnostico2023}.
%********************************Eliminado

%Estos casos subrayan la necesidad urgente de implementar tecnologías avanzadas para el monitoreo de la calidad del agua. Los métodos tradicionales, basados en muestreos manuales y análisis de laboratorio, son insuficientes para detectar oportunamente la presencia de contaminantes y residuos sólidos flotantes. La detección de estos residuos, como plásticos y otros desechos visibles, es crucial, ya que no solo obstruyen el flujo del agua, sino que también liberan sustancias tóxicas al descomponerse, agravando la contaminación y afectando a los ecosistemas acuáticos.\cite{monitoreo2023}.


%****************************Agregando del planteamiento*******

En las últimas décadas, los métodos de monitoreo de la calidad del agua han avanzado, pero aún dependen en gran medida de técnicas manuales y análisis de laboratorio. En México, diversos programas buscan involucrar a las comunidades en la detección de cambios en la calidad del agua mediante mediciones rápidas y variables básicas, apoyándose en guías que estandarizan metodologías y validan resultados. Sin embargo, estas mediciones son limitadas, este proceso tarda en realizarse y validarse, y se llevan a cabo en periodos definidos, lo que impide un monitoreo constante.


Para realizar este tipo de monitoreos participativos los interesados necesitan seguir una serie de pasos para poder ser aceptados por las instituciones responsables. Estos pasos incluyen presentar una solicitud formal, cumplir con los requisitos de capacitación y validación técnica, y comprometerse a seguir los protocolos de medición establecidos. Además, es necesario garantizar que los datos recopilados sean fiables y compatibles con los estándares definidos, lo que implica la evaluación continua por parte de capacitadores acreditados y la supervisión institucional. Este enfoque busca no solo fomentar la participación de la sociedad, sino también garantizar que los resultados sean útiles para la gestión y preservación de los ecosistemas acuáticos, sin embargo, la participación comunitaria puede disminuir con el tiempo debido a la falta de incentivos, recursos, reconocimiento, los datos recolectados pueden no ser aceptados oficialmente si no cumplen con los estándares requeridos y para generar la "primera alerta" se necesita de un historial de los datos acumulados en el monitoreo y evidencias de respaldo del muestro para corroborar.


Una situación similar sucede para los monitoreos realizados por las diferentes instituciones encargadas de la calidad del agua y que garantizan el cumplimiento de estándares ambientales, son reportes más detallados y formales sobre el tipo de contaminantes y el nivel de contaminación, no obstante, son procesos tardados, dependiendo del tipo de cuerpo de agua los tiempos de monitoreo varían(bimestral, semestral o anual), los resultados de los monitoreos son difíciles de comprender debido a su complejidad técnica para el público y, además, su publicación puede retrasarse durante largos periodos \cite{guiaMonitoreo2024}.


La falta de monitoreo constante en los cuerpos de agua que abastecen al país tiene un impacto severo en los ecosistemas y en la salud pública. Aunque las tecnologías actuales permiten medir parámetros clave como el pH, la oxigenación y la conductividad, estas mediciones se realizan principalmente a través de muestreos manuales y no de forma automatizada. Además, contaminantes como la residuos sólidos no solo obstruyen los cuerpos de agua, sino que también actúan como vectores para la dispersión de contaminantes más peligrosos, como metales pesados y productos químicos.

%referencia: https://www.gob.mx/cms/uploads/attachment/file/925491/Gu_a_de_Monitoreo_Participativo_Primera_Alerta.pdf

%***************************************************añadido

%La combinación de redes inalámbricas de sensores (WSN, por sus siglas en inglés) y visión artificial representa una solución innovadora para superar estas limitaciones. Por un lado, las WSN permiten medir parámetros clave como pH, oxigenación y conductividad de manera distribuida, adaptándose a condiciones variables como cambios de temperatura, climas extremos. Por otro lado, la visión artificial facilita la detección y clasificación de basura flotante, optimizando la identificación de contaminantes visibles. Juntos, estos enfoques proporcionan datos confiables y frecuentes, lo que podría ayudar a autoridades y comunidades locales a tomar decisiones mejor informadas y efectivas.
La combinación de redes inalámbricas de sensores (WSN, por sus siglas en inglés) y visión artificial ofrece una solución prometedora para abordar las limitaciones de los métodos tradicionales. Las WSN permiten la automatización de la medición de parámetros clave como pH, oxigenación y conductividad de manera distribuida, lo que facilita un monitoreo más continuo y preciso del estado de la calidad del agua. Por su parte, la visión artificial mejora la detección de residuos sólidos, contribuyendo a identificar contaminantes visibles en los ríos. En conjunto, estos enfoques podrían generar datos útiles que, eventualmente, apoyarían a las personas interesadas en comprender mejor las condiciones de los cuerpos de agua y fomentar acciones para su preservación.


%Esta investigación tiene como objetivo diseñar e implementar un sistema automatizado que integre redes inalámbricas de sensores y visión artificial para monitorear la calidad del agua y detectar residuos sólidos flotantes en cuerpos de agua. La solución propuesta busca generar información sobre las condiciones hídricas y la presencia de residuos sólidos flotantes, los cuales serán accesibles a través de una página web. Este sistema permitirá a las personas interesadas mantenerse informadas y, al mismo tiempo, apoyará a las autoridades en la toma de decisiones fundamentadas para mejorar la gestión de los recursos hídricos, proteger los ecosistemas acuáticos y promover el bienestar de las comunidades afectadas.

%Esta investigación tiene como objetivo diseñar e implementar un sistema automatizado que integre redes inalámbricas de sensores y visión artificial para monitorear la calidad del agua y detectar residuos sólidos flotantes en cuerpos de agua. La solución propuesta busca generar información relevante sobre las condiciones hídricas y la presencia de residuos sólidos flotantes, accesible a través de una página web. Este sistema no solo permitirá a las personas interesadas mantenerse informadas, sino que también apoyará la toma de decisiones fundamentadas, contribuyendo a mejorar la gestión de los recursos hídricos, proteger los ecosistemas acuáticos y promover el bienestar de las comunidades afectadas. 
Con base en esta propuesta, el sistema busca integrar tecnologías avanzadas para automatizar y ofrecer un monitoreo constante de la calidad del agua en cuerpos de agua lóticos\footnote{Lótico: Masa de agua con flujo constante en una dirección específica, como ríos, arroyos y corrientes\cite{conagua2024}.} (ríos, riachuelos y arroyos). A través de la recopilación de datos clave y su visualización en una plataforma accesible, se espera proporcionar información que facilite el análisis de las condiciones hídricas y la identificación de residuos flotantes. Este enfoque podría ser un recurso valioso para quienes deseen comprender el estado de los cuerpos de agua monitoreados y fomentar acciones orientadas a su conservación así como apoyar a la difusión abierta de información ambiental.



%Este enfoque integrará tecnología avanzada con un diseño orientado a la difusión abierta de información ambiental.

%El estudio detallado completo y la seleccion de las tecnologías 


%Esta investigación tiene como objetivo diseñar e implementar un sistema automatizado que integre redes inalámbricas de sensores y visión artificial para monitorear la calidad del agua y detectar residuos sólidos flotantes en cuerpos de agua. La solución propuesta busca mejorar la gestión de los recursos hídricos, proteger los ecosistemas acuáticos y promover el bienestar de las comunidades afectadas, abordando los desafíos ambientales mediante soluciones concretas y aplicables.
