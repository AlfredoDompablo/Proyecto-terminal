\renewcommand{\arraystretch}{1.5}
\small
\begin{longtable}{|p{3cm}|p{4cm}|p{4cm}|p{4cm}|}
\caption{Comparativa transpuesta de microcontroladores para el sistema de monitoreo}
\label{tab:comparativa_microcontroladores_transpuesta} \\
\hline
\multicolumn{4}{|c|}{Parte 1 Microcontroladores}\\
\hline
\textbf{Característica} 
& \textbf{LilyGO T-HaLow (ESP32-S3 N16R8)} \cite{lilygo_t-halow2025}
& \textbf{ESP8266 NodeMCU} \cite{espressif2023_ESP8266EX}
& \textbf{Raspberry Pi Pico}  \cite{raspberrypi2024_Pico_datasheet} \\
\hline
\endfirsthead

\hline
\textbf{Característica} 
& \textbf{LilyGO T-HaLow (ESP32-S3 N16R8)} \cite{lilygo_t-halow2025}
& \textbf{ESP8266 NodeMCU} \cite{espressif2023_ESP8266EX}
& \textbf{Raspberry Pi Pico} \cite{raspberrypi2024_Pico_datasheet} \\
\hline
\endhead

\hline
\multicolumn{4}{r}{\textit{Continúa en la siguiente página}} \\
\endfoot

\hline
\endlastfoot

Arquitectura 
& Xtensa dual 32-bit LX7 
& 32-bit RISC 
& ARM Cortex \\ \hline

Procesamiento 
& 240 MHz 
& 160 MHz 
& 133 MHz \\ \hline

RAM 
& 520KB + 16MB PSRAM 
& 160KB SRAM 
& 264KB SRAM \\ \hline

Flash 
& 16MB 
& 4MB 
& 2MB \\ \hline

Consumo Activo 
& 91 mA& 80 mA & 50 mA \\
Consumo Sleep & 240 \unit{\uA} & 20 \unit{\uA} & 390 \unit{\uA} \\ \hline

Voltaje &3.3V & 3.3V & 3.3V \\ \hline

WiFi 
& 2.4GHz 
& 2.4GHz 
& No \\ \hline

Bluetooth 
& Classic + BLE 
& No 
& No \\ \hline

Sistema operativo (OS) 
& FreeRTOS 
& FreeRTOS 
& Bare-metal \\ \hline

Costo (USD) 
& \$20–30 
& \$3–6 
& \$10 \\ \hline

Periféricos 
& \shortstack[l]{\\• 36 GPIOs\\• ADC (20 canales)\\• UART\\• SPI}
& \shortstack[l]{\\• 17 GPIOs\\• ADC (1 canal)\\• UART\\• SPI}
& \shortstack[l]{\\• 23 GPIOs\\• ADC (3 canales)\\• 2×UART\\• SPI} \\ \hline

Ventajas 
& \shortstack[l]{\\• RAM extensa (16MB)\\• Alto rendimiento\\• Multiperiféricos\\• Soporte RTOS}
& \shortstack[l]{\\• Bajo costo\\• Suficiente para\\ aplicaciones básicas}
& \shortstack[l]{\\• Dual-core\\• Buen balance\\ precio/rendimiento} \\ \hline

Desventajas 
& \shortstack[l]{\\• Costo elevado\\• Complejidad de \\desarrollo\\• Solo 1 modo de \\operación (AP o STA)}
& \shortstack[l]{\\• RAM limitada\\• Un solo canal ADC}
& \shortstack[l]{\\• Sin PSRAM\\• Canales ADC\\ insuficientes} \\ \hline

Imagen 
& 
    \shortstack{\\ \includegraphics[width=1 \linewidth]{Documento/Imagenes/Análisis/Microcontroladores/lilygo-T-Halow.png}}
&
    \includegraphics[width=1 \linewidth]{Documento/Imagenes/Análisis/Microcontroladores/NodeMCU.png}
&  
    \includegraphics[width=0.9 \linewidth]{Documento/Imagenes/Análisis/Microcontroladores/RB Pico.jpg}   
\\ \hline

\end{longtable}

%=====================================================================
%=====================================================================
%=====================================================================


\begin{longtable}{|p{3cm}|p{4cm}|p{4cm}|p{4cm}|}
\hline
\multicolumn{4}{|c|}{Parte 2: Microcontroladores (Alto Rendimiento)}\\
\hline
\textbf{Característica} 
& \textbf{Xiao ESP32-S3 R8} \cite{espressif2023_ESP32-S3}
& \textbf{STM32F103} \cite{stmicroelectronics2025_STM32F103ZE_datasheet}
& \textbf{Freenove ESP32-S3} \cite{espressif2023_ESP32-S3} \\ % Cita actualizada
\hline
\endfirsthead

\hline
\textbf{Característica} 
& \textbf{Xiao ESP32-S3 R8} \cite{espressif2023_ESP32-S3}
& \textbf{STM32F103} \cite{stmicroelectronics2025_STM32F103ZE_datasheet}
& \textbf{Freenove ESP32-S3} \cite{espressif2023_ESP32-S3} \\
\hline
\endhead

\hline
\multicolumn{4}{r}{\textit{Continúa en la siguiente página}} \\
\endfoot

\hline
\endlastfoot

Arquitectura 
& Xtensa dual 32-bit LX7 
& ARM 32-bit M3 
& Xtensa dual 32-bit LX7 \\ \hline

Procesamiento 
& 240 MHz 
& 72 MHz 
& 240 MHz \\ \hline

RAM 
& 512KB + 8MB PSRAM 
& 64KB SRAM 
& 512KB + 8MB PSRAM \\ \hline

Flash 
& 8MB (Externo)
& 256–512KB 
& 16MB (N16) \\ \hline

Consumo Activo 
& 91 mA 
& 66 mA 
& 91 mA \\
Consumo Sleep 
& 240 \unit{\uA} 
& 8.5 mA 
& 240 \unit{\uA} \\ \hline

Voltaje & 3.3V & 3.3V / 5V & 3.3V (5V USB) \\ \hline

WiFi 
& 2.4GHz 
& No 
& 2.4GHz \\ \hline

Bluetooth 
& BLE 5 
& No 
& BLE 5 \\ \hline

Sistema operativo 
& FreeRTOS 
& Bare-metal/RTOS 
& FreeRTOS \\ \hline

Costo (USD) 
& \$10–15 
& \$5 
& \$15–20 \\ \hline

Periféricos 
& \shortstack[l]{\\• 11 GPIOs\\• ADC (8 canales)\\• UART\\• SPI}
& \shortstack[l]{\\• 51 GPIOs\\• ADC (3 canales)\\• 5×UART\\• SPI}
& \shortstack[l]{\\• Puerto Cámara\\• Slot MicroSD\\• 20 GPIOs\\• USB Nativo} \\ \hline

Ventajas 
& \shortstack[l]{\\• Factor de forma\\ultra compacto\\• 8MB PSRAM\\• Bajo consumo}
& \shortstack[l]{\\• Múltiples UARTs\\• Amplios GPIOs\\• Robustez industrial}
& \shortstack[l]{\\• 16MB Flash\\• Cámara integrada\\• Slot SD integrado} \\ \hline

Desventajas 
& \shortstack[l]{\\• GPIOs limitados\\• Requiere expansión\\para cámara}
& \shortstack[l]{\\• RAM limitada\\• Sin PSRAM\\• Sin conectividad}
& \shortstack[l]{\\• Tamaño mayor\\• Alto consumo\\• Calentamiento} \\ \hline

Imagen 
& 
  \shortstack{\\ \includegraphics[width=1 \linewidth]{Documento/Imagenes/Análisis/Microcontroladores/xiao-esp32-s3-sense.jpg}}
&
  \includegraphics[width=1 \linewidth]{Documento/Imagenes/Análisis/Microcontroladores/STM32f103.jpg}
&  
  \includegraphics[width=0.9 \linewidth]{Documento/Imagenes/Análisis/Microcontroladores/EspCam.jpeg}   
\\ \hline

\end{longtable}

%=====================================================================
%=====================================================================
%=====================================================================


\renewcommand{\arraystretch}{1.5}
\begin{longtable}{|p{3cm}|p{4cm}|p{4cm}|p{4cm}|}
\hline
\multicolumn{4}{|c|}{Parte 3: Microcontroladores y Computadoras de Placa Reducida}\\
\hline
\textbf{Característica} 
& \textbf{Raspberry Pi Zero} \cite{raspberrypi_computers2025}
& \textbf{Raspberry Pi Zero 2 W} \cite{raspberrypi_computers2025}
& \textbf{Heltec Wireless Tracker V1.1} \cite{heltec_wireless_tracker_2023}
\\
\hline
\endfirsthead

\hline
\textbf{Característica} 
& \textbf{Raspberry Pi Zero \cite{raspberrypi_computers2025}}
& \textbf{Raspberry Pi Zero 2 W} \cite{raspberrypi_computers2025}
& \textbf{Heltec Wireless Tracker V1.1} \cite{heltec_wireless_tracker_2023}
\\
\hline
\endhead

\hline
\multicolumn{4}{r}{\textit{Continúa en la siguiente página}} \\
\endfoot

\hline
\endlastfoot

Arquitectura 
& ARM1176JZF-S 
& ARM Cortex-A53 (quad-core) 
& Xtensa 32-bit LX7 (Dual Core)
\\ \hline

Procesamiento 
& 1.0 GHz (single-core) 
& 1.0 GHz (quad-core) 
& 240 MHz
\\ \hline

RAM 
& 512MB LPDDR2 
& 512MB LPDDR2 
& 512KB SRAM
\\ \hline

Flash 
& microSD externa 
& microSD externa 
& 8MB (Integrada)
\\ \hline

Consumo Activo 
& $\sim$ 150 mA
& $\sim$ 250 mA
& $\sim$ 120 mA (Promedio)
\\ \hline

Voltaje & 5V (microUSB) & 5V (microUSB) & 3.3V / 3.7V (LiPo) / 5V (USB-C) \\ \hline

Chipset Principal & BCM2835 & RP3A0 & ESP32-S3 + SX1262 + UC6580 \\ \hline

WiFi 
& Solo en modelo W: 2.4GHz (802.11n) 
& 2.4GHz (802.11n) 
& 2.4GHz (802.11b/g/n)
\\ \hline

Bluetooth 
& BLE 4.0 
& BLE 4.2 
& BLE 5.0 + Mesh
\\ \hline

Sistema Operativo 
& Raspberry Pi OS (Linux) 
& Raspberry Pi OS (Linux) 
& FreeRTOS (Arduino)
\\ \hline

Costo (USD) 
& \$10 
& \$15 
& $\sim$ \$23
\\ \hline

Periféricos 
& \shortstack[l]{\\• 40 GPIOs\\• UART\\• SPI\\• I2C}
& \shortstack[l]{\\• 40 GPIOs\\• UART\\• SPI\\• I2C}
& \shortstack[l]{\\• ADC, UART, SPI, I2C\\• LoRa (SX1262)\\• GNSS (UC6580)\\• Pantalla TFT}
\\ \hline

Ventajas 
& \shortstack[l]{\\• Bajo costo\\• Ecosistema Linux\\• Compacto}
& \shortstack[l]{\\• Multiprocesamiento\\• Ideal para visión\\ artificial compleja}
& \shortstack[l]{\\• \textbf{Alta Integración}\\(LoRa + GPS)\\• Bajo consumo Sleep\\• Ideal para IoT\\ Outdoor}
\\ \hline

Desventajas 
& \shortstack[l]{\\• Procesador antiguo\\• Alto consumo vs MCU}
& \shortstack[l]{\\• Alto consumo energé- \\tico\\• Sin ADC nativo}
& \shortstack[l]{\\• Menos GPIO libres\\• Menor RAM que RPi}
\\ \hline

Imagen 
& 
  \shortstack{\\ \includegraphics[width=0.9\linewidth]{Documento/Imagenes/Análisis/Microcontroladores/zerow.jpg}}
&
  \shortstack{\\ \includegraphics[width=0.9\linewidth]{Documento/Imagenes/Análisis/Microcontroladores/zero2w.jpg}}
&
  \shortstack{\\ \includegraphics[width=0.9\linewidth]{Documento/Imagenes/Análisis/Microcontroladores/micro-heltec.png}}

\\ \hline

\end{longtable}
