\renewcommand{\arraystretch}{1.2}

\begin{longtable}{l l l p{6cm}}
\caption{Diccionario de datos: Tabla \texttt{nodes}.}
\label{tab:dd_nodes} \\

\toprule
\textbf{Columna} & \textbf{Tipo de Dato} & \textbf{Restricción} & \textbf{Descripción} \\
\midrule
\endfirsthead

\toprule
\textbf{Columna} & \textbf{Tipo de Dato} & \textbf{Restricción} & \textbf{Descripción} \\
\midrule
\endhead

\midrule
\multicolumn{4}{r}{\textit{Continúa en la siguiente página}} \\
\endfoot

\bottomrule
\endlastfoot

\texttt{node\_id}     & VARCHAR(20)   & PK                 & Identificador único del nodo (ej. "NODO\_01"). \\
\texttt{description}  & VARCHAR(100)  & NULL               & Descripción o ubicación descriptiva (ej. "Puente Norte"). \\
\texttt{latitude}     & DECIMAL(10,8) & NOT NULL           & Coordenada de latitud fija del nodo. \\
\texttt{longitude}    & DECIMAL(11,8) & NOT NULL           & Coordenada de longitud fija del nodo. \\
\texttt{status}       & VARCHAR(20)   & DEFAULT 'ACTIVE'   & Estado operativo (ACTIVE, INACTIVE, MAINT). \\
\texttt{last\_seen}   & TIMESTAMP     & NULL               & Fecha/hora de la última comunicación recibida. \\

\end{longtable}
