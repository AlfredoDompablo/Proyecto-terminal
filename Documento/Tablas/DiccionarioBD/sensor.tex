\renewcommand{\arraystretch}{1.2}

\begin{longtable}{l l l p{6cm}}
\caption{Diccionario de datos: Tabla \texttt{sensor\_readings}.}
\label{tab:dd_readings} \\

\toprule
\textbf{Columna} & \textbf{Tipo de Dato} & \textbf{Restricción} & \textbf{Descripción} \\
\midrule
\endfirsthead

\toprule
\textbf{Columna} & \textbf{Tipo de Dato} & \textbf{Restricción} & \textbf{Descripción} \\
\midrule
\endhead

\midrule
\multicolumn{4}{r}{\textit{Continúa en la siguiente página}} \\
\endfoot

\bottomrule
\endlastfoot

\texttt{reading\_id}       & BIGSERIAL      & PK         & Identificador autoincremental único de la lectura. \\
\texttt{node\_id}          & VARCHAR(20)    & FK         & Referencia al nodo que generó el dato. \\
\texttt{timestamp}         & TIMESTAMP      & NOT NULL   & Fecha y hora exacta de la toma de muestra. \\
\texttt{battery\_level}    & DECIMAL(4,2)   & NOT NULL   & Nivel de voltaje de la batería (V). \\
\texttt{ph}                & DECIMAL(4,2)   & NULL       & Valor de pH (0.00 -- 14.00). \\
\texttt{temperature}       & DECIMAL(5,2)   & NULL       & Temperatura del agua (\si{\celsius}). \\
\texttt{turbidity}         & DECIMAL(6,2)   & NULL       & Turbidez (NTU). \\
\texttt{dissolved\_oxygen} & DECIMAL(5,2)   & NULL       & Oxígeno disuelto (mg/L). \\
\texttt{conductivity}      & DECIMAL(8,2)   & NULL       & Conductividad eléctrica (\si{\micro\siemens\per\centi\meter}). \\

\end{longtable}
