%=================== Basado en UPIITA ===========================
%=================== Plantilla de informe EPN ===================

\documentclass[12pt,a4paper]{book}
\title{Plantilla para trabajos EPN}
\usepackage{EPN}
\usepackage{STY/slashbox}
%=================== Añade Otro nivel de sección ===================
\usepackage{titlesec}
\setcounter{secnumdepth}{4}
\titleformat{\paragraph}
{\normalfont\normalsize\bfseries}{\theparagraph}{1em}{}
\titlespacing*{\paragraph}
{0pt}{3.25ex plus 1ex minus .2ex}{1.5ex plus .2ex}

\renewcommand{\cleardoublepage}{\clearpage}

\makeatletter
\renewcommand\paragraph{\@startsection{paragraph}{4}{\z@}%
            {-2.5ex\@plus -1ex \@minus -.25ex}%
            {1.25ex \@plus .25ex}%
            {\normalfont\normalsize\bfseries}}
\makeatother
\setcounter{secnumdepth}{4} % how many sectioning levels to assign numbers to
\setcounter{tocdepth}{4}    % how many sectioning levels to show in ToC
%=================== Paquetes para DF ===================
\usepackage{tikz}
\usetikzlibrary{shapes.geometric, arrows}

\tikzstyle{startstop} = [rectangle, rounded corners, 
minimum width=3cm, 
minimum height=1cm,
text centered, 
draw=black, 
fill=red!30]
\tikzstyle{io} = [trapezium, 
trapezium stretches=true, % A later addition
trapezium left angle=70, 
trapezium right angle=110, 
minimum width=3cm, 
minimum height=1cm, text centered, 
draw=black, fill=blue!30]
\tikzstyle{process} = [rectangle, 
minimum width=3cm, 
minimum height=1cm, 
text centered, 
text width=3cm, 
draw=black, 
fill=orange!30]
\tikzstyle{decision} = [diamond, 
minimum width=3cm, 
minimum height=1cm, 
text centered, 
draw=black, 
fill=green!30]
\tikzstyle{arrow} = [thick,->,>=stealth]
%======================================
\asignatura{Asignatura} %Texto en la parte derecha abajo
\carrera{Ing. Ciencias de Computación}
\titulo{Titulo del informe}
\grupo{Grupo}
\alumno{Nombre Estudiante}
\profesor{Nombre Profesor}
\begin{document}
\maketitlepage % Portada
\newpage
\textbf{Resumen:}
Este documento describe el desarrollo de un sistema automatizado para monitorear parámetros de calidad del agua en cuerpos lóticos basado en red inalámbrica de sensores distribuida que mide parámetros esenciales de calidad del agua, como pH, oxigenación, temperatura, turbidez y conductividad, y emplea visión artificial para detectar residuos sólidos en la superficie. Los datos recopilados por los sensores se transmiten a una central de procesamiento y se visualizan en un periodo de tiempo a través de una página web. Esta solución proporcionará datos que permitirán una respuesta más informada ante emergencias ambientales y contribuirá a la sostenibilidad de los cuerpos de agua.

\textbf{Palabras clave}: Red inalámbrica de sensores, parámetros de calidad, monitoreo, visión artificial, página web, recolección de datos.
\newpage


%=============================================================
%=================== Inicio del Documeunto ===================
%=============================================================

%=================== Indices ===================
\tableofcontents %Índice
\listoffigures   % Índice de Figuras
\listoftables    % Índice de Tablas

%\listoffigures   %Índice de figuras
%\newpage        
%\listoftables   %Índice de tablas
%\newpage
%=================== Cuerpo del Documento ===================

\chapter{Introducción}

\section{Introducción}

El agua es un recurso esencial para la vida en el planeta, y su calidad tiene un impacto directo en la salud de los ecosistemas, la biodiversidad y las actividades humanas. A nivel mundial, la contaminación hídrica ha alcanzado niveles críticos, comprometiendo el bienestar de millones de personas y el equilibrio de los ecosistemas acuáticos. En México, la situación es particularmente alarmante: menos del 25\% de las aguas residuales descargadas en cuerpos de agua reciben tratamiento adecuado, según el Fondo para la Comunicación y la Educación Ambiental, agravando significativamente el deterioro de la calidad del agua \cite{agua2006}.

%*************************************Eliminado
%Entre los ejemplos más representativos de esta crisis se encuentra el río Atoyac, uno de los cuerpos de agua más contaminados del país. Desde la década de 1980, este río ha sido severamente impactado por descargas industriales y agrícolas, lo que ha provocado la acumulación de residuos tóxicos, metales pesados y aguas residuales no tratadas. Estas condiciones han incrementado notablemente las enfermedades en las comunidades cercanas, incluyendo casos de cáncer y enfermedades renales \cite{hernandez2021}.

%De manera similar, el río Sonora fue escenario de uno de los desastres ambientales más graves en la historia reciente de México. En 2014, un fallo en el sistema de represas de la mina Buenavista del Cobre, propiedad de Grupo México, liberó 40,000 metros cúbicos de ácido sulfúrico y sulfato de cobre al río. Este derrame contaminó el agua, el suelo y el aire con metales pesados como arsénico, plomo y mercurio, superando los límites establecidos por las normas nacionales e internacionales. A pesar de los esfuerzos de remediación, los efectos persisten, afectando a más de 22,000 personas y deteriorando los medios de vida de las comunidades dependientes del río \cite{diagnostico2023}.
%********************************Eliminado

%Estos casos subrayan la necesidad urgente de implementar tecnologías avanzadas para el monitoreo de la calidad del agua. Los métodos tradicionales, basados en muestreos manuales y análisis de laboratorio, son insuficientes para detectar oportunamente la presencia de contaminantes y residuos sólidos flotantes. La detección de estos residuos, como plásticos y otros desechos visibles, es crucial, ya que no solo obstruyen el flujo del agua, sino que también liberan sustancias tóxicas al descomponerse, agravando la contaminación y afectando a los ecosistemas acuáticos.\cite{monitoreo2023}.


%****************************Agregando del planteamiento*******

En las últimas décadas, los métodos de monitoreo de la calidad del agua han avanzado, pero aún dependen en gran medida de técnicas manuales y análisis de laboratorio. En México, diversos programas buscan involucrar a las comunidades en la detección de cambios en la calidad del agua mediante mediciones rápidas y variables básicas, apoyándose en guías que estandarizan metodologías y validan resultados. Sin embargo, estas mediciones son limitadas, este proceso tarda en realizarse y validarse, y se llevan a cabo en periodos definidos, lo que impide un monitoreo constante.


Para realizar este tipo de monitoreos participativos los interesados necesitan seguir una serie de pasos para poder ser aceptados por las instituciones responsables. Estos pasos incluyen presentar una solicitud formal, cumplir con los requisitos de capacitación y validación técnica, y comprometerse a seguir los protocolos de medición establecidos. Además, es necesario garantizar que los datos recopilados sean fiables y compatibles con los estándares definidos, lo que implica la evaluación continua por parte de capacitadores acreditados y la supervisión institucional. Este enfoque busca no solo fomentar la participación de la sociedad, sino también garantizar que los resultados sean útiles para la gestión y preservación de los ecosistemas acuáticos, sin embargo, la participación comunitaria puede disminuir con el tiempo debido a la falta de incentivos, recursos, reconocimiento, los datos recolectados pueden no ser aceptados oficialmente si no cumplen con los estándares requeridos y para generar la ``primera alerta'' se necesita de un historial de los datos acumulados en el monitoreo y evidencias de respaldo del muestro para corroborar.


Una situación similar sucede para los monitoreos realizados por las diferentes instituciones encargadas de la calidad del agua y que garantizan el cumplimiento de estándares ambientales, son reportes más detallados y formales sobre el tipo de contaminantes y el nivel de contaminación, no obstante, son procesos tardados, dependiendo del tipo de cuerpo de agua los tiempos de monitoreo varían(bimestral, semestral o anual), los resultados de los monitoreos son difíciles de comprender debido a su complejidad técnica para el público y, además, su publicación puede retrasarse durante largos periodos \cite{guiaMonitoreo2024}.


La falta de monitoreo constante en los cuerpos de agua que abastecen al país tiene un impacto severo en los ecosistemas y en la salud pública. Aunque las tecnologías actuales permiten medir parámetros clave como el pH, la oxigenación y la conductividad, estas mediciones se realizan principalmente a través de muestreos manuales y no de forma automatizada. Además, contaminantes como la residuos sólidos no solo obstruyen los cuerpos de agua, sino que también actúan como vectores para la dispersión de contaminantes más peligrosos, como metales pesados y productos químicos.

%referencia: https://www.gob.mx/cms/uploads/attachment/file/925491/Gu_a_de_Monitoreo_Participativo_Primera_Alerta.pdf

%***************************************************añadido

%La combinación de redes inalámbricas de sensores (WSN, por sus siglas en inglés) y visión artificial representa una solución innovadora para superar estas limitaciones. Por un lado, las WSN permiten medir parámetros clave como pH, oxigenación y conductividad de manera distribuida, adaptándose a condiciones variables como cambios de temperatura, climas extremos. Por otro lado, la visión artificial facilita la detección y clasificación de basura flotante, optimizando la identificación de contaminantes visibles. Juntos, estos enfoques proporcionan datos confiables y frecuentes, lo que podría ayudar a autoridades y comunidades locales a tomar decisiones mejor informadas y efectivas.
La combinación de redes inalámbricas de sensores (WSN, por sus siglas en inglés) y visión artificial ofrece una solución prometedora para abordar las limitaciones de los métodos tradicionales. Las WSN permiten la automatización de la medición de parámetros clave como pH, oxigenación y conductividad de manera distribuida, lo que facilita un monitoreo más continuo y preciso del estado de la calidad del agua. Por su parte, la visión artificial mejora la detección de residuos sólidos, contribuyendo a identificar contaminantes visibles en los ríos. En conjunto, estos enfoques podrían generar datos útiles que, eventualmente, apoyarían a las personas interesadas en comprender mejor las condiciones de los cuerpos de agua y fomentar acciones para su preservación.


%Esta investigación tiene como objetivo diseñar e implementar un sistema automatizado que integre redes inalámbricas de sensores y visión artificial para monitorear la calidad del agua y detectar residuos sólidos flotantes en cuerpos de agua. La solución propuesta busca generar información sobre las condiciones hídricas y la presencia de residuos sólidos flotantes, los cuales serán accesibles a través de una página web. Este sistema permitirá a las personas interesadas mantenerse informadas y, al mismo tiempo, apoyará a las autoridades en la toma de decisiones fundamentadas para mejorar la gestión de los recursos hídricos, proteger los ecosistemas acuáticos y promover el bienestar de las comunidades afectadas.

%Esta investigación tiene como objetivo diseñar e implementar un sistema automatizado que integre redes inalámbricas de sensores y visión artificial para monitorear la calidad del agua y detectar residuos sólidos flotantes en cuerpos de agua. La solución propuesta busca generar información relevante sobre las condiciones hídricas y la presencia de residuos sólidos flotantes, accesible a través de una página web. Este sistema no solo permitirá a las personas interesadas mantenerse informadas, sino que también apoyará la toma de decisiones fundamentadas, contribuyendo a mejorar la gestión de los recursos hídricos, proteger los ecosistemas acuáticos y promover el bienestar de las comunidades afectadas. 
Con base en esta propuesta, el sistema busca integrar tecnologías avanzadas para automatizar y ofrecer un monitoreo constante de la calidad del agua en cuerpos de agua lóticos\footnote{Lótico: Masa de agua con flujo constante en una dirección específica, como ríos, arroyos y corrientes\cite{conagua2024}.} (ríos, riachuelos y arroyos). A través de la recopilación de datos clave y su visualización en una plataforma accesible, se espera proporcionar información que facilite el análisis de las condiciones hídricas y la identificación de residuos flotantes. Este enfoque podría ser un recurso valioso para quienes deseen comprender el estado de los cuerpos de agua monitoreados y fomentar acciones orientadas a su conservación así como apoyar a la difusión abierta de información ambiental.



%Este enfoque integrará tecnología avanzada con un diseño orientado a la difusión abierta de información ambiental.

%El estudio detallado completo y la seleccion de las tecnologías 


%Esta investigación tiene como objetivo diseñar e implementar un sistema automatizado que integre redes inalámbricas de sensores y visión artificial para monitorear la calidad del agua y detectar residuos sólidos flotantes en cuerpos de agua. La solución propuesta busca mejorar la gestión de los recursos hídricos, proteger los ecosistemas acuáticos y promover el bienestar de las comunidades afectadas, abordando los desafíos ambientales mediante soluciones concretas y aplicables.

%\newpage
\section{Planteamiento del Problema}

Los métodos tradicionales, como los muestreos manuales y los análisis de laboratorio, son limitados para evaluar de forma continua y precisa la salud de los ríos. Estas técnicas, aunque útiles para obtener mediciones puntuales, no permiten monitorear de forma continua la variación de parámetros de calidad del agua o la acumulación de residuos sólidos flotantes. Esta deficiencia compromete la capacidad de detectar oportunamente problemas ambientales que afectan el flujo natural de los ríos y la calidad del agua, intensificando la liberación de sustancias tóxicas por la descomposición de residuos y poniendo en riesgo la biodiversidad y las comunidades que dependen de estos cuerpos de agua \cite{monitoreo2023}.

Aunque México ha tenido avances significativos en materia de transparencia de datos sobre el agua, en gran parte gracias a colectivos e individuos comprometidos, persiste un reto fundamental: traducir la información técnica disponible en plataformas, bases de datos y registros en formatos comprensibles para el público general. Tal como se señala en el artículo “Calidad del agua en México, un reto vital”, publicado por la revista \textit{Este País}, es necesario transformar dichos datos en mensajes claros que cualquier persona pueda interpretar fácilmente y entender sus implicaciones para la salud y el medio ambiente \cite{estepais2024}.

Ante este desafío, es esencial desarrollar un sistema integral que combine monitoreo automatizado con herramientas accesibles para el procesamiento e interpretación de los datos. Este sistema debe ser capaz de capturar información precisa sobre la calidad del agua y los residuos sólidos flotantes, presentándola de manera clara y útil tanto para las autoridades encargadas de la gestión hídrica como para las comunidades cercanas a los cuerpos de agua. De esta forma, se busca facilitar la toma de decisiones informadas y mejorar la capacidad de respuesta ante riesgos ambientales\cite{monitoreo2023}.

Actualmente se están explorando soluciones basadas en redes de sensores inalámbricos e inteligencia artificial, estas tecnologías aún enfrentan desafíos significativos. Entre los principales desafíos técnicos destacan: (i) la precisión de los sensores en ambientes con alta turbidez; (ii) la capacidad de los sistemas de visión artificial para identificar residuos sólidos bajo condiciones variables de iluminación y visibilidad; y (iii) el manejo eficiente de los grandes volúmenes de datos generados \cite{monitoreo2023}.

Los sensores superan las limitaciones de los métodos manuales al ofrecer mediciones continuas y representativas de las condiciones del agua. Realizan mediciones precisas que minimizan los errores típicos de la recolección, transporte y almacenamiento de muestras, como los cambios en los parámetros causados por el contacto con el aire, la agitación o los retrasos en el análisis. Además, eliminan la necesidad de conservar muestras y permiten un monitoreo constante de los parámetros de calidad del agua, garantizando la recolección continua de datos sin interrupciones.

La visión artificial proporciona una solución automatizada mediante cámaras y algoritmos avanzados para la detección de residuos sólidos flotantes. Estos sistemas procesan imágenes, detectando patrones y acumulaciones de desechos que podrían pasar inadvertidos con métodos tradicionales. Dicha capacidad resulta particularmente valiosa en cuerpos de agua lóticos, donde la presencia de materiales flotantes puede variar rápidamente. Asimismo, permiten el registro y almacenamiento de imágenes para análisis posteriores, facilitando la construcción de series históricas y el estudio de patrones de contaminación.

Este reto trasciende la generación de información, abarcando críticamente su accesibilidad. Por consiguiente, los sistemas de monitoreo deben garantizar no solo la precisión en la recolección de datos, sino también su presentación mediante interfaces visuales, intuitivas y contextualizadas que permitan a autoridades y ciudadanía tomar decisiones basadas en evidencia.

En este contexto, surge la siguiente pregunta de investigación:

\textbf{¿Cómo desarrollar un sistema automatizado de monitoreo, basado en una red inalámbrica de sensores y visión artificial, que permita medir parámetros clave de calidad del agua y detectar residuos sólidos en cuerpos de agua lóticos afectados por actividades industriales y agrícolas?}

%\newpage
\section{Propuesta de Solución}

En respuesta al contexto planteado anteriormente, se propone desarrollar un sistema automatizado para el monitoreo periódico de los parámetros de calidad del agua en cuerpos de agua que se mueven en una sola dirección, como ríos, riachuelos y arroyos. Este sistema integrará una red inalámbrica de sensores distribuidos, junto con un módulo de visión artificial para la detección de posibles residuos sólidos flotantes, como plásticos (PET), envases Tetra Pak, latas o desechos de unicel de un solo uso. La visión artificial procesará imágenes capturadas en intervalos de tiempo predefinidos, lo que permitirá estimar la cantidad de contaminantes flotantes presentes en la superficie del agua. Los sensores medirán parámetros clave de calidad del agua, incluyendo pH, oxigenación, temperatura, turbidez y conductividad, proporcionando información sobre su composición físico-química de forma periódica.

En la figura \ref{Dig:diagrama general} se presenta un diagrama a que muestra el funcionamiento general del sistema, destacando los diferentes módulos que lo componen: nodos sensores, nodo concentrador, almacenamiento en un servidor, así como la visualización a través de la página web.

\begin{figure}[H]
    \centering
    \includegraphics[width=0.9\textwidth]{Documento/Imagenes/Propuesta Sol/Dig_gen.pdf}
    \caption{Diagrama general del sistema.}
    \label{Dig:diagrama general}
\end{figure}


El diseño del sistema  considera las recomendaciones establecidas con las normas \textit{NTC-ISO 5667-1} y \textit{NTC-ISO 5667-10}, que recomiendan el uso de instrumentos automáticos en línea o muestreos regulares para obtener datos representativos. En este marco, la red de sensores con transmisión remota asegura mayor frecuencia y calidad en la información recolectada.  

La frecuencia inicial de monitoreo propúesta es mensual, con posibilidad de reducir intervalos según resultados y recursos, en coherencia con la \textit{NOM-001-SEMARNAT-2021} cuyas exigencias para descargas (municipales y no municipales) varían entre frecuencias mensuales, trimestrales o semestrales según el tipo de carga contaminante \cite{nom001}. Los datos se enviarán a un servidor central para su análisis y publicación en una plataforma web accesible a autoridades y comunidad, esta página permitirá la visualización de la información recopilada sobre la calidad del agua y las condiciones del cuerpo hídrico monitoreado.  

El sistema requiere conectividad de red estable, capacidad de almacenamiento y sensores con protección robusta (IP68/NEMA 6P) frente a condiciones adversas. Además, responde al mandato de la \textit{Ley de Aguas Nacionales} (Art. 86), que designa a la Autoridad del Agua promover, ejecutar y operar sistemas de monitoreo para preservar y mejorar la calidad del agua en cuencas y acuíferos \cite{LAN2024}, al habilitar un monitoreo hídrico constante, confiable y distribuido espacialmente.


\subsection{Módulo 1. Red Inalámbrica de Sensores}
Este módulo consiste en una red inalámbrica de cuatro nodos sensores distribuidos a lo largo del cuerpo de agua. Su función principal es realizar mediciones continuas de parámetros de calidad del agua y detectar residuos sólidos flotantes. La distancia entre nodos dependerá de la tecnología de transmisión empleada.

\subsubsection*{Red}
En aplicaciones ribereñas con cobertura longitudinal, la topología lineal/semi-lineal es esencial, requiriendo comunicación confiable entre nodos adyacentes para transferencias eficientes a lo largo del cauce.

\subsubsection*{Nodos Sensores}
\begin{itemize}  
  \item \textbf{Sensores:} Para medir pH, temperatura, turbidez, oxígeno disuelto y conductividad.
  \item \textbf{Microcontrolador:}Gestiona la adquisición de datos, comunicación y un sistema de visión artificial para detectar residuos sólidos flotantes.
  \item \textbf{Comunicación inalámbrica:} El dispositivo permite enviar y recibir datos para comunicarse con otros dispositivos dentro de su rango.
  \item \textbf{Visión artificial:} Módulo óptico para detección de residuos, con efectividad condicionada a la luz natural. Se evaluarán algoritmos eficientes entrenados para entornos fluviales.
  \item \textbf{Alimentación:} Sistema autónomo mediante baterías y/o paneles solares para alimentar todos los componentes.
\end{itemize}

\subsection{Módulo 2. Nodo Concentrador}
El nodo concentrador actúa como puente entre la red de sensores y el servidor central, cumpliendo una función dual: opera como un nodo sensor estándar y como punto de agregación y comunicación con el servidor.

\begin{itemize}
  \item \textbf{Hardware}
  \item \textbf{Control}
  \item \textbf{Flujo}
\end{itemize}

\subsection{Módulo 3. Servidor}
Es el centro de procesamiento y gestión, compuesto por:
\begin{itemize}
\item \textbf{Base de Datos:} Almacena las series temporales de los parámetros medidos.
\item \textbf{Backend:} Realiza el procesamiento analítico (ej., detección de anomalías).
\item \textbf{Frontend Web:} Sirve la interfaz de usuario para la visualización.
\end{itemize}

\subsection{Módulo 4. Interfaz Web de Monitoreo}
Plataforma web pública y de acceso libre que muestra:
\begin{itemize}
\item \textbf{Calidad del Agua:} Gráficos de series temporales de los parámetros con indicadores de estado.
\item \textbf{Basura Flotante:} Frecuencia de detecciones e histórico de eventos.
\end{itemize}


\begin{comment}
El diseño del sistema considera las recomendaciones establecidas en las normas \textit{NTC-ISO 5667-1} y \textit{NTC-ISO 5667-10}, las cuales reconocen que la variabilidad aleatoria y sistemática en las concentraciones de contaminantes dificulta obtener resultados representativos mediante muestreo manual puntual. Por ello, se recomienda el uso de instrumentos automáticos en línea (on-line) capaces de suministrar análisis continuos \cite{iso5667-10}, o bien realizar muestreos compuestos a intervalos regulares cuando no se dispone de dicha tecnología \cite{iso5667}. En este sentido, la implementación de una red de sensores con capacidad de transmisión remota representa una alternativa alineada a estas directrices normativas y mejora significativamente la frecuencia, oportunidad y calidad de los datos obtenidos.

La frecuencia de monitoreo inicial propuesta es mensual, sustentada en criterios técnicos, operativos y normativos. No obstante, la arquitectura automatizada del sistema permite operar en intervalos menores si se requieren. Esta periodicidad se ajustará conforme al análisis de resultados y disponibilidad de recursos durante la implementación, y resulta coherente con la \textit{NOM-001-SEMARNAT-2021}, cuyas exigencias para descargas (municipales y no municipales) varían entre frecuencias mensuales, trimestrales o semestrales según el tipo de carga contaminante \cite{nom001}.


El sistema transmitirá periódicamente los datos a una central de procesamiento, donde serán analizados y posteriormente publicados en una plataforma web accesible. Esta página permitirá la visualización de la información recopilada sobre la calidad del agua y las condiciones del cuerpo hídrico monitoreado a los interesados como autoridades, personas que residen cerca de los cuerpos de agua y público en general.
La información recopilada se procesará y almacenará en un servidor seguro, para luego publicarse en una plataforma web de interfaz intuitiva y accesibilidad universal.


La implementación exitosa exige dos prerrequisitos funcionales: conectividad de red estable para la transmisión continua de datos e infraestructura de almacenamiento con capacidad de procesamiento. Además, los módulos sensores deben integrar protección robusta (IP68/NEMA 6P) contra agentes hidrodinámicos (corrientes rápidas), exposición química (pH extremo, contaminantes), variables termohigrométricas (humedad, fluctuaciones térmicas) y turbidez elevada, garantizando operatividad confiable y durabilidad en ambientes acuáticos adversos.

Además de abordar necesidades técnicas y ambientales, este sistema cumple con lo establecido en la \textit{Ley de Aguas Nacionales} (Artículo 86), que designa a la Autoridad del Agua promover, ejecutar y operar sistemas de monitoreo para preservar y mejorar la calidad del agua en cuencas y acuíferos \cite{LAN2024}. La propuesta aquí desarrollada contribuye directamente a este mandato legal, ofreciendo una herramienta automatizada que habilita una supervisión hídrica constante, precisa y distribuida espacialmente.

\subsection{Módulo 1. Red Inalámbrica de Sensores}

Este módulo está conformado por una red inalámbrica de sensores, integrada por cuatro nodos sensores distribuidos a lo largo del cuerpo de agua, cuya distancia entre ellos dependerá de la tecnología empleada para la transmisión de datos. Su función principal es realizar mediciones continuas de parámetros de calidad del agua y la detección de residuos sólidos flotantes.


\subsubsection*{\textbf{Red}}

En aplicaciones ribereñas que demandan cobertura longitudinal, la topología lineal/semi-lineal resulta estratégica, influyendo tanto en el diseño del hardware como en la transmisión de datos. En tales escenarios, garantizar una comunicación confiable entre nodos adyacentes es requisito indispensable para transferencias de datos eficientes a lo largo del cauce.

\subsubsection*{\textbf{Nodos Sensores}}

Cada nodo sensor integrará sensores especializados para medir parámetros esenciales de calidad del agua como pH, temperatura, turbidez, oxígeno disuelto y conductividad. El nodo sensor incluye los componentes fundamentales: un microcontrolador, un transceptor inalámbrico, una fuente de alimentación y una cámara, todos interconectados y con capacidad de transmisión de datos remota. 

\begin{itemize}
    \item \textbf{Microcontrolador:} Opera como unidad central gestionando: la adquisición multisensorial, captura de imágenes para visión artificial, y comunicación inter-nodal bidireccional; coordinando el preprocesamiento local y enrutamiento de datos hacia nodos vecinos.
    \item \textbf{Sensores:} Se tendrán sensores que producen una respuesta medible de los parámetros a sensar como el pH, oxigenación, turbidez, temperatura y conductividad en el área de monitoreo.
    
Algunas características a considerar para los sensores: 
    \begin{itemize}
        \item Precisión y sensibilidad: Deben proporcionar mediciones exactas y detectar pequeñas variaciones en los parámetros.
        \item Tamaño compacto: Permite su instalación en un entorno natural variante y manipulación para ajuste.
        \item Materiales resistentes: garantiza la durabilidad del sensor en condiciones adversas.
        \item Bajo consumo energético: Fundamental para su uso en sistemas remotos o alimentados por baterías.
        \item Compatibilidad con sistemas inalámbricos: Para transmitir datos a un servidor o estación base de forma eficiente.
    \end{itemize}    


\item \textbf{Comunicación Inalámbrica:} El dispositivo permite enviar y recibir datos para comunicarse con otros dispositivos dentro de su rango.
    
\item \textbf{Sistema de Visión Artificial:} Emplea un módulo de captura óptica para monitorear la superficie acuática, especializado en la detección de residuos sólidos flotantes. La efectividad del proceso está condicionada a la disponibilidad de luz diurna, factor crítico para garantizar la confiabilidad métrica. Dada esta dependencia, la selección del algoritmo de clasificación resulta determinante para optimizar la precisión del sistema. Se evaluarán arquitecturas eficientes en entornos embebidos (e.g., YOLO, MobileNet) y modelos basados en redes convolucionales, entrenados con datasets específicos de residuos en contextos fluviales.


\item \textbf{Alimentación:} La fuente de alimentación debe ser capaz de alimentar a los sensores, las cámaras, el procesamiento y la comunicación, ya que no se dispone de red eléctrica, hay opciones como baterías y/o mediante placas solares.

\end{itemize}
\subsection{Módulo 2. Nodo Concentrador}
El nodo concentrador opera como punto terminal de la red inalámbrica y nodo sensor avanzado, integrando capacidades adicionales de gestión y comunicación. Cumple una función dual: como elemento sensorial equivalente a los nodos del Módulo 1, y como puente de comunicaciones hacia el servidor central. Su diseño garantiza interoperabilidad completa con la arquitectura de red establecida.

\begin{itemize}
    \item \textbf{Funcionalidad Dual:} 
    Opera simultáneamente como: 
    (i) nodo sensor estándar (adquiriendo parámetros de calidad del agua mediante sensores especializados y detectando residuos con visión artificial), 
    (ii) punto de agregación de datos de la red, y 
    (iii) interfaz de comunicación con el servidor central.
    
    \item \textbf{Arquitectura Hardware:} 
    Mantiene los componentes básicos de los nodos sensores (microcontrolador, transceptor inalámbrico, fuente de alimentación, sensores y módulo de visión artificial), pero incorpora:
    \begin{itemize}
        \item Módulo de comunicación backhaul para conectividad con el servidor.
        \item Memoria ampliada para almacenamiento temporal de datos agregados.
    \end{itemize}
    
    \item \textbf{Microcontrolador:} 
    Requiere especificaciones mejoradas para gestionar:
    \begin{itemize}
        \item Procesamiento concurrente de datos locales y remotos.
        \item Protocolos de comunicación múltiples (intra-red + backhaul).
        \item Mecanismos de priorización de tráfico.
    \end{itemize}
    
    \item \textbf{Flujo de Datos:} 
    Implementa un pipeline de procesamiento donde:
    \begin{enumerate}
        \item Recibe datos de nodos precedentes mediante el protocolo de red.
        \item Agrega información local (sensórica y/o visual).
        \item Organiza paquetes mediante esquemas de codificación eficiente.
        \item Transmite datos consolidados al servidor mediante el canal backhaul.
    \end{enumerate}
    
    \item \textbf{Comunicaciones:} 
    Emplea dos interfaces diferenciadas:
    \begin{itemize}
        \item \textbf{Intra-red:} Transceptor para comunicación con nodos sensores en topología lineal.
        \item \textbf{Backhaul:} Módulo para transmisión al servidor, con mecanismos de retransmisión y QoS adaptativos
    \end{itemize}
\end{itemize}


\subsection{Módulo 3. Servidor}
El servidor central constituye la capa de gestión y presentación del sistema, integrando tres componentes fundamentales: (i) servidor de base de datos para almacenamiento estructurado, (ii) backend de procesamiento analítico, y (iii) frontend web para visualización interactiva. Opera como repositorio centralizado y punto de acceso para monitoreo ambiental.

\begin{itemize}
    \item \textbf{Arquitectura Integrada:}
    \begin{itemize}
        \item \textbf{Servidor de Base de Datos:} Aloja el sistema de gestión de bases de datos (SGBD) relacional para almacenamiento estructurado de:
        \begin{itemize}
            \item Series temporales de parámetros fisicoquímicos
            \item Metadatos de detección de residuos
        \end{itemize}
       \item \textbf{Servidor de Aplicaciones y Web:} 
            Combina la lógica de procesamiento y la capa de presentación en una arquitectura unificada:
            \begin{itemize}
                \item \textit{Backend (Lógica de Negocio):}
                \begin{itemize}
                    \item Ejecuta procesamiento analítico avanzado (correlación de datos, detección de anomalías)
                    \item Gestiona APIs RESTful para comunicación interna/externa
                    \item Administra servicios de autenticación y seguridad
                \end{itemize}
                
                \item \textit{Frontend (Interfaz de Usuario):}
                \begin{itemize}
                    \item Hospeda aplicación web basada en tecnologías modernas (HTML5/CSS3/JavaScript)
                    \item Implementa dashboards interactivos para visualización de datos
                    \item Garantiza experiencia responsive para múltiples dispositivos
                \end{itemize}
        \end{itemize}
    \end{itemize}

    \item \textbf{Flujo de Información:}
    \begin{enumerate}
        \item Recepción de datos consolidados desde el nodo concentrador
        \item Validación y almacenamiento en el SGBD
        \item Procesamiento analítico en el servidor de aplicaciones
        \item Presentación mediante el servidor web a usuarios finales
    \end{enumerate}
\end{itemize}


\subsection{Módulo 4. Interfaz Web de Monitoreo}
Plataforma web de acceso público diseñada para visualizar parámetros de calidad del agua y eventos de detección de basura flotante. Proporciona un dashboard intuitivo, con acceso libre sin requerimiento de autenticación.

\begin{itemize}
    \item \textbf{Arquitectura y Acceso:}
    \begin{itemize}
        \item Diseño responsive basado en frameworks
        \item Acceso universal sin registro
        \item Interfaz simplificada con navegación intuitiva
    \end{itemize}
    
    \item \textbf{Visualización de Parámetros de Calidad del Agua:}
    \begin{itemize}
        \item Gráficos dinámicos de series temporales (pH, temperatura, turbidez, oxígeno disuelto, conductividad)
        \item Paneles comparativos entre diferentes periodos
        \item Indicadores de estado actual con semaforización (normal/advertencia/crítico)
    \end{itemize}
    
    \item \textbf{Monitoreo de Basura Flotante:}
    \begin{itemize}
        \item Indicadores binarios de presencia/ausencia de residuos
        \item Gráficos de frecuencia de detecciones (eventos/hora, eventos/día)
        \item Histórico de eventos con timestamps
        \item Indicadores de confianza promedio en las detecciones
    \end{itemize}
\end{itemize}
\end{comment}

\section{Alcances}
Los aspectos por considerar para el funcionamiento del sistema son los siguientes:

\begin{itemize}

    \item La red inalámbrica contará con al menos 4 nodos sensores, los cuales serán ubicados estratégicamente según la distancia de cobertura entre sensores. Estos nodos permitirán monitorear parámetros como pH, temperatura, turbidez, oxígeno disuelto y conductividad en diferentes puntos del cuerpo de agua.

    \item La comunicación entre los nodos sensores se establecerá con una distancia de al menos 50 metros, mientras que la distancia máxima dependerá de las características topográficas del área de implementación. Se propone el uso de una topología lineal o semi-lineal, dependiendo de la zona del cuerpo de agua elegida para monitorear y su entorno, adaptada al entorno, con la posibilidad de ampliar la cobertura de la red mediante configuraciones más complejas.
    
    \item El sistema de visión artificial detectará exclusivamente residuos sólidos en la superficie del agua; no se identificarán tipos de contaminantes sumergidos o invisibles al rango de la cámara.
    
    \item El sistema no reemplazará la toma de decisiones humanas; su objetivo será únicamente proporcionar información accesible a través de una página web, facilitando al público la comprensión del estado de los cuerpos de agua monitoreados.
    \item El sistema se limitará únicamente al monitoreo de parámetros de calidad del agua y a la detección de residuos sólidos flotantes. No realizará ningún tipo de separación, recolección o tratamiento de los residuos identificados.

\end{itemize}

%\newpage
\section{Objetivo General}
Diseñar un sistema automatizado basado en una red inalámbrica de sensores distribuidos y visión artificial para medir parámetros de calidad del agua y detectar residuos sólidos en cuerpos de agua que se mueven siempre en una misma dirección como ríos, riachuelos y arroyos. 

%Este sistema permitirá la transmisión periódica de los datos a una central, donde serán procesados para facilitar la toma de decisiones informadas por parte de las autoridades competentes, contribuyendo a una gestión más eficiente y precisa de los recursos hídricos.%

\subsection{Objetivos específicos}
\begin{itemize}
 %   \item Diseñar la arquitectura del sistema de monitoreo automatizado. 
    \item Diseñar e implementar una red inalámbrica de sensores equipada con nodos sensores capaces de medir parámetros clave de la calidad del agua, como pH, oxigenación, temperatura, turbidez y conductividad. 
    %Cada nodo sensor contará con una estructura resistente que proteja los sensores y un módulo de visión artificial para detectar basura flotante. 
    %\item Diseñar e implementar una red inalámbrica de sensores para medir parámetros de la calidad del agua, como pH, oxigenación y conductividad así como detectar basura flotante.
    \item Implementar un sistema de visión artificial para analizar imágenes capturadas por las cámaras y detectar residuos flotantes en el agua.
    \item Diseñar nodo concentrador para recolectar y gestionar la información de los sensores, y transmitirla al servidor de manera estructurada.
   % \item Implementar el servidor para almacenar y gestionar los datos recopilados, asegurando su organización y confiabilidad.
    %\item Diseñar una base de datos estructurada para almacenar y organizar los datos provenientes de los nodos sensores, garantizando su accesibilidad y disponibilidad para su posterior análisis.
    \item Diseñar e implementar un servidor con una base de datos estructurada para almacenar, gestionar y organizar los datos recopilados por los nodos sensores, buscando facilitar su confiabilidad, accesibilidad y disponibilidad para su posterior análisis.
    
   % \item Desarrollar una página web para presentar los datos recolectados de forma clara y comprensible, facilitando el acceso del público a la información sobre la calidad del agua y la detección de residuos sólidos.
    \item Desarrollar una página web para presentar los datos recolectados de forma clara y comprensible, facilitando el acceso del público a la información sobre la calidad del agua y la detección de residuos sólidos, la cual será alojada en un servidor local o en la nube.

\end{itemize}

%\newpage
\chapter{Estado del arte}
Esta sección analiza críticamente soluciones existentes en monitoreo acuático, identificando brechas tecnológicas que justifican nuestra propuesta integrada de WSN y visión artificial para ríos.

\section{Sistema telemático para el monitoreo y control de huertos urbanos basado en una red inalámbrica de sensores (Instituto Politécnico Nacional)}

Este proyecto propone el desarrollo de un sistema telemático para el monitoreo y control de huertos urbanos, utilizando una red inalámbrica de sensores (WSN). El sistema monitorea parámetros como temperatura, humedad, pH y la cantidad de nutrientes en el sustrato, con el objetivo de optimizar el riego y el suministro de nutrientes. Está compuesto por tres módulos: una red de sensores que recopila los datos, un servidor para almacenar la información y una aplicación web que permite al usuario visualizar y controlar los actuadores de riego y nutrientes en los cultivos. La implementación de este sistema busca facilitar la agricultura urbana al permitir a los usuarios interactuar de manera remota y tomar decisiones informadas sobre el cuidado de los cultivos, optimizando así el uso de los recursos en entornos urbanos \cite{benitez2024}.

Aunque este sistema se centra en el monitoreo de huertos urbanos, no aborda la complejidad de los entornos acuáticos ni considera parámetros específicos de calidad del agua. El sistema propuesto amplía este enfoque al incorporar tecnologías diseñadas para cuerpos de agua como los ríos, adaptadas a la detección de residuos flotantes y parámetros físicos y químicos del agua.
\newpage

\section{Sistema de detección de nivel de agua basado en una red inalámbrica de sensores (Instituto Politécnico Nacional)}

Este proyecto presenta un sistema basado en sensores inalámbricos que recopila y envía información sobre los niveles de agua a una plataforma en la nube. Utiliza métodos de regresión para predecir futuros niveles de agua y alertar sobre posibles inundaciones. El sistema incluye un tablero de control que muestra los datos de manera gráfica y una aplicación móvil que proporciona alertas. La aplicación clasifica los niveles de agua mediante un sistema de colores (verde, amarillo, naranja y rojo) e informa sobre las opciones de tránsito disponibles. Este enfoque busca mejorar la prevención de inundaciones y optimizar la toma de decisiones por parte de servicios de emergencia mediante la información del nivel del agua y alertas en determinado rango de tiempo basadas en modelos predictivos\cite{perez2024}.

Si bien este sistema utiliza técnicas predictivas para anticipar inundaciones, no incluye la capacidad de monitorear la calidad del agua ni la detección de residuos sólidos flotantes. La propuesta presentada aborda la necesidad de un monitoreo más integral al incluir sensores multiparamétricos y un sistema de visión artificial.

\section{Sistema de detección de basura en la superficie del agua basada en lightweight YOLOv5(Artículo Cientific Reports)}

Se desarrolló un sistema de detección de basura flotante en cuerpos de agua utilizando una versión optimizada del algoritmo YOLOv5, con el objetivo de mejorar la eficiencia en la identificación de residuos en tiempo real. Este sistema, implementado en barcos no tripulados y diseñado para operar en entornos con recursos computacionales limitados, busca incrementar la capacidad de detectar residuos plásticos, envases y otros desechos visibles que pueden dispersar contaminantes peligrosos como metales pesados. Para lograrlo, se empleó una red ligera basada en Shufflenetv2 junto con mecanismos de atención SE (Squeeze and Excitation), mejorando tanto la precisión como la velocidad de detección. Las pruebas realizadas por los investigadores en entornos controlados demostraron que el sistema mantiene una alta precisión incluso en condiciones adversas, haciéndolo adecuado para aplicaciones en el monitoreo de cuerpos de agua \cite{chen2024}.

Este sistema está diseñado para entornos computacionales limitados, pero no contempla la integración con sensores de calidad del agua ni una plataforma web para la visualización de los datos recolectados. La solución planteada combina la detección de residuos con el monitoreo continuo de parámetros de calidad del agua, proporcionando información más completa.

\section{Sistema de monitoreo para cuerpos de agua basado en IoT(Universidad de Cartagena)}

Se desarrolló un sistema de monitoreo para medir la calidad del agua en la ciudad de Cartagena, Colombia, utilizando una red de Internet de las Cosas (IoT). Este sistema emplea sensores que monitorean continuamente parámetros como temperatura, pH, turbidez, conductividad, oxígeno disuelto y sólidos suspendidos, transmitiendo la información en tiempo real a una plataforma central. La solución busca apoyar a las autoridades locales en la toma de decisiones más informadas para la conservación del agua, especialmente en actividades agropecuarias. El sistema fue diseñado para ser accesible y de bajo costo, con boyas flotantes equipadas con paneles solares para garantizar la autonomía energética. Los datos recopilados permiten prevenir la contaminación del agua y facilitar una gestión sostenible de los recursos hídricos \cite{lechuga2023}.

Aunque este sistema incluye monitoreo multiparamétrico, se centra en soluciones de bajo costo y no aborda la detección de residuos flotantes. El sistema que se esta proponiendo complementa esta limitación al integrar un sistema de visión artificial para identificar residuos sólidos visibles.


\section{Evaluación del rendimiento de redes LoRa para sistemas de monitoreo de la calidad del agua (Artículo University of Malaya)}

Se implementó y evaluó un sistema de monitoreo de la calidad del agua basado en la red LoRa (Long Range) para medir parámetros como pH, turbidez, temperatura, sólidos disueltos totales y oxígeno disuelto en dos ríos del campus de la Universidad de Malaya. El sistema utilizó estaciones de monitoreo conectadas a una red LoRa, evaluando el rendimiento en la transmisión de datos en tiempo real bajo diferentes factores de dispersión y condiciones de antena. Los resultados demostraron que LoRa es eficiente para aplicaciones de monitoreo de ríos en entornos urbanos, con una buena calidad de señal y baja pérdida de paquetes, incluso en condiciones sin línea de vista (NLOS) \cite{syed2024}.

Este sistema se enfoca en la eficiencia de transmisión en redes LoRa, pero no incluye un análisis de residuos sólidos flotantes. Esta propuesta amplía este enfoque al integrar sensores de calidad del agua con un sistema de detección de residuos flotantes.

\newpage

\section{Monitoreo de residuos sólidos en las riberas del río San Pedro mediante el reconocimiento de objetos con un dron (Universidad Politécnica Salesiana)}

Este proyecto se centra en la implementación de un sistema de reconocimiento de residuos sólidos en las riberas del río San Pedro utilizando un dron y tecnología de visión artificial. Se desarrolló un modelo que permite identificar tres tipos principales de residuos: contenedores de comida, envases Tetrapak y botellas PET. Mediante la captura de imágenes aéreas con el dron, el sistema procesa los datos utilizando algoritmos de inteligencia artificial para la detección de estos residuos. Se lograron resultados de precisión superiores al 90\% en las pruebas realizadas, lo que demuestra la efectividad del sistema para el monitoreo y la detección de basura en zonas ribereñas\cite{ortega2024}.

Aunque este sistema es eficaz para la detección de residuos sólidos en riberas, su alcance es limitado a imágenes aéreas. El sistema propuesto expande este enfoque al implementar una solución integrada para la calidad del agua y la detección de residuos en tiempo real.

\section{Sistema de detección y clasificación de contaminantes plásticos en entornos acuáticos utilizando un ASV (Universidad de Sevilla)}

Se implementó un sistema de detección y clasificación de basura flotante en entornos acuáticos mediante el uso de un vehículo autónomo de superficie (ASV) equipado con una cámara. El sistema emplea un algoritmo de visión artificial basado en YOLOv5, optimizado para identificar y clasificar residuos plásticos en tiempo real. El ASV fue diseñado para navegar de manera autónoma y recopilar imágenes de la superficie del agua, las cuales son procesadas para detectar objetos como botellas de plástico y otros desechos flotantes. Los resultados obtenidos durante las pruebas demostraron una alta precisión en la identificación de contaminantes, lo que hace que el sistema sea una herramienta efectiva para la limpieza y preservación de cuerpos de agua. Este enfoque contribuye significativamente a la mitigación de la contaminación plástica en los entornos acuáticos \cite{fernandez2024}.

Aunque este sistema utiliza un vehículo autónomo de superficie (ASV) y un algoritmo de visión artificial para detectar contaminantes plásticos flotantes con alta precisión, se limita a residuos visibles y no incluye el monitoreo de parámetros de calidad del agua. El presente trabajo amplía este enfoque al integrar sensores multiparamétricos y una plataforma web, ofreciendo un monitoreo más completo y accesible de los cuerpos de agua.

\section{Sistema de monitoreo de calidad de agua con WSN en el Lago Victoria (Universidad de Dodoma)}  

Se implementó una red de sensores inalámbricos (WSN) para monitorear parámetros de calidad del agua en el Lago Victoria (Tanzania). El sistema utiliza nodos sensores basados en Arduino, equipados con sensores de \textbf{pH}, \textbf{conductividad eléctrica}, \textbf{oxígeno disuelto} y \textbf{temperatura}, transmitiendo datos mediante módulos XBee (ZigBee) a una pasarela central. Esta pasarela emplea GPRS para enviar información a un portal web que visualiza datos en tiempo real. La solución destaca por su bajo costo (USD \$1250) y uso de paneles solares para autonomía energética, demostrando viabilidad en entornos remotos \cite{faustine2014}.


Si bien este sistema ofrece monitoreo multiparamétrico continuo, presenta limitaciones relevantes: carece de capacidades de detección de residuos sólidos flotantes, su arquitectura de red no está optimizada para topologías lineales extensas (ej. ríos), depende de cobertura celular para transmisión de datos, y no integra procesamiento edge para análisis visual. La presente propuesta supera estas limitaciones mediante una topología lineal adaptada a ríos y la integración de visión artificial para detección de basura flotante.

\section{Artículo de redes de sensores inalámbricos para monitoreo de calidad de agua (López-Ramírez y Aragón-Zavala, 2023)}

Este artículo analiza redes de sensores inalámbricos (WSN) aplicadas al monitoreo de calidad de agua (WQM), destacando su superioridad frente a métodos tradicionales en velocidad de respuesta, costo y confiabilidad. Examina arquitecturas de nodos sensores con énfasis en microcontroladores de bajo consumo (STM32L0, ATmega) y protocolos LPWAN (LoRaWAN, Sigfox, NB-IoT) para cobertura extensa. También considera sensores para medir \textbf{pH}, \textbf{turbidez} y \textbf{oxígeno disuelto}, variables clave en la evaluación de cuerpos de agua. El estudio incluye casos de aplicación de tecnologías LPWAN y discute técnicas de \textit{machine learning} para predicción de calidad de agua, aunque no propone un sistema específico de implementación \cite{lopez2023}.


Si bien este artículo ofrece un análisis exhaustivo de tecnologías WSN, presenta limitaciones relevantes para aplicaciones fluviales: omite la integración de visión artificial para detección de contaminantes sólidos, no considera topologías lineales adaptadas a ríos (asumiendo configuraciones puntuales o en malla), y centraliza el procesamiento de datos sin evaluar \textit{edge computing} para análisis visual. La presente propuesta supera estas brechas mediante la integración de YOLOv5 optimizado para detección \textit{on-edge} de basura flotante, una topología lineal jerárquica para cuerpos de agua alargados, y mecanismos adaptativos de procesamiento local/remoto de imágenes.



En la tabla \ref{tab:comparacion_tecnologias} se presenta una comparación de las principales características entre diversos proyectos relacionados con el monitoreo de la calidad del agua y la detección de basura flotante en cuerpos de agua. Se destacan los aspectos clave como el uso de sensores para la medición de parámetros de calidad del agua, la implementación de sistemas de visión artificial para la identificación de residuos, el empleo de redes inalámbricas para la transmisión de datos, y la existencia de plataformas de visualización de la información. El propósito de esta comparación es evidenciar las similitudes y diferencias entre los proyectos previos y el sistema propuesto en este trabajo.



\begin{table}[H]
    \caption{Comparativa de soluciones en monitoreo acuático}
    \label{tab:comparacion_tecnologias}
    \centering
    \renewcommand{\arraystretch}{1.5}
    \scalebox{0.8}{
    \begin{tabular}{|p{3.5cm}|c|c|c|c|p{3.2cm}|}
        \hline
        \textbf{Proyecto} & \textbf{Detección de Basura} & \textbf{Sensores de Agua} & \textbf{Red Inalámbrica} & \textbf{Web} & \textbf{Tecnologías Clave} \\
        \hline
        Sistema telemático para huertos urbanos  \cite{benitez2024} & No & Sí & Sí & Sí & Zigbee, actuadores \\ 
        \hline
        Sistema de detección de nivel de agua  \cite{perez2024} & No & No & Sí & Sí & Sensores ultrasónicos, modelos predictivos, app móvil \\
        \hline
        Lightweight YOLOv5 para detección de basura flotante \cite{chen2024} & Sí & No & No & No & YOLOv5, Shufflenetv2, SE \\ 
        \hline
        Sistema de monitoreo basado en IoT  \cite{lechuga2023} & No & Sí & Sí & Sí & ESP32, LoRa, paneles solares \\ 
        \hline
        Monitoreo de calidad del agua con LoRa  \cite{syed2024} & No & Sí & Sí & No & LoRaWAN, sensores multiparamétricos \\ 
        \hline
        Monitoreo de residuos sólidos con dron  \cite{ortega2024} & Sí & No & No & No & Visión artificial, drones \\ 
        \hline
        Detección de contaminantes con ASV  \cite{fernandez2024} & Sí & No & No & No & ASV, YOLOv5 \\ 
        \hline
        Monitoreo Lago Victoria \cite{faustine2014} & No & Sí & Sí & Sí & Arduino, XBee (ZigBee), GPRS, paneles solares \\
        \hline
        Artículo de WSN para calidad de agua \cite{lopez2023} & No & Sí & Sí & No & LoRaWAN, Sigfox, NB-IoT, STM32, técnicas de ML \\
        \hline
       
    \end{tabular}
    }
\end{table}

En comparación con los proyectos previamente analizados, este trabajo propone una solución integral que combina sensores multiparamétricos y un sistema de visión artificial para monitorear la calidad del agua y detectar residuos flotantes en ríos. También incluye un servidor centralizado y una página web para visualizar los datos, superando las limitaciones de enfoques aislados o unidimensionales observados en otros proyectos. Este enfoque ofrece una herramienta más completa y adaptable a las condiciones específicas de los cuerpos de agua monitoreados.
%\newpage
\chapter{Marco Teórico}

\section{Cuerpo de Agua}
Un cuerpo de agua es cualquier extensión que se encuentran en la superficie terrestre (ríos y lagos) o en el subsuelo (acuíferos, ríos subterráneos); tanto en estado líquido, como sólido (glaciares, casquetes polares); tanto naturales como artificiales (embalses) y pueden ser de agua salada o dulce\cite{agua2024}.

La RENAMECA establece que cuerpos de agua pueden clasificarse como lénticos (agua en reposo como lagos y estanques), lóticos (masas de agua en movimiento constante, como ríos y arroyos), costeros (aguas costeras como playas, bahías, marismas y estuarios) y subterráneos (aguas del subsuelo como pozos, cenotes, galerías y manantiales)

%referencia RENAMECA https://app.conagua.gob.mx/ICA/Contenido?n1=1&n2=1 . 
Este proyecto se enfoca principalmente en cuerpos de agua lóticos debido a la necesidad de monitoreo continuo, ya que en ellos suelen realizarse descargas industriales y agrícolas, albergan una vida acuática importante y, en muchas ocasiones, actúan como proveedores de agua para diversos fines. 

\section{Calidad del Agua}
La calidad del agua se refiere a la condición general y a las características físicas, químicas y biológicas que permiten su uso para un determinado fin, ayudaran a determinar si es apta, por ejemplo, para consumo humano, riego agrícola, protección de la vida acuática o uso recreativo. 
La evaluación de la calidad del agua requiere el monitoreo de parámetros clave como pH, oxígeno disuelto, turbidez, temperatura y conductividad, entre otros parámetros químicos, los cuales proporcionan información crítica sobre el estado del ecosistema acuático y la posible presencia de contaminantes.
La solución propuesta integra una Red Inalámbrica de Sensores (WSN) y un sistema de visión artificial, permitiendo un monitoreo continuo y automatizado que mejora la precisión al evitar errores asociados al transporte y almacenamiento de muestras.

\section{Contaminación}
El término contaminación se refiere a la introducción de cualquier agente químico, físico o biológico cuya presencia o acumulación tiene efectos nocivos en el entorno natural,  la salud y el bienestar de las personas.La magnitud de su impacto generalmente depende de una combinación de aspectos como la cantidad, el tipo de contaminante, la vía de ingreso  y el tipo de medio al que se incorporan.

Se dice que el agua está contaminada cuando los agentes contaminantes repercuten negativamente en su calidad para el consumo humano, para usos posteriores o para el bienestar de los ecosistemas. Es la contaminación que ocurre en cualquier espacio que alberga agua: ríos, lagos, acuíferos o incluso el mar \cite{agua2024contaminacion}. 


\section{Parámetros de la Calidad del Agua}

\subsection{pH}
El pH del agua afecta procesos como la corrosión y las incrustaciones en redes de distribución. Aunque no tiene un impacto directo en la salud, influye en el tratamiento del agua, especialmente en la coagulación y desinfección. Las aguas naturales suelen tener un pH entre 5 y 9 \cite{uno2020}. 


En México, los niveles recomendados de pH para agua potable están regulados por la NOM-127-SSA1-2021, que establece los límites de calidad del agua para uso y consumo humano. Según esta norma, el pH del agua debe encontrarse dentro de un rango de 6.5 a 8.5 para garantizar su potabilidad y evitar problemas como corrosión o incrustaciones en las redes de distribución \cite{uno2020}.

Este rango busca no solo proteger los equipos de distribución, sino también asegurar la eficiencia en los procesos de tratamiento, como la desinfección, y garantizar que el agua sea apta para el consumo humano \cite{uno2020}.

\subsection{Turbiedad}
La turbidez del agua se produce por partículas en suspensión, como arcillas, limo y tierra fina, que reducen su transparencia. Se mide usando turbidímetros o nefelómetros, en Unidades Nefelométricas de Turbidez (UNT). Para mantener la calidad del agua potable, las normas recomiendan no superar 5 UNT, y para una desinfección eficiente, el agua filtrada debería tener menos de 1 UNT\cite{uno2020}

\subsection{Conductividad Eléctrica}
El agua pura es un mal conductor de la electricidad, pero cuando contiene sales se convierte en un buen conductor debido a la presencia de iones con cargas eléctricas. La conductividad electrolítica es una expresión numérica de la capacidad de una solución para transportar una corriente eléctrica. Esta capacidad depende de la presencia de iones, de su concentración total, de su movilidad, valencia y concentraciones relativas, así como de la temperatura \cite{nmx2000}.

La unidad de medición estándar en el Sistema Internacional (SI) es el \textbf{Siemens por metro} ($S/m$). Sin embargo, dado que la conductividad en cuerpos de agua naturales suele presentar valores bajos, en la práctica es más común utilizar el \textbf{microSiemens por centímetro} ($\mu S/cm$) o el \textbf{miliSiemens por centímetro} ($mS/cm$). Existe una relación directa entre la conductividad y la cantidad de Sólidos Disueltos Totales (TDS), utilizándose frecuentemente la medida de conductividad como un indicador indirecto de la salinidad del agua.

\subsection{Temperatura}
La temperatura es un parámetro físico clave en el agua, ya que afecta procesos biológicos y químicos como la actividad microbiana, la absorción de oxígeno, la precipitación de compuestos y la eficiencia en la desinfección. También influye en operaciones de tratamiento como la mezcla, floculación, sedimentación y filtración. La temperatura del agua varía constantemente debido a factores ambientales \cite{uno2020}.

Para el monitoreo de la calidad del agua, la unidad de medida universalmente aceptada es el **grado Celsius** ($^\circ$C). Aunque en el Sistema Internacional (SI) la unidad base de temperatura es el Kelvin (K), en aplicaciones ambientales y normativas (como las establecidas por la SEMARNAT o la EPA) se utilizan los grados Celsius debido a su practicidad para describir condiciones ambientales. La precisión típica requerida para estos sensores en estudios hidrológicos oscila entre $\pm 0.1^\circ$C y $\pm 0.5^\circ$C.

La integración de sensores en la solución propuesta permite monitorear continuamente la temperatura, ajustando las mediciones de otros parámetros como la conductividad, que depende directamente de este valor. Esto asegura un análisis más preciso y adaptado a las condiciones dinámicas de los cuerpos de agua monitoreados.

\subsection{Oxigeno Disuelto}
El oxígeno disuelto (OD) es fundamental para la supervivencia de los organismos acuáticos, ya que la mayoría de las especies dependen de niveles adecuados de oxígeno en el agua. Factores como la temperatura y la actividad biológica influyen en su concentración. A mayor temperatura, menor capacidad del agua para retener oxígeno. Además, la medición del OD puede proporcionar información valiosa sobre la presencia de contaminantes, como materia orgánica en descomposición o vertidos industriales, que disminuyen los niveles de oxígeno en el agua, afectando negativamente a la vida acuática. Los valores recomendados para cuerpos de agua que soportan vida acuática oscilan entre 5 y 8 mg/L, dependiendo de las condiciones locales \cite{swamp2023}.


%\subsection{Sensor de pH}
%Es un dispositivo que se utiliza para medir la acidez o alcalinidad de una solución, mediante la detección de la concentración de iones de hidrógeno.
Los parámetros básicos, como pH, oxígeno disuelto, turbidez, conductividad y temperatura, son indicadores fundamentales del estado de los cuerpos de agua. Estos permiten identificar la presencia de contaminantes, evaluar la salud de los ecosistemas acuáticos y determinar la aptitud del agua para diferentes usos. Su monitoreo es clave para detectar cambios asociados a actividades humanas, como descargas industriales y agrícolas, y para garantizar la sostenibilidad de los recursos hídricos.

\section{Monitoreo del Agua}
Se entiende por el proceso constante de observación, medición y análisis de parámetros que dan indicio de la calidad del agua. El monitoreo del agua permite conocer su calidad a través del tiempo, la cual se determina analíticamente por diferentes parámetros físicos, químicos y biológicos, en función del uso al cual va a ser destinada \cite{monitoreo2023}.


\section{Sensores e Instrumentación}
\label{sec:marco_sensores}

Un sensor se define como un dispositivo de hardware capaz de detectar cambios físicos o químicos en el entorno monitoreado (como variaciones de temperatura, presión o acidez) y producir una respuesta medible. Esta señal, que generalmente es analógica y continua, es digitalizada posteriormente mediante un convertidor analógico-digital (ADC) para ser procesada por un microcontrolador.

Para la implementación eficiente en redes de monitoreo, los sensores deben cumplir con características críticas como un tamaño reducido, bajo consumo energético para garantizar la autonomía, capacidad de operación desatendida y adaptabilidad a las condiciones ambientales \cite{fernandez2009}.

Generalmente, los sensores pueden clasificarse en tres categorías según su método de interacción con el medio:
\begin{itemize}
    \item \textbf{Sensores pasivos omnidireccionales:} Captan datos del entorno sin manipularlo y sin una dirección de enfoque específica. Son autoalimentados y utilizan la energía únicamente para amplificar la señal captada.
    \item \textbf{Sensores pasivos unidireccionales:} Similares a los anteriores, pero con una dirección de captura bien definida; un ejemplo típico son las cámaras de visión artificial.
    \item \textbf{Sensores activos:} Sondean el ambiente emitiendo energía para medir la respuesta del entorno, como los radares o sensores sísmicos que generan ondas expansivas.
\end{itemize}

\subsection{Parámetros de Calidad del Agua}
En el contexto del monitoreo hídrico, se emplean sensores específicos diseñados para cuantificar las propiedades físico-químicas que determinan la salud del ecosistema. A continuación, se describen los principios de operación de los sensores utilizados en este proyecto \cite{waterQualitySensors}:

\begin{itemize}
    \item \textbf{Sensores de pH:} 
    Detectan variaciones en la acidez o alcalinidad del agua, las cuales suelen estar asociadas a vertidos industriales o agrícolas que alteran el equilibrio químico. Estos dispositivos utilizan comúnmente un electrodo de vidrio que mide la actividad de los iones de hidrógeno, generando un voltaje proporcional al nivel de pH detectado.
    
    \item \textbf{Sensores de Turbidez:} 
    Miden la claridad del agua para identificar la presencia de residuos sólidos o partículas en suspensión (sedimentos), a menudo relacionados con actividades de minería o construcción. Su funcionamiento se basa en principios ópticos: emiten un haz de luz a través del agua y miden la cantidad de luz dispersada por las partículas; a mayor dispersión, mayor es el nivel de turbidez.
    
    \item \textbf{Sensores de Temperatura:} 
    Estos dispositivos, generalmente basados en termistores o detectores de temperatura de resistencia (RTD), miden los cambios de resistencia eléctrica en función del calor. Su uso es doble: primero, para detectar contaminación térmica (vertidos industriales) que afecta la fauna; y segundo, para compensar las lecturas de otros sensores, como el de conductividad, que son sensibles a las fluctuaciones térmicas.
    
    \item \textbf{Sensores de Oxígeno Disuelto (OD):} 
    Cuantifican la concentración de oxígeno en el agua, un parámetro vital para la supervivencia acuática cuya disminución indica contaminación orgánica. Utilizan métodos electroquímicos (difusión de oxígeno a través de una membrana generando corriente) o métodos ópticos (medición de la extinción de luminescencia en un tinte sensible al oxígeno).

    \item \textbf{Sensores de Conductividad Eléctrica (EC):} 
    Determinan la capacidad del agua para conducir corriente eléctrica, lo cual es un indicador directo de la cantidad de sólidos disueltos (sales y minerales). Funcionan aplicando una pequeña corriente entre electrodos y midiendo la caída de voltaje resultante para calcular la conductividad del medio.
\end{itemize}

Los sensores representan una solución innovadora para medir parámetros de calidad del agua de forma continua y precisa. A diferencia de los métodos manuales, estos dispositivos permiten la recolección de datos en tiempo real, incluso en condiciones ambientales adversas. Al integrarse en redes inalámbricas, los sensores facilitan la transmisión inmediata de datos a sistemas de procesamiento, lo que optimiza su análisis y visualización
    
\section{Red Inalámbrica de Sensores}

Las redes inalámbricas de sensores (WSN, por sus siglas en inglés) están conformadas por un conjunto de nodos sensores autónomos que se comunican entre sí de manera inalámbrica para monitorear variables del entorno y transmitir los datos recolectados hacia un nodo central o estación base. Estos sistemas son ampliamente utilizados en aplicaciones como monitoreo ambiental, control de infraestructuras, agricultura inteligente, seguridad, entre otros.

Uno de los estándares más utilizados en estas redes es el IEEE 802.15.4, el cual define especificaciones para la capa física y la capa de enlace. Sobre estas capas operan diversos protocolos, como Zigbee, WirelessHART, ISA100.11a y 6LowPAN, que permiten establecer la comunicación entre nodos sensores de manera eficiente y escalable.

En contextos de monitoreo de infraestructuras lineales a gran escala —como ríos, oleoductos o fronteras— se emplea un tipo particular de WSN con topología lineal. Estas redes se caracterizan por disponer los nodos a lo largo de una línea o trayectoria determinada, donde cada nodo actúa como retransmisor hacia el siguiente, formando una cadena de comunicación hasta llegar al nodo colector. Dado que la mayoría de los nodos se encuentra fuera del rango directo del nodo frontera, la comunicación se realiza mediante múltiples saltos (\textit{multi-hop}) \cite{egas2021red}.

\begin{figure}[H]
    \centering
    \includegraphics[width=0.8\linewidth]{Documento/Imagenes/Marco Teorico/Topologia lineal.pdf}
    \caption{Topología Lineal}
    \label{fig:topologia_lineal}
\end{figure}


\subsection{Características de las Redes con Topología Lineal}

Las redes inalámbricas de sensores con topología lineal presentan desafíos y ventajas específicas:

\begin{itemize}
    \item \textbf{Bajo consumo energético:} Los nodos operan con baterías, por lo que los protocolos deben minimizar el procesamiento y las retransmisiones innecesarias para extender la vida útil de la red.
    \item \textbf{Infraestructura fija y predictable:} La ubicación secuencial y estática de los nodos permite asignar identificadores automáticamente sin necesidad de protocolos complejos de direccionamiento.
    \item \textbf{Rutas definidas por defecto:} Al existir una única trayectoria de comunicación hacia el nodo frontera, no es necesario implementar capas de red completas ni algoritmos de enrutamiento convencionales.
    \item \textbf{Escalabilidad:} Estas redes pueden extenderse fácilmente a lo largo de grandes distancias sin necesidad de rediseñar la topología.
    \item \textbf{Limitaciones de cómputo:} Dado que los nodos son de bajo costo y con recursos limitados, se recomienda eliminar funciones innecesarias, como la capa de red, para reducir complejidad y consumo.
\end{itemize}

\begin{comment}
\subsection*{Ventajas para Monitoreo Ambiental}

Este tipo de redes es especialmente adecuado para el monitoreo de cuerpos de agua como ríos, ya que:

\begin{itemize}
    \item Permiten cubrir grandes tramos longitudinales con una cantidad razonable de nodos.
    \item Pueden operar durante largos periodos sin intervención humana gracias a su bajo consumo energético.
    \item Su arquitectura simplificada reduce los costos de implementación y mantenimiento.
\end{itemize}

Por estas razones, las WSN con topología lineal representan una solución viable y eficiente para proyectos de monitoreo ambiental distribuido, como el que se propone en este trabajo.
\end{comment}



\begin{comment}
    

\subsection{Nodos}
Las WSN o redes inalámbricas de sensores están compuestas por nodos de bajo costo y bajo consumo de energía que pueden recolectar información del medio ambiente, procesarla y enviarla a través de conexiones inalámbricas a un nodo central de coordinación. Los nodos actúan como parte de la 
infraestructura de comunicaciones, retransmitiendo los mensajes de los nodos más alejados hasta llegar al centro de coordinación.  La red de sensores inalámbricos está formada por numerosos dispositivos distribuidos espacialmente, que utilizan sensores para monitorear diversas condiciones en distintos puntos, entre ellas la temperatura, el sonido, la vibración, la presión y movimiento o los contaminan
tes. Los sensores pueden ser fijos o móviles\cite{fernandez2008}. Los dispositivos son unidades autónomas que constan de un microcontrolador, una fuente de energía (casi siempre una batería), un
 radio transceptor(RF), un elemento sensor y en de manera opcional actuadores. 

En la figura [\ref{fig:Componentes_nodo}] se muestran los componentes que conforman a un nodo\cite{fernandez2009}.
\begin{figure}[H]
    \centering
    %\includegraphics[width=0.5\linewidth]{nodos.png}
    \includegraphics[width=0.6\linewidth]{Documento/Imagenes/compNodoSen.png}
    %\caption{Componentes que conforman a un nodo \cite{fernandez2008}}
    \caption{Componentes que conforman a un nodo}
    \label{fig:Componentes_nodo}
\end{figure}

\begin{enumerate}
    \item \textbf{Procesador:} 
    Es el componente que interpreta y procesa los datos para transmitirlos a otra estación. También gestiona el almacenamiento de datos en la memoria. Puesto que de un nodo sensor se espera una comunicación y una recogida de datos mediante sensores, debe existir una unidad de procesado que se encargue de gestionar todas estas operaciones \cite{fernandez2009}.
    \item \textbf{Sensor:} 
    Los sensores son dispositivos hardware que producen una respuesta medible ante un cambio en un estado físico, como puede ser temperatura o presión. Los sensores detectan o miden cambios físicos en el área que están monitoreando. La señal analógica continua detectada es digitalizada por un convertidor analógico-digital y enviada a un controlador para ser procesada. Las características y requerimientos que un sensor debe tener son un pequeño tamaño, un consumo bajo de energía, operar en densidades volumétricas altas, ser autónomo y funcionar desatendidamente, además de tener capacidad para adaptarse al ambiente \cite{fernandez2009}.

    \item \textbf{Transceptor:} 
    El dispositivo de comunicación utilizado por los nodos en una WSN es un dispositivo vía radio que se comunica con otros dispositivos dentro de su rango de transmisión. Los nodos utilizan la banda ISM, que es una banda reservada para uso no comercial de radiofrecuencia electromagnética en áreas industriales, científicas y médicas. Esta banda de frecuencia está disponible para todo el mundo sin necesidad de licencia, siempre que se respeten las regulaciones que limitan los niveles de potencia transmitida \cite{fernandez2009}. Los medios a elegir para realizar una comunicación inalámbrica son varios: radiofrecuencia, comunicación óptica mediante láser e infrarrojos. La radiofrecuencia (RF) es la más adecuada para usar en aplicaciones inalámbricas. Las WSN usan las frecuencias de comunicación que van entre 433 MHz y 2.4 GHz. El transceptor es un dispositivo que combina las funciones de emisión y recepción. Tiene diferentes estados de operación, incluyendo el modo de emisión, el modo de recepción, el modo de dormir y el modo de inactividad. En los modelos actuales de transceptor, el modo de inactividad consume casi la misma cantidad de energía que el modo de recepción, por lo que es mejor apagar completamente las comunicaciones de radio cuando no se están emitiendo ni recibiendo. Además, el cambio de modo de dormir a transmisión de datos también consume una cantidad significativa de energía \cite{fernandez2009}.
\end{enumerate}
\end{comment}
\subsection{Tipos de Nodos en una WSN}

Los nodos son los elementos fundamentales de una red inalámbrica de sensores (WSN), y están diseñados para operar de manera autónoma en entornos distribuidos. Cada nodo es capaz de sensar parámetros físicos, realizar un procesamiento básico local y transmitir la información recolectada hacia otros nodos o un nodo recolector (sink), a través de enlaces inalámbricos.
\subsubsection*{Nodo Sensor}
La arquitectura típica de un nodo sensor está compuesta por cuatro módulos principales:

\begin{itemize}
    \item \textbf{Módulo de sensado:} Incluye sensores encargados de capturar variables del entorno, como temperatura, humedad, presión, vibraciones, entre otros. Los datos analógicos son digitalizados mediante un convertidor ADC (Analógico-Digital) para su posterior procesamiento.
    
    \item \textbf{Unidad de procesamiento:} Se encarga de controlar el nodo, ejecutar rutinas de sensado, realizar tareas básicas de filtrado o compresión de datos, y gestionar las operaciones de transmisión.
    
    \item \textbf{Módulo de comunicación inalámbrica:} Utiliza tecnologías como Zigbee, 6LowPAN o protocolos basados en IEEE 802.15.4. Este módulo permite enviar y recibir datos entre nodos en configuraciones ad hoc de múltiples saltos.
    
    \item \textbf{Fuente de energía:} Usualmente se emplean baterías de larga duración. Dado que el reemplazo o recarga puede no ser viable, la eficiencia energética es un factor crítico. Algunas aplicaciones avanzadas incorporan sistemas de recolección de energía ambiental (harvesting).
\end{itemize}

Dependiendo de la aplicación, los nodos pueden estar complementados con otros elementos como sistemas de localización o actuadores. Además, pueden desplegarse de forma fija o móvil, en espacios abiertos, ambientes industriales o entornos hostiles.

En el contexto de aplicaciones como el monitoreo ambiental de ríos, los nodos se despliegan linealmente a lo largo del cauce, formando una red multisalto donde cada nodo transmite sus datos y actúa como repetidor de otros. Esta función requiere una arquitectura eficiente, ya que cada retransmisión incrementa el consumo energético y los retardos de comunicación.

Dado que los nodos operan frecuentemente en condiciones de aislamiento, deben ser tolerantes a fallos. Esto implica que la red debe seguir funcionando aun si algunos nodos fallan por pérdida de energía, daño físico o interferencias ambientales. Por esta razón, los protocolos utilizados deben ser robustos, autoorganizados y adaptables al entorno \cite{perez2014metodologia}.

El diseño del nodo y la elección de sus componentes deben considerar: tamaño, autonomía, capacidad de cómputo, tipo de sensor, alcance de transmisión, y compatibilidad con la topología de red definida (estrella, malla, lineal, etc.). Estas decisiones impactan directamente en la escalabilidad, fiabilidad y costo del sistema completo.

\begin{figure}[H]
    \centering
    \includegraphics[width=0.6\linewidth]{Documento/Imagenes/Marco Teorico/Nodo sensor.pdf}
    \caption{Componentes que conforman a un nodo sensor inalámbrico.}
    \label{fig:Componentes_nodo}
\end{figure}
\subsubsection*{Nodo Concentrador o Puerta de Enlace}

El nodo coordinador, también conocido como \textit{sink node} o puerta de enlace, es el elemento central encargado de recibir los datos transmitidos por los nodos sensores distribuidos en el campo. A diferencia de los nodos comunes, el coordinador suele contar con mayores capacidades de procesamiento, almacenamiento y conectividad, ya que su función principal es consolidar la información generada por la red y transmitirla hacia una red de mayor alcance, como una red local (LAN) o la Internet \cite{perez2014metodologia}.

Este nodo tiene la responsabilidad de:

\begin{itemize}
    \item Establecer y mantener la red inalámbrica, incluyendo la asignación del canal de comunicación y el identificador de red (PAN ID).
    \item Actuar como intermediario entre la red de sensores y los sistemas de supervisión o análisis centralizados.
    \item Coordinar el tráfico de datos, minimizando colisiones y asegurando que los paquetes lleguen de forma ordenada.
    \item En algunos casos, participar en el enrutamiento de los datos cuando la red lo requiere.
\end{itemize}

Dependiendo del diseño de la red, este nodo puede estar físicamente conectado a una computadora, un servidor o una interfaz gráfica de usuario que permite visualizar y procesar los datos en tiempo real. También puede incorporar funcionalidades adicionales como almacenamiento local (mediante dataloggers), sincronización horaria, o reglas de gestión de eventos.

En redes con topología estrella, el nodo coordinador es el punto único de conexión con todos los nodos sensores. En redes lineales o en malla, su papel puede incluir funciones de supervisión del estado de los nodos intermedios, garantizando la integridad de la red.

Debido a su papel crítico, este nodo suele alimentarse con una fuente de energía permanente y estar protegido físicamente, ya que una falla en este componente puede comprometer la operatividad de toda la red.


\begin{comment}

\section{Tecnologías de transmision de datos}
Las redes inalámbricas de sensores utilizan diversos protocolos para la transmisión de datos. A continuación, se describen los más relevantes y ampliamente utilizados:
%https://biblat.unam.mx/hevila/Gerenciatecnologicainformatica/2013/vol12/no33/6.pdf no se si ya esta pero son todos los protocolos
\begin{itemize}
    \item Zigbee (802.15.4):  Este protocolo de radiofrecuencia, basado en el estándar IEEE 802.15.4, se emplea principalmente en aplicaciones domóticas. Ofrece velocidades teóricas entre 40 Kbps y 250 Kbps y es conocido por su bajo costo. Los dispositivos Zigbee tienen un rango de conexión de 10 a 75 metros, dependiendo de la potencia de salida. Opera en tres bandas libres: 868 MHz, 915 MHz y 2.4 GHz \cite{gerenciatecnologicainformatica2013}.
    \item Bluetooth (802.15.1): Tecnología inalámbrica de corto alcance diseñada para eliminar cables entre dispositivos, excepto los de alimentación. Funciona en la banda ISM de 2.4 GHz, lo que permite su uso en cualquier parte del mundo. Es comúnmente utilizado en dispositivos portátiles y fijos, ofreciendo una conexión sencilla y confiable \cite{gerenciatecnologicainformatica2013}.
    \item IrDA (Infrarrojos): Protocolo de comunicación punto a punto que destaca por su bajo costo y bajo consumo energético. Proporciona tasas de transferencia desde 115 Kbps (estándar) hasta 4 Mbps en su versión Fast IR (FIR). Sin embargo, tiene un alcance limitado a un metro y requiere una línea visual directa entre emisor y receptor, con un ángulo de incidencia máximo de 15 grados. No puede atravesar paredes ni obstáculos, y es susceptible a interferencias de luz infrarroja, especialmente bajo luz solar directa \cite{gerenciatecnologicainformatica2013}.
    \item 802.11b/g/n (Wi-Fi): Este conjunto de protocolos permite la transmisión de datos de manera inalámbrica a mayores distancias y velocidades en comparación con los anteriores. Uno de los estándares más comunes, el IEEE 802.11g, introducido en 2003, permite transmisiones de hasta 54 Mbps en la banda de frecuencia de 2.4 GHz, utilizando tecnología OFDM (Orthogonal Frequency Division Multiplexing). Es ampliamente utilizado en redes locales y aplicaciones de alto rendimiento \cite{gerenciatecnologicainformatica2013}.
    
\end{itemize}

Estos protocolos ofrecen diversas capacidades y limitaciones, lo que permite su adaptación a diferentes escenarios y necesidades dentro de las WSN.

La implementación de una red inalámbrica de sensores es fundamental para lograr un monitoreo continuo de los cuerpos de agua. Esta tecnología permite la conexión de múltiples sensores distribuidos en áreas críticas, facilitando la recolección y transmisión de datos de manera continua. Las redes inalámbricas eliminan la necesidad de infraestructura física extensa, lo que las hace ideales para cuerpos de agua dinámicos y de difícil acceso, como ríos y arroyos.
La integración de una WSN asegura que los datos recopilados por los sensores, se transmitan de manera confiable a un servidor central para su procesamiento.
\end{comment}

\section{Tecnologías de Transmisión de Datos Inalámbricos para Redes de Sensores}
Esta sección describe los fundamentos técnicos de protocolos inalámbricos relevantes para redes de sensores, centrándose en sus características en la capa física y de acceso al medio (MAC).

\subsection*{Caracterización Técnica de Protocolos}
\begin{itemize}
    \item \textbf{Zigbee (IEEE 802.15.4)}:
        \begin{itemize}
            \item \textit{Técnica de acceso al medio}: CSMA/CA con slots temporales (beacon-enabled mode)
            \item \textit{Modulación}: DSSS con O-QPSK (2.4 GHz), BPSK (868/915 MHz)
            \item \textit{Tasa de transmisión}: 20-250 kbps (según banda)
            \item \textit{Alcance}: 10-100 m (interiores), hasta 1 km (exteriores con visibilidad)
            \item \textit{Potencia típica}: 0-20 dBm (1-100 mW)
            \item \textit{Aplicaciones}: Domótica, monitoreo ambiental, redes de baja tasa \cite{gerenciatecnologicainformatica2013}
        \end{itemize}
    
    \item \textbf{Bluetooth Low Energy - BLE (IEEE 802.15.1)}:
        \begin{itemize}
            \item \textit{Técnica de acceso al medio}: FHSS con 40 canales + TDMA
            \item \textit{Modulación}: GFSK (1 Mbps), $\pi/4$-DQPSK (2 Mbps en EDR)
            \item \textit{Tasa de transmisión}: 1-3 Mbps (BLE 5.0: 2 Mbps modo corto alcance)
            \item \textit{Alcance}: 10-100 m (dependiendo de clase de potencia)
            \item \textit{Potencia típica}: Clase 1: 20 dBm (100 mW), Clase 2: 4 dBm (2.5 mW)
            \item \textit{Aplicaciones}: Dispositivos portátiles, salud, IoT de consumo \cite{Collotta2017}
        \end{itemize}
    
    \item \textbf{Wi-Fi (IEEE 802.11b/g/n)}:
        \begin{itemize}
            \item \textit{Técnica de acceso al medio}: CSMA/CA con DCF (Distributed Coordination Function)
            \item \textit{Modulación}: DSSS/CCK (802.11b), OFDM (802.11g/n)
            \item \textit{Tasa de transmisión}: 11 Mbps (b), 54 Mbps (g), 600 Mbps (n)
            \item \textit{Alcance}: 35 m (interiores), 100 m (exteriores)
            \item \textit{Potencia típica}: 15-20 dBm (30-100 mW)
            \item \textit{Aplicaciones}: Video vigilancia, transmisión de alto ancho de banda \cite{gerenciatecnologicainformatica2013}
        \end{itemize}
    
    \item \textbf{Wi-Fi HaLow (IEEE 802.11ah)}:
        \begin{itemize}
            \item \textit{Técnica de acceso al medio}: CSMA/CA mejorado con TIM (Target Wake Time) y RAW (Restricted Access Window)
            \item \textit{Modulación}: OFDMA con MIMO (hasta 4 flujos espaciales) y modulaciones BPSK a 256-QAM
            \item \textit{Tasa de transmisión}: 150 kbps - 347 Mbps (dependiendo de ancho de canal)
            \item \textit{Alcance}: Hasta 1 km (exteriores), mayor penetración en obstáculos
            \item \textit{Potencia típica}: 14-20 dBm (25-100 mW) con modos de sueño profundo
            \item \textit{Aplicaciones}: Smart cities, agricultura de precisión, WSN de largo alcance \cite{Adame2014}
        \end{itemize}
    \item \textbf{LoRa (Long Range)}:
        \begin{itemize}
            \item \textit{Técnica de acceso al medio}: ALOHA (en redes LoRaWAN) o CAD (Channel Activity Detection) en implementaciones P2P
            \item \textit{Modulación}: CSS (Chirp Spread Spectrum)
            \item \textit{Tasa de transmisión}: 292 bps - 50 kbps (Adaptable según Factor de Dispersión SF7-SF12 y Ancho de Banda)
            \item \textit{Alcance}: 2-5 km (entornos urbanos), 15 km o más (rural/línea de vista)
            \item \textit{Potencia típica}: +14 dBm a +22 dBm (25-150 mW), altamente eficiente en receptor
            \item \textit{Aplicaciones}: Agricultura inteligente, medición de servicios públicos (metering), rastreo de activos, IoT industrial \cite{TTN_LoRaWAN}
        \end{itemize}
\end{itemize}


 %%%%%%%%%%%%%%%%%%%%%%%%%%%%%%%%%%%%%%%%%%
 %      SECCION: VISION ARTIFICIAL        %
 %%%%%%%%%%%%%%%%%%%%%%%%%%%%%%%%%%%%%%%%%%
 \begin{comment}
 \section{Visión Artificial}
 
 
La \textbf{Inteligencia Artificial } o \textbf{\textit{Intelligent Artificial (IA)}} busca diseñar mecanismos inteligentes y tiene como objetivo desarrollar sistemas que realicen tareas que normalmente requieren de inteligencia humana, como el razonamiento, la toma de decisiones y la percepción. Tiene diferentes técnicas cuando un sistema informático requiere simular el razonamiento humano, toma de decisiones y percepción\cite{iaunam}.
%ref: http://fcaenlinea1.unam.mx/apuntes/interiores/docs/98/7/Apuntes%20de%20Inteligencia%20Artificial.pdf %

La Inteligencia Artificial ha propiciado al aparición de términos: Aprendizaje Profundo(tambien conocido como \textit{Machine Learning, ML}) y Aprendizaje Automático (tambien conocido como \textit{Deep Learning, DL}).que procesan una gran cantidad de datos de modo que permite al algoritmo reconocer, aprender y tomar decisiones mediante la creación de patrones\cite{avila2020plant}\cite{centeno2019deep}. 
%ref : http://ricaxcan.uaz.edu.mx/jspui/bitstream/20.500.11845/1623/1/Articulo%20AMIA_%20STUDY%20AND%20COMPARISON%20OF%20OBJECTS%20DETECTION%20ALGORITHMS%20USING%20CONVOLUTIONAL%20NEURAL%20NETWORKS%20FOR%20PLANT%20DISEASES%20DETECTION%20IN%20LEAVES.pdf 
%

    \begin{figure}[H]
        \centering
        \includegraphics[width=0.5\linewidth]{Documento/Imagenes/Marco Teorico/CamposysubcamposIA.pdf}
        \caption{Campo \textit{Machine Learning} y subcampo \textit{Deep Learning} de la \textit{Inteligencia Artificial}}
        \label{fig:MLDL}
    \end{figure}

El \textbf{\textit{Machine Learning}} es un enfoque dentro de la IA que permite a las máquinas analizar y aprender de los datos mediante el uso de algoritmos sin ser programadas explícitamente para realizar una tarea. Utiliza algoritmos para reconocer patrones en grandes volúmenes de datos y hacer predicciones o tomar decisiones basadas en esos datos \cite{sanchez2020evaluacion}.  
%ref: https://uvadoc.uva.es/bitstream/handle/10324/43277/TFG-G4450.pdf?sequence=1 %

Técnicas de Machine Learning más utilizadas: 
\begin{itemize}
    \item SVM (Máquinas de Soporte Vectorial): Se usan para la clasificación de imágenes al identificar márgenes que separan diferentes clases de datos.
    \item Árboles de Decisión: Utilizados para clasificar y segmentar imágenes basándose en características específicas.
    \item k-NN (k-Nearest Neighbors): Un algoritmo de clasificación que asigna etiquetas a las imágenes según la proximidad a los ejemplos de entrenamiento.
\end{itemize}


El \textbf{\textit{Deep Learning}} es una técnica o subconjunto dentro del campo de \textit{Machine Learning} que emplea redes neuronales profundas para aprender representaciones jerárquicas de los datos, mejorando la capacidad de realizar tareas complejas como el reconocimiento de imágenes y la traducción automática \cite{avila2020plant}.
%ref: http://ricaxcan.uaz.edu.mx/jspui/bitstream/20.500.11845/1623/1/Articulo%20AMIA_%20STUDY%20AND%20COMPARISON%20OF%20OBJECTS%20DETECTION%20ALGORITHMS%20USING%20CONVOLUTIONAL%20NEURAL%20NETWORKS%20FOR%20PLANT%20DISEASES%20DETECTION%20IN%20LEAVES.pdf %
A diferencia de la IA simbólica, donde los humanos proporcionan reglas para procesar datos, en el Aprendizaje Automático los humanos solo suministran datos y las respuestas esperadas. El sistema aprende las reglas que relacionan las entradas con sus salidas, y luego puede aplicarlas a nuevos datos para generar respuestas automáticamente, sin intervención directa de los programadores. El \textit{Deep Learning} se ha convertido en la técnica predominante en visión artificial especialmente adecuada para tareas complejas como la detección de objetos en imágenes \cite{Goodfellow-et-al-2016}\cite{centeno2019deep}.
%ref: file:///C:/Users/itzel_dlz6lv9/Downloads/52%20Deep%20Learning%20autor%20Alba%20Centeno%20Franco.pdf %

La inteligencia artificial (IA) busca dotar a las máquinas de capacidades cognitivas humanas, como la percepción visual. Dentro de la IA, el aprendizaje automático (machine learning) permite a los sistemas aprender de los datos, y el aprendizaje profundo (deep learning), una subdisciplina del aprendizaje automático, utiliza redes neuronales profundas, como las CNNs, para detectar patrones complejos en grandes volúmenes de datos visuales.

Una \textbf{Red Neuronal Artificial (RNA)} es un modelo matemático inspirado en el comportamiento biológico de las neuronas y la estructura del cerebro, utilizado para resolver una amplia variedad de problemas. En una RNA, las neuronas artificiales procesan información de manera similar a las neuronas biológicas. Cada neurona recibe entradas ponderadas, las suma, y pasa el resultado a través de una función de activación. Este proceso permite que la red aprenda patrones y tome decisiones basadas en los datos que recibe\cite{centeno2019deep}.

Las neuronas en una RNA se organizan en capas que facilitan el procesamiento de la información. Las tres capas principales son:
\begin{enumerate}
    \item Capa de entrada: Recibe los datos iniciales y los transmite a la siguiente capa.
    \item Capas ocultas: Realizan el procesamiento intermedio mediante la transformación de las entradas a través de funciones de activación. Pueden ser múltiples en redes profundas.
    \item Capa de salida: Produce el resultado final del modelo, como una clasificación o predicción.
\end{enumerate}

\begin{figure}[H]
        \centering
        \includegraphics[width=0.6\linewidth]{Documento/Imagenes/Marco Teorico/RNA.pdf}
        \caption{Capas de una Red Neuronal Artificial}
        \label{fig:RNA}
\end{figure}

Cada neurona dentro de una capa recibe señales ponderadas de la capa anterior, las procesa y pasa el resultado a la siguiente capa. Este flujo de información permite que las redes neuronales aprendan a reconocer patrones y realizar tareas complejas.
En una RNA, los enlaces sinápticos (flechas que conectan las neuronas) indican el flujo de información entre las neuronas de diferentes capas. Estos enlaces tienen asignados pesos sinápticos, que controlan la influencia de las entradas. El número de capas de una RNA se calcula sumando las capas ocultas y la capa de salida.
En una Red Neuronal Artificial, el algoritmo o método de aprendizaje se refiere al procedimiento que asigna valores a los coeficientes sinápticos (pesos y umbral de activación). 
Existen diferentes tipos de aprendizaje en las Redes Neuronales Artificiales\cite{centeno2019deep}:
\begin{itemize}
    \item \textbf{Aprendizaje Supervisado}: La red aprende a partir de ejemplos con entradas y salidas correctas.
    \item \textbf{Aprendizaje No Supervisado}: La red aprende solo de las entradas, sin salidas correctas.
    \item \textbf{Aprendizaje Híbrido}: Combina aprendizaje supervisado en algunas capas y no supervisado en otras.
    \item \textbf{Aprendizaje por Refuerzo}: La red aprende a través de recompensas o castigos, sin ejemplos de salidas correctas.
\end{itemize}

%ref: Goodfellow, Ian, Yoshua Bengio, and Aaron Courville. Deep Learning. MIT Press, 2016.%
%ref: file:///C:/Users/itzel_dlz6lv9/Downloads/52%20Deep%20Learning%20autor%20Alba%20Centeno%20Franco.pdf %

Las redes neuronales son técnicas que pueden utilizarse tanto en \textit{Machine Learning (ML)} como en \textit{Deep Learning (DL)}. Sin embargo, la diferencia clave radica en su profundidad y capacidad de aprendizaje, en DL tenemos la disponibilidad de conjuntos de datos masivos.

La \textbf{\textit{Visión Artificial}} permite a las máquinas comprender el mundo a partir de imágenes. A través de procesos como la adquisición, el preprocesamiento, la segmentación y el reconocimiento de patrones, las máquinas pueden identificar y clasificar objetos. Esto permite automatizar tareas repetitivas de inspección, controlar la calidad de productos, realizar procesos de inspección sin contacto físico, reducir el tiempo de ciclo en procesos automatizados \cite{visionartificial}.

Se trata de deducir automáticamente la estructura y propiedades de un entorno tridimensional a partir de imágenes bidimensionales. Estas propiedades incluyen tanto aspectos geométricos (como forma, tamaño y ubicación) como materiales (como color, textura e iluminación) de los objetos \cite{visionartificialunirioja}. 

En \cite{visionporcomputador} se definen cuatro fases principales de un sistema de Visión Artificial:
\begin{itemize}
    \item \textbf{Fase sensorial}: Captura de imágenes mediante sensores.
    \item \textbf{Preprocesamiento}: Eliminación de ruido y realce de características importantes.
    \item \textbf{Segmentación}: Aislamiento de regiones de interés en la imagen mediante técnicas de umbralización, detección de bordes o clustering, permitiendo la identificación de objetos dentro de la escena.
    \item \textbf{Reconocimiento/Clasificación}: Identificación de objetos segmentados mediante análisis de características y algoritmos de aprendizaje automático
\end{itemize}

Como muestra la Figura \ref{fig:proceso_vision}, este flujo no es estrictamente secuencial sino iterativo: cuando falla la clasificación, se retrocede a etapas anteriores (segmentación o preprocesamiento), e incluso se repite la captura si se detectan artefactos irreparables en la imagen fuente \cite{visionporcomputador}.

\begin{figure}[H]
    \centering
    \includegraphics[width=0.8\linewidth]{Documento/Imagenes/Marco Teorico/Dig_blo_VA.pdf}
    \caption{Diagrama de bloques de las etapas típicas en un sistema de visión artificial\cite{visionporcomputador}}
    \label{fig:proceso_vision}
\end{figure}

%ref: https://publicaciones.unirioja.es/catalogo/online/VisionArtificial.pdf
%%%%%%%%%%%%%%%%%%%%%%%%%%%%%%%%
%          Tecnicas            %
%%%%%%%%%%%%%%%%%%%%%%%%%%%%%%%%


\subsection{Técnicas y Algoritmos de Visión Artificial}

El documento \cite{visionArtificial2024} presenta diversas metodologías empleadas en el procesamiento y análisis de imágenes. A continuación, se resumen los principales algoritmos y técnicas:

\begin{itemize}
    \item \textbf{Preprocesamiento de Imágenes}:
    \begin{itemize}
        \item \textbf{Filtrado espacial}: Uso de máscaras para resaltar características o reducir el ruido.
        \item \textbf{Transformaciones geométricas}: Ajustes como traslación, rotación y escalado.
        \item \textbf{Corrección de iluminación}: Mejora de la iluminación para uniformidad en los datos visuales.
    \end{itemize}

    \item \textbf{Segmentación de Imágenes}:
    \begin{itemize}
        \item \textbf{Umbralización}: Separación de objetos del fondo.
        \item \textbf{Detección de bordes}: Identificación de contornos con operadores como Sobel o Canny.
        \item \textbf{Regiones de interés (ROI)}: Aislamiento de áreas específicas para análisis detallado.
    \end{itemize}

    \item \textbf{Extracción de Características}:
    \begin{itemize}
        \item \textbf{Análisis de formas}: Cálculo de propiedades geométricas.
        \item \textbf{Textura}: Evaluación de patrones de intensidad.
        \item \textbf{Color}: Uso de la cromática para distinguir y clasificar elementos.
    \end{itemize}

    \item \textbf{Reconocimiento de Patrones}:
    \begin{itemize}
        \item \textbf{Clasificación supervisada}: Uso de algoritmos como SVM o redes neuronales.
        \item \textbf{Clasificación no supervisada}: Técnicas como clustering para agrupar datos.
    \end{itemize}

    \item \textbf{Seguimiento de Objetos}:
    \begin{itemize}
        \item \textbf{Filtros de Kalman}: Estimación de la posición y movimiento en secuencias de imágenes.
        \item \textbf{Algoritmos de correlación}: Comparación de regiones a lo largo del tiempo.
    \end{itemize}
\end{itemize}

En el Reconocimiento de Patrones, la Clasificación Supervisada es una tarea clave dentro del Aprendizaje Supervisado, donde el modelo asigna una etiqueta a cada entrada según las salidas proporcionadas durante el entrenamiento.

Dentro de las Redes Neuronales Artificiales (RNA), las \textbf{Redes Neuronales Profundas} o \textbf{\textit{Deep Neural Network(DNN)}} son fundamentales, están compuestas por múltiples capas que permiten a las máquinas aprender representaciones jerárquicas de los datos. En el caso de las imágenes, las primeras capas pueden detectar bordes simples, mientras que las capas posteriores pueden reconocer patrones más complejos, como formas o incluso objetos completos, sin intervención humana directa en el diseño de esas características\cite{centeno2019deep}.
%ref: https://www.researchgate.net/publication/277411157_Deep_Learning%

Existen diferentes redes dentro de las DNN, como las RNN y sus variantes LSTM/GRU son útiles para tareas secuenciales como reconocimiento de voz, traducción automática y escritura a mano. Las CNN se destacan en reconocimiento de imágenes, análisis de video y procesamiento de lenguaje natural. Las DBN son eficaces en recuperación de información y predicción de fallas, mientras que las DSN ayudan en reconocimiento continuo de voz y clasificación de imágenes. Las GAN son ideales para generación de imágenes y síntesis de datos, y los Transformers son fundamentales en procesamiento de lenguaje natural, traducción automática y generación de texto. Cada una de estas redes está optimizada para tareas específicas y tiene ventajas particulares en diversos campos de la inteligencia artificial\cite{centeno2019deep}.
%ref: file:///C:/Users/itzel_dlz6lv9/Downloads/52%20Deep%20Learning%20autor%20Alba%20Centeno%20Franco.pdf %

En la etapa de Segmentación en un sistema de Visión Artificial podemos encontrar la detección de objetos, es una tarea clave que busca identificar y localizar objetos dentro de una imagen, utilizando métodos como cuadros delimitadores o segmentación. Antes del deep learning, se usaban técnicas como  SIFT (Scale-Invariant Feature Transform) y HOG (Histogram of Oriented Gradients) para extraer características y comparar imágenes con plantillas de objetos. Con la llegada de las Redes Neuronales Convolucionales (CNN), este proceso se simplificó y mejoró, permitiendo la detección automática de características complejas\cite{sanchez2020evaluacion}.
%ref: https://uvadoc.uva.es/bitstream/handle/10324/43277/TFG-G4450.pdf?sequence=1%

\subsection*{Algoritmos para la detección de objetos} 

\textbf{CNN (Redes Neuronales Convolucionales)}: Las redes neuronales convolucionales (CNN, por sus siglas en inglés) son una técnica dentro del campo del aprendizaje profundo (\textit{Deep Learning}). Específicamente, son un tipo de red neuronal artificial que se utiliza para procesar datos con una estructura en forma de rejilla, como las imágenes, la diferencia fundamental entre las redes neuronales convencionales y las redes neuronales convolucionales es que estas últimas están específicamente diseñadas para que los datos de entrada sean imágenes. Son ampliamente utilizadas en tareas de clasificación, detección y segmentación de objetos\cite{centeno2019deep}\cite{sanchez2020evaluacion}\cite{iaavanzada}.

Su arquitectura se compone de múltiples capas, incluyendo capas convolucionales, capas de activación (como ReLU), capas de agrupamiento (\textbf{pooling}) y capas completamente conectadas. Las capas convolucionales aplican filtros a las imágenes para detectar características como bordes, formas y texturas, mientras que las capas profundas permiten la identificación de patrones complejos de alto nivel.
    
    \begin{figure}[H]
        \centering
        \includegraphics[width=0.8\linewidth]{Documento/Imagenes/Marco Teorico/CNN.pdf}
        \caption{Ejemplo del proceso de convolución entre una matriz de entrada y un filtro 3×3. Se ilustra el cálculo del primer elemento del mapa de características resultante (output array).}
        \label{fig:cnnIBM}
    \end{figure}
    
    Gracias a su capacidad de aprendizaje automático y a su robustez frente a variaciones en iluminación, escala o rotación, las CNN se han convertido en el estándar para muchas aplicaciones de visión por computadora con tareas complejas como el reconocimiento y la clasificación, incluyendo el reconocimiento facial, la conducción autónoma y el análisis médico por imágenes \cite{cnnIBM}.

\textbf{R-CNN}: Fue uno de los primeros modelos de detección de objetos basados en aprendizaje profundo. R-CNN es un método de detección de objetos de dos etapas que divide el proceso en dos fases: generación de propuestas de objetos y clasificación de esas propuestas. Primero, utiliza el algoritmo de búsqueda selectiva para generar alrededor de 2000 propuestas de regiones en la imagen. Luego, estas propuestas son procesadas por una red neuronal convolucional (CNN) para extraer características de cada una. Después de esto, se utiliza un clasificador SVM para identificar los objetos en cada región. Aunque R-CNN representó un avance importante en la detección de objetos basada en aprendizaje profundo, su principal desventaja es la lentitud en el proceso, debido a la redundancia en el cálculo de características a través de un gran número de propuestas de regiones superpuestas, lo que hace que el modelo sea muy lento, incluso con aceleración por GPU\cite{wang2024yolosurvey}\cite{sapkota2025yolo}.

\textbf{Fast R-CNN}: Mejoró la eficiencia de R-CNN al integrar la extracción de características y la clasificación en un solo paso, reduciendo el tiempo de procesamiento al eliminar los cálculos redundantes. Esto permitió una detección más rápida y eficiente \cite{sapkota2025yolo}.

\textbf{Faster R-CNN}: Es la evolución de R-CNN y Fast R-CNN. Comenzó con "Rich feature hierarchies for accurate object detection and semantic segmentation" (R-CNN), utilizando un algoritmo de búsqueda selectiva para proponer posibles regiones de interés y una CNN estándar para clasificarlas y ajustarlas. Fast R-CNN, introdujo "Region of Interest Pooling", que permitió compartir cálculos costosos, haciendo el modelo mucho más rápido. Finalmente, Faster R-CNN propuso el primer modelo de esta arquitectura, mejorando la eficiencia en la detección de objetos\cite{centeno2019deep}\cite{sanchez2020evaluacion}.

\textbf{SSD (Single Shot MultiBox Detector)}: SSD realiza tanto la localización como la clasificación de objetos en un solo paso de la red. Utiliza una técnica llamada "MultiBox" para la regresión del cuadro delimitador, haciendo que la red clasifique y detecte objetos de manera eficiente. Mejoró la detección al utilizar cajas delimitadoras predeterminadas de diversas escalas y proporciones, y predecir las puntuaciones de los objetos y ajustar las formas de las cajas. Utiliza mapas de características multiescala para manejar objetos de diferentes tamaños, eliminando la necesidad de un paso separado de generación de propuestas y mejorando el rendimiento en la detección de objetos pequeños. Revolucionó la detección de objetos al simplificar el proceso con un enfoque de una sola etapa, inspirando desarrollos posteriores en los modelos YOLO. A diferencia de los modelos de dos etapas como R-CNN, que dependen de una etapa de propuesta de región antes de la detección real, SSD y, por extensión, las variantes de YOLO, realizan la detección y clasificación en una sola pasada por la imagen. Este cambio de paradigma mejora el proceso de detección al eliminar pasos intermedios, lo que facilita una detección de objetos más rápida y eficiente, adecuada para aplicaciones en tiempo real. La arquitectura de SSD, que los modelos YOLO han adaptado, utiliza múltiples mapas de características a diferentes resoluciones para detectar objetos de varios tamaños, empleando una variedad de cajas ancla en cada ubicación del mapa de características para mejorar la precisión de la localización.

YOLO incorpora los principios arquitectónicos de SSD para mejorar las capacidades de detección en tiempo real mediante una mejor extracción de características utilizando capas de atención multi-cabeza. Esta adopción de la metodología SSD mejora significativamente la velocidad de procesamiento y la precisión de detección de modelos como YOLOv8, YOLOv9 y YOLOv10, haciéndolos ideales para la detección rápida y confiable de objetos en entornos con recursos limitados. El mecanismo eficiente de detección de un solo paso, que clasifica y localiza objetos directamente, destaca la evolución continua de la serie YOLO para cumplir con los requisitos de precisión y velocidad en diversos escenarios del mundo real\cite{sapkota2025yolo}\cite{sanchez2020evaluacion}.

\textbf{YOLO (You Only Look Once)}: Es un algoritmo de detección de objetos que se distingue por su capacidad para realizar esta tarea en un solo paso, a diferencia de métodos previos como R-CNN que emplean un pipeline de múltiples etapas. Al tomar una imagen de entrada y pasarla a través de una red neuronal convolucional (CNN), YOLO predice simultáneamente las coordenadas de las cajas delimitadoras y las clases de los objetos. Este enfoque simplifica el proceso de detección y mejora su eficiencia, abordando la tarea como un único problema de regresión.
Este enfoque permite que YOLO es extremadamente rápido, lo que lo convierte en una opción ideal para aplicaciones en tiempo real, como vehículos autónomos, vigilancia de video y robótica. Además, ha mejorado con el tiempo, ofreciendo mayor precisión, especialmente en la detección de objetos pequeños y en diferentes escalas, sin sacrificar su rapidez.
Si bien su principal ventaja es la detección en tiempo real, también se emplea en áreas donde la velocidad no es crítica. En campos como la agricultura y la medicina, se prioriza la precisión y fiabilidad del modelo, ayudando en tareas como la clasificación de cultivos o la detección de enfermedades, donde la rapidez y procesamiento en tiempo real no es un factor esencial\cite{sanchez2020evaluacion}\cite{terven2023yolo}.

Las versiones de YOLO para la detección de objetos, destacando sus innovaciones clave, son\cite{sapkota2025yolo}\cite{terven2023yolo}:

\begin{itemize}
     
    \item \textbf{YOLOv1 (2016):} Introdujo la detección de objetos en tiempo real con una única red neuronal convolucional (CNN), permitiendo realizar la detección en un solo paso. Fue revolucionario en cuanto a velocidad, pero presentaba limitaciones en la precisión de localización.
    \textbf{Arquitectura:} 24 capas convolucionales, 2 capas completamente conectadas. Utiliza Leaky ReLU - ativación, última capa - activación lineal. Backbone: Darknet24, AP(\%): 63.4.
    
    \item \textbf{YOLOv2/YOLO9000 (2017):} Mejoró la precisión con la introducción de anclajes (anchor boxes) y normalización por lotes, lo que aumentó la estabilidad y rendimiento del modelo. También adoptó una arquitectura completamente convolucional, mejorando la flexibilidad y desempeño. La arquitectura de YOLOv2 mantiene la estructura de YOLOv1, pero con mejoras clave. 
    \textbf{Arquitectura:} 19 capas convolucionales y cinco capas de max pooling capas convolucionales, implementación de anchor boxes para precision. Utiliza Leaky ReLU - ativación, última capa - activación lineal. Backbone: Darknet19, AP(\%): 78.6.
    
    \item \textbf{YOLOv3 (2018):} Incorporó una arquitectura más profunda (Darknet-53) y la capacidad de realizar predicciones multi-escala, mejorando la detección de objetos pequeños y en diferentes escalas. Se enfocó en equilibrar precisión y velocidad sin sacrificar rendimiento. YOLOv3 introdujo tres variantes principales, cada una diseñada para equilibrar el tamaño del modelo y el rendimiento: YOLOv3-spp (Small), Standard y YOLOv3-tiny (Tiny), adaptadas a diferentes compensaciones entre velocidad y precisión.  Este modelo introduce predicción multiescala, utilizando tres diferentes tamaños de grilla para detectar objetos en distintas escalas. El neck es una combinación de PANet (Path Aggregation Network), lo que mejora la fusión de características a través de diferentes niveles. El head predice las cajas y clases en tres escalas diferentes.
    \textbf{Arquitectura:} 53 capas convolucionales y residual connections para facilitar el flujo de gradientes y mejorar el aprendizaje profundo; utiliza Leaky ReLU - ativación, última capa - activación lineal. Backbone: Darknet53; AP(\%):36.2.
    
    \item \textbf{YOLOv4 (2020):} Implementó CSPNet, la función de activación Mish y técnicas avanzadas como SPP (Spatial Pyramid Pooling) y PANet (Path Aggregation Network), lo que mejoró la precisión y eficiencia computacional sin sacrificar la velocidad. YOLOv4 introdujo cuatro variantes principales: la versión estándar, YOLOv4-CSP, que incorpora redes Cross-Stage Partial (CSP) para mejorar el rendimiento y reducir los costos computacionales; YOLOv4x-mish, que utiliza la función de activación Mish para mejorar la precisión mientras mantiene la eficiencia; y YOLOv4-tiny, una versión ligera optimizada para aplicaciones en tiempo real y dispositivos de borde, sacrificando algo de precisión por velocidad. YOLOv4 utiliza el backbone CSPDarknet-53, con Cross-Stage Partial connections que permiten una mayor eficiencia computacional. Esta versión introduce técnicas avanzadas como Mish activation, DropBlock regularization, y Bag of Freebies (BoF) para mejorar la precisión sin aumentar la carga computacional. El neck está compuesto por PANet para una mejor fusión de características multiescala, mientras que el head optimiza la salida usando técnicas como Label Smoothing y CutMix.   
    \textbf{Arquitectura:} 53 capas convolucionales y residual connections para facilitar el flujo de gradientes y mejorar el aprendizaje profundo. Backbone: CSPDarknet-53; AP(\%):43.5.
   
     \item \textbf{Scaled-YOLOv4 (2020):} Utilizó un enfoque de escalado para generar modelos más grandes y pequeños según las necesidades de la aplicación. Scaled-YOLOv4 logró un AP de 56\% en MS COCO con la versión más grande, mientras que su versión más pequeña, YOLOv4-tiny, se ejecutó a 440 FPS en un RTX2080Ti. Mejoró la detección de objetos al eliminar los pasos de preentrenamiento y entrenar desde cero, lo que permitió obtener resultados de alta calidad. Introdujo mejoras clave como CSPNet en PAN, modelos escalables (P5, P6, P7) para dispositivos de borde, y técnicas de escala de modelos combinada para optimizar la eficiencia. Además, utilizó una arquitectura amigable con el hardware y resolvió la inconsistencia de resolución entre modelos preentrenados y datos de entrada mediante un método de escalado eficiente. Estas mejoras le permitieron alcanzar la más alta precisión y velocidad de inferencia en su categoría.
     \textbf{Arquitectura:} Backbone: CSPDarknet; AP(\%):56.0.
     
    \item \textbf{YOLOv5 (2020):} Aunque no es una versión oficial de la serie, se destacó por su facilidad de uso, optimización en Pytorch y un excelente rendimiento en tiempo real. Ofrece versiones escalables (como YOLOv5n, YOLOv5s, etc.) para adaptarse a diferentes necesidades. Introdujo cinco variantes principales para satisfacer diversas necesidades de rendimiento: YOLOv5s (pequeña), optimizada para velocidad y eficiencia en entornos con recursos limitados; YOLOv5m (media), ofreciendo un balance entre velocidad y precisión; YOLOv5l (grande), diseñada para mayor precisión a costa de recursos; YOLOv5x (extra grande), enfocada en la precisión de alto nivel para hardware potente; y YOLOv5n (nano), una versión ligera adaptada para inferencia rápida y bajas demandas computacionales, ideal para aplicaciones en tiempo real y dispositivos de borde.
    \textbf{Arquitectura:} Backbone: ModifiedCSPv7; AP(\%):55.8.
     
    \item \textbf{PP-YOLO (2020):} Basado en YOLOv3, mejoró con aumentos como Mixup y distorsión de colores, logrando un rendimiento sólido en tiempo real. Se mejora sobre YOLOv3, utilizando diversas técnicas de entrenamiento de YOLOv4, además de añadir métodos como CoordConv [39], Matrix NMS [40], y un mejor modelo preentrenado de ImageNet.
    \textbf{Arquitectura:} Backbone: ResNet50-vd; AP(\%):45.9.
    
    \item \textbf{PP-YOLOv2 (2021):} Mejoró PP-YOLO con un cambio de backbone de ResNet50 a ResNet101 y la implementación de una red de agregación de rutas (PAN) en lugar de FPN.  Introduce además el CSPPAN de YOLOv4 escalado y otros mecanismos.
    \textbf{Arquitectura:} Backbone: ResNet50-vd; AP(\%):50.3.
    
    \item \textbf{YOLOR (2021):} Introdujo un enfoque de aprendizaje multitarea, donde se crea un solo modelo para diversas tareas como clasificación, detección y estimación de poses. Utiliza el conocimiento implícito de las redes neuronales para mejorar la eficiencia en múltiples tareas.
    \textbf{Arquitectura:} Backbone: CSPDarknet; AP(\%):55.4.
    
    \item \textbf{YOLOX (2021):} Revertió a una arquitectura sin anclajes y mejoró la precisión mediante center sampling y un head desacoplado para separar las tareas de clasificación y localización. Utilizó aumentaciones fuertes como MixUp y Mosaic, lo que aumentó el AP en 2.4 puntos. 
    \textbf{Arquitectura:} Backbone:  ResNet50-vd; AP(\%):50.3.
    
    \item \textbf{PP-YOLOE (2022):} Mejoró sobre PP-YOLOv2 utilizando una arquitectura sin anclajes y un nuevo backbone y neck con RepResBlocks. Implementó un aprendizaje de alineación de tareas (TAL) y una pérdida focal Varifocal (VFL). Realiza cambios importantes, modificando RepVGG y diseñando el CSPRepResStage, además de usar regresión de cajas delimitadoras en el proceso de regresión basado en distribución de TOOD. 
    \textbf{Arquitectura:} Backbone: ModifiedCSPv5; AP(\%):51.2.

    \item \textbf{YOLOv6 (2020):}  Introdujo un detector sin anclajes y un backbone basado en EfficientRep, optimizado para eficiencia. Mejoró las pérdidas de clasificación y regresión y utilizó auto-destilación para optimizar la precisión. Implementó un esquema de cuantización para acelerar el proceso sin sacrificar precisión, y fue diseñado para dispositivos con bajos recursos. Las variantes incluyen: YOLOv6 (precisión y velocidad equilibradas), YOLOv6-Nano (optimizado para velocidad en tiempo real), y YOLOv6-Tiny (inferencias rápidas en hardware limitado). RepVGG + QAT (Quantization Aware Training).
    \textbf{Arquitectura:} Backbone: EfficientRep; AP(\%):52.5.

    \item \textbf{YOLOv7 (2022):} Superó a otros detectores en velocidad y precisión, introduciendo la E-ELAN para un aprendizaje eficiente y un modelo de escalado adaptativo. Implementó RepConv para mejorar la eficiencia de las convoluciones y YOLOR para mejor generalización. Sus variantes son: YOLOv7 (equilibrio entre velocidad y precisión), YOLOv7-X (mayor rendimiento, más recursos), y YOLOv7-Tiny (ligero para aplicaciones en tiempo real).
    \textbf{Arquitectura:} Backbone: RepConvN; AP(\%):56.8.
    
    \item \textbf{DAMO-YOLO (2022):} Introdujo la búsqueda de arquitectura neuronal (NAS) y optimizó la arquitectura con un neck eficiente llamado Efficient-RepGFPN. Implementó un enfoque de destilación de conocimiento para mejorar la precisión. 
    \textbf{Arquitectura:} Backbone:   MAE-NAS; AP(\%):50.0.
    
    \item \textbf{YOLOv8 (2023):}  Presenta una arquitectura más eficiente, técnicas de entrenamiento mejoradas y soporte para conjuntos de datos más grandes. Su implementación fácil de usar en PyTorch lo hace accesible tanto para la investigación como para la producción, que optimiza el proceso de NMS (Supresión de No Máximos), mejora la precisión y la velocidad, además de implementar una arquitectura sin anclajes, facilitando su rendimiento y flexibilidad. También utiliza aumentos de datos avanzados y técnicas de optimización para mejorar la generalización del modelo. Tiene cuatro variantes YOLOv8-S, optimizada para inferencia rápida en dispositivos de borde con algunas compensaciones en precisión; YOLOv8-M, equilibrando precisión y velocidad para tareas generales; YOLOv8-L, priorizando la precisión a costa de la demanda computacional; y YOLOv8-Tiny, una versión ligera para aplicaciones en tiempo real.
    \textbf{Arquitectura:} Backbone: YOLOv8; AP(\%):53.8.

    \item \textbf{YOLOv9 (2024):} Propone el concepto de información de gradiente programable (PGI) para enfrentar los diversos cambios requeridos por las redes profundas para lograr múltiples objetivos. PGI puede proporcionar información completa de entrada para la tarea objetivo para calcular la función objetivo, de modo que se pueda obtener información de gradiente confiable para actualizar los pesos de la red. Además, se diseñó una nueva arquitectura de red ligera, Generalized Efficient Layer Aggregation Network (GELAN), basada en la planificación de rutas de gradiente. 
    \textbf{Arquitectura:} Backbone:  ResNet50-vd; AP(\%):50.3.

    \item \textbf{YOLOv10 (2024):} Introduce un enfoque novedoso para la detección de objetos en tiempo real, abordando las limitaciones tanto del postprocesamiento como de la arquitectura del modelo en versiones anteriores de YOLO. Al eliminar la supresión de no máximos (NMS) y optimizar componentes clave del modelo, ofrece mejoras significativas en eficiencia y rendimiento. Esta versión introduce seis variantes distintas: YOLOv10-N, YOLOv10-S, YOLOv10-M, YOLOv10-B, YOLOv10-L y YOLOv10-X. Es destacable que YOLOv10-N y YOLOv10-S presentan las latencias más bajas, de 1.84 ms y 2.49 ms, respectivamente, lo que los hace altamente adecuados para aplicaciones que requieren baja latencia. YOLOv10-X logra el mAP más alto de 54.4\% y una latencia de 10.70 ms, reflejando una mejora equilibrada tanto en precisión como en velocidad de inferencia.

    \item \textbf{YOLOv11 (2024): } Avance en la detección de objetos, por su arquitectura sofisticada de backbone y neck para una mejor extracción de características. Optimiza la velocidad y eficiencia mientras mantiene una alta precisión. Equilibra precisión y eficiencia computacional, siendo adecuado para diversas aplicaciones, desde sistemas embebidos hasta implementaciones a gran escala. Tiene cinco variantes: YOLOv11n, YOLOv11s, YOLOv11m, YOLOv11L y YOLOv11x, basadas en la profundidad de la red.

    \item \textbf{YOLOv12 (2025):} Enfoque centrado en la atención. Introduce el módulo de Atención de Área (A2) y Redes de Agregación de Capa Eficiente Residual (R-ELAN) para un procesamiento mejorado de características. Utilizando estos cambios arquitectónicos,logra un rendimiento de vanguardia mientras mantiene las capacidades de detección en tiempo real. Sus variantes YOLOv12-N logró un mAP del 40.6\% con una latencia de inferencia de 1.64 ms en una GPU T4, superando a YOLOv10-N y YOLOv11-N por 2.1 mAP con una velocidad comparable. Mejor definición de contornos de objetos y activación de primeros planos en comparación con sus predecesores.
    
\end{itemize}

\subsection*{Metricas de detección de objetos}
\subsubsection*{mAP}
La Precisión Promedio (AP), tradicionalmente llamada Precisión Promedio Media (mAP), es la métrica comúnmente utilizada para evaluar el rendimiento de los modelos de detección de objetos. Mide la precisión promedio a través de todas las categorías, proporcionando un valor único para comparar diferentes modelos. 

En YOLOv1 y YOLOv2, el conjunto de datos utilizado para el entrenamiento y la evaluación fue PASCAL VOC 2007, y VOC 2012. Sin embargo, desde YOLOv3 en adelante, el conjunto de datos utilizado es Microsoft COCO (Common Objects in Context). La AP se calcula de manera diferente para estos conjuntos de datos. 

La métrica AP se basa en las métricas de precisión y recall, manejando múltiples categorías de objetos y utilizando la Intersección sobre Unión (IoU) para definir predicciones positivas.

\begin{itemize}
    \item Precisión y Recall: La precisión mide la exactitud de las predicciones positivas, mientras que el recall mide la proporción de objetos reales detectados. Hay un compromiso entre ambas, ya que aumentar el recall puede disminuir la precisión. La AP balancea estos factores utilizando la curva precisión-recall, que evalúa la precisión para diferentes umbrales de confianza.
    
    \item Manejo de múltiples categorías: Para evaluar el rendimiento en múltiples categorías de objetos, la AP calcula la precisión promedio de cada categoría y luego promedia estos valores, proporcionando una evaluación más completa del modelo.
    
    \item Intersección sobre Unión (IoU): IoU mide la superposición entre las cajas delimitadoras predichas y las reales, y se utiliza para evaluar la calidad de la localización de los objetos. El conjunto de datos COCO considera múltiples umbrales de IoU para evaluar el rendimiento del modelo en diferentes niveles de precisión en la localización.
\end{itemize}

La AP se calcula de manera diferente en los conjuntos de datos VOC y COCO:

\textbf{VOC}: Para calcular la AP en VOC (20 categorías de objetos), se sigue este proceso:
\begin{enumerate}
    \item Calcular la curva de precisión-recall variando el umbral de confianza.
    \item Calcular la precisión promedio de cada categoría usando una interpolación de 11 puntos.
    \item Promediar las APs de todas las categorías para obtener la AP final.
\end{enumerate}

\textbf{COCO}: Para COCO (80 categorías de objetos), se utiliza un método más complejo:

\begin{enumerate}
    \item Calcular la curva de precisión-recall variando el umbral de confianza.
    \item Calcular la precisión promedio usando 101 umbrales de recall.
    \item Calcular la AP en diferentes umbrales de IoU (de 0.5 a 0.95 con un paso de 0.05).
    \item Promediar las APs de las 80 categorías para cada umbral de IoU.
    \item Calcular el AP general promediando los resultados de todos los umbrales de IoU.
\end{enumerate}

La diferencia en el cálculo de AP hace difícil comparar el rendimiento entre los conjuntos de datos. El estándar actual usa COCO AP debido a su evaluación más detallada en diferentes niveles de IoU.

\textbf{Supresión de No Máximos (NMS)}:
La NMS es una técnica de postprocesamiento que reduce las cajas delimitadoras superpuestas y mejora la calidad de la detección:

\begin{itemize}
    \item Se filtran las cajas predichas usando un umbral de confianza.
    \item Se ordenan las cajas por sus puntajes de confianza en orden descendente.
    \item Se selecciona la caja con el puntaje más alto y se eliminan las cajas restantes si su IoU con la seleccionada excede un umbral predefinido.
\end{itemize}

\textbf{Backbone}: Es responsable de extraer características útiles de la imagen de entrada. Típicamente es una red neuronal convolucional (CNN) entrenada en una tarea de clasificación de imágenes a gran escala, como ImageNet. El backbone captura características jerárquicas en diferentes escalas: características de bajo nivel (como bordes y texturas) en las primeras capas, y características de alto nivel (como partes de objetos e información semántica) en las capas más profundas.

\textbf{Neck}: Es un componente intermedio que conecta el backbone con el head. Su función es agregar y refinar las características extraídas por el backbone, mejorando la información espacial y semántica a través de diferentes escalas. El neck puede incluir capas convolucionales adicionales, redes piramidales de características (FPN), o mecanismos similares para mejorar la representación de las características.

\textbf{Head}: Es el componente final del detector de objetos. Se encarga de realizar las predicciones basadas en las características proporcionadas por el backbone y el neck. Normalmente consiste en una o más subredes específicas para tareas como clasificación, localización y, más recientemente, segmentación de instancias y estimación de poses. El head procesa las características proporcionadas por el neck, generando predicciones para cada objeto candidato. Finalmente, un paso de postprocesamiento, como la supresión de no máximos (NMS), filtra las predicciones superpuestas y retiene solo las detecciones más confiables.

En los modelos YOLO, las arquitecturas se describen utilizando estas tres partes: backbone, neck y head.


% refpara yolo: https://uvadoc.uva.es/bitstream/handle/10324/43277/TFG-G4450.pdf?sequence=1
%file:///C:/Users/itzel_dlz6lv9/Downloads/2304.00501v1.pdf
%https://arxiv.org/pdf/2406.19407
\end{comment}


\section{Visión Artificial}
\label{sec:vision_artificial}

La Visión Artificial es una disciplina dentro de la Inteligencia Artificial (IA) que busca desarrollar sistemas capaces de interpretar y comprender información del mundo visual \cite{szeliski2022computer}. De manera análoga a la visión humana, su objetivo es deducir automáticamente la estructura y propiedades de un entorno a partir de imágenes bidimensionales, permitiendo a las máquinas "ver" e identificar objetos \cite{visionartificialunirioja}.

Un sistema de visión artificial típicamente opera en cuatro fases, como se ilustra en la Figura \ref{fig:proceso_vision}: 1) \textbf{Adquisición} de la imagen, 2) \textbf{Preprocesamiento} para mejorar su calidad, 3) \textbf{Segmentación} para aislar objetos de interés, y 4) \textbf{Reconocimiento y Clasificación} para identificar dichos objetos \cite{visionporcomputador}. Para esta última fase, las técnicas modernas se basan fundamentalmente en el Aprendizaje Profundo.

\begin{figure}[H]
    \centering
    \includegraphics[width=0.8\linewidth]{Documento/Imagenes/Marco Teorico/Dig_blo_VA.pdf}
    \caption{Diagrama de bloques de las etapas típicas en un sistema de visión artificial \cite{visionporcomputador}.}
    \label{fig:proceso_vision}
\end{figure}

\subsection{Del Aprendizaje Automático al Aprendizaje Profundo}
\label{subsec:ml_dl}

La \textbf{Inteligencia Artificial} es el campo general que engloba la creación de máquinas que pueden simular la inteligencia humana \cite{iaunam}. Dentro de la IA, el \textbf{Aprendizaje Automático (\textit{Machine Learning}, ML)} es un subcampo que se enfoca en algoritmos que permiten a los sistemas aprender patrones a partir de datos sin ser programados explícitamente \cite{sanchez2020evaluacion}.

El \textbf{Aprendizaje Profundo (\textit{Deep Learning}, DL)} es, a su vez, una especialización del \textit{Machine Learning} que utiliza \textbf{Redes Neuronales Artificiales (RNA)} con múltiples capas para aprender representaciones jerárquicas de los datos, como se conceptualiza en la Figura \ref{fig:MLDL} \cite{Goodfellow-et-al-2016}. En el contexto de la visión artificial, esto significa que las primeras capas de la red pueden aprender a detectar características simples como bordes o colores, mientras que las capas más profundas aprenden a reconocer patrones complejos como formas, texturas y, finalmente, objetos completos \cite{centeno2019deep}.

\begin{figure}[H]
    \centering
    \includegraphics[width=0.4\linewidth]{Documento/Imagenes/Marco Teorico/CamposysubcamposIA.pdf}
    \caption{Relación jerárquica entre Inteligencia Artificial, \textit{Machine Learning} y \textit{Deep Learning}.}
    \label{fig:MLDL}
\end{figure}

\subsection{Redes Neuronales Convolucionales (CNN)}
\label{subsec:cnn}

Las \textbf{Redes Neuronales Convolucionales (\textit{Convolutional Neural Networks}, CNN)} son una clase de redes neuronales profundas que se han convertido en el estándar de facto para el análisis de imágenes \cite{lecun1998gradient}. Su arquitectura está inspirada en el córtex visual humano y está especialmente diseñada para procesar datos con una estructura de rejilla, como una imagen.

La característica distintiva de una CNN es la \textbf{capa convolucional}, que aplica una serie de filtros (o kernels) a la imagen de entrada. Cada filtro está diseñado para detectar una característica específica. Al deslizar estos filtros por toda la imagen, como se muestra en la Figura \ref{fig:cnn_convolution}, la red crea "mapas de características" que indican dónde se han detectado dichos patrones \cite{cnnIBM}. Esta operación permite a la red ser invariante a la posición de los objetos en la imagen.

\begin{figure}[H]
    \centering
    \includegraphics[width=0.5\linewidth]{Documento/Imagenes/Marco Teorico/CNN.pdf}
    \caption{Ejemplo del proceso de convolución entre una matriz de entrada y un filtro 3×3 \cite{cnnIBM}.}
    \label{fig:cnn_convolution}
\end{figure}

\subsection{Entrenamiento de Modelos y Aprendizaje por Transferencia}
\label{subsec:transfer_learning}

El rendimiento de un modelo de aprendizaje profundo depende intrínsecamente de la calidad y cantidad de los datos con los que se entrena. Entrenar una red neuronal convolucional desde cero para una tarea de visión artificial requiere conjuntos de datos masivos (a menudo con millones de imágenes) y una enorme capacidad de cómputo, lo cual es inviable para la mayoría de los proyectos \cite{pan2009survey}. Para superar este desafío, la práctica estándar es utilizar el \textbf{aprendizaje por transferencia (\textit{Transfer Learning})}.

El aprendizaje por transferencia es una técnica de \textit{Machine Learning} que consiste en tomar un modelo previamente entrenado para una tarea y reutilizarlo como punto de partida para una tarea nueva pero relacionada \cite{bengio2012deep}. La intuición es que un modelo entrenado en un conjunto de datos grande y general, como \textbf{ImageNet} o \textbf{COCO (Common Objects in Context)}, ya ha aprendido a reconocer características visuales universales (bordes, texturas, formas) en sus primeras capas.

El proceso típicamente sigue estos pasos:
\begin{enumerate}
    \item \textbf{Modelo Pre-entrenado:} Se toma un modelo de última generación (como una arquitectura YOLO) que ya ha sido entrenado en un \textit{dataset} a gran escala como COCO, que contiene cientos de miles de imágenes con objetos de 80 clases diferentes \cite{lin2014microsoft}.
    \item \textbf{Afinamiento (\textit{Fine-Tuning}):} En lugar de entrenar la red completa desde cero, solo se re-entrenan las últimas capas (o se entrena toda la red con una tasa de aprendizaje muy baja) utilizando un conjunto de datos mucho más pequeño y específico para el problema a resolver (en este caso, imágenes de basura flotante en ríos).
\end{enumerate}

Esta metodología permite desarrollar modelos de alta precisión con una fracción de los datos y los recursos computacionales que se necesitarían para un entrenamiento desde cero, haciendo que la aplicación de la visión artificial sea accesible y viable para problemas de nicho como el de este proyecto \cite{pan2009survey}.

\subsection{Detección de Objetos con YOLO}
\label{subsec:yolo}
Si bien YOLO fue concebido originalmente para la detección de objetos, las arquitecturas más recientes, como YOLOv8, han evolucionado para convertirse en un \textit{framework} unificado capaz de realizar múltiples tareas de visión artificial. Esto permite utilizar una misma base de código y una metodología de entrenamiento similar para diferentes problemas, optimizando el desarrollo \cite{ultralyticsYOLOv8}. Las tareas principales soportadas son:

\begin{itemize}
    \item \textbf{Detección de Objetos (\textit{Object Detection}):} Es la tarea fundamental y más conocida de YOLO. Consiste en identificar la presencia y la ubicación de múltiples objetos en una imagen, delimitando cada uno con un cuadro (\textit{bounding box}) y asignándole una etiqueta de clase \cite{terven2023yolo}. El resultado es un conjunto de coordenadas del cuadro, una clase y una puntuación de confianza para cada objeto detectado.

    \item \textbf{Segmentación de Instancias (\textit{Instance Segmentation}):} Proporciona una localización mucho más precisa que la detección. En lugar de un cuadro delimitador, el modelo genera una máscara a nivel de píxel que perfila la forma exacta de cada instancia de objeto \cite{ultralyticsYOLOv8}. Esta tarea es computacionalmente más intensiva, pero es crucial para aplicaciones que requieren un análisis preciso de la forma y el área de los objetos, como en la imaginería médica.

    \item \textbf{Clasificación de Imágenes (\textit{Image Classification}):} Es la tarea más simple. Consiste en asignar una única etiqueta a toda la imagen para describir su contenido principal, sin proporcionar información sobre la ubicación del objeto \cite{sapkota2025yolo}. Por ejemplo, determinar si una imagen contiene un "río contaminado" o un "río limpio".

    \item \textbf{Estimación de Pose (\textit{Pose Estimation}):} Se enfoca en la detección de puntos clave (\textit{keypoints}) en un objeto, comúnmente en el cuerpo humano para identificar la posición de las articulaciones (codos, rodillas, etc.). El resultado no es un cuadro ni una máscara, sino un conjunto de coordenadas que definen el "esqueleto" o la postura del sujeto \cite{ultralyticsYOLOv8}. Es ampliamente utilizada en análisis de movimiento, deportes y realidad aumentada.
\end{itemize}

Esta versatilidad consolida a YOLO como una herramienta integral para una amplia gama de aplicaciones en visión por computadora, más allá de la simple detección de objetos.

\subsection{Métricas para la Detección de Objetos}
\label{subsec:metrics}

Para evaluar cuantitativamente el rendimiento de un modelo de detección de objetos, se utiliza un conjunto de métricas estandarizadas que miden la precisión de las localizaciones y las clasificaciones.

\begin{itemize}
    \item \textbf{Intersección sobre Unión (\textit{Intersection over Union}, IoU):} Es la métrica fundamental para evaluar la calidad de una localización. Mide el grado de superposición entre el cuadro delimitador predicho por el modelo ($B_p$) y el cuadro real anotado en los datos (\textit{ground truth}, $B_{gt}$). Se calcula como el área de la intersección dividida por el área de la unión de ambos cuadros \cite{google2024iou}.
    
    $$ \text{IoU} = \frac{\text{Área}(B_p \cap B_{gt})}{\text{Área}(B_p \cup B_{gt})} $$
    
    Una detección se considera un \textbf{Verdadero Positivo (TP)} si su IoU supera un umbral predefinido (comúnmente 0.5 o 50\%). Si no lo supera, se considera un \textbf{Falso Positivo (FP)} \cite{terven2023yolo}.

    \item \textbf{Precisión y Exhaustividad (\textit{Precision \& Recall}):} Son dos métricas clave que evalúan el rendimiento de la clasificación desde diferentes perspectivas \cite{google2024pr}:
    \begin{itemize}
        \item La \textbf{Precisión} responde a la pregunta: de todas las detecciones que hizo el modelo, ¿cuántas fueron correctas? Un valor alto indica una baja tasa de falsos positivos.
        $$ \text{Precisión} = \frac{\text{TP}}{\text{TP} + \text{FP}} $$
        \item La \textbf{Exhaustividad (Recall)} responde a: de todos los objetos que realmente había, ¿cuántos encontró el modelo? Un valor alto indica una baja tasa de falsos negativos.
        $$ \text{Exhaustividad} = \frac{\text{TP}}{\text{TP} + \text{FN}} $$
    \end{itemize}
    
    \item \textbf{Precisión Media Promedio (\textit{Mean Average Precision}, mAP):} Es la métrica estándar para evaluar el rendimiento general de un detector de objetos. El mAP resume la curva de Precisión-Exhaustividad en un solo valor, representando la precisión promedio en todos los niveles de exhaustividad. Este valor se calcula para cada clase de objeto y luego se promedia entre todas las clases para obtener el mAP del modelo \cite{sapkota2025yolo}. En conjuntos de datos como COCO, el mAP se reporta a menudo promediado sobre varios umbrales de IoU (ej., de 0.5 a 0.95), lo que proporciona una evaluación aún más robusta \cite{terven2023yolo}.

    \item \textbf{Supresión de No Máximos (\textit{Non-Maximum Suppression}, NMS):} Aunque no es una métrica, es un paso de post-procesamiento indispensable. Un modelo a menudo detecta el mismo objeto varias veces con cuadros delimitadores ligeramente diferentes. NMS se encarga de eliminar estas detecciones redundantes, conservando únicamente la detección con la puntuación de confianza más alta para cada objeto \cite{sapkota2025yolo}.
\end{itemize}












%%%%%%%%%%%%%%%%%%%%%%%%%%%%


\subsection{Cámaras}
\label{subsec:camaras}

En el contexto de la visión artificial, la cámara es el dispositivo de adquisición de imagen, actuando como el ``ojo" del sistema. Su función principal es capturar la radiación electromagnética (luz) reflejada por los objetos en la escena y convertirla en una señal electrónica que puede ser procesada por un sistema digital \cite{szeliski2022computer}.

Un módulo de cámara, especialmente en sistemas embebidos, se compone de varios elementos clave:

\begin{itemize}
    \item \textbf{La Óptica (Lente):} Es el conjunto de lentes que enfoca la luz de la escena sobre el sensor. Sus propiedades, como la \textbf{distancia focal} y la \textbf{apertura}, determinan el campo de visión (\textit{Field of View}, FoV) y la cantidad de luz que se captura, afectando directamente el área del río que el sistema puede monitorear \cite{szeliski2022computer}.

    \item \textbf{El Sensor de Imagen:} Es el componente semiconductor que convierte los fotones en una señal eléctrica. Los dos tipos de sensores más comunes son \textbf{CCD} (\textit{Charge-Coupled Device}) y \textbf{CMOS} (\textit{Complementary Metal-Oxide-Semiconductor}). En aplicaciones de IoT y sistemas embebidos como este proyecto, los sensores CMOS son los más utilizados debido a su menor consumo de energía, mayor integración de circuitos en el chip y menor costo \cite{fossum1997cmos}.

    \item \textbf{La Interfaz de Datos:} Es el protocolo mediante el cual el sensor transmite el flujo de datos de imagen al microcontrolador o procesador. Las interfaces comunes en sistemas embebidos incluyen \textbf{MIPI CSI-2} (Mobile Industry Processor Interface Camera Serial Interface), DVP (Digital Video Port) o, en algunos casos, USB \cite{mipi2019csi}.
\end{itemize}

La selección de una cámara adecuada, por lo tanto, depende de un balance entre la resolución del sensor, las características de la óptica y la compatibilidad de su interfaz con la unidad de procesamiento del nodo sensor.

%%%%%%%%%%%%%%%%%%%%%%%%%%%%%%%%%%%%%
%       SECCION PAGINA WEB          %
%%%%%%%%%%%%%%%%%%%%%%%%%%%%%%%%%%%%%

\section{Página Web y Aplicación Web}
\label{sec:pagina_web}

Una \textbf{página web} es un documento digital, comúnmente escrito en Lenguaje de Marcado de Hipertexto (HTML), diseñado para ser accesible a través de un servidor de Internet mediante un navegador \cite{niebla2013creación}. Una colección de páginas web interrelacionadas bajo un mismo dominio constituye un \textbf{sitio web}, cuyo propósito principal es la diseminación de información.

Sin embargo, para un sistema de monitoreo interactivo como el que se propone, es más preciso hablar de una \textbf{Aplicación Web} (\textit{Web Application}). A diferencia de un sitio web estático, una aplicación web es un sistema cliente-servidor diseñado para ser interactivo. El navegador (el cliente) obtiene la interfaz inicial del servidor, pero luego puede modificar su contenido y comunicarse con el servidor de forma dinámica sin necesidad de recargar la página por completo \cite{lujan2011programacion, mdn2024web}.

Las aplicaciones web modernas se construyen sobre tres tecnologías fundamentales:

\begin{itemize}
    \item \textbf{HTML (\textit{HyperText Markup Language}):} Proporciona la estructura semántica y el contenido base de la aplicación (títulos, párrafos, botones, contenedores de gráficos) \cite{mdn2024html}.
    \item \textbf{CSS (\textit{Cascading Style Sheets}):} Se encarga de la presentación, el diseño y el estilo visual (colores, tipografías, posicionamiento) para hacer que la interfaz sea responsiva y clara \cite{mdn2024css}.
    \item \textbf{JavaScript:} Es el lenguaje de programación del lado del cliente que dota a la aplicación de interactividad y comportamiento. En este proyecto, JavaScript será responsable de tareas clave como: solicitar datos al servidor (vía API), actualizar dinámicamente los valores de los sensores y renderizar las gráficas de series temporales \cite{mdn2024javascript}.
\end{itemize}

En el contexto de este proyecto, la aplicación web permitirá centralizar y presentar de forma clara y accesible los datos recopilados sobre calidad del agua y detección de residuos. Facilitará la consulta dinámica por parámetros, fechas o ubicaciones, fomentando la transparencia, la conciencia pública y la participación ciudadana.

%%%%%%%%%%%%%%%%%%%%%%%%%%%%%%%%%%%%%
%       SECCION Servdores           %
%%%%%%%%%%%%%%%%%%%%%%%%%%%%%%%%%%%%%

\section{Servidor}

En el ámbito de la informática, un servidor es un sistema de cómputo —compuesto por hardware y software— que provee recursos, datos, servicios o programas a otros dispositivos, denominados clientes, a través de una red \cite{Espana2003}. Esta interacción se rige por el \textbf{modelo cliente-servidor}, una arquitectura de software distribuida que centraliza la gestión de recursos y la lógica de negocio en el servidor, mientras que los clientes se encargan de la interfaz de usuario y la solicitud de servicios \cite{Britannica2024}.

Las características esenciales que definen la robustez y fiabilidad de un servidor incluyen:

\begin{itemize}
    \item \textbf{Alta disponibilidad}: Se refiere a la capacidad del servidor para operar de forma continua y sin interrupciones durante largos periodos. Esto se logra mediante hardware redundante (ej. fuentes de alimentación, discos) y software de conmutación por error (\textit{failover}) \cite{Stanek2014}.
    \item \textbf{Escalabilidad}: Es la capacidad del sistema para incrementar su rendimiento y capacidad para manejar una mayor carga de trabajo, ya sea de forma vertical (añadiendo más recursos a un solo servidor, como CPU o RAM) o horizontal (distribuyendo la carga entre múltiples servidores) \cite{Mancera2015}.
    \item \textbf{Seguridad}: Implica la implementación de múltiples capas de protección, incluyendo firewalls de red, sistemas de detección de intrusiones, cifrado de datos en tránsito y en reposo, y políticas de control de acceso para proteger la integridad y confidencialidad de la información \cite{Espana2003}.
\end{itemize}

\subsection{Modelos de Despliegue de Servidores}
\label{subsec:modelos_despliegue_marco}

La implementación de una arquitectura cliente-servidor se puede realizar mediante dos modelos de despliegue principales, cuya diferencia radica en la propiedad y gestión de la infraestructura física.

\begin{itemize}
    \item \textbf{Servidores locales (On-Premise)}: En este modelo, la organización adquiere, instala y mantiene su propia infraestructura de hardware en sus instalaciones físicas. Esto implica una inversión inicial significativa en capital (\textbf{CapEx}) y la responsabilidad total sobre la administración, el mantenimiento y la seguridad de los servidores \cite{Mullins2012}. Ofrece un control granular máximo sobre los datos y el entorno operativo.

    \item \textbf{Servidores en la nube (Cloud)}: Este modelo se basa en el acceso a una infraestructura de cómputo virtualizada, proporcionada y gestionada por un tercero a través de internet. Se caracteriza por un modelo de costos basado en el pago por uso, lo que transforma la inversión en un gasto operativo (\textbf{OpEx}) \cite{softteco2024}. La principal referencia para definir este modelo es la publicación especial del NIST (National Institute of Standards and Technology), que lo describe como un modelo para el acceso bajo demanda a un conjunto compartido de recursos de cómputo configurables \cite{mell2011nist}.
\end{itemize}

\subsection{Modelos de Servicio en la Nube}
\label{subsec:modelos_servicio_nube}

El NIST también define tres modelos de servicio fundamentales que describen cómo se ofrecen los recursos de la nube a los usuarios \cite{mell2011nist}:

\begin{itemize}
    \item \textbf{Infraestructura como Servicio (IaaS - Infrastructure as a Service)}: El proveedor ofrece los recursos de cómputo fundamentales, como servidores virtuales, almacenamiento y redes. El usuario no gestiona la infraestructura física subyacente, pero tiene control sobre los sistemas operativos, el almacenamiento y las aplicaciones desplegadas. Ejemplos: Amazon EC2, Google Compute Engine.

    \item \textbf{Plataforma como Servicio (PaaS - Platform as a Service)}: El proveedor ofrece una plataforma que permite a los clientes desarrollar, ejecutar y gestionar aplicaciones sin la complejidad de construir y mantener la infraestructura asociada. El usuario controla las aplicaciones desplegadas, pero no la infraestructura de red, servidores o sistemas operativos. Ejemplos: AWS Lambda, Google App Engine, Microsoft Azure App Service.

    \item \textbf{Software como Servicio (SaaS - Software as a Service)}: El proveedor ofrece una aplicación completa que es consumida directamente por el usuario final a través de un navegador web o una interfaz de programa. El usuario no gestiona ni controla la infraestructura subyacente de la nube. Ejemplos: Gmail, Microsoft 365.
\end{itemize}

Para el presente proyecto, la arquitectura del servidor se basará en una combinación de servicios IaaS y PaaS, aprovechando la flexibilidad de la infraestructura virtual y la eficiencia de las plataformas gestionadas para bases de datos y procesamiento de datos.



%%%%%%%%%%%%%%%%%%%%%%%%%%%%%%%%%%%%%
%       SECCION: Base de datos      %
%%%%%%%%%%%%%%%%%%%%%%%%%%%%%%%%%%%%%
\section{Base de Datos}
\label{sec:base_de_datos}

Una base de datos (DB) es un conjunto organizado de datos estructurados, almacenados electrónicamente en un sistema informático, que permite su fácil acceso, gestión y actualización \cite{mullins2012database}. Su función principal es permitir el almacenamiento sistemático de grandes volúmenes de información para su posterior consulta, modificación o análisis. Las bases de datos son gestionadas por un Sistema de Gestión de Bases de Datos (DBMS), que actúa como una interfaz entre la base de datos y los usuarios o aplicaciones.

Además de almacenar datos, un DBMS robusto garantiza la \textbf{integridad} y \textbf{consistencia} de los datos mediante la aplicación de un conjunto de reglas y restricciones definidas en su diseño. Una base de datos bien diseñada no solo mejora el rendimiento de las consultas, sino que también optimiza la toma de decisiones basada en datos confiables.

\subsection{Modelos Lógicos de Bases de Datos}
\label{subsec:modelos_logicos_db}

Las bases de datos se clasifican principalmente según su modelo lógico, que define la estructura de los datos y las relaciones entre ellos. Los dos modelos predominantes en la actualidad son el relacional (SQL) y el no relacional (NoSQL).

\subsubsection{Bases de Datos Relacionales (SQL)}
El modelo relacional, propuesto por E. F. Codd en 1970, organiza los datos en tablas (o relaciones) compuestas por filas (tuplas) y columnas (atributos). Cada tabla tiene un esquema predefinido que dicta el tipo de datos para cada columna, y las relaciones entre tablas se establecen mediante claves primarias y foráneas \cite{oracle2024sql}.

Estas bases de datos utilizan el Lenguaje de Consulta Estructurado (SQL) para la manipulación y definición de los datos. Su principal fortaleza es la garantía de consistencia transaccional a través de las propiedades \textbf{ACID} (Atomicidad, Consistencia, Aislamiento y Durabilidad) \cite{oracle2024acid}.
\begin{itemize}
    \item \textbf{Ejemplos de DBMS Relacionales:} PostgreSQL, MySQL, Microsoft SQL Server, Oracle Database.
\end{itemize}

\subsubsection{Bases de Datos No Relacionales (NoSQL)}
Las bases de datos NoSQL (a menudo interpretado como "Not Only SQL") surgieron para abordar las limitaciones de escalabilidad y flexibilidad de los modelos relacionales, especialmente para aplicaciones web a gran escala y el manejo de datos no estructurados \cite{aws2024nosql}. No utilizan un esquema fijo, lo que permite una mayor flexibilidad en el almacenamiento de datos.

En lugar de la consistencia estricta de ACID, muchos sistemas NoSQL se adhieren al teorema CAP (Consistencia, Disponibilidad, Tolerancia a particiones) y a menudo garantizan un modelo de consistencia eventual a través de las propiedades \textbf{BASE} (Basically Available, Soft state, Eventually consistent) \cite{mongodb2024sqlvsnosql}.

Existen varios tipos de bases de datos NoSQL, cada una optimizada para un caso de uso específico \cite{aws2024nosql}:
\begin{itemize}
    \item \textbf{De Documentos:} Almacenan datos en documentos, comúnmente en formato JSON o BSON. Cada documento puede tener su propia estructura. Ejemplo: \textbf{MongoDB}.
    \item \textbf{Clave-Valor:} Es el modelo más simple, donde cada dato se almacena como un par de clave y valor. Son extremadamente rápidas para lecturas y escrituras simples. Ejemplo: \textbf{Redis}, \textbf{Amazon DynamoDB}.
    \item \textbf{Columnares:} Optimizadas para consultas rápidas sobre grandes conjuntos de datos, almacenando los datos por columnas en lugar de por filas. Ejemplo: \textbf{Apache Cassandra}.
    \item \textbf{De Grafos:} Diseñadas para almacenar y navegar relaciones complejas entre entidades. Ejemplo: \textbf{Neo4j}.
\end{itemize}

\subsection{Bases de Datos como Servicio (DBaaS) en la Nube}
\label{subsec:dbaas}

Independientemente del modelo lógico (SQL o NoSQL), las bases de datos pueden ser desplegadas en infraestructuras locales (\textit{on-premise}) o, más comúnmente en la actualidad, consumidas como un servicio en la nube.

Las \textbf{Bases de Datos como Servicio (DBaaS)} son servicios gestionados ofrecidos por proveedores de nube como Amazon Web Services (AWS), Microsoft Azure y Google Cloud. En este modelo, el proveedor se encarga de todas las tareas de administración de la infraestructura, como el aprovisionamiento de hardware, la instalación de software, la aplicación de parches, la configuración y las copias de seguridad \cite{azure2024dbaas}.

Esto permite a las organizaciones enfocarse en el desarrollo de aplicaciones y el uso de los datos, beneficiándose de la escalabilidad, la alta disponibilidad global y la seguridad que ofrece la plataforma en la nube \cite{amazon2024rds}.
\begin{itemize}
    \item \textbf{Ejemplos de servicios DBaaS:} Amazon RDS, Azure SQL Database, Google Cloud SQL, MongoDB Atlas.
\end{itemize}
%\newpage
\chapter{Análisis}

\section{Análisis del sistema y requerimientos del prototipo}

A continuación, se presenta una lista con los posibles requerimientos identificados para el desarrollo del prototipo de monitoreo ambiental. Estos requerimientos están organizados en subsistemas funcionales y servirán como guía para el diseño e implementación del sistema.


En la figura \ref{Dig:diagramaAB} se presenta un diagrama a bloques que muestra el funcionamiento del sistema, la interconexión entre módulos, los componentes dentro de cada uno y las funciones de cada módulo dentro del sistema de monitoreo.

\begin{figure}[H]
    \centering
    \includegraphics[width=1\textwidth]{Documento/Imagenes/Análisis/DiagramaAB2 (3).pdf}
    \caption{Diagrama a bloques del sistema.}
    \label{Dig:diagramaAB}
\end{figure}
   

\subsubsection*{Módulo 1. Red inalámbrica de sensores}

\begin{enumerate}
    \item Subsistema – Sensado de parámetros de calidad del agua \\
    \begin{figure}[H]
    \centering
    \includegraphics[width=1\textwidth]{Documento/Imagenes/Análisis/DiagramaAB2 (2).pdf}
    \caption{Diagrama a bloques del Nodeo sensor.}
    \label{Dig:diagramaAB2}
\end{figure}
    Este subsistema se encargará de obtener mediciones periódicas sobre las condiciones físico-químicas del agua mediante sensores especializados.
    \begin{itemize}
        \item Sensor de pH: permite medir el nivel de acidez del agua en intervalos definidos.
        \item Sensor de oxígeno disuelto (OD): monitorea la oxigenación del agua.
        \item Sensor de turbidez: indica la presencia de partículas suspendidas.
        \item Sensor de conductividad eléctrica: mide la concentración de iones disueltos.
        \item Sensor de temperatura: contextualiza las demás variables, ya que la temperatura afecta los valores de pH, OD y conductividad.
        \item Integración en un nodo sensor flotante: los sensores estarán integrados en un nodo con protección IP adecuada al entorno fluvial.
    \end{itemize}
    La Tabla \ref{tab:parametros_calidad_agua} presenta los rangos de medición establecidos según las recomendaciones de la Comisión Nacional del Agua (CONAGUA) \cite{conagua2022} y las Normas Oficiales Mexicanas (NOM) para análisis de agua \cite{nmx0082016, nmx0122001, nmx0382001, nmx0932000}.
    
    \begin{table}[H]
\caption{Parámetros de calidad del agua, unidades y rangos normativos}
\centering
\renewcommand{\arraystretch}{1.5}
\begin{tabular}{|p{4cm}|p{4cm}|p{4cm}|}
    \hline
    \textbf{Parámetro} & \textbf{Unidad} & \textbf{Rango normativo} \\ 
    \hline
    pH & Unidades de pH (adimensional) & 5.0 - 8.0 \\ 
    \hline
    Oxígeno disuelto (OD) & Miligramos por litro (mg/L) & $> 10$ \\ 
    \hline
    Turbidez & Unidades Nefelométricas de Turbidez (UNT) & $\leq 5$ \\ 
    \hline
    Conductividad eléctrica & Microsiemens por centímetro (\textmu S/cm) & $\leq 1,000$ \\ 
    \hline
    Temperatura & Grados Celsius (°C) & $5 - 30$ \\ 
    \hline
\end{tabular}
\label{tab:parametros_calidad_agua}
\end{table}

    \item Subsistema – Cámara \\
    Encargado de capturar imágenes periódicas de la superficie del agua para identificar residuos sólidos flotantes.
    \begin{itemize}
        \item Captura de imágenes: se utilizará una cámara compatible con el microcontrolador.
        \item Montaje en posición fija: las cámaras estarán orientadas hacia la superficie del agua.
        \item Iluminación diurna natural: se aprovechará la luz solar para evitar consumo energético adicional.
    \end{itemize}

    \item Subsistema – Microcontrolador \\
    Gestiona el funcionamiento de sensores, adquisición de datos, almacenamiento temporal y envío de información al servidor.
    \begin{itemize}
        \item Tarjeta de desarrollo: se empleará una tarjeta de bajo consumo energético y con capacidad de conectividad de largo alcance.
        \item Organización y empaquetado de datos: procesa, organiza y empaqueta los datos obtenidos de sensores y cámara antes de su transmisión.
        \item Manejo eficiente de energía: optimiza el consumo de energía en los nodos, considerando las limitaciones de alimentación en entornos remotos.
        \item Memoria: gestiona la memoria interna que almacenará temporalmente los datos antes de su transmisión.
    \end{itemize}

    \item Subsistema – Transceptor \\
    Encargado de transmitir los datos obtenidos por el nodo sensor al nodo vecino, y posteriormente al servidor central.
    \begin{itemize}
        \item Transmisión de datos: utiliza tecnologías de comunicación inalámbrica para garantizar conectividad en áreas remotas.
        \item Optimización del consumo energético: minimiza el consumo durante la transmisión mediante protocolos de bajo consumo.
        \item Enrutamiento de datos: gestiona la comunicación entre nodos mediante técnicas de reenvío.
    \end{itemize}

    \item Subsistema – Fuente de alimentación \\
    Este subsistema proporciona energía a todos los componentes del nodo.
    \begin{itemize}
        \item Batería: proporciona la energía necesaria para el funcionamiento completo del nodo.
        \item Panel solar: recarga la batería utilizando energía solar para garantizar autonomía en entornos remotos.
    \end{itemize}
\end{enumerate}

\subsubsection*{Módulo 2. Nodo concentrador}

El nodo concentrador cuenta con los mismos componentes que los nodos sensores, pero con la función adicional de recibir la información recolectada a través de la red. Este nodo centraliza los datos, los organiza y los transmite al servidor central para su procesamiento.

\begin{enumerate}
    \item Subsistema – Recepción de datos de nodos sensores
    \begin{itemize}
        \item Recepción de datos inalámbricos: debe recibir múltiples transmisiones periódicas para su almacenamiento temporal.
    \end{itemize}

    \item Subsistema – Almacenamiento temporal y organización de datos
    \begin{itemize}
        \item Almacenamiento temporal en buffer: los datos de los nodos se almacenan temporalmente en el buffer del concentrador.
        \item Organización de datos: se estructuran para facilitar su transmisión posterior.
    \end{itemize}

    \item Subsistema – Gateway
    \begin{itemize}
        \item Reenvío al servidor: envía los datos al servidor central mediante WLAN o red celular.
    \end{itemize}
\end{enumerate}

\subsubsection*{Módulo 3. Servidor}

\begin{enumerate}
    \item Subsistema – Recepción de datos del gateway
    \begin{itemize}
        \item Clasificación de datos: organiza y asigna metadatos a la información de sensores para su análisis.
    \end{itemize}

    \item Subsistema – Procesamiento de imágenes
    \begin{itemize}
        \item Detección de residuos: las imágenes son analizadas mediante visión artificial para identificar residuos flotantes.
        \item Marcas de detección: se agregan metadatos visuales a las imágenes para facilitar su interpretación.
    \end{itemize}

    \item Subsistema – Almacenamiento en base de datos
    \begin{itemize}
        \item Gestión de datos históricos: almacena la información adquirida para su análisis a largo plazo.
        \item Optimización de consultas: estructura la base de datos para recuperación rápida y eficiente.
    \end{itemize}

    \item Subsistema – API / Comunicación con página web
    \begin{itemize}
        \item API de comunicación: permite acceder a los datos e imágenes desde la página web.
        \item Enlace con la página web: proporciona una interfaz para que los usuarios consulten los resultados de monitoreo.
    \end{itemize}
\end{enumerate}

\subsubsection*{Módulo 4. Página web}

La página web permite consultar los datos recopilados por los nodos sensores y procesados en el servidor. Presenta información sobre calidad del agua y residuos flotantes en una interfaz responsiva.

\begin{enumerate}
    \item Subsistema – Conexión al servidor (API / Backend)
    \begin{itemize}
        \item API RESTful: gestiona solicitudes y respuestas de datos para visualización web.
        \item Conexión segura: implementa protocolos como HTTPS para proteger la transmisión de datos.
        \item Manejo de solicitudes y respuestas: estructura los datos enviados al frontend.
    \end{itemize}

    \item Subsistema – Interfaz de usuario (Frontend)
    \begin{itemize}
        \item HTML (estructura): define las secciones de visualización de datos.
        \item CSS (estilo): adapta el diseño para distintos dispositivos.
        \item JavaScript (interactividad): permite la actualización dinámica y navegación fluida.
    \end{itemize}
\end{enumerate}



%%%%%%%%%%%%%%%%%%%%%%%%%%%%%%%%%%%%%%%%%%%%%%%%%%%%
%          ANALISIS REQUERIMIENTOS                 %
%%%%%%%%%%%%%%%%%%%%%%%%%%%%%%%%%%%%%%%%%%%%%%%%%%%%


%%%%%%%%%%%%%%%%%%%%%%%%%%%%%%%%%%%%%%%%%%%%%%%%%%%%%
%          ANALISIS REQUERIMIENTOS                 %
%%%%%%%%%%%%%%%%%%%%%%%%%%%%%%%%%%%%%%%%%%%%%%%%%%%%


\newcounter{RF}

% Comando con contador y etiqueta opcional
\makeatletter
\newcommand{\RF}[1][]{%
  \refstepcounter{RF}%
  RF\@tempcnta=\value{RF}%
  \ifnum\@tempcnta<10 0\fi%
  \arabic{RF}%
  \if\relax#1\relax\else\label{#1}\fi%
}
\makeatother

\section{Análisis de Requerimientos}
A continuación, se presentan las tablas de los requerimientos funcionales y no funcionales identificados para el proyecto, los cuales servirán como base de referencia a lo largo de las etapas de diseño, implementación y pruebas.
\subsection{Requerimientos Funcionales}
A manera de resumen, la siguiente lista contiene los títulos de todos los requerimientos funcionales documentados a lo largo de los diferentes módulos del sistema:

\subsubsection*{Módulo 1. Red Inalámbrica de Sensores}
\begin{itemize}
    \item \textbf{RF01}: Obtención de Parámetros de Calidad del Agua.
    \item \textbf{RF02}: Protección del Nodo Sensor.
    \item \textbf{RF03}: Captura de Imágenes para Detección de Residuos.
    \item \textbf{RF04}: Coordinación del Ciclo de Operación de Sensores y Cámara
    \item \textbf{RF05}: Adquisición de Datos y Metadatos de Imagen.
    \item \textbf{RF06}: Gestión de Búfer y Memoria Temporal del Nodo Sensor.
    \item \textbf{RF07}: Estructuración y Empaquetado de Tramas de Datos.
    \item \textbf{RF08}: Garantía de Integridad de Tramas Antes de la Transmisión
    \item \textbf{RF09}: Gestión de Energía.
    \item \textbf{RF10}: Transmisión Inalámbrica de Datos.
    \item \textbf{RF11}: Proporcionar Energía a los Componentes del Nodo.
    \item \textbf{RF12}: Recarga de la Batería con Energía Solar.
\end{itemize}

\subsubsection*{Módulo 2. Nodo Concentrador}
\begin{itemize}
    \item \textbf{RF13}: Recepción de Datos de Nodos Sensores.
    \item \textbf{RF14}: Almacenamiento Temporal y Organización de Datos.
    \item \textbf{RF15}: Transmisión de Datos Organizados al Servidor Central
    \item \textbf{RF16}: Implementación de Protocolo de Detección y Corrección de Errores.
    \item \textbf{RF17}: Retransmisión de Datos en Caso de Fallo de Conexión.
\end{itemize}

\subsubsection*{Módulo 3. Servidor}
\begin{itemize}
    \item \textbf{RF18}: Recepción de Datos de Gateway.
    \item \textbf{RF19}: Organización de Datos Recibidos.
    \item \textbf{RF20}: Procesamiento de Imágenes para Cuantificación de Área de Basura.
    \item \textbf{RF21}: Almacenamiento Persistente de Datos y Metadatos.
    \item \textbf{RF22}: Exposición de Datos y Servicios Mediante API (RESTful).
\end{itemize}

\subsubsection*{Módulo 4. Página Web}
\begin{itemize}
    \item \textbf{RF23}: Consumo de Servicios del Servidor API.
    \item \textbf{RF24}: Visualización y Presentación Interactiva de Datos (Frontend).
    \item \textbf{RF25}: Permitir la Visualización de Datos Históricos.
\end{itemize}


\subsection*{Tablas de requerimientos funcionales}

%%%%%%%%%%%%%%       MODULO 1        %%%%%%%%%%%%%%%%%%%%%%%%%%%%
%%%%%%%%%%%%%%%%%%%%%%%%%%%%%%%%% subsistema sensado de parametros
\subsubsection*{Módulo 1. Red Inalámbrica de Sensores}

%%%%%%%%
% RF01
%%%%%%%%
\begin{longtable}{|l|p{12cm}|}
\hline
\textbf{\RF} & \textbf{Obtención de Parámetros de Calidad del Agua.} \\
\hline
\endfirsthead
\hline
\textbf{Versión} & 1.0 - Fecha de versión: 06/06/2025 \\
\hline
\textbf{Autor} & Equipo de Desarrollo. \\ 
\hline
\textbf{Fuente} & Documentación técnica del subsistema - Sensado de Parámetros de Calidad del Agua.\\
\hline
\textbf{Propósito} & Obtener mediciones periódicas de la calidad del agua a través de sensores especializados. \\
\hline
\textbf{Descripción} & El subsistema de sensado incluirá sensores de pH, oxígeno disuelto, turbidez, conductividad y temperatura, los cuales tomarán mediciones de los parámetros de calidad del agua en intervalos predefinidos. \\
\hline
\textbf{Especificación} & Los sensores deben medir parámetros de calidad del agua, y deben proporcionar mediciones precisas a intervalos regulares. Los datos deben ser transmitidos periódicamente al Nodo concentrador a través de una red de sensores inalámbricos para ser procesados. \\
\hline
\textbf{Prioridad} & Alta \\
\hline
\textbf{Comentarios} & La obtención de parámetros de calidad del agua es crucial para la información y prevención de problemas de contaminación en cuerpos de agua utilizados para consumo humano o en procesos industriales. El monitoreo periódico permitirá la intervención temprana. \\
\hline
\end{longtable}


%%%%%%%%
% RF02
%%%%%%%%
\begin{longtable}{|l|p{12cm}|}
\hline
\textbf{\RF} & \textbf{Protección del Nodo Sensor.} \\
\hline
\endfirsthead
\hline
\textbf{Versión} & 1.0 - Fecha de versión: 06/06/2025 \\
\hline
\textbf{Autor} & Equipo de Desarrollo \\
\hline
\textbf{Fuente} & Documentación técnica del subsistema - Sensado de Parámetros. \\
%\hline

%\textbf{Propósito} & Proteger todos los componentes en una estructura flotante que sea resistente a las condiciones acuáticas. \\
\hline
\textbf{Propósito} & Proteger los componentes del nodo sensor mediante una boya flotante con protección IP adecuada para ambientes acuáticos.  \\
\hline
\textbf{Descripción} & La boya debe estar diseñada con materiales resistentes para ser hermético al agua y debe proteger todos los componentes internos, los sensores y otros elementos electrónicos contra la humedad y el contacto con el agua. Esta estructura debe proteger los componentes internos y permitir el funcionamiento confiable cuando se utilice en cuerpos de agua. \\
\hline
\textbf{Especificación} & La estructura del nodo sensor debe tener una protección mínima de IP68, lo que asegura que es impermeable al agua y polvo. Además, el nodo debe ser capaz de soportar el movimiento brusco en condiciones desfavorables para el sistema, sin comprometer su funcionalidad. \\
\hline
\textbf{Prioridad} & Alta \\
\hline
\textbf{Comentarios} & La protección adecuada de los sensores es esencial para asegurar su rendimiento y vida útil en ambientes acuáticos. \\
\hline
\end{longtable}

%%%%%%%%
% RF03
%%%%%%%%
%%%%%%%%%%%%%%%%%%%%%%%%%%%%%%%%%% subsistema camara
\begin{longtable}{|l|p{12cm}|}
\hline
\textbf{\RF} & \textbf{Captura de Imágenes para Detección de Residuos.} \\
\hline
\endfirsthead
\hline
\textbf{Versión} & 1.0 - Fecha de versión: 06/06/2025 \\
\hline
\textbf{Autor} & Equipo de Desarrollo \\
\hline
\textbf{Fuente} & Documentación técnica del subsistema - Camára. \\
\hline
\textbf{Propósito} & Capturar imágenes periódicas de la superficie del agua para permitir la cuantificación del área cubierta por cúmulos de residuos flotantes. \\
\hline
\textbf{Descripción} & El sistema debe utilizar una cámara de bajo consumo energético para capturar imágenes de la superficie del agua, las cuales serán transmitidas al Servidor para su procesamiento. El nodo sensor solo gestionará una imagen a la vez en su memoria antes de su transmisión. \\
\hline
\textbf{Especificación} & La imagen original debe ser almacenada temporalmente y transmitida al nodo concentrador para su posterior envío al servidor central sin procesamiento ni clasificación previo en el nodo sensor. El nodo sensor no debe almacenar más de una imagen de 640×640 simultáneamente para su procesamiento y transmisión. \\
\hline
\textbf{Prioridad} & Alta \\
\hline
\textbf{Comentarios} & La captura precisa de imágenes es esencial para identificar residuos flotantes en la superficie del agua. La cámara debe ser capaz de operar en condiciones de luz natural y estar protegida de las condiciones ambientales adversas. \\
\hline
\end{longtable}

%%%%%%%%
% RF04
%%%%%%%%
%%%%%%%%%%%%%%%%%%%%%%%%%%%%%%%%%%% subsistema-Microcontrolador
\begin{longtable}{|l|p{12cm}|}
\hline
\textbf{\RF[rf:coordinacion_sensores]} & \textbf{Coordinación del Ciclo de Operación de Sensores y Cámara} \\
\hline
\endfirsthead
\hline
\textbf{Versión} & 1.0 - Fecha de versión: 06/06/2025 \\
\hline
\textbf{Autor} & Equipo de Desarrollo \\
\hline
\textbf{Fuente} & Documentación técnica del subsistema - Microcontrolador. \\
\hline
\textbf{Propósito} & Coordinar y sincronizar la activación y desactivación de los componentes, optimizando el consumo energético, para asegurar la adquisición de datos de manera eficiente. \\
\hline
\textbf{Descripción} & El microcontrolador deberá gestionar un ciclo de operación predefinido que activará secuencialmente los sensores y la cámara solo cuando sea necesario, y los pondrá en modo de reposo. \\
\hline
\textbf{Especificación} &  El microcontrolador deberá ser capaz de activar y desactivar los componentes eficientemente. El sistema debe asegurar que los componentes no esenciales entren en modo de reposo o baja energía cuando el ciclo de adquisición no esté activo. \\
\hline
\textbf{Prioridad} & Alta \\
\hline
\textbf{Comentarios} & La coordinación eficiente es esencial para garantizar que los datos y las imágenes se adquieran sin interferencias ni retrasos y que la autonomía del sistema se mantenga. \\
\hline
\end{longtable}

%%%%%%%%
% RF05
%%%%%%%%
%%%%%%%%%%%%%%%%%%%%%%%%%%%%%%%%%%% subsistema-Microcontrolador
\begin{longtable}{|l|p{12cm}|}
\hline
\textbf{\RF} & \textbf{Adquisición de Datos y Metadatos de Imagen.} \\
\hline
\endfirsthead
\hline
\textbf{Versión} & 1.0 - Fecha de versión: 06/06/2025 \\
\hline
\textbf{Autor} & Equipo de Desarrollo \\
\hline
\textbf{Fuente} & Documentación técnica del subsistema - Microcontrolador. \\
\hline
\textbf{Propósito} & Obtener las mediciones de los sensores y la captura de la imagen, asegurando su entrega al búfer para su posterior empaquetado y transmisión. \\
\hline
\textbf{Descripción} & Una vez activados, los sensores y la cámara deberán entregar sus datos y capturas, los cuales serán almacenados en el búfer del microcontrolador. \\
\hline
\textbf{Especificación} &  El sistema deberá adquirir mediciones de pH, turbidez, oxígeno disuelto, conductividad y temperatura conforme al ciclo de operación coordinado y almacenar los datos de los sensores y las imágenes temporalmente en el búfer del nodo sensor hasta su transmisión al nodo concentrador. \\
\hline
\textbf{Prioridad} & Alta \\
\hline
\textbf{Comentarios} & La correcta adquisición y almacenamiento temporal son pasos críticos para garantizar la integridad y disponibilidad de toda la información (parámetros e imágenes) antes de que se empaqueten y se envíen al servidor. \\
\hline
\end{longtable}

%%%%%%%%
% RF06
%%%%%%%%
\begin{longtable}{|l|p{12cm}|}
\hline
\textbf{\RF} & \textbf{Gestión de Búfer y Memoria Temporal del Nodo Sensor.} \\
\hline
\endfirsthead
\hline
\textbf{Versión} & 1.0 - Fecha de versión: 06/06/2025 \\
\hline
\textbf{Autor} & Equipo de Desarrollo \\
\hline
\textbf{Fuente} & Documentación técnica del subsistema - Microcontrolador. \\
\hline
\textbf{Propósito} & Gestionar la memoria interna del microcontrolador para almacenar temporalmente los datos adquiridos (sensores y la imagen única) antes de su transmisión, evitando la sobrecarga y asegurando la no pérdida de datos. \\
\hline
\textbf{Descripción} & 	El microcontrolador debe utilizar su memoria interna para almacenar de forma temporal los datos adquiridos de los sensores y una única imagen antes de que sean transmitidos al nodo concentrador. La memoria debe gestionarse eficientemente para evitar la sobrecarga. \\
\hline
\textbf{Especificación} & El microcontrolador debe ser capaz de almacenar los datos de todos los sensores y una sola imagen de 640×640 antes de la transmisión. El sistema debe permitir el borrado o sobrescritura de datos antiguos cuando la memoria esté llena, para garantizar la disponibilidad de espacio para nuevas mediciones. \\
\hline
\textbf{Prioridad} & Alta \\
\hline
\textbf{Comentarios} & 	La gestión de memoria es crucial para asegurar que los datos no se pierdan antes de ser transmitidos, lo que podría comprometer el análisis posterior. \\
\hline
\end{longtable}


%%%%%%%%
% RF07
%%%%%%%%

\begin{longtable}{|l|p{12cm}|}
\hline
\textbf{\RF[rf:Empaquetado_datos]} & \textbf{Estructuración y Empaquetado de Tramas de Datos.} \\
\hline
\endfirsthead
\hline
\textbf{Versión} & 1.0 - Fecha de versión: 06/06/2025 \\
\hline
\textbf{Autor} & Equipo de Desarrollo \\
\hline
\textbf{Fuente} & Documentación técnica del subsistema - Microcontrolador. \\
\hline
\textbf{Propósito} & Organizar y empaquetar los datos obtenidos de los sensores y cámara en tramas de información de manera estructurada para su eficiente transmisión y posterior procesamiento. \\
\hline
\textbf{Descripción} & El microcontrolador debe organizar los datos adquiridos de los sensores y cámara en tramas de información, incluyendo campos como tipo de medición, valor, fecha y hora. \\
\hline
\textbf{Especificación} & Las tramas deben estar estructuradas de manera que contengan campos definidos como: tipo de medición, valor de la medición, fecha y hora de la medición, y cualquier metadato necesario. \\
\hline
\textbf{Prioridad} & Alta \\
\hline
\textbf{Comentarios} & El correcto empaquetado de las tramas de información es esencial para garantizar la fiabilidad de los datos. La estructura de las tramas debe ser clara y uniforme. \\
\hline
\end{longtable}

%%%%%%%%
% RF08
%%%%%%%%
\begin{longtable}{|l|p{12cm}|}
\hline
\textbf{\RF} & \textbf{Garantía de Integridad de Tramas Antes de la Transmisión} \\
\hline
\endfirsthead
\hline
\textbf{Versión} & 1.0 - Fecha de versión: 06/06/2025 \\
\hline
\textbf{Autor} & Equipo de Desarrollo \\
\hline
\textbf{Fuente} & Documentación técnica del subsistema - Microcontrolador. \\
\hline
\textbf{Propósito} & Asegurar que las tramas de información empaquetadas estén completas, sin errores, y listas para su transmisión al nodo concentrador. \\
\hline
\textbf{Descripción} & Una vez empaquetados (RF0\ref{rf:Empaquetado_datos}), el microcontrolador debe aplicar métodos de verificación (ej. checksum) para asegurar que las tramas de datos están completas y sin datos faltantes o errores internos antes de iniciar la transmisión. \\
\hline
\textbf{Especificación} & El sistema debe asegurar que las tramas sean completas, sin datos faltantes, y sin errores de transmisión , y estas tramas deben ser enviadas al servidor o nodo concentrador de manera periódica, en intervalos predefinidos, asegurando que todos los datos sean correctamente transmitidos y almacenados para su análisis posterior. \\
\hline
\textbf{Prioridad} & Alta \\
\hline
\textbf{Comentarios} & Esta validación previa es un paso crítico para reducir la tasa de error en la transmisión inalámbrica y garantizar la calidad de los datos. \\
\hline
\end{longtable}



%%%%%%%%
% RF09
%%%%%%%%
\begin{longtable}{|l|p{12cm}|}
\hline
\textbf{\RF} & \textbf{Gestión de Energía.} \\
\hline
\endfirsthead
\hline
\textbf{Versión} & 1.0 - Fecha de versión: 06/06/2025 \\
\hline
\textbf{Autor} & Equipo de Desarrollo \\
\hline
\textbf{Fuente} & Documentación técnica del subsistema - Microcontrolador. \\
\hline
\textbf{Propósito} & Optimizar el consumo energético de los sensores y otros componentes del sistema para garantizar su funcionamiento eficiente en entornos remotos. \\
\hline
\textbf{Descripción} & El microcontrolador debe gestionar el consumo energético de los sensores y otros componentes, asegurando que solo estén activos cuando sea necesario y que los componentes entren en modo de reposo para maximizar la duración de la batería. \\
\hline
\textbf{Especificación} & El microcontrolador debe ser capaz de activar y desactivar los sensores y otros componentes del sistema de forma eficiente, gestionando el consumo de energía para operar durante largos períodos sin recargar. \\
\hline
\textbf{Prioridad} & Alta \\
\hline
\end{longtable}

%%%%%%%%
% RF10
%%%%%%%%
%%%%%%%%%%%%%%%%%%%%%%%%%% subsistema transceptor
\begin{longtable}{|l|p{12cm}|}
\hline
\textbf{\RF} & \textbf{Transmisión Inalámbrica de Datos.} \\
\hline
\endfirsthead
\hline
\textbf{Versión} & 1.0 - Fecha de versión: 06/06/2025 \\
\hline
\textbf{Autor} & Equipo de Desarrollo \\
\hline
\textbf{Fuente} & Documentación técnica del subsistema - Transceptor. \\
\hline
\textbf{Propósito} & Asegurar que los datos obtenidos por el nodo sensor sean transmitidos de manera inalámbrica al nodo vecinos para su procesamiento y análisis. \\
\hline
\textbf{Descripción} & Los nodos sensores deben transmitir los datos adquiridos a través de tecnologías inalámbricas, al nodo vecino, posteriormente al nodo concentrador. La transmisión debe ser confiable y asegurar que no haya pérdida de información. \\
\hline
\textbf{Especificación} & Los datos deben ser transmitidos en paquetes de datos estructurados al nodo concentrador. El sistema debe operar eficientemente en distancias de al menos 50 metros entre nodos y sin pérdida de datos. Se deben utilizar tecnologías de comunicación inalámbrica confiables para mantener la calidad de la transmisión. \\
\hline
\textbf{Prioridad} & Alta \\
\hline
\textbf{Comentarios} & La transmisión inalámbrica proporciona flexibilidad y reduce los costos de instalación, ya que no requiere cableado físico. Es esencial garantizar que los datos se transmitan sin pérdidas y que la red sea resistente a interferencias para asegurar la continuidad del monitoreo. \\
\hline
\end{longtable}


%%%%%%%%
% RF11
%%%%%%%%
%%%%%%%%%%%%%%%% subsistema fuente de alimentacion 

\begin{longtable}{|l|p{12cm}|}
\hline
\textbf{\RF} & \textbf{Proporcionar Energía a los Componentes del Nodo.} \\
\hline
\endfirsthead
\hline
\textbf{Versión} & 1.0 - Fecha de versión: 06/06/2025 \\
\hline
\textbf{Autor} & Equipo de Desarrollo \\
\hline
\textbf{Fuente} & Documentación técnica del subsistema - Fuente de Alimentación. \\
\hline
\textbf{Propósito} & Asegurar que todos los componentes del nodo, incluidos el microcontrolador, los sensores, la cámara y el transceptor, reciban la energía necesaria para su funcionamiento. \\
\hline
\textbf{Descripción} & El subsistema debe ser capaz de proporcionar energía de manera continua a todos los componentes del nodo, de modo que el sistema funcione de forma autónoma sin depender de fuentes externas de energía. \\
\hline
\textbf{Especificación} & La batería debe proporcionar energía suficiente para el funcionamiento de los componentes del nodo durante un período mínimo de 24 horas sin necesidad de recarga. La batería debe ser lo suficientemente potente para soportar el consumo combinado de todos los dispositivos conectados en el nodo. \\
\hline
\textbf{Prioridad} & Alta \\
\hline
\textbf{Comentarios} & Este subsistema es fundamental para que el sistema funcione de manera autónoma en entornos remotos donde el acceso a fuentes de energía externas es limitado o inexistente. \\
\hline
\end{longtable}


%%%%%%%%
% RF12
%%%%%%%%
\begin{longtable}{|l|p{12cm}|}
\hline
\textbf{\RF} & \textbf{Recarga de la Batería con Energía Solar.} \\
\hline
\endfirsthead
\hline
\textbf{Versión} & 1.0 - Fecha de versión: 06/06/2025 \\
\hline
\textbf{Autor} & Equipo de Desarrollo \\
\hline
\textbf{Fuente} & Documentación técnica del subsistema de Fuente de Alimentación. \\
\hline
\textbf{Propósito} & Asegurar que la batería se recargue utilizando energía solar de manera continua para mantener el funcionamiento autónomo del nodo. \\
\hline
\textbf{Descripción} & El panel solar debe ser capaz de recargar la batería durante el día, aprovechando la energía solar, para que el nodo siga funcionando de manera autónoma. \\
\hline
\textbf{Especificación} & El panel solar debe recargar completamente la batería durante aproximadamente 6 horas de luz solar directa. El sistema debe tener circuitos de protección para evitar sobrecarga o daño a la batería. \\
\hline
\textbf{Prioridad} & Alta \\
\hline
\textbf{Comentarios} & Este subsistema es esencial para garantizar el funcionamiento continuo del sistema en entornos remotos y ayuda a reducir la dependencia de fuentes externas de energía. \\
\hline
\end{longtable}


%%%%%%%%%%%%%%%%%%           MODULO 2        %%%%%%%%%%%%%%%%%%%%%%%

\subsubsection*{Módulo 2. Nodo Concentrador}

%%%%%%%%
% RF13
%%%%%%%%
\begin{longtable}{|l|p{12cm}|}
\hline
\textbf{\RF} & \textbf{Recepción de Datos de Nodos Sensores.} \\
\hline
\endfirsthead
\hline
\textbf{Versión} & 1.0 - Fecha de versión: 06/06/2025 \\
\hline
\textbf{Autor} & Equipo de Desarrollo \\
\hline
\textbf{Fuente} & Documentación técnica del subsistema - Recepción de datos de nodos sensores. \\
\hline
\textbf{Propósito} & Asegurar que el nodo concentrador pueda recibir datos de los nodos sensores de manera confiable y periódica. \\
\hline
\textbf{Descripción} & El subsistema debe ser capaz de recibir los datos transmitidos desde el último nodo sensor a través de la red inalámbrica. \\
\hline
\textbf{Especificación} & El nodo concentrador debe ser capaz de recibir múltiples transmisiones de datos periódicas del nodo sensor anterior a él sin perder información. Los datos deben ser recibidos en un formato estructurado y organizado para su almacenamiento temporal. \\
\hline
\textbf{Prioridad} & Alta \\
\hline
\textbf{Comentarios} & La confiabilidad en la recepción de los datos es esencial para evitar pérdidas de información y garantizar que todos los parámetros monitoreados sean procesados correctamente. \\
\hline
\end{longtable}

%%%%%%%%
% RF14
%%%%%%%%
\begin{longtable}{|l|p{12cm}|}
\hline
\textbf{\RF} & \textbf{Almacenamiento Temporal y Organización de Datos.} \\
\hline
\endfirsthead
\hline
\textbf{Versión} & 1.0 - Fecha de versión: 06/06/2025 \\
\hline
\textbf{Autor} & Equipo de Desarrollo \\
\hline
\textbf{Fuente} & Documentación técnica del subsistema - Almacenamiento Temporal y Organización de Datos. \\
\hline
\textbf{Propósito} & Almacenar temporalmente los datos recibidos de los nodos sensores en el buffer del nodo concentrador y organizarlos para su posterior transmisión al servidor central. \\
\hline
\textbf{Descripción} & El subsistema debe ser capaz de recibir los datos de los nodos sensores transmitidos por el nodo anterior y almacenarlos de manera temporal en el buffer del nodo concentrador. Además, debe organizar los datos de forma estructurada, asegurando que sean fácilmente accesibles y listos para ser transmitidos al servidor central. \\
\hline
\textbf{Especificación} & Los datos recibidos deben ser almacenados temporalmente en el buffer del nodo concentrador, y deben ser organizados en una estructura que facilite su acceso y transmisión posterior. \\
\hline
\textbf{Prioridad} & Alta \\
\hline
\textbf{Comentarios} & El almacenamiento temporal y la organización eficiente de los datos son esenciales para asegurar que la información se mantenga accesible y lista para su transmisión rápida al servidor sin pérdida de datos. \\
\hline
\end{longtable}

%%%%%%%%
% RF15
%%%%%%%%
\begin{longtable}{|l|p{12cm}|}
\hline
\textbf{\RF} & \textbf{Transmisión de Datos Organizados al Servidor Central} \\
\hline
\endfirsthead
\hline
\textbf{Versión} & 1.0 - Fecha de versión: 06/06/2025 \\
\hline
\textbf{Autor} & Equipo de Desarrollo \\
\hline
\textbf{Fuente} & Documentación técnica del subsistema - Gateway. \\
\hline
\textbf{Propósito} & Enviar los datos procesados y organizados al servidor central para su almacenamiento y análisis. \\
\hline
\textbf{Descripción} & El nodo concentrador debe transmitir los datos organizados y procesados al servidor central utilizando una red confiable (WLAN o enlace celular). La transmisión debe ser eficiente y sin pérdidas de datos. \\
\hline
\textbf{Especificación} & El nodo concentrador debe enviar los datos al servidor central utilizando una conexión estable y confiable , y si es necesario, se deben utilizar enlaces celulares para garantizar la conectividad remota , y el sistema debe garantizar que la transmisión sea segura y sin errores con algún método de conexión de errores. \\
\hline
\textbf{Prioridad} & Alta \\
\hline
\textbf{Comentarios} & La transmisión de los datos al servidor es esencial para el procesamiento y almacenamiento a largo plazo. Debe asegurarse que la red utilizada sea confiable y no haya pérdidas de datos durante el envío. \\
\hline
\end{longtable}

%%%%%%%%
% RF16
%%%%%%%%
\begin{longtable}{|l|p{12cm}|}
\hline
\textbf{\RF} & \textbf{Implementación de Protocolo de Detección y Corrección de Errores.} \\
\hline
\endfirsthead
\hline
\textbf{Versión} & 1.0 - Fecha de versión: 06/06/2025 \\
\hline
\textbf{Autor} & Equipo de Desarrollo \\
\hline
\textbf{Fuente} & Documentación técnica del subsistema - Gateway. \\
\hline
\textbf{Propósito} &  Asegurar que la transmisión de datos al servidor central sea segura y sin errores mediante la aplicación de un método de corrección de errores. \\
\hline
\textbf{Descripción} & El nodo concentrador debe implementar un mecanismo o protocolo (ej. control de flujo, CRC) que permita la detección y corrección de errores durante la transmisión al servidor central, garantizando la integridad de los datos. \\
\hline
\textbf{Especificación} & El sistema debe garantizar que la transmisión sea segura y sin errores con algún método de conexión de errores, y deberá realizar una validación de la trama recibida por el servidor (ACK/NACK) para confirmar la entrega exitosa de los datos. \\
\hline
\textbf{Prioridad} & Alta \\
\hline
\textbf{Comentarios} & Este requerimiento es la base técnica que soporta el requisito funcional de retransmisión en caso de fallo, mejorando la fiabilidad general del sistema. \\
\hline
\end{longtable}

%%%%%%%%
% RF17
%%%%%%%%
\begin{longtable}{|l|p{12cm}|}
\hline
\textbf{\RF} & \textbf{Retransmisión de Datos en Caso de Fallo de Conexión.} \\
\hline
\endfirsthead
\hline
\textbf{Versión} & 1.0 - Fecha de versión: 06/06/2025 \\
\hline
\textbf{Autor} & Equipo de Desarrollo \\
\hline
\textbf{Fuente} & Documentación técnica del subsistema - Gateway. \\
\hline
\textbf{Propósito} & Asegurar que los datos se retransmitan en caso de fallo de conexión hasta lograr su entrega exitosa al servidor. \\
\hline
\textbf{Descripción} & El sistema debe ser capaz de retransmitir los datos cada vez que se detecte un fallo de conexión con el servidor, hasta que la transmisión se realice con éxito. \\
\hline
\textbf{Especificación} & El sistema debe realizar retransmisiones automáticas de los datos en intervalos predefinidos si no recibe confirmación de entrega exitosa. La retransmisión debe continuar hasta que el servidor confirme la recepción exitosa de los datos. \\
\hline
\textbf{Prioridad} & Alta \\
\hline
\textbf{Comentarios} & La retransmisión de datos es crítica para asegurar que los datos no se pierdan en caso de fallos de comunicación, garantizando la integridad de la información. \\
\hline
\end{longtable}



%%%%%%%%%%%%%%%     MODULO 3. SERVIDOR
\subsubsection*{Módulo 3. Servidor}

%   subsistema 1
%%%%%%%%
% RF18
%%%%%%%%
\begin{longtable}{|l|p{12cm}|}
\hline
\textbf{\RF} & \textbf{Recepción de Datos de Gateway.} \\
\hline
\endfirsthead
\hline
\textbf{Versión} & 1.0 - Fecha de versión: 06/06/2025 \\
\hline
\textbf{Autor} & Equipo de Desarrollo \\
\hline
\textbf{Fuente} & Documentación técnica del subsistema - Recepción de datos de gateway. \\
\hline
\textbf{Propósito} & Recibir los datos enviados por el gateway para su posterior organización y almacenamiento. \\
\hline
\textbf{Descripción} & El servidor debe ser capaz de recibir los datos transmitidos por el gateway desde los nodos sensores. \\
\hline
\textbf{Especificación} & El servidor debe ser capaz de recibir múltiples transmisiones de datos de manera periódica. Los datos deben ser almacenados en una base de datos estructurada para su posterior análisis. \\
\hline
\textbf{Prioridad} & Alta \\
\hline
\textbf{Comentarios} & La recepción de datos confiable es esencial para evitar pérdidas de información que puedan afectar el análisis posterior. \\
\hline
\end{longtable}

%%%%%%%%
% RF19
%%%%%%%%
\begin{longtable}{|l|p{12cm}|}
\hline
\textbf{\RF} & \textbf{Organización de Datos Recibidos.} \\
\hline
\endfirsthead
\hline
\textbf{Versión} & 1.0 - Fecha de versión: 06/06/2025 \\
\hline
\textbf{Autor} & Equipo de Desarrollo \\
\hline
\textbf{Fuente} & Documentación técnica del subsistema - Recepción de  datos de Gateway. \\
\hline
\textbf{Propósito} & Organizar los datos recibidos para su almacenamiento y análisis eficiente. \\
\hline
\textbf{Descripción} & El servidor debe organizar los datos procesados de manera estructurada, clasificando los datos por tipo de medición, fecha, hora y otros metadatos. \\
\hline
\textbf{Especificación} & Los datos deben ser organizados en una base de datos estructurada que permita consultas rápidas y eficientes. \\
\hline
\textbf{Prioridad} & Alta \\
\hline
\textbf{Comentarios} & La organización eficiente de los datos permite un acceso rápido y preciso para análisis futuros y toma de decisiones. \\
\hline
\end{longtable}
 
%   subsistema 2 proceso de imagenes
%%%%%%%%
% RF20
%%%%%%%%
\begin{longtable}{|l|p{12cm}|}
\hline
\textbf{\RF} & \textbf{Procesamiento de Imágenes para Cuantificación de Área de Basura.} \\
\hline
\endfirsthead
\hline
\textbf{Versión} & 1.0 - Fecha de versión: 06/06/2025 \\
\hline
\textbf{Autor} & Equipo de Desarrollo \\
\hline
\textbf{Fuente} & Documentación técnica del subsistema - Procesamiento de Imágenes. \\
\hline
\textbf{Propósito} & Procesar las imágenes enviadas por el nodo concentrador, aplicar algoritmos de visión artificial y generar un valor numérico que represente el área cubierta por los cúmulos de residuos flotantes. \\
\hline
\textbf{Descripción} & El sistema debe procesar las imágenes enviadas desde el nodo concentrador, aplicar algoritmos de visión artificial para detectar cúmulos de residuos sólidos flotantes, y calcular el área que cubren en la superficie del agua. El resultado debe ser un metadato numérico que se adjuntará a la imagen procesada. \\
\hline
\textbf{Especificación} & El sistema debe ser capaz de procesar imágenes, aplicar algoritmos de visión artificial (ej. YOLOv8 o segmentación) para detectar los cúmulos de residuos flotantes y generar un metadato numérico del área cubierta para su análisis y posterior almacenamiento. \\
\hline
\textbf{Prioridad} & Alta \\
\hline
\textbf{Comentarios} & La captura precisa de imágenes es esencial para la posterior cuantificación del área de residuos flotantes en el servidor. El uso de la imagen original de 640×640 requiere una alta eficiencia de compresión en el nodo sensor para mitigar el riesgo de sobrecarga de la red inalámbrica de baja potencia. \\
\hline
\end{longtable}

%%%%%%%%
% RF21
%%%%%%%%
%   subsistema 3. Almacenamiento en base de datos
\begin{longtable}{|l|p{12cm}|}
\hline
\textbf{\RF} & \textbf{Almacenamiento Persistente de Datos y Metadatos.} \\
\hline
\endfirsthead
\hline
\textbf{Versión} & 1.0 - Fecha de versión: 06/06/2025 \\
\hline
\textbf{Autor} & Equipo de Desarrollo \\
\hline
\textbf{Fuente} & Documentación técnica del subsistema - Almacenamiento en Base de Datos. \\
\hline
\textbf{Propósito} & Almacenar los datos y metadatos de los sensores y las imágenes de residuos en una base de datos para su posterior consulta. \\
\hline
\textbf{Descripción} & El servidor debe almacenar los datos procesados y organizados de los sensores y las imágenes de residuos flotantes de manera estructurada en una base de datos organizada. \\
\hline
\textbf{Especificación} & Los datos y metadatos deben ser almacenados de manera eficiente, garantizando la integridad y accesibilidad de la información histórica para su consulta. \\
\hline
\textbf{Prioridad} & Alta \\
\hline
\textbf{Comentarios} & El almacenamiento adecuado de los datos es fundamental para su acceso a largo plazo y para el análisis posterior. \\
\hline
\end{longtable}

%%%%%%%%
% RF22
%%%%%%%%
%   subsistema 4. API/Comunicación (Enlace con página web)
\begin{longtable}{|l|p{12cm}|}
\hline
\textbf{\RF} & \textbf{Exposición de Datos y Servicios Mediante API (RESTful).} \\
\hline
\endfirsthead
\hline
\textbf{Versión} & 1.0 - Fecha de versión: 06/06/2025 \\
\hline
\textbf{Autor} & Equipo de Desarrollo \\
\hline
\textbf{Fuente} & Documentación técnica del subsistema - API/Comunicación (Enlace con página web). \\
\hline
\textbf{Propósito} & Gestionar la comunicación entre el servidor y la página web, permitiendo la recuperación eficiente de los datos procesados. \\
\hline
\textbf{Descripción} & El servidor debe proporcionar una API para que la página web acceda a los datos de calidad del agua y las imágenes de residuos flotantes. \\
\hline
\textbf{Especificación} & La API RESTful debe permitir el acceso eficiente a los datos y imágenes almacenados en el servidor. La comunicación debe ser segura (utilizando HTTPS). \\
\hline
\textbf{Prioridad} & Alta \\
\hline
\textbf{Comentarios} & La integración con la página web es esencial para que los usuarios puedan consultar los datos sobre el estado del agua y los residuos flotantes. \\
\hline
\end{longtable}

%%%%%%%%%%      MODULO 3         %%%%%%%%%%%%%%%
\subsubsection*{Módulo 4. Página Web}

%%%%%%%%
% RF23
%%%%%%%%
%       subistema 1. Conexion al servidor API backend
\begin{longtable}{|l|p{12cm}|}
\hline
\textbf{\RF} & \textbf{Consumo de Servicios del Servidor API.} \\
\hline
\endfirsthead
\hline
\textbf{Versión} & 1.0 - Fecha de versión: 06/06/2025 \\
\hline
\textbf{Autor} & Equipo de Desarrollo \\
\hline
\textbf{Fuente} & Documentación técnica del subsistema - Conexión al Servidor (API/Backend). \\
\hline
\textbf{Propósito} & Gestionar la comunicación entre la página web y el servidor para la recuperación y visualización de datos. \\
\hline
\textbf{Descripción} & El subsistema debe gestionar la comunicación entre el frontend (página web) y el backend (servidor). Utilizará una API RESTful para recuperar los datos de calidad del agua y los residuos sólidos, enviando solicitudes y respondiendo con la información necesaria para la visualización en la página web. \\
\hline
\textbf{Especificación} & La API RESTful debe permitir que la página web solicite y reciba datos del servidor de forma eficiente y segura. La conexión debe ser segura (HTTPS), y el backend debe manejar las solicitudes de datos, procesarlas y responder con la información organizada y estructurada. \\
\hline
\textbf{Prioridad} & Alta \\
\hline
\textbf{Comentarios} & La seguridad en la comunicación entre la página web y el servidor es fundamental para proteger los datos y garantizar la privacidad. \\
\hline
\end{longtable}

%%%%%%%%
% RF24
%%%%%%%%
\begin{longtable}{|l|p{12cm}|}
\hline
\textbf{\RF} & \textbf{Visualización y Presentación Interactiva de Datos (Frontend).} \\
\hline
\endfirsthead
\hline
\textbf{Versión} & 1.0 - Fecha de versión: 06/06/2025 \\
\hline
\textbf{Autor} & Equipo de Desarrollo \\
\hline
\textbf{Fuente} & Documentación técnica del subsistema - Interfaz de usuario (Frontend). \\
\hline
\textbf{Propósito} & Permitir la visualización clara y atractiva de los datos de calidad del agua y residuos sólidos flotantes en la página web, incluyendo la generación de gráficos interactivos. \\
\hline
\textbf{Descripción} & La página web debe presentar los datos de forma clara, utilizando gráficos interactivos, tablas y representaciones visuales de los datos obtenidos. El diseño debe ser responsive y accesible desde diferentes dispositivos (móviles, tabletas y computadoras). \\
\hline
\textbf{Especificación} & Los gráficos deben ser actualizados dinámicamente sin necesidad de recargar la página. La visualización debe ser clara y los usuarios deben poder interactuar con los gráficos para obtener información detallada de los datos (ej. mediante tooltips o filtros). \\
\hline
\textbf{Prioridad} & Alta \\
\hline
\textbf{Comentarios} & La usabilidad es clave para garantizar que los usuarios puedan acceder a la información sin dificultades, por lo que la interactividad y la adaptabilidad de la página son esenciales. \\
\hline
\end{longtable}



%%%%%%%%
% RF25
%%%%%%%%
\begin{longtable}{|l|p{12cm}|}
\hline
\textbf{\RF} & \textbf{Permitir la Visualización de Datos Históricos.} \\
\hline
\endfirsthead
\hline
\textbf{Versión} & 1.0 - Fecha de versión: 06/06/2025 \\
\hline
\textbf{Autor} & Equipo de Desarrollo \\
\hline
\textbf{Fuente} & Documentación técnica del subsistema - Interfaz de usuario (Frontend). \\
\hline
\textbf{Propósito} & Permitir que los usuarios visualicen los datos históricos de calidad del agua y residuos sólidos flotantes. \\
\hline
\textbf{Descripción} & La página web debe permitir que los usuarios visualicen los datos históricos de los parámetros de calidad del agua y los residuos sólidos flotantes a lo largo del tiempo con la capacidad de interactuar con las fechas de monitoreo. \\
\hline
\textbf{Especificación} & Los usuarios deben poder acceder a los datos históricos mediante una interfaz que les permita ver cómo evolucionaron los parámetros de calidad del agua y los residuos flotantes. \\
\hline
\textbf{Prioridad} & Alta \\
\hline
\textbf{Comentarios} & La visualización de los datos históricos es crucial para realizar un análisis de tendencias y evaluar la evolución de la calidad del agua y la presencia de residuos. \\
\hline
\end{longtable}


%%%%%%%%%%%%%%%%%%%%%%%%%%%%%%%%%%%%%%%%%%%%%
%       Requerimientos NO FUNCIONALES       %
%%%%%%%%%%%%%%%%%%%%%%%%%%%%%%%%%%%%%%%%%%%%%
\subsection{Requerimientos No Funcionales}
Los requerimientos no funcionales definen los criterios de calidad y restricciones del sistema. A continuación, se presenta un resumen de sus títulos:

\subsubsection*{Módulo 1. Red Inalámbrica de Sensores}
\begin{itemize}
    \item \textbf{RNF01}: Precisión de los Sensores.
    \item \textbf{RNF02}: Certificación IP de la Carcasa del Nodo Sensor.
    \item \textbf{RNF03}: Calidad de la Imagen.
    \item \textbf{RNF04}: Bajo Consumo Energético del Microcontrolador.
    \item \textbf{RNF05}: Compatibilidad con Comunicaciones de Largo Alcance.
    \item \textbf{RNF06}: Gestión de Memoria para Evitar Pérdida de Datos.
    \item \textbf{RNF07}: Fiabilidad de la Comunicación Inalámbrica.
    \item \textbf{RNF08}: Consumo Energético del Transceptor.
    \item \textbf{RNF09}: Eficiencia Energética de la Fuente de Alimentación.
\end{itemize}

\subsubsection*{Módulo 2. Nodo Concentrador}
\begin{itemize}
    \item \textbf{RNF10}: Fiabilidad de la Recepción de Datos.
    \item \textbf{RNF11}: Eficiencia en el Almacenamiento Temporal.
    \item \textbf{RNF12}: Organización de Datos para Transmisión.
    \item \textbf{RNF13}: Prevención de Sobrecargas o Pérdidas por Desbordamiento de Buffer.
    \item \textbf{RNF14}: Fiabilidad en el Reenvío de Datos al Servidor.
\end{itemize}

\subsubsection*{Módulo 3. Servidor}
\begin{itemize}
    \item \textbf{RNF15}: Validación Automática de Datos Recibidos.
    \item \textbf{RNF16}: Optimización del Procesamiento y Trazabilidad del Algoritmo de Área de Basura.
    \item \textbf{RNF17}: Acceso a Datos Históricos de al Menos un Año.
    \item \textbf{RNF18}: Rendimiento y Seguridad de la API.
\end{itemize}

\subsubsection*{Módulo 4. Página Web}
\begin{itemize}
    \item \textbf{RNF19}: Accesibilidad y Responsividad de la Interfaz.
    \item \textbf{RNF20}: Diseño Claro y Estético de la Interfaz.
\end{itemize}


\subsection*{Tablas de requerimientos no funcionales}
%%%%%%%%%%      Modulo 1        %%%%%%%%%%%
%   subsistema 1. Sensado de parametros de calidad del agua

\subsubsection*{Módulo 1. Red Inalámbrica de Sensores}

%%%%%%%%
% RNF01
%%%%%%%%
\begin{longtable}{|l|p{12cm}|}
\hline
\textbf{RNF01} & \textbf{Precisión de los Sensores.} \\
\hline
\endfirsthead
\hline
\textbf{Versión} & 1.0 - Fecha de versión: 06/06/2025 \\
\hline
\textbf{Autor} & Equipo de Desarrollo \\
\hline
\textbf{Fuente} & Documentación técnica del subsistema - Sensado de Parámetros de Calida del Agua. \\
\hline
\textbf{Propósito} & Garantizar que los sensores proporcionen mediciones precisas y fiables de los parámetros de calidad del agua. \\
\hline
\textbf{Descripción} & Los sensores deben ser capaces de medir los parámetros de calidad del agua: pH, oxígeno disuelto, turbidez, conductividad, con un margen de error mínimo en la medición, cumpliendo con las normas de precisión establecidas (NOM). \\
\hline
\textbf{Especificación} & Los sensores deben cumplir con la precisión adecuada. El margen de error máximo debe ser del 2\% para turbidez y de ±0.1 unidades para el pH, y estar dentro de los límites de las Normas Oficiales Mexicanas (NOM). \\
\hline
\textbf{Prioridad} & Alta \\
\hline
\textbf{Comentarios} & La precisión de los sensores es fundamental para asegurar que los datos obtenidos sean confiables y útiles para la toma de decisiones. \\
\hline
\end{longtable}

%%%%%%%%
% RNF02
%%%%%%%%
\begin{longtable}{|l|p{12cm}|}
\hline
\textbf{RNF02} & \textbf{Certificación IP de la Carcasa del Nodo Sensor.} \\
\hline
\endfirsthead
\hline
\textbf{Versión} & 1.0 - Fecha de versión: 06/06/2025 \\
\hline
\textbf{Autor} & Equipo de Desarrollo \\
\hline
\textbf{Fuente} & Documentación técnica del subsistema - Sensado de Parámetros de Calidad del Agua. \\
\hline
\textbf{Propósito} & Garantizar que la carcasa del nodo sensor tenga una clasificación de protección IP adecuada para soportar inmersión en cuerpos de agua. \\
\hline
\textbf{Descripción} & La carcasa del nodo sensor debe contar con una clasificación IP adecuada que garantice su resistencia al agua y su capacidad para operar bajo condiciones de inmersión en cuerpos de agua, como ríos, arroyos y lagos, considerando pesos y ubicación de los componentes. \\
\hline
\textbf{Especificación} & La carcasa debe tener al menos una clasificación IP68, lo que asegura que el nodo sensor sea completamente impermeable al agua y resistente al polvo. La boya debe ser capaz de flotar sin que la funcionalidad de los componentes se vea comprometida. \\
\hline
\textbf{Prioridad} & Alta \\
\hline
\textbf{Comentarios} & La resistencia al agua es esencial para garantizar que el nodo sensor funcione correctamente en entornos acuáticos sin riesgo de fallos debido a la exposición al agua o la humedad. \\
\hline
\end{longtable}

%%%%%%%%
% RNF03
%%%%%%%%
%       subsistema 2. Camara
\begin{longtable}{|l|p{12cm}|}
\hline
\textbf{RNF03} & \textbf{Calidad de la Imagen.} \\
\hline
\endfirsthead
\hline
\textbf{Versión} & 1.0 - Fecha de versión: 06/06/2025 \\
\hline
\textbf{Autor} & Equipo de Desarrollo \\
\hline
\textbf{Fuente} & Documentación técnica del subsistema - Cámara. \\
\hline
\textbf{Propósito} & Garantizar que la cámara capture imágenes de la calidad necesaria para el cálculo preciso del área de residuos flotantes en el servidor. \\
\hline
\textbf{Descripción} & La cámara debe ser capaz de capturar imágenes nítidas y claras bajo diversas condiciones de iluminación y con la resolución 640×640 necesaria para el análisis de segmentación. \\
\hline
\textbf{Especificación} & La cámara debe ser capaz de capturar imágenes con una resolución mínima de 640×640 y una calidad de imagen que permita identificar claramente los cúmulos de residuos flotantes en el agua. La calidad de la imagen debe ser consistente incluso con variaciones de luz diurna. \\
\hline
\textbf{Prioridad} & Alta \\
\hline
\textbf{Comentarios} & La precisión en la captura de imágenes es esencial para una detección precisa de los residuos flotantes y para garantizar la calidad del análisis posterior. \\
\hline
\end{longtable}

%%%%%%%%
% RNF04
%%%%%%%%
%       subssitema 3. Microcontrolador
\begin{longtable}{|l|p{12cm}|}
\hline
\textbf{RNF04} & \textbf{Bajo Consumo Energético del Microcontrolador.} \\
\hline
\endfirsthead
\hline
\textbf{Versión} & 1.0 - Fecha de versión: 06/06/2025 \\
\hline
\textbf{Autor} & Equipo de Desarrollo \\
\hline
\textbf{Fuente} & Documentación técnica del subsistema de Microcontrolador. \\
\hline
\textbf{Propósito} & Optimizar el consumo energético del microcontrolador para maximizar la duración de la batería. \\
\hline
\textbf{Descripción} & El microcontrolador debe operar con bajo consumo energético, utilizando técnicas de bajo consumo durante períodos de inactividad (como el modo reposo). \\
\hline
\textbf{Especificación} & El microcontrolador debe consumir no más de 100 mA en su modo activo y menos de 5 mA en el modo de reposo. \\
\hline
\textbf{Prioridad} & Alta \\
\hline
\textbf{Comentarios} & El bajo consumo energético es crucial para asegurar la autonomía del sistema, especialmente en entornos remotos. \\
\hline
\end{longtable}

%%%%%%%%
% RNF05
%%%%%%%%
\begin{longtable}{|l|p{12cm}|}
\hline
\textbf{RNF05} & \textbf{Compatibilidad con Comunicaciones de Largo Alcance.} \\
\hline
\endfirsthead
\hline
\textbf{Versión} & 1.0 - Fecha de versión: 06/06/2025 \\
\hline
\textbf{Autor} & Equipo de Desarrollo \\
\hline
\textbf{Fuente} & Documentación técnica del subsistema de Microcontrolador. \\
\hline
\textbf{Propósito} & Garantizar que el microcontrolador sea compatible con tecnologías de comunicación de largo alcance. \\
\hline
\textbf{Descripción} & El microcontrolador debe ser compatible con tecnologías de comunicación de largo alcance, para garantizar la transmisión eficiente de datos entre nodos y el servidor central en áreas remotas. \\
\hline
\textbf{Especificación} & El microcontrolador debe ser compatible con LoRa o tecnologías similares como WiFI de bajo consumo y largo alcance, garantizando una transmisión de datos confiable hasta distancias de 50 m en entornos abiertos. \\
\hline
\textbf{Prioridad} & Alta \\
\hline
\textbf{Comentarios} & La compatibilidad con comunicaciones de largo alcance es clave para conectar los nodos sensores en áreas remotas sin necesidad de infraestructura de red adicional. \\
\hline
\end{longtable}

%%%%%%%%
% RNF06
%%%%%%%%
\begin{longtable}{|l|p{12cm}|}
\hline
\textbf{RNF06} & \textbf{Gestión de Memoria para Evitar Pérdida de Datos.} \\
\hline
\endfirsthead
\hline
\textbf{Versión} & 1.0 - Fecha de versión: 06/06/2025 \\
\hline
\textbf{Autor} & Equipo de Desarrollo \\
\hline
\textbf{Fuente} & Documentación técnica del subsistema de Microcontrolador. \\
\hline
\textbf{Propósito} & Garantizar que no se pierdan datos de los sensores ni la imagen única antes de ser transmitidos al servidor. \\
\hline
\textbf{Descripción} & El microcontrolador debe gestionar eficientemente la memoria para almacenar los datos temporalmente hasta su transmisión al servidor central, asegurando que no haya pérdida de información. \\
\hline
\textbf{Especificación} & La memoria interna o externa debe ser suficiente para almacenar los datos de los sensores y al menos una imagen de 640×640 antes de la transmisión. El sistema debe asegurar que esta capacidad sea suficiente para un ciclo de operación completo sin desbordamiento. \\
\hline
\textbf{Prioridad} & Alta \\
\hline
\textbf{Comentarios} & La gestión de memoria es crucial para asegurar que los datos no se pierdan antes de ser transmitidos, lo que podría comprometer el análisis posterior. \\
\hline
\end{longtable}


%%%%%%%%
% RNF07
%%%%%%%%
%       subsistema 4. Transceptor
\begin{longtable}{|l|p{12cm}|}
\hline
\textbf{RNF07} & \textbf{Fiabilidad de la Comunicación Inalámbrica.} \\
\hline
\endfirsthead
\hline
\textbf{Versión} & 1.0 - Fecha de versión: 06/06/2025 \\
\hline
\textbf{Autor} & Equipo de Desarrollo \\
\hline
\textbf{Fuente} & Documentación técnica del subsistema - Transceptor. \\
\hline
\textbf{Propósito} & Asegurar que los datos transmitidos entre los nodos sensores y el nodo concentrador lleguen de manera confiable y sin pérdidas. \\
\hline
\textbf{Descripción} & La comunicación inalámbrica entre nodos debe ser estable y confiable, garantizando que los datos transmitidos no se pierdan ni corran el riesgo de ser corrompidos. \\
\hline
\textbf{Especificación} & El sistema debe operar con una fiabilidad de transmisión mínima del 99\%, y los datos deben ser enviados sin pérdidas a través de protocolos de comunicación confiables. \\
\hline
\textbf{Prioridad} & Alta \\
\hline
\textbf{Comentarios} & La confiabilidad de la red inalámbrica es crítica, especialmente en entornos remotos, para asegurar que los datos siempre lleguen al nodo concentrador sin interrupciones. \\
\hline
\end{longtable}

%%%%%%%%
% RNF08
%%%%%%%%
\begin{longtable}{|l|p{12cm}|}
\hline
\textbf{RNF08} & \textbf{Consumo Energético del Transceptor.} \\
\hline
\endfirsthead
\hline
\textbf{Versión} & 1.0 - Fecha de versión: 06/06/2025 \\
\hline
\textbf{Autor} & Equipo de Desarrollo \\
\hline
\textbf{Fuente} & Documentación técnica del subsistema - Transceptor. \\
\hline
\textbf{Propósito} & Minimizar el consumo energético del transceptor para garantizar un funcionamiento autónomo durante períodos largos sin necesidad de recarga. \\
\hline
\textbf{Descripción} & El transceptor debe operar con un bajo consumo de energía durante el proceso de transmisión de datos, utilizando protocolos de comunicación de bajo consumo energético. \\
\hline
\textbf{Especificación} & El transceptor debe optimizar su consumo energético, limitando la transmisión de datos solo cuando sea necesario. El consumo de energía no debe exceder los 200 mA durante la transmisión de datos. \\
\hline
\textbf{Prioridad} & Alta \\
\hline
\textbf{Comentarios} & El bajo consumo energético es fundamental para maximizar la autonomía de los nodos en entornos remotos donde la energía es limitada. \\
\hline
\end{longtable}

%%%%%%%%
% RNF09
%%%%%%%%
%       subsistema 5. Fuente de Alimentacion
\begin{longtable}{|l|p{12cm}|}
\hline
\textbf{RNF09} & \textbf{Eficiencia Energética de la Fuente de Alimentación.} \\
\hline
\endfirsthead
\hline
\textbf{Versión} & 1.0 - Fecha de versión: 06/06/2025 \\
\hline
\textbf{Autor} & Equipo de Desarrollo \\
\hline
\textbf{Fuente} & Documentación técnica del subsistema - Fuente de Alimentación. \\
\hline
\textbf{Propósito} & Asegurar que la fuente de alimentación proporcione energía de manera eficiente, con la capacidad de recargar la batería utilizando energía solar. \\
\hline
\textbf{Descripción} & La fuente de alimentación debe ser capaz de proporcionar suficiente energía para todos los componentes del nodo, optimizando la duración de la batería y permitiendo su recarga utilizando energía solar. \\
\hline
\textbf{Especificación} & El panel solar debe ser capaz de recargar completamente la batería en aproximadamente 6 horas de luz solar directa, y la batería debe soportar al menos 24 horas de funcionamiento sin recarga. \\
\hline
\textbf{Prioridad} & Alta \\
\hline
\textbf{Comentarios} & El consumo eficiente de energía es crucial para garantizar el funcionamiento autónomo de los nodos en entornos remotos. La recarga solar optimiza la sostenibilidad del sistema. \\
\hline
\end{longtable}

%%%%%%%%%%%%%%%%%       Modulo 2        %%%%%%%%%%%%%%%%%%
%       subsistema 1. Recepcion de datos de nodos sensores
\subsubsection*{Módulo 2. Nodo concentrador}

%%%%%%%%
% RNF10
%%%%%%%%
\begin{longtable}{|l|p{12cm}|}
\hline
\textbf{RNF10} & \textbf{Fiabilidad de la Recepción de Datos.} \\
\hline
\endfirsthead
\hline
\textbf{Versión} & 1.0 - Fecha de versión: 06/06/2025 \\
\hline
\textbf{Autor} & Equipo de Desarrollo \\
\hline
\textbf{Fuente} & Documentación técnica del subsistema - Recepción de datos de nodos concentradores. \\
\hline
\textbf{Propósito} & Garantizar que los datos recibidos de los nodos sensores lleguen de manera confiable y sin pérdida de información. \\
\hline
\textbf{Descripción} & El subsistema debe ser capaz de recibir los datos transmitidos desde los nodos sensores de manera eficiente y sin pérdidas de información, incluso si los nodos están enviando datos de forma simultánea. \\
\hline
\textbf{Especificación} & La recepción de datos debe garantizar que no haya pérdida de paquetes ni interrupciones en la transmisión, y debe permitir la recepción de múltiples transmisiones de datos periódicas sin errores. \\
\hline
\textbf{Prioridad} & Alta \\
\hline
\textbf{Comentarios} & La fiabilidad es esencial para evitar la pérdida de datos, lo cual podría afectar el análisis y la toma de decisiones. \\
\hline
\end{longtable}

%%%%%%%%
% RNF11
%%%%%%%%
%       subsistema 2. Almacenamiento temporal y organización de datos
\begin{longtable}{|l|p{12cm}|}
\hline
\textbf{RNF11} & \textbf{Eficiencia en el Almacenamiento Temporal.} \\
\hline
\endfirsthead
\hline
\textbf{Versión} & 1.0 - Fecha de versión: 06/06/2025 \\
\hline
\textbf{Autor} & Equipo de Desarrollo \\
\hline
\textbf{Fuente} & Documentación técnica del subsistema - Almacenamiento temporal y organización de datos. \\
\hline
\textbf{Propósito} & Asegurar que los datos sean almacenados temporalmente de forma eficiente y sin sobrecargar el buffer del nodo concentrador. \\
\hline
\textbf{Descripción} & El subsistema debe ser capaz de almacenar los datos temporalmente en un buffer de manera eficiente para evitar que se pierdan o se sobrecargue el sistema. Los datos deben estar disponibles para ser procesados y enviados al servidor en el momento adecuado. \\
\hline
\textbf{Especificación} & El buffer del nodo concentrador debe tener suficiente capacidad para almacenar los datos de al menos 24 horas de monitoreo continuo. Además, el almacenamiento temporal debe tener una latencia mínima para garantizar que los datos se transmitan rápidamente al servidor una vez procesados. \\
\hline
\textbf{Prioridad} & Alta \\
\hline
\textbf{Comentarios} & La gestión eficiente del almacenamiento temporal es crucial para asegurar que no haya pérdida de datos y que el sistema opere sin interrupciones, incluso cuando los nodos sensores envíen datos de manera periódica. \\
\hline
\end{longtable}

%%%%%%%%
% RNF12
%%%%%%%%
\begin{longtable}{|l|p{12cm}|}
\hline
\textbf{RNF12} & \textbf{Organización de Datos para Transmisión.} \\
\hline
\endfirsthead
\hline
\textbf{Versión} & 1.0 - Fecha de versión: 06/06/2025 \\
\hline
\textbf{Autor} & Equipo de Desarrollo \\
\hline
\textbf{Fuente} & Documentación técnica del subsistema de Nodo Concentrador. \\
\hline
\textbf{Propósito} & Organizar los datos de forma eficiente para su transmisión posterior al servidor. \\
\hline
\textbf{Descripción} & Los datos almacenados temporalmente deben ser organizados en una estructura que permita su transmisión rápida y eficiente al servidor. La organización de los datos debe facilitar su acceso y reducir el tiempo de procesamiento. \\
\hline
\textbf{Especificación} & Los datos deben ser organizados en una estructura jerárquica o por categorías que faciliten su transmisión y procesamiento posterior, con un tiempo de acceso no mayor a 5 segundos. \\
\hline
\textbf{Prioridad} & Alta \\
\hline
\textbf{Comentarios} & Una buena organización de los datos permite reducir la latencia de transmisión y asegura que los datos se mantengan estructurados y coherentes. \\
\hline
\end{longtable}

%%%%%%%%
% RNF13
%%%%%%%%
\begin{longtable}{|l|p{12cm}|}
\hline
\textbf{RNF13} & \textbf{Prevención de Sobrecargas o Pérdidas por Desbordamiento de Buffer.} \\
\hline
\endfirsthead
\hline
\textbf{Versión} & 1.0 - Fecha de versión: 06/06/2025 \\
\hline
\textbf{Autor} & Equipo de Desarrollo \\
\hline
\textbf{Fuente} & Documentación técnica del subsistema de Gestión de Memoria. \\
\hline
\textbf{Propósito} & Evitar la sobrecarga o pérdida de datos debido a desbordamientos de buffer en el sistema. \\
\hline
\textbf{Descripción} & El sistema debe estar diseñado para prevenir sobrecargas y desbordamientos de buffer en cualquier etapa del proceso de almacenamiento temporal de datos, garantizando que no haya pérdidas de información debido a la incapacidad de la memoria para manejar grandes cantidades de datos. \\
\hline
\textbf{Especificación} & El sistema debe implementar mecanismos de gestión de memoria que aseguren que el buffer no se sobrecargue, como gestión dinámica de buffer o descarga automática de los datos cuando se acerca el límite de capacidad. Además, el sistema debe ser capaz de alertar cuando se detecte que el buffer está a punto de desbordarse. \\
\hline
\textbf{Prioridad} & Alta \\
\hline
\textbf{Comentarios} & La prevención de sobrecarga o pérdidas de datos es crucial para mantener la integridad de la información recopilada y para evitar fallos en el sistema de monitoreo. \\
\hline
\end{longtable}

%%%%%%%%
% RNF14
%%%%%%%%
%       subsistema 3. Gateway
\begin{longtable}{|l|p{12cm}|}
\hline
\textbf{RNF14} & \textbf{Fiabilidad en el Reenvío de Datos al Servidor.} \\
\hline
\endfirsthead
\hline
\textbf{Versión} & 1.0 - Fecha de versión: 06/06/2025 \\
\hline
\textbf{Autor} & Equipo de Desarrollo \\
\hline
\textbf{Fuente} & Documentación técnica del subsistema - Gateway. \\
\hline
\textbf{Propósito} & Garantizar que los datos procesados y organizados sean transmitidos al servidor central sin pérdidas ni errores. \\
\hline
\textbf{Descripción} & El subsistema Gateway debe ser capaz de transmitir los datos organizados desde el nodo concentrador al servidor central a través de una red confiable. La transmisión debe ser segura y libre de errores. \\
\hline
\textbf{Especificación} & El Gateway debe transmitir datos sin pérdidas a través de una red confiable, garantizando que la integridad de los datos se mantenga durante todo el proceso de transmisión. El tiempo de transmisión debe ser mínimo, asegurando que los datos lleguen al servidor sin retrasos significativos. \\
\hline
\textbf{Prioridad} & Alta \\
\hline
\textbf{Comentarios} & La fiabilidad de la transmisión es crucial para asegurar que los datos sean recibidos por el servidor central y puedan ser procesados y almacenados sin errores. \\
\hline
\end{longtable}

%%%%%%%%%%%%%%%%%%%%%% Modulo 3. Servidor

%%%%%%%%
% RNF15
%%%%%%%%
%       subsistema 1. Recepcion de datos de gateway
\begin{longtable}{|l|p{12cm}|}
\hline
\textbf{RNF15} & \textbf{Validación Automática de Datos Recibidos.} \\
\hline
\endfirsthead
\hline
\textbf{Versión} & 1.0 - Fecha de versión: 06/06/2025 \\
\hline
\textbf{Autor} & Equipo de Desarrollo \\
\hline
\textbf{Fuente} & Documentación técnica del subsistema - Recepción de datos Gateway. \\
\hline
\textbf{Propósito} & Asegurar que los datos recibidos estén en el formato adecuado antes de ser procesados. \\
\hline
\textbf{Descripción} & El sistema debe validar automáticamente que los datos entrantes cumplan con la estructura JSON, los tipos de datos esperados, y que los campos obligatorios estén presentes y completos. \\
\hline
\textbf{Especificación} & Los datos deben ser validados automáticamente para asegurarse de que el formato JSON es correcto, los tipos de datos coinciden con los especificados y los campos obligatorios están presentes. \\
\hline
\textbf{Prioridad} & Alta \\
\hline
\textbf{Comentarios} & La validación asegura que solo datos correctos y estructurados sean procesados, evitando errores en el análisis y almacenamiento. \\
\hline
\end{longtable}

%%%%%%%%
% RNF16
%%%%%%%%
%       subsistema 2. Procesamiento de imagenes 
\begin{longtable}{|l|p{12cm}|}
\hline
\textbf{RNF16} & \textbf{Optimización del Procesamiento y Trazabilidad del Algoritmo de Área de Basura.} \\
\hline
\endfirsthead
\hline
\textbf{Versión} & 1.0 - Fecha de versión: 06/06/2025 \\
\hline
\textbf{Autor} & Equipo de Desarrollo \\
\hline
\textbf{Fuente} & Documentación técnica del subsistema de Visión Artificial. \\
\hline
\textbf{Propósito} & Garantizar que las imágenes procesadas sean almacenadas de manera eficiente y que el algoritmo de cuantificación de área sea trazable para auditoría y mejora continua. \\
\hline
\textbf{Descripción} & Las imágenes procesadas deben ser almacenadas en un formato comprimido (como JPEG o PNG) sin perder la calidad necesaria para el análisis. El sistema debe asegurar la trazabilidad del algoritmo de segmentación utilizado para calcular el área de basura. \\
\hline
\textbf{Especificación} & Las imágenes deben ser almacenadas en formato comprimido sin pérdida de calidad significativa para el análisis, y los resultados de la segmentación deben incluir un nivel de confianza (probabilidad$>$0.85) para filtrar resultados poco fiables. Además, el sistema debe incluir un control de versiones para el algoritmo de cuantificación de área, lo que permite la trazabilidad de mejoras. \\
\hline
\textbf{Prioridad} & Alta \\
\hline
\textbf{Comentarios} & La optimización de la imagen y la trazabilidad de la versión del algoritmo de cuantificación son esenciales para asegurar la fiabilidad de la métrica de área calculada a lo largo del tiempo. \\
\hline
\end{longtable}

%%%%%%%%
% RNF17
%%%%%%%%
%       subsistema 3. 
\begin{longtable}{|l|p{12cm}|}
\hline
\textbf{RNF17} & \textbf{Acceso a Datos Históricos de al Menos un Año.} \\
\hline
\endfirsthead
\hline
\textbf{Versión} & 1.0 - Fecha de versión: 06/06/2025 \\
\hline
\textbf{Autor} & Equipo de Desarrollo \\
\hline
\textbf{Fuente} & Documentación técnica del subsistema de Base de Datos. \\
\hline
\textbf{Propósito} & Permitir el acceso a los datos históricos de al menos un año para su análisis y consulta. \\
\hline
\textbf{Descripción} & El sistema debe garantizar que los datos históricos (como pH, turbidez, temperatura, etc.) estén disponibles para su consulta en un período de al menos un año. Este acceso debe realizarse mediante consultas eficientes a la base de datos. \\
\hline
\textbf{Especificación} & Los datos históricos deben ser almacenados y accesibles durante al menos un año, y las consultas deben ser capaces de recuperar datos de manera eficiente (menos de 2 segundos por consulta). \\
\hline
\textbf{Prioridad} & Alta \\
\hline
\end{longtable}

%%%%%%%%
% RNF18
%%%%%%%%
%       subsistema 4. 
\begin{longtable}{|l|p{12cm}|}
\hline
\textbf{RNF18} & \textbf{Rendimiento y Seguridad de la API.} \\
\hline
\endfirsthead
\hline
\textbf{Versión} & 1.0 - Fecha de versión: 06/06/2025 \\
\hline
\textbf{Autor} & Equipo de Desarrollo \\
\hline
\textbf{Fuente} & Documentación técnica del subsistema de API. \\
\hline
\textbf{Propósito} & Garantizar el rendimiento, seguridad y capacidad de la API para manejar múltiples usuarios concurrentes. \\
\hline
\textbf{Descripción} & La API debe responder en menos de 1 segundo en condiciones normales. Además, debe cifrar toda la comunicación entre el servidor y la página web (por ejemplo, mediante HTTPS) y debe soportar al menos 50 usuarios concurrentes sin que el rendimiento se degrade. \\
\hline
\textbf{Especificación} & La API debe responder en menos de 1 segundo en condiciones normales. La comunicación entre el servidor y la página web debe estar cifrada (HTTPS) y la API debe ser capaz de soportar al menos 50 usuarios concurrentes sin que se degrade el rendimiento. \\
\hline
\textbf{Prioridad} & Alta \\
\hline
\end{longtable}

%%%%%%%%%%%%%%%%%%%%%   Modulo 4. Página Web

%%%%%%%%
% RNF19
%%%%%%%%
%       subsistema 2
\begin{longtable}{|l|p{12cm}|}
\hline
\textbf{RNF19} & \textbf{Accesibilidad y Responsividad de la Interfaz.} \\
\hline
\endfirsthead
\hline
\textbf{Versión} & 1.0 - Fecha de versión: 06/06/2025 \\
\hline
\textbf{Autor} & Equipo de Desarrollo \\
\hline
\textbf{Fuente} & Documentación técnica del subsistema de Interfaz de Usuario. \\
\hline
\textbf{Propósito} & Asegurar que la interfaz sea accesible y responsiva en computadoras, brindando una experiencia de usuario óptima. \\
\hline
\textbf{Descripción} & La interfaz debe ser totalmente accesible en computadoras de escritorio y portátiles. Además, debe ajustarse automáticamente a diferentes tamaños de pantalla para asegurar que el contenido sea visible y fácil de usar en todos los dispositivos. \\
\hline
\textbf{Especificación} & La interfaz debe ser responsiva y adaptarse a pantallas de tamaños diversos, asegurando que el diseño se ajuste correctamente en computadoras. El contenido debe ser legible y funcional en pantallas pequeñas o grandes sin perder usabilidad. \\
\hline
\textbf{Prioridad} & Alta \\
\hline
\textbf{Comentarios} & La responsividad es esencial para asegurar que los usuarios puedan acceder al sistema de forma eficiente desde computadoras, independientemente del tamaño de la pantalla. \\
\hline
\end{longtable}

%%%%%%%%
% RNF20
%%%%%%%%
\begin{longtable}{|l|p{12cm}|}
\hline
\textbf{RNF20} & \textbf{Diseño Claro y Estético de la Interfaz.} \\
\hline
\endfirsthead
\hline
\textbf{Versión} & 1.0 - Fecha de versión: 06/06/2025 \\
\hline
\textbf{Autor} & Equipo de Desarrollo \\
\hline
\textbf{Fuente} & Documentación técnica del subsistema de Interfaz de Usuario. \\
\hline
\textbf{Propósito} & Garantizar que la interfaz sea visualmente clara, estética y agradable para el usuario. \\
\hline
\textbf{Descripción} & La interfaz debe ser diseñada con un estilo limpio, intuitivo y funcional, donde los elementos sean fáciles de encontrar y utilizar. El diseño debe adaptarse a las necesidades del usuario, proporcionando colores y tipografías que mejoren la legibilidad y la experiencia general. \\
\hline
\textbf{Especificación} & El sistema debe tener un diseño visual claro, con una estructura lógica de navegación y un estilo agradable que facilite el uso y la comprensión del sistema. La paleta de colores debe ser coherente y accesible, asegurando un contraste adecuado para facilitar la lectura y la interacción. \\
\hline
\textbf{Prioridad} & Alta \\
\hline
\textbf{Comentarios} & Un diseño claro y estético es clave para una experiencia de usuario exitosa, reduciendo la curva de aprendizaje y aumentando la satisfacción del usuario. \\
\hline
\end{longtable}

\newcounter{RF}
\makeatletter
\newcommand{\RF}[1][]{%
  \refstepcounter{RF}%
  RF\@tempcnta=\value{RF}%
  \ifnum\@tempcnta<10 0\fi%
  \arabic{RF}%
  \if\relax#1\relax\else\label{#1}\fi%
}
\makeatother

\newcounter{RNF}
\makeatletter
\newcommand{\RNF}[1][]{%
  \refstepcounter{RNF}%
  RNF\@tempcnta=\value{RNF}%
  \ifnum\@tempcnta<10 0\fi%
  \arabic{RNF}%
  \if\relax#1\relax\else\label{#1}\fi%
}
\makeatother




\section{Análisis de Requerimientos}
\label{sec:analisis_requerimientos} % Added section label

A continuación, se presentan los requerimientos funcionales (RF) y no funcionales (RNF) identificados para el proyecto, agrupados por el componente principal responsable de su cumplimiento. Estos servirán como base de referencia a lo largo de las etapas de diseño, implementación y pruebas.

\subsection{Requerimientos Funcionales}
\label{subsec:rf_summary} % Opcional: etiqueta para la subsección

A manera de resumen, la siguiente lista contiene los títulos de todos los requerimientos funcionales, agrupados por el componente principal responsable:

% --- RFs Nodo Sensor ---
\subsubsection*{Nodo Sensor}
\begin{itemize}
    \item \textbf{RF01}: Obtener parámetros de calidad del agua.
    \item \textbf{RF02}: Proteger el nodo sensor.
    \item \textbf{RF03}: Capturar imágenes para detección de residuos.
    \item \textbf{RF04}: Coordinar el ciclo de operación de sensores y cámara.
    \item \textbf{RF05}: Adquirir datos y metadatos de imagen.
    \item \textbf{RF06}: Gestionar búfer y memoria temporal del nodo sensor.
    \item \textbf{RF07}: Estructurar y entramar los datos.
    \item \textbf{RF08}: Garantizar integridad de tramas antes de la transmisión.
    \item \textbf{RF09}: Gestionar energía del nodo.
    \item \textbf{RF11}: Suministrar energía a los componentes del nodo.
    \item \textbf{RF12}: Recargar la batería con energía solar.
\end{itemize}

% --- RFs Nodo Concentrador / Gateway ---
\subsubsection*{Nodo Concentrador / Gateway}
\begin{itemize}
    \item \textbf{RF13}: Recibir datos de nodos sensores.
    \item \textbf{RF14}: Almacenar temporalmente y organizar datos en Gateway.
    \item \textbf{RF15}: Transmitir datos agregados al servidor central.
    \item \textbf{RF16}: Implementar protocolo de fiabilidad para transmisión al Servidor.
    \item \textbf{RF17}: Retransmitir datos al servidor en caso de fallo de conexión.
\end{itemize}

% --- RFs Red WSN ---
\subsubsection*{Red Inalámbrica de Sensores (Comunicación Inter-Nodo)}
\begin{itemize}
    \item \textbf{RF10}: Transmitir inalámbricamente datos inter-nodo.
\end{itemize}

% --- RFs Servidor ---
\subsubsection*{Servidor (Backend)}
\begin{itemize}
    \item \textbf{RF18}: Recibir datos de Gateway.
    \item \textbf{RF19}: Organizar y parsear datos recibidos.
    \item \textbf{RF20}: Procesar imágenes para cuantificar área de basura.
    \item \textbf{RF21}: Almacenar persistentemente datos y metadatos.
    \item \textbf{RF22}: Exponer datos y servicios mediante API (RESTful).
\end{itemize}

% --- RFs Aplicación Web ---
\subsubsection*{Aplicación Web (Frontend)}
\begin{itemize}
    \item \textbf{RF23}: Consumir servicios del servidor API.
    \item \textbf{RF24}: Visualizar datos actuales y resumen.
    \item \textbf{RF25}: Permitir la visualización de datos históricos.
\end{itemize}

% --- INICIO LISTA RNF ---
\subsection{Requerimientos No Funcionales}
\label{subsec:rnf_summary} % Opcional: etiqueta para la subsección

A continuación, se resumen los títulos de todos los requerimientos no funcionales, agrupados por el componente principal al que aplican:

% --- RNFs Nodo Sensor ---
\subsubsection*{Nodo Sensor}
\begin{itemize}
    \item \textbf{RNF01}: Precisión de los sensores.
    \item \textbf{RNF02}: Protección física y ambiental del nodo sensor.
    \item \textbf{RNF03}: Calidad de la imagen capturada.
    \item \textbf{RNF04}: Eficiencia energética del microcontrolador.
    \item \textbf{RNF05}: Capacidad de interfaz con transceptor de medio/largo alcance.
    \item \textbf{RNF06}: Capacidad de memoria del nodo sensor.
    \item \textbf{RNF08}: Eficiencia energética del transceptor.
    \item \textbf{RNF09}: Eficiencia y autonomía del subsistema de alimentación.
\end{itemize}

% --- RNFs Nodo Concentrador / Gateway ---
\subsubsection*{Nodo Concentrador / Gateway}
\begin{itemize}
    \item \textbf{RNF10}: Fiabilidad de la recepción de datos desde la WSN.
    \item \textbf{RNF11}: Capacidad y eficiencia del almacenamiento temporal del Gateway.
    \item \textbf{RNF12}: Formato de datos para transmisión al servidor.
    \item \textbf{RNF13}: Gestión del desbordamiento del buffer del Gateway.
    \item \textbf{RNF14}: Fiabilidad del enlace de comunicación Gateway-Servidor.
\end{itemize}

% --- RNFs Red WSN ---
\subsubsection*{Red Inalámbrica de Sensores (Comunicación Inter-Nodo)}
\begin{itemize}
    \item \textbf{RNF07}: Fiabilidad de la comunicación inalámbrica inter-nodo.
\end{itemize}

% --- RNFs Servidor ---
\subsubsection*{Servidor (Backend)}
\begin{itemize}
    \item \textbf{RNF15}: Validación de datos en la ingesta.
    \item \textbf{RNF16}: Eficiencia y trazabilidad del procesamiento de imágenes.
    \item \textbf{RNF17}: Retención y rendimiento de consultas de datos históricos.
    \item \textbf{RNF18}: Rendimiento, seguridad y escalabilidad de la API.
\end{itemize}

% --- RNFs Aplicación Web ---
\subsubsection*{Aplicación Web (Frontend)}
\begin{itemize}
    \item \textbf{RNF19}: Accesibilidad y diseño responsivo de la interfaz.
    \item \textbf{RNF20}: Claridad, estética y usabilidad de la interfaz (UI/UX).
\end{itemize}

% ======================================================
% Componente: Nodo Sensor (Individual)
% ======================================================
\subsection{Nodo Sensor}
\label{subsec:req_nodo_sensor}

Requerimientos asociados a las capacidades y características de cada nodo sensor individual desplegado en la red.

\subsubsection{Requerimientos Funcionales (Nodo Sensor)}

%%%%%%%% RF01 - Obtención Parámetros %%%%%%%%
\begin{longtable}{|l|p{0.8\textwidth}|}
\hline
\rowcolor{orange!15}
\textbf{\RF} & \textbf{Obtener parámetros de calidad del agua.} \\
\hline
\endfirsthead
\multicolumn{2}{r}{\textit{Continúa en la siguiente página}} \\
\endfoot
\endlastfoot
\textbf{Versión} & 1.0 - Fecha de versión: 06/06/2025 \\ \hline
\textbf{Fuente} & Documentación técnica del subsistema - Sensado de Parámetros de Calidad del Agua.\\ \hline
\textbf{Propósito} & Obtener mediciones periódicas de la calidad del agua a través de sensores especializados. \\ \hline
\textbf{Descripción} & El subsistema de sensado incluirá sensores de pH, oxígeno disuelto, turbidez, conductividad y temperatura, los cuales tomarán mediciones de los parámetros de calidad del agua en intervalos predefinidos.\\ \hline
\textbf{Especificación} & Los sensores deben medir parámetros de calidad del agua, y deben proporcionar mediciones precisas a intervalos regulares. Los datos deben ser transmitidos periódicamente al Nodo concentrador a través de una red de sensores inalámbricos para ser procesados. \\ \hline
\textbf{Prioridad} & Alta \\ \hline
\textbf{Comentarios} & La obtención de parámetros de calidad del agua es crucial para la información y prevención de problemas de contaminación en cuerpos de agua utilizados para consumo humano o en procesos industriales. El monitoreo periódico permitirá la intervención temprana. \\ \hline
\end{longtable}



%%%%%%%% RF02 - Protección Nodo %%%%%%%%
\begin{longtable}{|l|p{0.8\textwidth}|}
\hline
\textbf{\RF} & \textbf{Proteger el nodo sensor.} \\
\hline
\endfirsthead
\multicolumn{2}{r}{\textit{Continúa en la siguiente página}} \\
\endfoot
\endlastfoot
\textbf{Versión} & 1.0 - Fecha de versión: 06/06/2025 \\ \hline
\textbf{Fuente} & Documentación técnica del subsistema - Sensado de Parámetros. \\ \hline
\textbf{Propósito} & Proteger los componentes del nodo sensor mediante una boya flotante con protección IP adecuada para ambientes acuáticos. \\ \hline
\textbf{Descripción} & La boya debe estar diseñada con materiales resistentes para ser hermético al agua y debe proteger todos los componentes internos, los sensores y otros elementos electrónicos contra la humedad y el contacto con el agua. Esta estructura debe proteger los componentes internos y permitir el funcionamiento confiable cuando se utilice en cuerpos de agua. \\ \hline
\textbf{Especificación} & La estructura del nodo sensor debe tener una protección mínima de IP68, lo que asegura que es impermeable al agua y polvo. Además, el nodo debe ser capaz de soportar el movimiento brusco en condiciones desfavorables para el sistema, sin comprometer su funcionalidad. \\ \hline
\textbf{Prioridad} & Alta \\ \hline
\textbf{Comentarios} & La protección adecuada de los sensores es esencial para asegurar su rendimiento y vida útil en ambientes acuáticos. \\ \hline
\end{longtable}

%%%%%%%% RF03 - Captura Imágenes %%%%%%%%
\begin{longtable}{|l|p{0.8\textwidth}|}
\hline
\textbf{\RF} & \textbf{Capturar imágenes para detección de residuos.} \\
\hline
\endfirsthead
\multicolumn{2}{r}{\textit{Continúa en la siguiente página}} \\
\endfoot
\endlastfoot
\textbf{Versión} & 1.0 - Fecha de versión: 06/06/2025 \\ \hline
\textbf{Fuente} & Documentación técnica del subsistema - Camára. \\ \hline
\textbf{Propósito} & Capturar imágenes periódicas de la superficie del agua para permitir la cuantificación del área cubierta por cúmulos de residuos flotantes. \\ \hline
\textbf{Descripción} & El sistema debe utilizar una cámara de bajo consumo energético para capturar imágenes de la superficie del agua, las cuales serán transmitidas al Servidor para su procesamiento. El nodo sensor solo gestionará una imagen a la vez en su memoria antes de su transmisión. \\ \hline
\textbf{Especificación} & La imagen original debe ser almacenada temporalmente y transmitida al nodo concentrador para su posterior envío al servidor central sin procesamiento ni clasificación previo en el nodo sensor. El nodo sensor no debe almacenar más de una imagen de 640×640 simultáneamente para su procesamiento y transmisión. \\ \hline
\textbf{Prioridad} & Alta \\ \hline
\textbf{Comentarios} & La captura precise de imágenes es esencial para identificar residuos flotantes en la superficie del agua. La cámara debe ser capaz de operar en condiciones de luz natural y estar protegida de las condiciones ambientales adversas. \\ \hline
\end{longtable}

%%%%%%%% RF04 - Coordinación Ciclo %%%%%%%%
\begin{longtable}{|l|p{0.8\textwidth}|}
\hline
\textbf{\RF[rf:coordinacion_sensores]} & \textbf{Coordinar el ciclo de operación de sensores y cámara} \\
\hline
\endfirsthead
\multicolumn{2}{r}{\textit{Continúa en la siguiente página}} \\
\endfoot
\endlastfoot
\textbf{Versión} & 1.0 - Fecha de versión: 06/06/2025 \\ \hline
\textbf{Fuente} & Documentación técnica del subsistema - Microcontrolador.\\ \hline
\textbf{Propósito} & Coordinar y sincronizar la activación y desactivación de los componentes, optimizando el consumo energético, para asegurar la adquisición de datos de manera eficiente.\\ \hline
\textbf{Descripción} & El microcontrolador deberá gestionar un ciclo de operación predefinido que activará secuencialmente los sensores y la cámara solo cuando sea necesario, y los pondrá en modo de reposo.\\ \hline
\textbf{Especificación} &  El microcontrolador deberá ser capaz de activar y desactivar los componentes eficientemente.El sistema debe asegurar que los componentes no esenciales entren en modo de reposo o baja energía cuando el ciclo de adquisición no esté activo.\\ \hline
\textbf{Prioridad} & Alta \\ \hline
\textbf{Comentarios} & La coordinación eficiente es esencial para garantizar que los datos y las imágenes se adquieran sin interferencias ni retrasos y que la autonomía del sistema se mantenga.\\ \hline
\end{longtable}

%%%%%%%% RF05 - Adquisición Datos %%%%%%%%
\begin{longtable}{|l|p{0.8\textwidth}|}
\hline
\textbf{\RF} & \textbf{Adquirir datos y metadatos de imagen.} \\
\hline
\endfirsthead
\multicolumn{2}{r}{\textit{Continúa en la siguiente página}} \\
\endfoot
\endlastfoot
\textbf{Versión} & 1.0 - Fecha de versión: 06/06/2025 \\ \hline
\textbf{Fuente} & Documentación técnica del subsistema - Microcontrolador.\\ \hline
\textbf{Propósito} & Obtener las mediciones de los sensores y la captura de la imagen, asegurando su entrega al búfer para su posterior empaquetado y transmisión.\\ \hline
\textbf{Descripción} & Una vez activados, los sensores y la cámara deberán entregar sus datos y capturas, los cuales serán almacenados en el búfer del microcontrolador.\\ \hline
\textbf{Especificación} &  El sistema deberá adquirir mediciones de pH, turbidez, oxígeno disuelto, conductividad y temperatura conforme al ciclo de operación coordinado y almacenar los datos de los sensores y las imágenes temporalmente en el búfer del nodo sensor hasta su transmisión al nodo concentrador.\\ \hline
\textbf{Prioridad} & Alta \\ \hline
\textbf{Comentarios} & La correcta adquisición y almacenamiento temporal son pasos críticos para garantizar la integridad y disponibilidad de toda la información (parámetros e imágenes) antes de que se empaqueten y se envíen al servidor.\\ \hline
\end{longtable}

%%%%%%%% RF06 - Gestión Búfer Nodo %%%%%%%%
\begin{longtable}{|l|p{0.8\textwidth}|}
\hline
\textbf{\RF} & \textbf{Gestionar búfer y memoria temporal del nodo sensor.} \\
\hline
\endfirsthead
\multicolumn{2}{r}{\textit{Continúa en la siguiente página}} \\
\endfoot
\endlastfoot
\textbf{Versión} & 1.0 - Fecha de versión: 06/06/2025 \\ \hline
\textbf{Fuente} & Documentación técnica del subsistema - Microcontrolador.\\ \hline
\textbf{Propósito} & Gestionar la memoria interna del microcontrolador para almacenar temporalmente los datos adquiridos (sensores y la imagen única) antes de su transmisión, evitando la sobrecarga y asegurando la no pérdida de datos.\\ \hline
\textbf{Descripción} & 	El microcontrolador debe utilizar su memoria interna para almacenar de forma temporal los datos adquiridos de los sensores y una única imagen antes de que sean transmitidos al nodo concentrador.La memoria debe gestionarse eficientemente para evitar la sobrecarga. \\ \hline
\textbf{Especificación} & El microcontrolador debe ser capaz de almacenar los datos de todos los sensores y una sola imagen de 640×640 antes de la transmisión.El sistema debe permitir el borrado o sobrescritura de datos antiguos cuando la memoria esté llena, para garantizar la disponibilidad de espacio para nuevas mediciones.\\ \hline
\textbf{Prioridad} & Alta \\ \hline
\textbf{Comentarios} & 	La gestión de memoria es crucial para asegurar que los datos no se pierdan antes de ser transmitidos, lo que podría comprometer el análisis posterior.\\ \hline
\end{longtable}

%%%%%%%% RF07 - Empaquetado Tramas %%%%%%%%
\begin{longtable}{|l|p{0.8\textwidth}|}
\hline
\textbf{\RF[rf:Empaquetado_datos]} & \textbf{Estructurar y entramar los datos.} \\
\hline
\endfirsthead
\multicolumn{2}{r}{\textit{Continúa en la siguiente página}} \\
\endfoot
\endlastfoot
\textbf{Versión} & 1.0 - Fecha de versión: 06/06/2025 \\ \hline
\textbf{Fuente} & Documentación técnica del subsistema - Microcontrolador.\\ \hline
\textbf{Propósito} & Organizar y empaquetar los datos obtenidos de los sensores y cámara en tramas de información de manera estructurada para su eficiente transmisión y posterior procesamiento.\\ \hline
\textbf{Descripción} & El microcontrolador debe organizar los datos adquiridos de los sensores y cámara en tramas de información, incluyendo campos como tipo de medición, valor, fecha y hora.\\ \hline
\textbf{Especificación} & Las tramas deben estar estructuradas de manera que contengan campos definidos como: tipo de medición, valor de la medición, fecha y hora de la medición, y cualquier metadato necesario.\\ \hline
\textbf{Prioridad} & Alta \\ \hline
\textbf{Comentarios} & El correcto empaquetado de las tramas de información es esencial para garantizar la fiabilidad de los datos.La estructura de las tramas debe ser clara y uniforme.\\ \hline
\end{longtable}

%%%%%%%% RF08 - Integridad Tramas Pre-Tx %%%%%%%%
\begin{longtable}{|l|p{0.8\textwidth}|}
\hline
\textbf{\RF} & \textbf{Garantizar integridad de tramas antes de la transmisión.} \\
\hline
\endfirsthead
\multicolumn{2}{r}{\textit{Continúa en la siguiente página}} \\
\endfoot
\endlastfoot
\textbf{Versión} & 1.0 - Fecha de versión: 06/06/2025 \\ \hline
\textbf{Fuente} & Documentación técnica del subsistema - Microcontrolador.\\ \hline
\textbf{Propósito} & Asegurar que las tramas de información empaquetadas estén completas, sin errores, y listas para su transmisión al nodo concentrador.\\ \hline
\textbf{Descripción} & Una vez empaquetados (RF\ref{rf:Empaquetado_datos}), el microcontrolador debe aplicar métodos de verificación (ej. checksum) para asegurar que las tramas de datos están completas y sin datos faltantes o errores internos antes de iniciar la transmisión.\\ \hline
\textbf{Especificación} & El sistema debe asegurar que las tramas sean completas, sin datos faltantes, y sin errores de transmisión , y estas tramas deben ser enviadas al servidor o nodo concentrador de manera periódica, en intervalos predefinidos, asegurando que todos los datos sean correctamente transmitidos y almacenados para su análisis posterior.\\ \hline
\textbf{Prioridad} & Alta \\ \hline
\textbf{Comentarios} & Esta validación previa es un paso crítico para reducir la tasa de error en la transmisión inalámbrica y garantizar la calidad de los datos.\\ \hline
\end{longtable}

%%%%%%%% RF09 - Gestión Energía Propia %%%%%%%%
\begin{longtable}{|l|p{0.8\textwidth}|}
\hline
\textbf{\RF} & \textbf{Gestionar energía del nodo.} \\
\hline
\endfirsthead
\multicolumn{2}{r}{\textit{Continúa en la siguiente página}} \\
\endfoot
\endlastfoot
\textbf{Versión} & 1.0 - Fecha de versión: 06/06/2025 \\ \hline
\textbf{Fuente} & Documentación técnica del subsistema - Microcontrolador.\\ \hline
\textbf{Propósito} & Optimizar el consumo energético de los sensores y otros componentes del sistema para garantizar su funcionamiento eficiente en entornos remotos.\\ \hline
\textbf{Descripción} & El microcontrolador debe gestionar el consumo energético de los sensores y otros componentes, asegurando que solo estén activos cuando sea necesario y que los componentes entren en modo de reposo para maximizar la duración de la batería.\\ \hline
\textbf{Especificación} & El microcontrolador debe ser capaz de activar y desactivar los sensores y otros componentes del sistema de forma eficiente, gestionando el consumo de energía para operar durante largos períodos sin recargar.\\ \hline
\textbf{Prioridad} & Alta \\ \hline
% Comentario faltante en el original, se puede añadir si es necesario
\end{longtable}

%%%%%%%% RF11 - Proporcionar Energía %%%%%%%%
\begin{longtable}{|l|p{0.8\textwidth}|}
\hline
\textbf{\RF} & \textbf{Suministrar energía a los componentes del nodo.} \\
\hline
\endfirsthead
\multicolumn{2}{r}{\textit{Continúa en la siguiente página}} \\
\endfoot
\endlastfoot
\textbf{Versión} & 1.0 - Fecha de versión: 06/06/2025 \\ \hline
\textbf{Fuente} & Documentación técnica del subsistema - Fuente de Alimentación.\\ \hline
\textbf{Propósito} & Asegurar que todos los componentes del nodo, incluidos el microcontrolador, los sensores, la cámara y el transceptor, reciban la energía necesaria para su funcionamiento.\\ \hline
\textbf{Descripción} & El subsistema debe ser capaz de proporcionar energía de manera continua a todos los componentes del nodo, de modo que el sistema funcione de forma autónoma sin depender de fuentes externas de energía.\\ \hline
\textbf{Especificación} & La batería debe proporcionar energía suficiente para el funcionamiento de los componentes del nodo durante un período mínimo de 24 horas sin necesidad de recarga.La batería debe ser lo suficientemente potente para soportar el consumo combinado de todos los dispositivos conectados en el nodo.\\ \hline
\textbf{Prioridad} & Alta \\ \hline
\textbf{Comentarios} & Este subsistema es fundamental para que el sistema funcione de manera autónoma en entornos remotos donde el acceso a fuentes de energía externas es limitado o inexistente.\\ \hline
\end{longtable}

%%%%%%%% RF12 - Recarga Solar %%%%%%%%
\begin{longtable}{|l|p{0.8\textwidth}|}
\hline
\textbf{\RF} & \textbf{Recargar la batería con energía solar.} \\
\hline
\endfirsthead
\multicolumn{2}{r}{\textit{Continúa en la siguiente página}} \\
\endfoot
\endlastfoot
\textbf{Versión} & 1.0 - Fecha de versión: 06/06/2025 \\ \hline
\textbf{Fuente} & Documentación técnica del subsistema de Fuente de Alimentación.\\ \hline
\textbf{Propósito} & Asegurar que la batería se recargue utilizando energía solar de manera continua para mantener el funcionamiento autónomo del nodo.\\ \hline
\textbf{Descripción} & El panel solar debe ser capaz de recargar la batería durante el día, aprovechando la energía solar, para que el nodo siga funcionando de manera autónoma.\\ \hline
\textbf{Especificación} & El panel solar debe recargar completamente la batería durante aproximadamente 6 horas de luz solar directa.El sistema debe tener circuitos de protección para evitar sobrecarga o daño a la batería.\\ \hline
\textbf{Prioridad} & Alta \\ \hline
\textbf{Comentarios} & Este subsistema es esencial para garantizar el funcionamiento continuo del sistema en entornos remotos y ayuda a reducir la dependencia de fuentes externas de energía.\\ \hline
\end{longtable}

\subsubsection{Requerimientos No Funcionales (Nodo Sensor)}

%%%%%%%% RNF01 - Precisión Sensores %%%%%%%%
\begin{longtable}{|l|p{0.8\textwidth}|}
\hline
\textbf{\RNF} & \textbf{Precisión de los sensores.} \\ 
\hline
\endfirsthead
\multicolumn{2}{r}{\textit{Continúa en la siguiente página}} \\
\endfoot
\endlastfoot
\textbf{Versión} & 1.0 - Fecha de versión: 06/06/2025 \\ \hline
\textbf{Fuente} & Documentación técnica del subsistema - Sensado de Parámetros de Calida del Agua. \\ \hline
\textbf{Propósito} & Garantizar que los sensores proporcionen mediciones precisas y fiables de los parámetros de calidad del agua. \\ \hline
\textbf{Descripción} & Los sensores deben ser capaces de medir los parámetros de calidad del agua: pH, oxígeno disuelto, turbidez, conductividad, con un margen de error mínimo en la medición, cumpliendo con las normas de precisión establecidas (NOM). \\ \hline
\textbf{Especificación} & Los sensores deben cumplir con la precisión adecuada. El margen de error máximo debe ser del 2\% para turbidez y de ±0.1 unidades para el pH, y estar dentro de los límites de las Normas Oficiales Mexicanas (NOM). \\ \hline
\textbf{Prioridad} & Alta \\ \hline
\textbf{Comentarios} & La precisión de los sensores es fundamental para asegurar que los datos obtenidos sean confiables y útiles para la toma de decisiones. \\ \hline
\end{longtable}

%%%%%%%% RNF02 - Certificación IP %%%%%%%%
\begin{longtable}{|l|p{0.8\textwidth}|}
\hline
\textbf{\RNF} & \textbf{Protección física y ambiental del nodo sensor.} \\
\hline
\endfirsthead
\multicolumn{2}{r}{\textit{Continúa en la siguiente página}} \\
\endfoot
\endlastfoot
\textbf{Versión} & 1.0 - Fecha de versión: 06/06/2025 \\ \hline
\textbf{Fuente} & Documentación técnica del subsistema - Sensado de Parámetros de Calidad del Agua. \\ \hline
\textbf{Propósito} & Garantizar que la carcasa del nodo sensor tenga una clasificación de protección IP adecuada para soportar inmersión en cuerpos de agua. \\ \hline
\textbf{Descripción} & La carcasa del nodo sensor debe contar con una clasificación IP adecuada que garantice su resistencia al agua y su capacidad para operar bajo condiciones de inmersión en cuerpos de agua, como ríos, arroyos y lagos, considerando pesos y ubicación de los componentes. \\ \hline
\textbf{Especificación} & La carcasa debe tener al menos una clasificación IP68, lo que asegura que el nodo sensor sea completamente impermeable al agua y resistente al polvo. La boya debe ser capaz de flotar sin que la funcionalidad de los componentes se vea comprometida. \\ \hline
\textbf{Prioridad} & Alta \\ \hline
\textbf{Comentarios} & La resistencia al agua es esencial para garantizar que el nodo sensor funcione correctamente en entornos acuáticos sin riesgo de fallos debido a la exposición al agua o la humedad. \\ \hline
\end{longtable}

%%%%%%%% RNF03 - Calidad Imagen %%%%%%%%
\begin{longtable}{|l|p{0.8\textwidth}|}
\hline
\textbf{\RNF} & \textbf{Calidad de la imagen capturada.} \\ 
\hline
\endfirsthead
\multicolumn{2}{r}{\textit{Continúa en la siguiente página}} \\
\endfoot
\endlastfoot
\textbf{Versión} & 1.0 - Fecha de versión: 06/06/2025 \\ \hline
\textbf{Fuente} & Documentación técnica del subsistema - Cámara. \\ \hline
\textbf{Propósito} & Garantizar que la cámara capture imágenes de la calidad necesaria para el cálculo preciso del área de residuos flotantes en el servidor. \\ \hline
\textbf{Descripción} & La cámara debe ser capaz de capturar imágenes nítidas y claras bajo diversas condiciones de iluminación y con la resolución 640×640 necesaria para el análisis de segmentación. \\ \hline
\textbf{Especificación} & La cámara debe ser capaz de capturar imágenes con una resolución mínima de 640×640 y una calidad de imagen que permita identificar claramente los cúmulos de residuos flotantes en el agua. La calidad de la imagen debe ser consistente incluso con variaciones de luz diurna. \\ \hline
\textbf{Prioridad} & Alta \\ \hline
\textbf{Comentarios} & La precisión en la captura de imágenes es esencial para una detección precisa de los residuos flotantes y para garantizar la calidad del análisis posterior. \\ \hline
\end{longtable}

%%%%%%%% RNF04 - Consumo MCU %%%%%%%%
\begin{longtable}{|l|p{0.8\textwidth}|}
\hline
\textbf{\RNF} & \textbf{Eficiencia energética del microcontrolador.} \\ 
\hline
\endfirsthead
\multicolumn{2}{r}{\textit{Continúa en la siguiente página}} \\
\endfoot
\endlastfoot
\textbf{Versión} & 1.1 - Fecha de versión: 22/10/2025 \\ \hline
\textbf{Fuente} & Documentación técnica del subsistema de Microcontrolador. \\ \hline
\textbf{Propósito} & Asegurar que los microcontroladores seleccionados operen con un consumo de energía compatible con los objetivos de autonomía del nodo. \\ \hline
\textbf{Descripción} & Los microcontroladores (tanto principal como auxiliar, si aplica) deben presentar un consumo de corriente bajo en modo activo y, de forma crítica, un consumo ultra bajo en los modos de reposo (sleep/deep sleep) para maximizar la vida útil de la batería durante los largos periodos de inactividad del nodo. \\ \hline
\textbf{Especificación} & Los consumos de corriente de los microcontroladores seleccionados deben ser coherentes con los valores utilizados en el cálculo del presupuesto energético (Tabla \ref{tab:componentes_energia}). Específicamente, el consumo en modo deep sleep debe ser del orden de microamperios (\si{\micro\ampere}) o decenas/cientos de microamperios, permitiendo alcanzar la autonomía calculada en la Sección \ref{subsec:seleccion_hardware_alimentacion}. \\ \hline % TODO: Specification made general, removed specific MCU names and citations
\textbf{Prioridad} & Alta \\ \hline
\textbf{Comentarios} & El consumo del microcontrolador, especialmente en modo sleep/deep sleep, es un factor dominante en la autonomía total del nodo sensor, dado el bajo ciclo de trabajo del sistema. \\ \hline
\end{longtable}

%%%%%%%% RNF05 - Compatibilidad Largo Alcance %%%%%%%%
\begin{longtable}{|l|p{0.8\textwidth}|}
\hline
\textbf{\RNF} & \textbf{Capacidad de interfaz con transceptor de medio/largo alcance.} \\ 
\hline
\endfirsthead
\multicolumn{2}{r}{\textit{Continúa en la siguiente página}} \\
\endfoot
\endlastfoot
\textbf{Versión} & 1.1 - Fecha de versión: 22/10/2025 \\ \hline
\textbf{Fuente} & Documentación técnica del subsistema de Microcontrolador. \\ \hline
\textbf{Propósito} & Garantizar que el microcontrolador sea compatible con tecnologías de comunicación de largo alcance. \\ \hline
\textbf{Descripción} & El microcontrolador debe ser compatible con tecnologías de comunicación de largo alcance, para garantizar la transmisión eficiente de datos entre nodos y el servidor central en áreas remotas. \\ \hline
\textbf{Especificación} & El microcontrolador debe ser compatible con LoRa o tecnologías similares como WiFI, de bajo consumo y largo alcance, garantizando una transmisión de datos confiable sobre las distancias inter-nodo definidas por el diseño. \\ % TODO: Review/Revise Specification Text - Mentions specific tech/distance prematurely
\hline
\textbf{Prioridad} & Alta \\ \hline
\textbf{Comentarios} & La compatibilidad con comunicaciones de largo alcance es clave para conectar los nodos sensores en áreas remotas sin necesidad de infraestructura de red adicional. \\ \hline
\end{longtable}

%%%%%%%% RNF06 - Gestión Memoria Nodo %%%%%%%%
\begin{longtable}{|l|p{0.8\textwidth}|}
\hline
\textbf{\RNF} & \textbf{Capacidad de memoria del nodo sensor.} \\
\hline
\endfirsthead
\multicolumn{2}{r}{\textit{Continúa en la siguiente página}} \\
\endfoot
\endlastfoot
\textbf{Versión} & 1.0 - Fecha de versión: 06/06/2025 \\ \hline
\textbf{Fuente} & Documentación técnica del subsistema de Microcontrolador. \\ \hline
\textbf{Propósito} & Garantizar que no se pierdan datos de los sensores ni la imagen única antes de ser transmitidos al servidor. \\ \hline
\textbf{Descripción} & El microcontrolador debe gestionar eficientemente la memoria para almacenar los datos temporalmente hasta su transmisión al servidor central, asegurando que no haya pérdida de información. \\ \hline
\textbf{Especificación} & La memoria interna o externa debe ser suficiente para almacenar los datos de los sensores y al menos una imagen de 640×640 antes de la transmisión. El sistema debe asegurar que esta capacidad sea suficiente para un ciclo de operación completo sin desbordamiento. \\ \hline
\textbf{Prioridad} & Alta \\ \hline
\textbf{Comentarios} & La gestión de memoria es crucial para asegurar que los datos no se pierdan antes de ser transmitidos, lo que podría comprometer el análisis posterior. \\ \hline
\end{longtable}

%%%%%%%% RNF08 - Consumo Transceptor %%%%%%%%
\begin{longtable}{|l|p{0.8\textwidth}|}
\hline
\textbf{\RNF} & \textbf{Eficiencia energética del transceptor.} \\ 
\hline
\endfirsthead
\multicolumn{2}{r}{\textit{Continúa en la siguiente página}} \\
\endfoot
\endlastfoot
\textbf{Versión} & 1.1 - Fecha de versión: 22/10/2025 \\ \hline
\textbf{Fuente} & Documentación técnica del subsistema - Transceptor. \\ \hline
\textbf{Propósito} & Asegurar que el transceptor seleccionado opere con un consumo energético compatible con los objetivos de autonomía del nodo sensor. \\ \hline
\textbf{Descripción} & El transceptor debe presentar un bajo consumo de corriente en los modos activos (Transmisión - Tx, Recepción - Rx) y, de manera crítica, un consumo ultra bajo en el modo de reposo (Sleep/Standby) para permitir largos periodos de inactividad. \\ \hline
\textbf{Especificación} & Los consumos de corriente del transceptor seleccionado, tanto en modos activos (Tx/Rx) como en modo sleep, deben ser lo suficientemente bajos como para permitir que el nodo sensor alcance la autonomía calculada en el análisis de alimentación (Sección \ref{sec:analisis_alimentacion}). El consumo en modo sleep debe ser del orden de microamperios (\si{\micro\ampere}) o decenas de microamperios. \\ \hline
\textbf{Prioridad} & Alta \\ \hline
\textbf{Comentarios} & El consumo del transceptor, particularmente en modo Sleep, es un factor determinante para la viabilidad de un nodo autónomo alimentado por batería. La eficiencia durante Tx/Rx impacta el pico de demanda de corriente. \\ \hline
\end{longtable}

%%%%%%%% RNF09 - Eficiencia Fuente Alimentación %%%%%%%%
\begin{longtable}{|l|p{0.8\textwidth}|}
\hline
\textbf{\RNF} & \textbf{Eficiencia y autonomía del subsistema de alimentación.} \\ 
\hline
\endfirsthead
\multicolumn{2}{r}{\textit{Continúa en la siguiente página}} \\
\endfoot
\endlastfoot
\textbf{Versión} & 1.0 - Fecha de versión: 06/06/2025 \\ \hline
\textbf{Fuente} & Documentación técnica del subsistema - Fuente de Alimentación. \\ \hline
\textbf{Propósito} & Asegurar que la fuente de alimentación proporcione energía de manera eficiente, con la capacidad de recargar la batería utilizando energía solar. \\ \hline
\textbf{Descripción} & La fuente de alimentación debe ser capaz de proporcionar suficiente energía para todos los componentes del nodo, optimizando la duración de la batería y permitiendo su recarga utilizando energía solar. \\ \hline
\textbf{Especificación} & El panel solar debe ser capaz de recargar completamente la batería en aproximadamente 6 horas de luz solar directa, y la batería debe soportar al menos 24 horas de funcionamiento sin recarga. \\ \hline
\textbf{Prioridad} & Alta \\ \hline
\textbf{Comentarios} & El consumo eficiente de energía es crucial para garantizar el funcionamiento autónomo de los nodos en entornos remotos. La recarga solar optimiza la sostenibilidad del sistema. \\ \hline
\end{longtable}


% ======================================================
% Componente: Nodo Concentrador / Gateway
% ======================================================
\subsection{Nodo Concentrador / Gateway}
\label{subsec:req_gateway}

Requerimientos específicos para el nodo final de la WSN, responsable de agregar datos y conectarse al servidor central.

\subsubsection{Requerimientos Funcionales (Gateway)}

%%%%%%%% RF13 - Recepción Datos WSN %%%%%%%%
\begin{longtable}{|l|p{0.8\textwidth}|}
\hline
\textbf{\RF} & \textbf{Recibir datos de nodos sensores.} \\ 
\hline
\endfirsthead
\multicolumn{2}{r}{\textit{Continúa en la siguiente página}} \\
\endfoot
\endlastfoot
\textbf{Versión} & 1.0 - Fecha de versión: 06/06/2025 \\ \hline
\textbf{Fuente} & Documentación técnica del subsistema - Recepción de datos de nodos sensores.\\ \hline
\textbf{Propósito} & Asegurar que el nodo concentrador pueda recibir datos de los nodos sensores de manera confiable y periódica.\\ \hline
\textbf{Descripción} & El subsistema debe ser capaz de recibir los datos transmitidos desde el último nodo sensor a través de la red inalámbrica.\\ \hline
\textbf{Especificación} & El nodo concentrador debe ser capaz de recibir múltiples transmisiones de datos periódicas del nodo sensor anterior a él sin perder información.Los datos deben ser recibidos en un formato estructurado y organizado para su almacenamiento temporal.\\ \hline
\textbf{Prioridad} & Alta \\ \hline
\textbf{Comentarios} & La confiabilidad en la recepción de los datos es esencial para evitar pérdidas de información y garantizar que todos los parámetros monitoreados sean procesados correctamente.\\ \hline
\end{longtable}

%%%%%%%% RF14 - Almacenamiento Temporal Gateway %%%%%%%%
\begin{longtable}{|l|p{0.8\textwidth}|}
\hline
\textbf{\RF} & \textbf{Almacenar temporalmente y organizar datos en Gateway.} \\
\hline
\endfirsthead
\multicolumn{2}{r}{\textit{Continúa en la siguiente página}} \\
\endfoot
\endlastfoot
\textbf{Versión} & 1.0 - Fecha de versión: 06/06/2025 \\ \hline
\textbf{Fuente} & Documentación técnica del subsistema - Almacenamiento Temporal y Organización de Datos.\\ \hline
\textbf{Propósito} & Almacenar temporalmente los datos recibidos de los nodos sensores en el buffer del nodo concentrador y organizarlos para su posterior transmisión al servidor central.\\ \hline
\textbf{Descripción} & El subsistema debe ser capaz de recibir los datos de los nodos sensores transmitidos por el nodo anterior y almacenarlos de manera temporal en el buffer del nodo concentrador. Además, debe organizar los datos de forma estructurada, asegurando que sean fácilmente accesibles y listos para ser transmitidos al servidor central. \\ \hline
\textbf{Especificación} & Los datos recibidos deben ser almacenados temporalmente en el buffer del nodo concentrador, y deben ser organizados en una estructura que facilite su acceso y transmisión posterior. \\ \hline
\textbf{Prioridad} & Alta \\ \hline
\textbf{Comentarios} & El almacenamiento temporal y la organización eficiente de los datos son esenciales para asegurar que la información se mantenga accesible y lista para su transmisión rápida al servidor sin pérdida de datos. \\ \hline
\end{longtable}

%%%%%%%% RF15 - Transmisión Datos Servidor %%%%%%%%
\begin{longtable}{|l|p{0.8\textwidth}|}
\hline
\textbf{\RF} & \textbf{Transmitir datos agregados al servidor central.} \\ % RF15
\hline
\endfirsthead
\multicolumn{2}{r}{\textit{Continúa en la siguiente página}} \\
\endfoot
\endlastfoot
\textbf{Versión} & 1.0 - Fecha de versión: 06/06/2025 \\ \hline
\textbf{Fuente} & Documentación técnica del subsistema - Gateway. \\ \hline
\textbf{Propósito} & Enviar los datos procesados y organizados al servidor central para su almacenamiento y análisis. \\ \hline
\textbf{Descripción} & El nodo concentrador debe transmitir los datos organizados y procesados al servidor central utilizando una red confiable (WLAN o enlace celular). La transmisión debe ser eficiente y sin pérdidas de datos. \\ \hline
\textbf{Especificación} & El nodo concentrador debe enviar los datos al servidor central utilizando una conexión estable y confiable , y si es necesario, se deben utilizar enlaces celulares para garantizar la conectividad remota , y el sistema debe garantizar que la transmisión sea segura y sin errores con algún método de conexión de errores. \\ \hline
\textbf{Prioridad} & Alta \\ \hline
\textbf{Comentarios} & La transmisión de los datos al servidor es esencial para el procesamiento y almacenamiento a largo plazo. Debe asegurarse que la red utilizada sea confiable y no haya pérdidas de datos durante el envío. \\ \hline
\end{longtable}

%%%%%%%% RF16 - Protocolo Errores (To Server) %%%%%%%%
\begin{longtable}{|l|p{0.8\textwidth}|}
\hline
\textbf{\RF} & \textbf{Implementar protocolo de fiabilidad para transmisión al servidor.} \\ 
\hline
\endfirsthead
\multicolumn{2}{r}{\textit{Continúa en la siguiente página}} \\
\endfoot
\endlastfoot
\textbf{Versión} & 1.0 - Fecha de versión: 06/06/2025 \\ \hline
\textbf{Fuente} & Documentación técnica del subsistema - Gateway. \\ \hline
\textbf{Propósito} &  Asegurar que la transmisión de datos al servidor central sea segura y sin errores mediante la aplicación de un método de corrección de errores. \\ \hline
\textbf{Descripción} & El nodo concentrador debe implementar un mecanismo o protocolo (ej. control de flujo, CRC) que permita la detección y corrección de errores durante la transmisión al servidor central, garantizando la integridad de los datos. \\ \hline
\textbf{Especificación} & El sistema debe garantizar que la transmisión sea segura y sin errores con algún método de conexión de errores, y deberá realizar una validación de la trama recibida por el servidor (ACK/NACK) para confirmar la entrega exitosa de los datos. \\ \hline
\textbf{Prioridad} & Alta \\ \hline
\textbf{Comentarios} & Este requerimiento es la base técnica que soporta el requisito funcional de retransmisión en caso de fallo, mejorando la fiabilidad general del sistema. \\ \hline
\end{longtable}

%%%%%%%% RF17 - Retransmisión Datos Servidor %%%%%%%%
\begin{longtable}{|l|p{0.8\textwidth}|}
\hline
\textbf{\RF} & \textbf{Retransmitir datos al servidor en caso de fallo de conexión.} \\ 
\hline
\endfirsthead
\multicolumn{2}{r}{\textit{Continúa en la siguiente página}} \\
\endfoot
\endlastfoot
\textbf{Versión} & 1.0 - Fecha de versión: 06/06/2025 \\ \hline
\textbf{Fuente} & Documentación técnica del subsistema - Gateway. \\ \hline
\textbf{Propósito} & Asegurar que los datos se retransmitan en caso de fallo de conexión hasta lograr su entrega exitosa al servidor. \\ \hline
\textbf{Descripción} & El sistema debe ser capaz de retransmitir los datos cada vez que se detecte un fallo de conexión con el servidor, hasta que la transmisión se realice con éxito. \\ \hline
\textbf{Especificación} & El sistema debe realizar retransmisiones automáticas de los datos en intervalos predefinidos si no recibe confirmación de entrega exitosa. La retransmisión debe continuar hasta que el servidor confirme la recepción exitosa de los datos. \\ \hline
\textbf{Prioridad} & Alta \\ \hline
\textbf{Comentarios} & La retransmisión de datos es crítica para asegurar que los datos no se pierdan en caso de fallos de comunicación, garantizando la integridad de la información. \\ \hline
\end{longtable}

\subsubsection{Requerimientos No Funcionales (Gateway)}

%%%%%%%% RNF10 - Fiabilidad Recepción WSN %%%%%%%%
\begin{longtable}{|l|p{0.8\textwidth}|}
\hline
\textbf{\RNF} & \textbf{Fiabilidad de la recepción de datos desde la WSN.} \\ 
\hline
\endfirsthead
\multicolumn{2}{r}{\textit{Continúa en la siguiente página}} \\
\endfoot
\endlastfoot
\textbf{Versión} & 1.0 - Fecha de versión: 06/06/2025 \\ \hline
\textbf{Fuente} & Documentación técnica del subsistema - Recepción de datos de nodos concentradores. \\ \hline
\textbf{Propósito} & Garantizar que los datos recibidos de los nodos sensores lleguen de manera confiable y sin pérdida de información. \\ \hline
\textbf{Descripción} & El subsistema debe ser capaz de recibir los datos transmitidos desde los nodos sensores de manera eficiente y sin pérdidas de información, incluso si los nodos están enviando datos de forma simultánea. \\ \hline
\textbf{Especificación} & La recepción de datos debe garantizar que no haya pérdida de paquetes ni interrupciones en la transmisión, y debe permitir la recepción de múltiples transmisiones de datos periódicas sin errores. \\ \hline
\textbf{Prioridad} & Alta \\ \hline
\textbf{Comentarios} & La fiabilidad es esencial para evitar la pérdida de datos, lo cual podría afectar el análisis y la toma de decisiones. \\ \hline
\end{longtable}

%%%%%%%% RNF11 - Eficiencia Almacenamiento Gateway %%%%%%%%
\begin{longtable}{|l|p{0.8\textwidth}|}
\hline
\textbf{\RNF} & \textbf{Capacidad y eficiencia del almacenamiento temporal del Gateway.} \\ 
\hline
\endfirsthead
\multicolumn{2}{r}{\textit{Continúa en la siguiente página}} \\
\endfoot
\endlastfoot
\textbf{Versión} & 1.0 - Fecha de versión: 06/06/2025 \\ \hline
\textbf{Autor} & Equipo de Desarrollo \\ \hline
\textbf{Fuente} & Documentación técnica del subsistema - Almacenamiento temporal y organización de datos. \\ \hline
\textbf{Propósito} & Asegurar que los datos sean almacenados temporalmente de forma eficiente y sin sobrecargar el buffer del nodo concentrador. \\ \hline
\textbf{Descripción} & El subsistema debe ser capaz de almacenar los datos temporalmente en un buffer de manera eficiente para evitar que se pierdan o se sobrecargue el sistema. Los datos deben estar disponibles para ser procesados y enviados al servidor en el momento adecuado. \\ \hline
\textbf{Especificación} & El buffer del nodo concentrador debe tener suficiente capacidad para almacenar los datos de al menos 24 horas de monitoreo continuo. Además, el almacenamiento temporal debe tener una latencia mínima para garantizar que los datos se transmitan rápidamente al servidor una vez procesados. \\ \hline
\textbf{Prioridad} & Alta \\ \hline
\textbf{Comentarios} & La gestión eficiente del almacenamiento temporal es crucial para asegurar que no haya pérdida de datos y que el sistema opere sin interrupciones, incluso cuando los nodos sensores envíen datos de manera periódica. \\ \hline
\end{longtable}

%%%%%%%% RNF12 - Organización Datos Tx Servidor %%%%%%%%
\begin{longtable}{|l|p{0.8\textwidth}|}
\hline
\textbf{\RNF} & \textbf{Formato de datos para transmisión al servidor.} \\ 
\hline
\endfirsthead
\multicolumn{2}{r}{\textit{Continúa en la siguiente página}} \\
\endfoot
\endlastfoot
\textbf{Versión} & 1.0 - Fecha de versión: 06/06/2025 \\ \hline
\textbf{Fuente} & Documentación técnica del subsistema de Nodo Concentrador. \\ \hline
\textbf{Propósito} & Organizar los datos de forma eficiente para su transmisión posterior al servidor. \\ \hline
\textbf{Descripción} & Los datos almacenados temporalmente deben ser organizados en una estructura que permita su transmisión rápida y eficiente al servidor. La organización de los datos debe facilitar su acceso y reducir el tiempo de procesamiento. \\ \hline
\textbf{Especificación} & Los datos deben ser organizados en una estructura jerárquica o por categorías que faciliten su transmisión y procesamiento posterior, con un tiempo de acceso no mayor a 5 segundos. \\ \hline
\textbf{Prioridad} & Alta \\ \hline
\textbf{Comentarios} & Una buena organización de los datos permite reducir la latencia de transmisión y asegura que los datos se mantengan estructurados y coherentes. \\ \hline
\end{longtable}

%%%%%%%% RNF13 - Prevención Desbordamiento Buffer Gateway %%%%%%%%
\begin{longtable}{|l|p{0.8\textwidth}|}
\hline
\textbf{\RNF} & \textbf{Gestión del desbordamiento del buffer del Gateway.} \\ 
\hline
\endfirsthead
\multicolumn{2}{r}{\textit{Continúa en la siguiente página}} \\
\endfoot
\endlastfoot
\textbf{Versión} & 1.0 - Fecha de versión: 06/06/2025 \\ \hline
\textbf{Fuente} & Documentación técnica del subsistema de Gestión de Memoria. \\ \hline
\textbf{Propósito} & Evitar la sobrecarga o pérdida de datos debido a desbordamientos de buffer en el sistema. \\ \hline
\textbf{Descripción} & El sistema debe estar diseñado para prevenir sobrecargas y desbordamientos de buffer en cualquier etapa del proceso de almacenamiento temporal de datos, garantizando que no haya pérdidas de información debido a la incapacidad de la memoria para manejar grandes cantidades de datos. \\ \hline
\textbf{Especificación} & El sistema debe implementar mecanismos de gestión de memoria que aseguren que el buffer no se sobrecargue, como gestión dinámica de buffer o descarga automática de los datos cuando se acerca el límite de capacidad. Además, el sistema debe ser capaz de alertar cuando se detecte que el buffer está a punto de desbordarse. \\ \hline
\textbf{Prioridad} & Alta \\ \hline
\textbf{Comentarios} & La prevención de sobrecarga o pérdidas de datos es crucial para mantener la integridad de la información recopilada y para evitar fallos en el sistema de monitoreo. \\ \hline
\end{longtable}

%%%%%%%% RNF14 - Fiabilidad Reenvío Servidor %%%%%%%%
\begin{longtable}{|l|p{0.8\textwidth}|}
\hline
\textbf{\RNF} & \textbf{Fiabilidad del enlace de comunicación Gateway-Servidor.} \\
\hline
\endfirsthead
\multicolumn{2}{r}{\textit{Continúa en la siguiente página}} \\
\endfoot
\endlastfoot
\textbf{Versión} & 1.0 - Fecha de versión: 06/06/2025 \\ \hline
\textbf{Fuente} & Documentación técnica del subsistema - Gateway. \\ \hline
\textbf{Propósito} & Garantizar que los datos procesados y organizados sean transmitidos al servidor central sin pérdidas ni errores. \\ \hline
\textbf{Descripción} & El subsistema Gateway debe ser capaz de transmitir los datos organizados desde el nodo concentrador al servidor central a través de una red confiable. La transmisión debe ser segura y libre de errores. \\ \hline
\textbf{Especificación} & El Gateway debe transmitir datos sin pérdidas a través de una red confiable, garantizando que la integridad de los datos se mantenga durante todo el proceso de transmisión. El tiempo de transmisión debe ser mínimo, asegurando que los datos lleguen al servidor sin retrasos significativos. \\ \hline
\textbf{Prioridad} & Alta \\ \hline
\textbf{Comentarios} & La fiabilidad de la transmisión es crucial para asegurar que los datos sean recibidos por el servidor central y puedan ser procesados y almacenados sin errores. \\ \hline
\end{longtable}


% ======================================================
% Componente: Red WSN (Comunicación Inter-Nodo)
% ======================================================
\subsection{Red Inalámbrica de Sensores (Comunicación Inter-Nodo)}
\label{subsec:req_red_wsn}

Requerimientos asociados al enlace de comunicación y la interacción entre los nodos sensores dentro de la WSN.

\subsubsection{Requerimientos Funcionales (Red WSN)}

%%%%%%%% RF10 - Transmisión Inter-Nodo %%%%%%%%
\begin{longtable}{|l|p{0.8\textwidth}|}
\hline
\textbf{\RF} & \textbf{Transmitir inalámbricamente datos inter-nodo.} \\% RF10 (Adaptado)
\hline
\endfirsthead
\multicolumn{2}{r}{\textit{Continúa en la siguiente página}} \\
\endfoot
\endlastfoot
\textbf{Versión} & 1.1 - Fecha de versión: 22/10/2025 \\ \hline
\textbf{Fuente} & Documentación técnica del subsistema - Transceptor.\\ \hline
\textbf{Propósito} & Asegurar que los datos obtenidos por un nodo sensor sean transmitidos de manera inalámbrica al siguiente nodo en la cadena lineal (o al concentrador).\\ \hline
\textbf{Descripción} & Los nodos sensores deben transmitir los datos adquiridos (empaquetados según RF\ref{rf:Empaquetado_datos}) al siguiente nodo en la topología lineal utilizando la tecnología inalámbrica seleccionada.La transmisión debe ser confiable para asegurar que no haya pérdida de información entre saltos.\\ \hline
\textbf{Especificación} & Los datos deben ser transmitidos en paquetes estructurados al nodo adyacente.El sistema debe establecer y mantener un enlace de comunicación fiable entre nodos adyacentes separados por la distancia definida en el diseño (a determinar en análisis de transceptores).Se deben utilizar mecanismos (ej. ACK a nivel MAC) para asegurar la entrega entre saltos.\\ % TODO: Review/Revise Specification Text - Removed specific 50m distance
\hline
\textbf{Prioridad} & Alta \\ \hline
\textbf{Comentarios} & La transmisión inalámbrica inter-nodo es la base de la WSN multisalto.Es esencial garantizar la fiabilidad de cada salto para asegurar la conectividad extremo a extremo.\\ \hline
\end{longtable}

\subsubsection{Requerimientos No Funcionales (Red WSN)}

%%%%%%%% RNF07 - Fiabilidad Comunicación Inter-Nodo %%%%%%%%
\begin{longtable}{|l|p{0.8\textwidth}|}
\hline
\textbf{\RNF} & \textbf{Fiabilidad de la comunicación inalámbrica inter-nodo.} \\ % RNF07 (Adaptado)
\hline
\endfirsthead
\multicolumn{2}{r}{\textit{Continúa en la siguiente página}} \\
\endfoot
\endlastfoot
\textbf{Versión} & 1.1 - Fecha de versión: 22/10/2025 \\ \hline
\textbf{Fuente} & Documentación técnica del subsistema - Transceptor. \\ \hline
\textbf{Propósito} & Asegurar que los datos transmitidos *entre* los nodos sensores (y hacia el concentrador) lleguen de manera confiable y sin pérdidas significativas. \\ \hline
\textbf{Descripción} & El enlace de comunicación inalámbrica entre nodos debe ser estable y confiable, minimizando la pérdida de paquetes o la corrupción de datos. \\ \hline
\textbf{Especificación} & La Tasa de Entrega de Paquetes (Packet Delivery Ratio - PDR) entre nodos adyacentes debe ser superior al 98\% bajo las condiciones ambientales esperadas. Se deben emplear protocolos que incluyan mecanismos de retransmisión a nivel de enlace si es necesario. \\ % TODO: Adjusted specification for PDR instead of 99% reliability
\hline
\textbf{Prioridad} & Alta \\ \hline
\textbf{Comentarios} & La fiabilidad del enlace inter-nodo es crítica para la integridad de la WSN multisalto, especialmente en entornos remotos propensos a interferencias. \\ \hline
\end{longtable}


% ======================================================
% Componente: Servidor (Backend)
% ======================================================
\subsection{Servidor (Backend)}
\label{subsec:req_servidor}

Requerimientos para el sistema centralizado en la nube responsable del procesamiento, almacenamiento y exposición de los datos.

\subsubsection{Requerimientos Funcionales (Servidor)}

%%%%%%%% RF18 - Recepción Datos Gateway %%%%%%%%
\begin{longtable}{|l|p{0.8\textwidth}|}
\hline
\textbf{\RF} & \textbf{Recibir Datos de Gateway.} \\ 
\hline
\endfirsthead
\multicolumn{2}{r}{\textit{Continúa en la siguiente página}} \\
\endfoot
\endlastfoot
\textbf{Versión} & 1.0 - Fecha de versión: 06/06/2025 \\ \hline
\textbf{Fuente} & Documentación técnica del subsistema - Recepción de datos de gateway. \\ \hline
\textbf{Propósito} & Recibir los datos enviados por el gateway para su posterior organización y almacenamiento. \\ \hline
\textbf{Descripción} & El servidor debe ser capaz de recibir los datos transmitidos por el gateway desde los nodos sensores. \\ \hline
\textbf{Especificación} & El servidor debe ser capaz de recibir múltiples transmisiones de datos de manera periódica. Los datos deben ser almacenados en una base de datos estructurada para su posterior análisis. \\ \hline
\textbf{Prioridad} & Alta \\ \hline
\textbf{Comentarios} & La recepción de datos confiable es esencial para evitar pérdidas de información que puedan afectar el análisis posterior. \\ \hline
\end{longtable}

%%%%%%%% RF19 - Organización Datos Servidor %%%%%%%%
\begin{longtable}{|l|p{0.8\textwidth}|}
\hline
\textbf{\RF} & \textbf{Organizar y parsear datos recibidos.} \\ % RF19
\hline
\endfirsthead
\multicolumn{2}{r}{\textit{Continúa en la siguiente página}} \\
\endfoot
\endlastfoot
\textbf{Versión} & 1.0 - Fecha de versión: 06/06/2025 \\ \hline
\textbf{Fuente} & Documentación técnica del subsistema - Recepción de  datos de Gateway. \\ \hline
\textbf{Propósito} & Organizar los datos recibidos para su almacenamiento y análisis eficiente. \\ \hline
\textbf{Descripción} & El servidor debe organizar los datos procesados de manera estructurada, clasificando los datos por tipo de medición, fecha, hora y otros metadatos. \\ \hline
\textbf{Especificación} & Los datos deben ser organizados en una base de datos estructurada que permita consultas rápidas y eficientes. \\ \hline
\textbf{Prioridad} & Alta \\ \hline
\textbf{Comentarios} & La organización eficiente de los datos permite un acceso rápido y preciso para análisis futuros y toma de decisiones. \\ \hline
\end{longtable}

%%%%%%%% RF20 - Procesamiento Imágenes Servidor %%%%%%%%
\begin{longtable}{|l|p{0.8\textwidth}|}
\hline
\textbf{\RF} & \textbf{Procesar imágenes para cuantificar área de basura.} \\ 
\hline
\endfirsthead
\multicolumn{2}{r}{\textit{Continúa en la siguiente página}} \\
\endfoot
\endlastfoot
\textbf{Versión} & 1.0 - Fecha de versión: 06/06/2025 \\ \hline
\textbf{Fuente} & Documentación técnica del subsistema - Procesamiento de Imágenes. \\ \hline
\textbf{Propósito} & Procesar las imágenes enviadas por el nodo concentrador, aplicar algoritmos de visión artificial y generar un valor numérico que represente el área cubierta por los cúmulos de residuos flotantes. \\ \hline
\textbf{Descripción} & El sistema debe procesar las imágenes enviadas desde el nodo concentrador, aplicar algoritmos de visión artificial para detectar cúmulos de residuos sólidos flotantes, y calcular el área que cubren en la superficie del agua. El resultado debe ser un metadato numérico que se adjuntará a la imagen procesada. \\ \hline
\textbf{Especificación} & El sistema debe ser capaz de procesar imágenes, aplicar algoritmos de visión artificial (ej. YOLOv8 o segmentación) para detectar los cúmulos de residuos flotantes y generar un metadato numérico del área cubierta para su análisis y posterior almacenamiento. \\ \hline
\textbf{Prioridad} & Alta \\ \hline
\textbf{Comentarios} & La captura precisa de imágenes es esencial para la posterior cuantificación del área de residuos flotantes en el servidor. El uso de la imagen original de 640×640 requiere una alta eficiencia de compresión en el nodo sensor para mitigar el riesgo de sobrecarga de la red inalámbrica de baja potencia. \\ \hline
\end{longtable}

%%%%%%%% RF21 - Almacenamiento Persistente %%%%%%%%
\begin{longtable}{|l|p{0.8\textwidth}|}
\hline
\textbf{\RF} & \textbf{Almacenar persistentemente datos y metadatos.} \\ 
\hline
\endfirsthead
\multicolumn{2}{r}{\textit{Continúa en la siguiente página}} \\
\endfoot
\endlastfoot
\textbf{Versión} & 1.0 - Fecha de versión: 06/06/2025 \\ \hline
\textbf{Fuente} & Documentación técnica del subsistema - Almacenamiento en Base de Datos. \\ \hline
\textbf{Propósito} & Almacenar los datos y metadatos de los sensores y las imágenes de residuos en una base de datos para su posterior consulta. \\ \hline
\textbf{Descripción} & El servidor debe almacenar los datos procesados y organizados de los sensores y las imágenes de residuos flotantes de manera estructurada en una base de datos organizada. \\ \hline
\textbf{Especificación} & Los datos y metadatos deben ser almacenados de manera eficiente, garantizando la integridad y accesibilidad de la información histórica para su consulta. \\ \hline
\textbf{Prioridad} & Alta \\ \hline
\textbf{Comentarios} & El almacenamiento adecuado de los datos es fundamental para su acceso a largo plazo y para el análisis posterior. \\ \hline
\end{longtable}

%%%%%%%% RF22 - Exposición API %%%%%%%%
\begin{longtable}{|l|p{0.8\textwidth}|}
\hline
\textbf{\RF} & \textbf{Exponer datos y servicios mediante API (RESTful)} \\ 
\hline
\endfirsthead
\multicolumn{2}{r}{\textit{Continúa en la siguiente página}} \\
\endfoot
\endlastfoot
\textbf{Versión} & 1.0 - Fecha de versión: 06/06/2025 \\ \hline
\textbf{Fuente} & Documentación técnica del subsistema - API/Comunicación (Enlace con página web). \\ \hline
\textbf{Propósito} & Gestionar la comunicación entre el servidor y la página web, permitiendo la recuperación eficiente de los datos procesados. \\ \hline
\textbf{Descripción} & El servidor debe proporcionar una API para que la página web acceda a los datos de calidad del agua y las imágenes de residuos flotantes. \\ \hline
\textbf{Especificación} & La API RESTful debe permitir el acceso eficiente a los datos y imágenes almacenados en el servidor. La comunicación debe ser segura (utilizando HTTPS). \\ \hline
\textbf{Prioridad} & Alta \\ \hline
\textbf{Comentarios} & La integración con la página web es esencial para que los usuarios puedan consultar los datos sobre el estado del agua y los residuos flotantes. \\ \hline
\end{longtable}

\subsubsection{Requerimientos No Funcionales (Servidor)}

%%%%%%%% RNF15 - Validación Datos Servidor %%%%%%%%
\begin{longtable}{|l|p{0.8\textwidth}|}
\hline
\textbf{\RNF} & \textbf{Validación de datos en la ingesta.} \\ 
\hline
\endfirsthead
\multicolumn{2}{r}{\textit{Continúa en la siguiente página}} \\
\endfoot
\endlastfoot
\textbf{Versión} & 1.0 - Fecha de versión: 06/06/2025 \\ \hline
\textbf{Fuente} & Documentación técnica del subsistema - Recepción de datos Gateway. \\ \hline
\textbf{Propósito} & Asegurar que los datos recibidos estén en el formato adecuado antes de ser procesados. \\ \hline
\textbf{Descripción} & El sistema debe validar automáticamente que los datos entrantes cumplan con la estructura JSON, los tipos de datos esperados, y que los campos obligatorios estén presentes y completos. \\ \hline
\textbf{Especificación} & Los datos deben ser validados automáticamente para asegurarse de que el formato JSON es correcto, los tipos de datos coinciden con los especificados y los campos obligatorios están presentes. \\ \hline
\textbf{Prioridad} & Alta \\ \hline
\textbf{Comentarios} & La validación asegura que solo datos correctos y estructurados sean procesados, evitando errores en el análisis y almacenamiento. \\ \hline
\end{longtable}

%%%%%%%% RNF16 - Optimización Procesamiento Imagen Servidor %%%%%%%%
\begin{longtable}{|l|p{0.8\textwidth}|}
\hline
\textbf{\RNF} & \textbf{Eficiencia y trazabilidad del procesamiento de imágenes.} \\ 
\hline
\endfirsthead
\multicolumn{2}{r}{\textit{Continúa en la siguiente página}} \\
\endfoot
\endlastfoot
\textbf{Versión} & 1.0 - Fecha de versión: 06/06/2025 \\ \hline
\textbf{Fuente} & Documentación técnica del subsistema de Visión Artificial. \\ \hline
\textbf{Propósito} & Garantizar que las imágenes procesadas sean almacenadas de manera eficiente y que el algoritmo de cuantificación de área sea trazable para auditoría y mejora continua. \\ \hline
\textbf{Descripción} & Las imágenes procesadas deben ser almacenadas en un formato comprimido (como JPEG o PNG) sin perder la calidad necesaria para el análisis. El sistema debe asegurar la trazabilidad del algoritmo de segmentación utilizado para calcular el área de basura. \\ \hline
\textbf{Especificación} & Las imágenes deben ser almacenadas en formato comprimido sin pérdida de calidad significativa para el análisis, y los resultados de la segmentación deben incluir un nivel de confianza (probabilidad$>$0.85) para filtrar resultados poco fiables. Además, el sistema debe incluir un control de versiones para el algoritmo de cuantificación de área, lo que permite la trazabilidad de mejoras. \\ \hline
\textbf{Prioridad} & Alta \\ \hline
\textbf{Comentarios} & La optimización de la imagen y la trazabilidad de la versión del algoritmo de cuantificación son esenciales para asegurar la fiabilidad de la métrica de área calculada a lo largo del tiempo. \\ \hline
\end{longtable}

%%%%%%%% RNF17 - Acceso Datos Históricos %%%%%%%%
\begin{longtable}{|l|p{0.8\textwidth}|}
\hline
\textbf{\RNF} & \textbf{Retención y rendimiento de consultas de datos históricos.} \\
\hline
\endfirsthead
\multicolumn{2}{r}{\textit{Continúa en la siguiente página}} \\
\endfoot
\endlastfoot
\textbf{Versión} & 1.0 - Fecha de versión: 06/06/2025 \\ \hline
\textbf{Fuente} & Documentación técnica del subsistema de Base de Datos. \\ \hline
\textbf{Propósito} & Permitir el acceso a los datos históricos de al menos un año para su análisis y consulta. \\ \hline
\textbf{Descripción} & El sistema debe garantizar que los datos históricos (como pH, turbidez, temperatura, etc.) estén disponibles para su consulta en un período de al menos un año. Este acceso debe realizarse mediante consultas eficientes a la base de datos. \\ \hline
\textbf{Especificación} & Los datos históricos deben ser almacenados y accesibles durante al menos un año, y las consultas deben ser capaces de recuperar datos de manera eficiente (menos de 2 segundos por consulta). \\ \hline
\textbf{Prioridad} & Alta \\ \hline
% Comentario faltante en el original
\end{longtable}

%%%%%%%% RNF18 - Rendimiento/Seguridad API %%%%%%%%
\begin{longtable}{|l|p{0.8\textwidth}|}
\hline
\textbf{\RNF} & \textbf{Rendimiento, seguridad y escalabilidad de la API.} \\ 
\hline
\endfirsthead
\multicolumn{2}{r}{\textit{Continúa en la siguiente página}} \\
\endfoot
\endlastfoot
\textbf{Versión} & 1.0 - Fecha de versión: 06/06/2025 \\ \hline
\textbf{Fuente} & Documentación técnica del subsistema de API. \\ \hline
\textbf{Propósito} & Garantizar el rendimiento, seguridad y capacidad de la API para manejar múltiples usuarios concurrentes. \\ \hline
\textbf{Descripción} & La API debe responder en menos de 1 segundo en condiciones normales. Además, debe cifrar toda la comunicación entre el servidor y la página web (por ejemplo, mediante HTTPS) y debe soportar al menos 50 usuarios concurrentes sin que el rendimiento se degrade. \\ \hline
\textbf{Especificación} & La API debe responder en menos de 1 segundo en condiciones normales. La comunicación entre el servidor y la página web debe estar cifrada (HTTPS) y la API debe ser capaz de soportar al menos 50 usuarios concurrentes sin que se degrade el rendimiento. \\ \hline
\textbf{Prioridad} & Alta \\ \hline
% Comentario faltante en el original
\end{longtable}


% ======================================================
% Componente: Aplicación Web (Frontend)
% ======================================================
\subsection{Aplicación Web (Frontend)}
\label{subsec:req_webapp}

Requerimientos para la interfaz de usuario final que presenta los datos a los usuarios.

\subsubsection{Requerimientos Funcionales (Web App)}

%%%%%%%% RF23 - Consumo API %%%%%%%%
\begin{longtable}{|l|p{0.8\textwidth}|}
\hline
\textbf{\RF} & \textbf{Consumir servicios del servidor API.} \\ 
\hline
\endfirsthead
\multicolumn{2}{r}{\textit{Continúa en la siguiente página}} \\
\endfoot
\endlastfoot
\textbf{Versión} & 1.0 - Fecha de versión: 06/06/2025 \\ \hline
\textbf{Fuente} & Documentación técnica del subsistema - Conexión al Servidor (API/Backend). \\ \hline
\textbf{Propósito} & Gestionar la comunicación entre la página web y el servidor para la recuperación y visualización de datos. \\ \hline
\textbf{Descripción} & El subsistema debe gestionar la comunicación entre el frontend (página web) y el backend (servidor). Utilizará una API RESTful para recuperar los datos de calidad del agua y los residuos sólidos, enviando solicitudes y respondiendo con la información necesaria para la visualización en la página web. \\ \hline
\textbf{Especificación} & La API RESTful debe permitir que la página web solicite y reciba datos del servidor de forma eficiente y segura. La conexión debe ser segura (HTTPS), y el backend debe manejar las solicitudes de datos, procesarlas y responder con la información organizada y estructurada. \\ \hline
\textbf{Prioridad} & Alta \\ \hline
\textbf{Comentarios} & La seguridad en la comunicación entre la página web y el servidor es fundamental para proteger los datos y garantizar la privacidad. \\ \hline
\end{longtable}

%%%%%%%% RF24 - Visualización Datos Actuales %%%%%%%%
\begin{longtable}{|l|p{0.8\textwidth}|}
\hline
\textbf{\RF} & \textbf{Visualizar datos actuales y resumen.} \\ % RF24
\hline
\endfirsthead
\multicolumn{2}{r}{\textit{Continúa en la siguiente página}} \\
\endfoot
\endlastfoot
\textbf{Versión} & 1.0 - Fecha de versión: 06/06/2025 \\ \hline
\textbf{Fuente} & Documentación técnica del subsistema - Interfaz de usuario (Frontend). \\ \hline
\textbf{Propósito} & Permitir la visualización clara y atractiva de los datos de calidad del agua y residuos sólidos flotantes en la página web, incluyendo la generación de gráficos interactivos. \\ \hline
\textbf{Descripción} & La página web debe presentar los datos de forma clara, utilizando gráficos interactivos, tablas y representaciones visuales de los datos obtenidos. El diseño debe ser responsive y accesible desde diferentes dispositivos (móviles, tabletas y computadoras). \\ \hline
\textbf{Especificación} & Los gráficos deben ser actualizados dinámicamente sin necesidad de recargar la página. La visualización debe ser clara y los usuarios deben poder interactuar con los gráficos para obtener información detallada de los datos (ej. mediante tooltips o filtros). \\ \hline
\textbf{Prioridad} & Alta \\ \hline
\textbf{Comentarios} & La usabilidad es clave para garantizar que los usuarios puedan acceder a la información sin dificultades, por lo que la interactividad y la adaptabilidad de la página son esenciales. \\ \hline
\end{longtable}

%%%%%%%% RF25 - Visualización Datos Históricos %%%%%%%%
\begin{longtable}{|l|p{0.8\textwidth}|}
\hline
\textbf{\RF} & \textbf{Permitir la visualización de datos históricos.} \\ 
\hline
\endfirsthead
\multicolumn{2}{r}{\textit{Continúa en la siguiente página}} \\
\endfoot
\endlastfoot
\textbf{Versión} & 1.0 - Fecha de versión: 06/06/2025 \\ \hline
\textbf{Fuente} & Documentación técnica del subsistema - Interfaz de usuario (Frontend). \\ \hline
\textbf{Propósito} & Permitir que los usuarios visualicen los datos históricos de calidad del agua y residuos sólidos flotantes. \\ \hline
\textbf{Descripción} & La página web debe permitir que los usuarios visualicen los datos históricos de los parámetros de calidad del agua y los residuos sólidos flotantes a lo largo del tiempo con la capacidad de interactuar con las fechas de monitoreo. \\ \hline
\textbf{Especificación} & Los usuarios deben poder acceder a los datos históricos mediante una interfaz que les permita ver cómo evolucionaron los parámetros de calidad del agua y los residuos flotantes. \\ \hline
\textbf{Prioridad} & Alta \\ \hline
\textbf{Comentarios} & La visualización de los datos históricos es crucial para realizar un análisis de tendencias y evaluar la evolución de la calidad del agua y la presencia de residuos. \\ \hline
\end{longtable}

\subsubsection{Requerimientos No Funcionales (Web App)}

%%%%%%%% RNF19 - Accesibilidad/Responsividad %%%%%%%%
\begin{longtable}{|l|p{0.8\textwidth}|}
\hline
\textbf{\RNF} & \textbf{Accesibilidad y diseño responsivo de la interfaz.} \\
\hline
\endfirsthead
\multicolumn{2}{r}{\textit{Continúa en la siguiente página}} \\
\endfoot
\endlastfoot
\textbf{Versión} & 1.0 - Fecha de versión: 06/06/2025 \\ \hline
\textbf{Fuente} & Documentación técnica del subsistema de Interfaz de Usuario. \\ \hline
\textbf{Propósito} & Asegurar que la interfaz sea accesible y responsiva en computadoras, brindando una experiencia de usuario óptima. \\ \hline
\textbf{Descripción} & La interfaz debe ser totalmente accesible en computadoras de escritorio y portátiles. Además, debe ajustarse automáticamente a diferentes tamaños de pantalla para asegurar que el contenido sea visible y fácil de usar en todos los dispositivos. \\ \hline
\textbf{Especificación} & La interfaz debe ser responsiva y adaptarse a pantallas de tamaños diversos, asegurando que el diseño se ajuste correctamente en computadoras. El contenido debe ser legible y funcional en pantallas pequeñas o grandes sin perder usabilidad. \\ \hline
\textbf{Prioridad} & Alta \\ \hline
\textbf{Comentarios} & La responsividad es esencial para asegurar que los usuarios puedan acceder al sistema de forma eficiente desde computadoras, independientemente del tamaño de la pantalla. \\ \hline
\end{longtable}

%%%%%%%% RNF20 - Diseño UI/UX %%%%%%%%
\begin{longtable}{|l|p{0.8\textwidth}|}
\hline
\textbf{\RNF} & \textbf{Claridad, estética y usabilidad de la interfaz (UI/UX)} \\ 
\hline
\endfirsthead
\multicolumn{2}{r}{\textit{Continúa en la siguiente página}} \\
\endfoot
\endlastfoot
\textbf{Versión} & 1.0 - Fecha de versión: 06/06/2025 \\ \hline
\textbf{Fuente} & Documentación técnica del subsistema de Interfaz de Usuario. \\ \hline
\textbf{Propósito} & Garantizar que la interfaz sea visualmente clara, estética y agradable para el usuario. \\ \hline
\textbf{Descripción} & La interfaz debe ser diseñada con un estilo limpio, intuitivo y funcional, donde los elementos sean fáciles de encontrar y utilizar. El diseño debe adaptarse a las necesidades del usuario, proporcionando colores y tipografías que mejoren la legibilidad y la experiencia general. \\ \hline
\textbf{Especificación} & El sistema debe tener un diseño visual claro, con una estructura lógica de navegación y un estilo agradable que facilite el uso y la comprensión del sistema. La paleta de colores debe ser coherente y accesible, asegurando un contraste adecuado para facilitar la lectura y la interacción. \\ \hline
\textbf{Prioridad} & Alta \\ \hline
\textbf{Comentarios} & Un diseño claro y estético es clave para una experiencia de usuario exitosa, reduciendo la curva de aprendizaje y aumentando la satisfacción del usuario. \\ \hline
\end{longtable}




%%%%%%%%%%%%%%%%%%%%%%%%%%%%%%%%%%%%%%%%%%%%%%%%%%%%
%             SECCIÓN: Analisis sensores           %
%%%%%%%%%%%%%%%%%%%%%%%%%%%%%%%%%%%%%%%%%%%%%%%%%%%%

\section{Selección de Sensores}
\label{sec:analisis_sensores}
La selección de los transductores más adecuados para el sistema se rige por los requerimientos funcionales y no funcionales previamente definidos. Como base para la evaluación, cada sensor debe cumplir con las especificaciones mínimas derivadas de la Tabla~\ref{tab:parametros_calidad_agua} (Parámetros Normativos) y el requerimiento RNF01 (Precisión), las cuales se resumen a continuación:

\begin{itemize}
    \item Sensor de pH: Rango de medición mínimo de 5.0 a 8.0 unidades de pH, con una precisión de $\pm0.1$ unidades.
    
    \item Sensor de Oxígeno Disuelto: Capacidad para medir concentraciones superiores a 10~mg/L.
    
    \item Sensor de Turbidez: Rango que cubra hasta 5~UNT, con un margen de error máximo del 2\%.
    
    \item Sensor de Conductividad: Rango de hasta 1{,}000~\si{\micro\siemens\per\centi\meter}.
    
    \item Sensor de Temperatura: Cobertura del rango ambiental esperado en cuerpos de agua, típicamente de 5~a~30~\si{\celsius}.
\end{itemize}

Adicionalmente, se consideran como criterios clave la compatibilidad con una tensión de operación de 3.3--5.5~V, una interfaz de comunicación compatible con el microcontrolador seleccionado, un bajo consumo energético y un diseño robusto para inmersión continua. 

A continuación, se presenta el análisis comparativo de las opciones de mercado frente a estos criterios.


%%%%%%%%%%%%%%%%%%%%%%%%%%%%%%%%%%%%%%%%%%%%%%%%%%%%
%             SECCIÓN: pH                          %
%%%%%%%%%%%%%%%%%%%%%%%%%%%%%%%%%%%%%%%%%%%%%%%%%%%%

\subsection{Análisis sensores pH}
\renewcommand{\arraystretch}{1.5}
\begin{longtable}{
    |p{4cm}
    |p{3cm}
    |p{3cm}
    |p{3cm}|
}
\caption{Comparativa de sensores para medición de pH.}
\label{tab:sensores_ph} \\
\hline
\textbf{Característica} 
    & \textbf{Gravity: Analog pH Meter V2 \cite{DFRobot_pH_Sensor}} 
    & \textbf{Gravity: Industrial Analog pH Meter Pro Kit V2 \cite{DFRobot_pH_Sensor}} 
    & \textbf{Atlas Scientific EZO-pH™ \cite{AtlasScientific_pH_Kit}} \\ 
\hline
\endfirsthead

\hline
\textbf{Característica} 
    & \textbf{Gravity: Analog pH Meter V2 \cite{DFRobot_pH_Sensor}} 
    & \textbf{Gravity: Industrial Analog pH Meter Pro Kit V2 \cite{DFRobot_pH_Sensor}} 
    & \textbf{Atlas Scientific EZO-pH™ \cite{AtlasScientific_pH_Kit}} \\ 
\hline
\endhead

\hline
\multicolumn{4}{r}{\textit{Continúa en la siguiente página}} \\
\endfoot

\hline
\endlastfoot

Rango de medición 
    & 0--14 pH 
    & 0--14 pH 
    & 0--14 pH \\ \hline

Precisión 
    & $\pm$0.1 pH @ 25°C 
    & $\pm$0.1 pH @ 25°C 
    & $\pm$0.002 pH \\ \hline

Tipo de salida 
    & Analógica (0--3.0 V) 
    & Analógica (0--3.0 V) 
    & UART, I2C, Analógica \\ \hline

Voltaje de operación 
    & 3.3--5.5 V 
    & 3.3--5.5 V 
    & 3.3--5.5 V \\ \hline

Compatibilidad ESP32 
    & Sí (ADC) 
    & Sí (ADC) 
    & Sí (UART/I2C/ADC) \\ \hline

Tipo de sonda 
    & Grado laboratorio 
    & Grado industrial 
    & Grado laboratorio/industrial \\ \hline

Longitud del cable de sonda 
    & 100 cm 
    & 500 cm 
    & 100 cm (extensible) \\ \hline

Diseñado para monitoreo continuo 
    & No 
    & Sí 
    & Sí \\ \hline

Vida útil 
    & $>$0.5 años (Dependiendo frecuencia de uso) 
    & $>$0.5 años (uso 24/7) 
    & $>$2.5 años \\ \hline

Tiempo de respuesta 
    & $<$2 min 
    & $<$1 min 
    & $<$1 min \\ \hline

Costo aproximado 
    & \$39.50 USD 
    & \$64.90 USD 
    & \$159.99 USD \\ \hline

Ventajas 
    & Económico, fácil de usar 
    & Resistente, ideal para ambientes hostiles 
    & Alta precisión, múltiples interfaces de comunicación \\ \hline

Desventajas 
    & No apto para uso continuo 
    & Mayor costo que versión V2 
    & Muy costoso y compleja integración \\ \hline

Imagen
    & \shortstack{\\ \includegraphics[width=1\linewidth]{Documento/Imagenes/Análisis/sensores/pH_sensor.jpg}}
    & \shortstack{\\ \includegraphics[width=1\linewidth]{Documento/Imagenes/Análisis/sensores/pH pro sensor.jpg}}
    & \shortstack{\\ \includegraphics[width=1\linewidth]{Documento/Imagenes/Análisis/sensores/ph atlas sensor.png}} \\ \hline

\end{longtable}

Para la medición de pH, se han considerado dos alternativas principales de sensores de la línea Gravity de DFRobot: el \textit{Analog pH Meter V2} y el \textit{Industrial Analog pH Meter Pro Kit V2}. Ambos dispositivos ofrecen un rango de medición de 0 a 14 pH y una precisión de $\pm$0.1 pH a 25°C, con salidas analógicas compatibles con microcontroladores basados en ESP32.

La elección definitiva entre estos sensores dependerá de la evaluación del entorno específico donde se implementará el sistema. El \textit{Analog pH Meter V2} constituye una opción económica y de fácil integración, adecuada para aplicaciones en condiciones controladas o de baja exigencia ambiental. Por su parte, el \textit{Industrial Analog pH Meter Pro Kit V2} presenta características superiores para monitoreo continuo en ambientes hostiles, como un cable de sonda de 5 metros, un diseño robusto de grado industrial y una mayor resistencia a la exposición prolongada.

Considerando que el proyecto contempla la instalación de sensores en cuerpos de agua lóticos, donde se enfrentan condiciones variables de corriente y turbidez, el \textit{Industrial Analog pH Meter Pro Kit V2} se perfila como la opción más adecuada. No obstante, la selección final será confirmada tras la validación de las condiciones ambientales específicas del sitio de prueba.

%%%%%%%%%%%%%%%%%%%%%%%%%%%%%%%%%%%%%%%%%%%%%%%%%%%%
%             SECCIÓN: Oxigeno Disuelto            %
%%%%%%%%%%%%%%%%%%%%%%%%%%%%%%%%%%%%%%%%%%%%%%%%%%%%

\subsection{Análisis sensores oxigeno disuelto}
\renewcommand{\arraystretch}{1.5}
\begin{longtable}{
    |p{4cm}
    |p{5cm}
    |p{5cm}|
}
\caption{Comparativa de sensores para medición de oxígeno disuelto.}
\label{tab:sensores_od} \\
\hline
\textbf{Característica} 
    & \textbf{Gravity: Analog Dissolved Oxygen Sensor (SEN0237) \cite{DFRobot_DO_Sensor}} 
    & \textbf{Atlas Scientific EZO-DO™ \cite{Atlas_DO_Sensor_Manual}} \\
\hline
\endfirsthead

\hline
\textbf{Característica} 
    & \textbf{Gravity: Analog Dissolved Oxygen Sensor (SEN0237) \cite{DFRobot_DO_Sensor}} 
    & \textbf{Atlas Scientific EZO-DO™ \cite{Atlas_DO_Sensor_Manual}} \\
\hline
\endhead

\hline
\multicolumn{3}{r}{\textit{Continúa en la siguiente página}} \\
\endfoot

\hline
\endlastfoot

Tipo de tecnología 
    & Galvánica 
    & Óptica \\ \hline

Rango de medición 
    & 0--20 mg/L 
    & 0--50 mg/L \\ \hline

Precisión 
    & $\pm$0.2 mg/L 
    & $\pm$0.05 mg/L \\ \hline

Tipo de salida 
    & Analógica (0--3.0 V) 
    & UART / I2C / Analógica \\ \hline

Voltaje de operación 
    & 3.3--5.5 V 
    & 3.3--5.5 V \\ \hline

Compatibilidad con ESP32 
    & Sí (ADC) 
    & Sí (UART/I2C/ADC) \\ \hline

Diseñado para monitoreo continuo 
    & No recomendado 
    & Sí \\ \hline

Vida útil de la sonda 
    & 6--12 meses 
    & $>$2 años \\ \hline

Calibración necesaria 
    & Frecuente 
    & Muy baja frecuencia \\ \hline

Costo aproximado 
    & \$169 USD 
    & \$195+ USD \\ \hline

Ventajas 
    & Bajo costo, fácil de usar 
    & Alta precisión, mínima deriva, muy estable \\ \hline

Desventajas 
    & Afectado por temperatura y flujo, no ideal para monitoreo 24/7 
    & Muy costoso, integración más compleja \\ \hline

Imagen
    & \shortstack{\\ \includegraphics[width=0.8\linewidth]{Documento/Imagenes/Análisis/sensores/OD sensor.jpg}}
    & \shortstack{\\ \includegraphics[width=0.8\linewidth]{Documento/Imagenes/Análisis/sensores/OD Atlas Sensor.png}} \\ \hline

\end{longtable}


Para la medición de oxígeno disuelto (OD) en el agua, se consideran dos opciones principales: el sensor Gravity: Analog Dissolved Oxygen Sensor (SEN0237) de DFRobot y el Atlas Scientific EZO-DO™.

El sensor Gravity Analog es una opción de bajo costo, basado en tecnología galvánica, que ofrece un rango de medición de 0 a 20 mg/L con una precisión de aproximadamente $\pm$0.2 mg/L. Su salida analógica facilita la integración con microcontroladores basados en ESP32. No obstante, presenta ciertas limitaciones para monitoreo continuo, ya que sufre interferencias por temperatura y flujo, y requiere calibraciones más frecuentes.

Por otro lado, el sensor Atlas Scientific EZO-DO™ utiliza tecnología óptica, ofreciendo un rango extendido de 0 a 50 mg/L y una alta precisión de $\pm$0.05 mg/L. Además, es adecuado para operaciones de monitoreo continuo en ambientes variables, con una mínima necesidad de calibración y una vida útil superior a los dos años. Sin embargo, su costo significativamente más elevado y su integración más compleja deben ser considerados.

Debido a que el sistema propuesto busca un equilibrio entre costos razonables y confiabilidad operativa, el sensor Gravity Analog se perfila como la opción más viable para su implementación inicial. No obstante, se mantiene abierta la posibilidad de considerar el sensor Atlas Scientific EZO-DO™ en futuras fases de ampliación o mejora del sistema, en caso de que se requiera una mayor precisión y robustez frente a condiciones ambientales extremas.

%%%%%%%%%%%%%%%%%%%%%%%%%%%%%%%%%%%%%%%%%%%%%%%%%%%%
%             SECCIÓN: Turbidez                    %
%%%%%%%%%%%%%%%%%%%%%%%%%%%%%%%%%%%%%%%%%%%%%%%%%%%%

\subsection{Análisis sensor de turbidez}
\renewcommand{\arraystretch}{1.5}
\begin{longtable}{
    |p{4cm}
    |p{8cm}|
}
\caption{Características del sensor seleccionado para la medición de turbidez.}
\label{tab:sensor_turbidez} \\
\hline
\textbf{Característica} 
    & \textbf{Gravity: Analog Turbidity Sensor (SEN0189) \cite{DFRobot_Turbidity_Sensor}} \\ 
\hline
\endfirsthead

\hline
\textbf{Característica} 
    & \textbf{Gravity: Analog Turbidity Sensor (SEN0189) \cite{DFRobot_Turbidity_Sensor}} \\ 
\hline
\endhead

\hline
\multicolumn{2}{r}{\textit{Continúa en la siguiente página}} \\
\endfoot

\hline
\endlastfoot

Tipo de tecnología 
    & Sensor óptico infrarrojo (dispersión de luz) \\ \hline

Rango de medición 
    & 0--1000 NTU \\ \hline

Precisión 
    & Dependiente de la calibración (aproximada) \\ \hline

Tipo de salida 
    & Analógica (0--4.5 V) \\ \hline

Voltaje de operación 
    & 5 V \\ \hline

Compatibilidad con ESP32 
    & Sí (entrada ADC) \\ \hline

Diseñado para monitoreo continuo 
    & No recomendado (sellado básico) \\ \hline

Vida útil de la sonda 
    & 6--12 meses (dependiendo de condiciones ambientales) \\ \hline

Calibración necesaria 
    & Frecuente en ambientes exteriores \\ \hline

Costo aproximado 
    & \$10 USD \\ \hline

Ventajas 
    & Bajo costo, fácil de usar, rápida integración \\ \hline

Desventajas 
    & Precisión limitada, sensibilidad a suciedad y humedad \\ \hline

Imagen
    & \shortstack{\\ \includegraphics[width=0.7\linewidth]{Documento/Imagenes/Análisis/sensores/turbidez sensor.jpg}} \\ \hline

\end{longtable}


Para la medición de turbidez en el sistema propuesto se ha seleccionado el sensor Gravity: Analog Turbidity Sensor (SKU: SEN0189) de DFRobot. Este sensor permite medir niveles de turbidez en un rango de 0 a 1000 NTU, entregando una señal analógica proporcional a la concentración de partículas suspendidas en el agua. Su compatibilidad con microcontroladores basados en ESP32, su facilidad de integración mediante conversión analógica lo hacen adecuado para el presente proyecto.

Aunque su precisión es limitada en comparación con sensores digitales o de grado industrial, y su desempeño puede verse afectado por condiciones extremas de suciedad o humedad prolongada, su bajo costo y facilidad de implementación representan una alternativa viable para una primera fase de monitoreo distribuido en cuerpos de agua lóticos. La utilización de este sensor permitirá realizar una estimación práctica del nivel de sólidos suspendidos en el agua y, de ser necesario, futuras iteraciones del sistema podrán contemplar la adopción de tecnologías más robustas.

%%%%%%%%%%%%%%%%%%%%%%%%%%%%%%%%%%%%%%%%%%%%%%%%%%%%
%             SECCIÓN: Conductividad               %
%%%%%%%%%%%%%%%%%%%%%%%%%%%%%%%%%%%%%%%%%%%%%%%%%%%%

\subsection{Análisis sensores conductividad}
\renewcommand{\arraystretch}{1.5}
\begin{longtable}{
    |p{4cm}
    |p{5cm}
    |p{5cm}|
}
\caption{Comparativa de sensores para medición de conductividad eléctrica.}
\label{tab:sensor_conductividad} \\
\hline
\textbf{Característica} 
    & \textbf{Gravity: Analog EC Sensor V2 (DFR0300) \cite{DFRobot_EC_Sensor}} 
    & \textbf{Atlas Scientific EZO-EC™ \cite{AtlasScientific_Conductivity}} \\ 
\hline
\endfirsthead

\hline
\textbf{Característica} 
    & \textbf{Gravity: Analog EC Sensor V2 (DFR0300) \cite{DFRobot_EC_Sensor}} 
    & \textbf{Atlas Scientific EZO-EC™ \cite{AtlasScientific_Conductivity}} \\ 
\hline
\endhead

\hline
\multicolumn{3}{r}{\textit{Continúa en la siguiente página}} \\
\endfoot

\hline
\endlastfoot

Tipo de tecnología 
    & Electrodos de conductividad (K=1.0) 
    & Electrodos de conductividad \\ \hline

Parámetro principal 
    & Conductividad (EC) 
    & Conductividad (EC) \\ \hline

Rango de medición 
    & 0--20,000~$\mu$S/cm 
    & 0--200,000~$\mu$S/cm \\ \hline

Precisión 
    & $\pm$5\% F.S. 
    & $\pm$2\% F.S. \\ \hline

Tipo de salida 
    & Analógica (0--3.0 V) 
    & UART / I2C / Analógica \\ \hline

Voltaje de operación 
    & 3.0--5.0 V 
    & 3.3--5.5 V \\ \hline

Compatibilidad con ESP32 
    & Sí (ADC) 
    & Sí (UART/I2C/ADC) \\ \hline

Diseñado para monitoreo continuo 
    & Sí (recalibración periódica requerida) 
    & Sí (mínimo mantenimiento) \\ \hline

Vida útil de la sonda 
    & 6--12 meses 
    & $>$2 años \\ \hline

Calibración necesaria 
    & Periódica (solución estándar de conductividad) 
    & Muy baja frecuencia \\ \hline

Costo aproximado 
    & \$70 USD 
    & \$200+ USD \\ \hline

Ventajas 
    & Buen rango, estable, económico 
    & Alta precisión, muy robusto \\ \hline

Desventajas 
    & Requiere calibración manual periódica 
    & Alto costo, integración más compleja \\ \hline

Imagen
    & \shortstack{\\ \includegraphics[width=0.8\linewidth]{Documento/Imagenes/Análisis/sensores/conductividad sensor.jpg}}
    & \shortstack{\\ \includegraphics[width=0.8\linewidth]{Documento/Imagenes/Análisis/sensores/conductividad atlas sensor.png}} \\ \hline

\end{longtable}


Para la medición de conductividad eléctrica (EC) en el sistema propuesto se consideraron dos opciones: el sensor Gravity: Analog Electrical Conductivity Sensor V2 (SKU: DFR0300) de DFRobot y el Atlas Scientific EZO-EC™.

El sensor Gravity EC V2 ofrece un rango de medición de 0 a 20 mS/cm, con una precisión de $\pm$5\% de la escala completa. Su salida analógica facilita la integración directa con microcontroladores basados en ESP32, mediante el uso del convertidor analógico-digital (ADC) interno. Además, su costo accesible y su diseño robusto lo convierten en una opción viable para proyectos de monitoreo distribuido en cuerpos de agua lóticos. No obstante, requiere calibraciones periódicas utilizando soluciones estándar de conductividad, especialmente en aplicaciones de largo plazo.

En contraste, el sensor Atlas Scientific EZO-EC™ proporciona un rango extendido de hasta 200 mS/cm, alta precisión de $\pm$2\%, y mínimas necesidades de calibración. Sin embargo, su costo elevado y su integración más compleja lo hacen menos accesible para un sistema que busca un equilibrio entre funcionalidad y viabilidad económica.

Considerando los objetivos de este Proyecto y las restricciones presupuestarias, se selecciona el sensor Gravity: Analog Electrical Conductivity Sensor V2 como la opción más adecuada para la medición de conductividad, dejando abierta la posibilidad de futuras mejoras en etapas posteriores del proyecto.
%%%%%%%%%%%%%%%%%%%%%%%%%%%%%%%%%%%%%%%%%%%%%%%%%%%%
%             SECCIÓN: Temperatura                 %
%%%%%%%%%%%%%%%%%%%%%%%%%%%%%%%%%%%%%%%%%%%%%%%%%%%%

\subsection{Análisis sensores temperatura}
\renewcommand{\arraystretch}{1.5}
\begin{longtable}{
    |p{4cm}
    |p{5cm}
    |p{5cm}|
}
\caption{Comparativa de sensores para medición de temperatura.}
\label{tab:sensor_temperatura} \\
\hline
\textbf{Característica} 
    & \textbf{DS18B20 Sumergible \cite{DFRobot_DS18B20}} 
    & \textbf{Atlas Scientific EZO-RTD™ \cite{Atlas_EZO_RTD}} \\ 
\hline
\endfirsthead

\hline
\textbf{Característica} 
    & \textbf{DS18B20 Sumergible \cite{DFRobot_DS18B20}} 
    & \textbf{Atlas Scientific EZO-RTD™ \cite{Atlas_EZO_RTD}} \\ 
\hline
\endhead

\hline
\multicolumn{3}{r}{\textit{Continúa en la siguiente página}} \\
\endfoot

\hline
\endlastfoot

Tipo de tecnología 
    & Sensor digital 1-Wire 
    & RTD PT-1000 industrial \\ \hline

Rango de medición 
    & -55\,\textdegree C a +125\,\textdegree C 
    & -200\,\textdegree C a +850\,\textdegree C \\ \hline

Precisión típica 
    & $\pm$0.5\,\textdegree C (-10\,\textdegree C a +85\,\textdegree C) 
    & $\pm$0.1\,\textdegree C \\ \hline

Tipo de salida 
    & Digital (1-Wire) 
    & UART / I2C \\ \hline

Voltaje de operación 
    & 3.0--5.5 V 
    & 3.3--5.5 V \\ \hline

Compatibilidad con ESP32 
    & Sí (1-Wire) 
    & Sí (UART/I2C) \\ \hline

Diseñado para inmersión continua 
    & Sí 
    & Sí \\ \hline

Calibración necesaria 
    & No 
    & No \\ \hline

Costo aproximado 
    & \$7.5 USD 
    & \$30 USD \\ \hline

Ventajas 
    & Muy económico, fácil integración, adecuado para monitoreo de calidad de agua 
    & Altísima precisión, robustez industrial \\ \hline

Desventajas 
    & Menor precisión que RTD, rango limitado para aplicaciones extremas 
    & Alto costo, integración más compleja \\ \hline

Imagen
    & \shortstack{\\ \includegraphics[width=0.75\linewidth]{Documento/Imagenes/Análisis/sensores/Temp_Sensor.png}}
    & \shortstack{\\ \includegraphics[width=0.75\linewidth]{Documento/Imagenes/Análisis/sensores/Temp_Atlas_Sensor.png}} \\ \hline

\end{longtable}


Para la medición de temperatura en el sistema propuesto se consideraron dos alternativas: el sensor digital sumergible DS18B20 y el sensor industrial Atlas Scientific EZO-RTD™.

El sensor DS18B20, ampliamente utilizado en aplicaciones de monitoreo ambiental, ofrece un rango de medición de -55°C a 125°C con una precisión típica de ±0.5°C en el rango de -10°C a 85°C. Su comunicación mediante el protocolo digital 1-Wire facilita su integración directa con microcontroladores basados en ESP32. Además, su bajo costo, facilidad de adquisición y resistencia a la inmersión continua en agua lo convierten en una opción práctica y adecuada para sistemas de monitoreo distribuidos en cuerpos de agua.

Por su parte, el sensor Atlas Scientific EZO-RTD™ proporciona una mayor precisión (±0.1°C) y un rango de operación mucho más amplio, gracias al uso de sensores RTD de tipo PT-1000. Sin embargo, su alto costo y mayor complejidad de integración lo hacen menos adecuado para proyectos donde se busca un balance entre funcionalidad y viabilidad económica.

Considerando los objetivos de este Proyecto y las condiciones operativas previstas, se selecciona el sensor DS18B20 sumergible como la opción más adecuada para la medición de temperatura en cuerpos de agua lóticos.

%%%%%%%%%%%%%%%%%%%%%%%%%%%%%%%%%%%%%%%%%%%%%%%%%%%%
%             SECCIÓN: Topologia de red            %
%%%%%%%%%%%%%%%%%%%%%%%%%%%%%%%%%%%%%%%%%%%%%%%%%%%%

\section{Análisis de la topología de red}


El diseño de la Red Inalámbrica de Sensores (WSN) es fundamental para garantizar la conectividad fiable y eficiente entre los nodos distribuidos y el sistema central de procesamiento. Para este proyecto, se ha seleccionado una \textbf{topología lineal multisalto} (\textit{multi-hop linear topology}).

\subsection{Descripción de la Topología Seleccionada}
\label{subsec:descripcion_topologia}

La red se compone de cuatro nodos sensores (N=4) desplegados secuencialmente a lo largo del cuerpo de agua lótico y un nodo final denominado nodo concentrador o \textit{gateway} (\textit{sink node}), ubicado al final de la cadena. Cada nodo sensor ($n=1, ..., 4$) captura datos fisicoquímicos e imágenes. La comunicación hacia el \textit{gateway} se realiza mediante un esquema de retransmisión secuencial (\textit{store-and-forward}), donde cada nodo $n$ recibe los datos acumulados de los nodos $1$ a $n-1$ y los retransmite junto con sus propios datos hacia el nodo $n+1$ (o hacia el \textit{gateway} si $n=N$). La Figura \ref{fig:topologia_lineal_fluvial} ilustra esta configuración.

\begin{figure}[H]
   \centering
   \includegraphics[width=0.8\linewidth]{Documento/Imagenes/Análisis/Topologia/Topologia_Lineal.pdf}
   \caption{Topología lineal multisalto para el monitoreo fluvial. Los nodos sensores se comunican mediante enlaces inalámbricos (líneas discontinuas), y el Nodo Sink/Gateway se conecta a la nube mediante Wi-Fi estándar (línea sólida).}
   \label{fig:topologia_lineal_fluvial}
\end{figure}

El \textit{gateway} actúa como punto de agregación final. Recibe toda la información de la WSN a través del enlace inalámbrico proveniente del último nodo sensor (Nodo 4). Posteriormente, utiliza una interfaz de comunicación distinta, Wi-Fi estándar (IEEE 802.11n/g), para transmitir los datos consolidados al servidor central alojado en la nube. Esta elección para el enlace final (\textit{backhaul}) facilita la integración con infraestructuras de red comunes (routers Wi-Fi) para el acceso a Internet.


\subsection{Justificación de la Topología Lineal}
\label{subsec:justificacion_topologia}

La topología lineal se considera la más adecuada para el monitoreo de entornos fluviales por las siguientes razones:

\begin{itemize}
    \item \textbf{Adaptación al Entorno Geográfico:} Se alinea de forma natural con la morfología alargada de ríos y arroyos.
    \item \textbf{Monitoreo Longitudinal:} Facilita el estudio de la evolución espacial de parámetros a medida que el agua fluye.
    \item \textbf{Simplicidad de Enrutamiento:} El flujo de datos es determinista (de $n$ a $n+1$), simplificando la lógica en los nodos y ahorrando energía al evitar protocolos de enrutamiento complejos. 
\end{itemize}
Alternativas como topologías en estrella o malla presentarían desventajas en términos de alcance o complejidad para esta aplicación específica.

\subsection{Esquema de Comunicación Inter-Nodo}
\label{subsec:esquema_comunicacion_inter_nodo} 

La comunicación \textit{entre} los nodos sensores y hacia el \textit{gateway} utilizará una tecnología inalámbrica de medio alcance y bajo consumo (cuya selección se detalla en la Sección \ref{sec:analisis_tecnologias}). Esta tecnología deberá operar en modo \textit{half-duplex}, lo cual impacta el tiempo total de comunicación, como se analizó en la Sección \ref{ssubsec:tiempo_general_comm}. El esquema de retransmisión es \textit{store-and-forward}, requiriendo mecanismos básicos de fiabilidad como confirmaciones de recepción (ACK) para asegurar la entrega entre saltos.

\subsection{Implicaciones Técnicas y Limitaciones}
\label{subsec:implicaciones_topologia}

La topología lineal seleccionada presenta ciertas implicaciones:

\begin{itemize}
    \item \textbf{Carga Energética Desbalanceada:} Los nodos más cercanos al \textit{gateway} (Nodo 4) soportan mayor tráfico de retransmisión (Ecuación \eqref{eq:tiempo_comm_nodo_n}), requiriendo un dimensionamiento energético cuidadoso.
    \item \textbf{Retardo Acumulado:} Tolerable dada la baja frecuencia de muestreo semanal.
    \item \textbf{Tolerancia a Fallos Limitada:} La falla de un nodo intermedio interrumpe la cadena. Se considerarán mecanismos de detección de fallos o redundancia para futuras fases.
\end{itemize}

A pesar de la limitación en la tolerancia a fallos, la simplicidad, eficiencia y adecuación geográfica de la topología lineal la validan como la opción más pragmática para la fase actual del proyecto. La conexión final del \textit{gateway} mediante Wi-Fi estándar simplifica la integración con la infraestructura de nube.



%%%%%%%%%%%%%%%%%%%%%%%%%%%%%%%%%%%%%%%%%%%%%%%%%%%%
%             SECCIÓN: Transceptores               %
%%%%%%%%%%%%%%%%%%%%%%%%%%%%%%%%%%%%%%%%%%%%%%%%%%%%


\section{Análisis de Transceptores Inalámbricos}
\label{sec:analisis_tecnologias}

Para seleccionar la tecnología más adecuada para la transmisión de datos en el sistema de monitoreo, se evaluaron distintos estándares inalámbricos considerando tres criterios fundamentales:

\begin{enumerate}
    \item \textbf{Consumo energético}: Factor determinante para asegurar una operación prolongada mediante baterías, minimizando la necesidad de mantenimiento en zonas de difícil acceso.
    \item \textbf{Alcance efectivo}: Capacidad del protocolo para cubrir grandes distancias en una topología lineal a lo largo del río, considerando la presencia de obstáculos naturales como vegetación densa y la falta de línea de vista directa (NLOS).
    \item \textbf{Sensibilidad y Robustez}: Capacidad del receptor para decodificar señales débiles, lo cual es crítico en entornos húmedos y con vegetación que atenúan las señales de radiofrecuencia.
\end{enumerate}

La Tabla~\ref{tab:redes_corto_alcance} presenta un resumen comparativo de soluciones de red de corto alcance comúnmente utilizadas en WSN. Se incluyen tecnologías basadas en los estándares IEEE 802.15.4 y 802.15.1. Estas soluciones, aunque eficientes energéticamente, presentan limitaciones críticas de alcance ($<100$ m) para un despliegue fluvial extenso, lo que requeriría una cantidad excesiva de nodos repetidores.

\begin{table}[H]
\centering
\caption{Resumen de soluciones de red de corto alcance \cite{lopez2023}}
\label{tab:redes_corto_alcance}
\renewcommand{\arraystretch}{1.3}
\resizebox{\textwidth}{!}{
\begin{tabular}{p{2.5cm}p{2.5cm}p{2cm}p{2.5cm}p{2cm}p{1.5cm}p{2.3cm}}
\toprule
\textbf{Solution} & \textbf{Standard} & \textbf{Data rate} & \textbf{Coverage} & \textbf{Topology} & \textbf{Carrier freq} & \textbf{Energy cost} \\
\midrule
IEEE 802.15.4      & IEEE 802.15.4     & 250 kbps           & $<$100 m          & Star, tree, cluster tree, mesh & 868/915 MHz, 2.4 GHz & Low \\
ZigBee             & IEEE 802.15.4     & 250, 40, 20 kbps    & $>$100 m          & Mesh, star, tree               & 868/915 MHz, 2.4 GHz & Low \\
6LoWPAN            & IEEE 802.15.4     & 250 kbps           & $<$100 m          & Mesh, star                     & 868/915 MHz, 2.4 GHz & Low \\
Bluetooth          & IEEE 802.15.1     & 1 Mbps             & Up to 100 m       & Star, P2P                      & 2.4 GHz              & Low \\
BLE                & IEEE 802.15.1     & 1 Mbps             & Up to 200 m       & Star, P2P                      & 2.4 GHz              & Very low \\
\bottomrule
\end{tabular}
}
\end{table}

La Tabla~\ref{tab:redes_largo_alcance} complementa el análisis enfocándose en redes de medio y largo alcance (LPWAN). Se compara el estándar Wi-Fi (alto ancho de banda pero corto alcance/alto consumo) contra la tecnología LoRa. A diferencia del Wi-Fi, LoRa utiliza una técnica de modulación de espectro ensanchado (Chirp Spread Spectrum) que permite alcances de varios kilómetros y una alta inmunidad a interferencias, a costa de una tasa de transmisión reducida. Esta característica es vital para superar la atenuación por vegetación en las riberas del río.

\begin{table}[H]
\centering
\caption{Comparativa de tecnologías de medio y largo alcance\cite{lopez2023}}
\label{tab:redes_largo_alcance}
\renewcommand{\arraystretch}{1.3}
\resizebox{\textwidth}{!}{
\begin{tabular}{p{2.5cm}p{3cm}p{3cm}p{2cm}p{2.5cm}p{2cm}p{2.3cm}}
\toprule
\textbf{Tecnología} & \textbf{Estándar/Modulación} & \textbf{Tasa de Datos} & \textbf{Cobertura} & \textbf{Topología} & \textbf{Frecuencia} & \textbf{Costo Energético} \\
\midrule
Wi-Fi & IEEE 802.11 & Up to 54 Mbps & 200 m & Infrastructure, mesh & 2.4 GHz, 5 GHz & High \\
Wi-Fi HaLow & IEEE 802.11ah   & 0.15 to 4 Mbps @ 1 MHz bandwidth, an 0.65 to 7.8 Mbps @ 2 MHz bandwidth & 1000m & star, mesh & ISM Sub-GHz & Low \\ 
LoRa & CSS (Chirp Spread Spectrum) & 0.3 kbps a 50 kbps (Adaptativo) & $>$ 10 km (Rural) & Star, Mesh, P2P & Sub-GHz (915 MHz) & Muy Bajo \\ 
\bottomrule
\end{tabular}
}
\end{table}

Considerando las restricciones geográficas, en la Tabla~\ref{tab:comparativa_transceptores} se comparan módulos comerciales específicos. Se destaca el transceptor \textbf{SX1262} (integrado en la placa Heltec Wireless Tracker), que representa la generación más reciente de radios LoRa, ofreciendo mayor potencia de transmisión (+22 dBm) y menor consumo en recepción comparado con generaciones anteriores (como el SX1276) o tecnologías como Wi-Fi HaLow.

\begin{table}[H]
\renewcommand{\arraystretch}{1.5}
\centering
\caption{Comparativa de transceptores comerciales para WSN}
\label{tab:comparativa_transceptores}
\resizebox{\textwidth}{!}{
\begin{tabular}{|p{3.2cm}|C{3cm}|C{3cm}|C{3cm}|C{3cm}|C{3cm}|}
\hline
\textbf{Parámetro} 
& \shortstack{\\ \textbf{LoRa (SX1262)}\\ \textbf{(Propuesto)}} 
& \shortstack{\\ \textbf{Wi-Fi HaLow}\\ \textbf{(802.11ah)}} 
& \shortstack{\\ \textbf{Wi-Fi}\\ \textbf{(802.11n)}} 
& \shortstack{\\ \textbf{Zigbee}\\ \textbf{(802.15.4)}}
& \shortstack{\\ \textbf{Bluetooth} \\ \textbf{(802.15.1)}} \\ \hline

\textbf{Módulo/Chip} & Heltec (SX1262) \cite{semtech_sx1261_2017} & Quectel FGH100M \cite{QuectelFGH100M} & ESP32-S3 \cite{EspressifESP32S3} & XBee S2C \cite{DigiXBeeS2C} & HC-05 \cite{HC05Datasheet} \\ \hline
\textbf{Tasa Máx.} & $\sim$ 22 kbps (SF7) & 32.5 Mbps & 150 Mbps & 0.25 Mbps & 2.1 Mbps \\ \hline
\textbf{Potencia Tx} & +22 dBm & +20 dBm & +19 dBm & +5 dBm & +4 dBm \\ \hline
\textbf{Sensibilidad Rx} & -148 dBm & -98 dBm & -98 dBm & -100 dBm & -84 dBm \\ \hline
\textbf{Alcance LOS\footnote{Line Of Sight}} & $>$ 10 km & $\sim$ 1 km & 120 m & 1200 m & 40 m \\ \hline
\textbf{Consumo Tx} & 118 mA (@22dBm) & 286 mA & 286 mA & 33 mA & 30 mA \\ \hline
\textbf{Consumo Rx} & 4.2 mA & 43 mA & 95 mA & 28 mA & 30 mA \\ \hline
\textbf{Consumo Sleep} & 0.6 µA & TBD & 240 µA & 1 µA & TBD \\ \hline

\end{tabular}
}
\end{table}


En la Tabla \ref{tab:imagenes_transceptore} se presentan imágenes representativas de los módulos evaluados.

\begin{table}[H]
\renewcommand{\arraystretch}{1.5}
\centering
\caption{Ilustraciones de transceptores evaluados}
\label{tab:imagenes_transceptore}
\resizebox{\textwidth}{!}{
\begin{tabular}{|C{3.5cm}|C{3.5cm}|C{3.5cm}|C{3.5cm}|C{3.5cm}|}
\hline
 \textbf{Heltec (SX1262)} & \textbf{FGH100M} & \textbf{ESP32-S3} & \textbf{XBee S2C} & \textbf{HC-05} \\ \hline

\shortstack{\\ \includegraphics[width=1 \linewidth]{Documento/Imagenes/Análisis/Transceptores/heltec_lora.png}} 
&
\shortstack{\\ \includegraphics[width=1 \linewidth]{Documento/Imagenes/Análisis/Transceptores/Xiao Halow.jpg}}
&
\shortstack{\\ \includegraphics[width=1 \linewidth]{Documento/Imagenes/Análisis/Transceptores/Esp32.jpg}}
&
\shortstack{\\ \includegraphics[width=1 \linewidth]{Documento/Imagenes/Análisis/Transceptores/zigbee.jpg}}
&
\shortstack{\\ \includegraphics[width=1 \linewidth]{Documento/Imagenes/Análisis/Transceptores/HC-05.jpg}}
\\ \hline
\end{tabular}
}
\end{table}

\subsection{Justificación y Estrategia de Implementación}
\label{subsec:justificacion_lora}

Con base en la comparativa anterior, se selecciona el transceptor \textbf{Semtech SX1262} como el elemento de comunicación central del sistema. Esta decisión se fundamenta en dos ventajas técnicas decisivas para el entorno fluvial: su \textbf{presupuesto de enlace superior} y su \textbf{eficiencia energética}.

Con una potencia de salida de \SI{+22}{dBm} y una sensibilidad de \SI{-148}{dBm}, el SX1262 proporciona un margen de enlace de \SI{170}{dB}, indispensable para superar la atenuación severa causada por la vegetación ribereña y la falta de línea de vista directa, condiciones donde tecnologías como Wi-Fi o Zigbee fallarían. Además, su consumo en recepción de solo \SI{4.2}{\milli\ampere} extiende significativamente la vida útil de la batería en los nodos de la red lineal, que deben permanecer en modo de escucha para retransmitir datos.

No obstante, la limitación crítica de LoRa es su ancho de banda, lo que dificulta la transmisión de archivos grandes como imágenes. Para ilustrar este desafío, se calcula el Tiempo en el Aire (ToA) requerido para transmitir una imagen de \SI{150}{\kilo\byte} (equivalente a \SI{1.23}{Mbits}) asumiendo el escenario más optimista permitido por el transceptor SX1262 ($SF=7$, $BW=\SI{500}{\kilo\hertz}$, $CR=4/5$), el cual ofrece una tasa máxima teórica de \SI{21.9}{kbps}:

\begin{equation} \label{eq:calculo_toa_imagen}
T_{ToA} \approx \frac{S_{img}}{R_{b}} = \frac{\SI{1.23}{Mbit}}{\SI{21.9}{kbit/s}} \approx \SI{56}{\second}
\end{equation}

Donde:

\begin{itemize}
    \item $T_{\mathrm{ToA}}$: Tiempo en el Aire (\textit{Time on Air}).
    \item $S_{\mathrm{img}}$: Tamaño de la imagen comprimida ($\approx \SI{150}{kB}$).
    \item $R_{b}$: Tasa de transmisión de datos efectiva (Bitrate).
\end{itemize}

Aunque un tiempo de transmisión de \SI{56}{\second} es inviable para video en tiempo real, estudios experimentales como el realizado por Jebril et al. \cite{jebril2018overcoming} han demostrado la viabilidad de transmitir imágenes JPEG sobre la capa física (PHY) de LoRa en entornos de vegetación densa (manglares), superando las limitaciones de la zona de Fresnel.

Por lo tanto, para equilibrar la robustez del enlace con la eficiencia operativa, se adopta la siguiente estrategia:
\begin{enumerate}
    \item \textbf{Transmisión de Datos de Sensores:} La transmisión periódica de los parámetros físico-químicos (pH, oxígeno disuelto, temperatura, etc.) constituye la operación base del sistema. Dado el reducido tamaño de estos datos, su transmisión se realiza con máxima eficiencia energética y mínima ocupación del canal.
    \item \textbf{Transmisión de Imágenes (Fragmentación):} Para el envío de la evidencia visual, se utiliza el protocolo de fragmentación hexadecimal validado por Jebril et al. \cite{jebril2018overcoming} para transmitir la imagen completa a través de la capa física LoRa. Se acepta el mayor costo energético temporal (\SI{\sim 118}{\milli\ampere} durante la transmisión) a cambio de garantizar que la información llegue al destino a pesar de los obstáculos del entorno.
\end{enumerate}
























\begin{comment}
\section{Análisis de Transceptores Inalámbricos}
\label{sec:analisis_tecnologias}

Para seleccionar la tecnología más adecuada para la transmisión de datos en el sistema de monitoreo, se evaluaron distintos estándares inalámbricos considerando tres criterios fundamentales:

\begin{enumerate}
    \item \textbf{Consumo energético}: Factor determinante para asegurar una operación prolongada mediante baterías, minimizando la necesidad de mantenimiento.
    \item \textbf{Alcance efectivo}: Capacidad del protocolo para cubrir distancias en una topología lineal, considerando la presencia de obstáculos naturales como vegetación y accidentes geográficos.
    \item \textbf{Tasa máxima de transmisión}: Velocidad teórica máxima con la que se pueden enviar datos, clave para asegurar el envío oportuno de lecturas y archivos, como imágenes capturadas por el sistema.
\end{enumerate}

La Tabla~\ref{tab:redes_corto_alcance} presenta un resumen comparativo de soluciones de red de corto alcance comúnmente utilizadas en aplicaciones de monitoreo ambiental con WSN. Se incluyen tecnologías basadas en los estándares IEEE 802.15.4 y 802.15.1, como ZigBee, 6LoWPAN, Bluetooth y BLE, analizando sus tasas de transmisión, topologías soportadas, rango típico de cobertura, frecuencia de operación y consumo energético. Estas soluciones priorizan eficiencia energética y bajo costo de despliegue, aunque con limitaciones en velocidad o alcance.

\input{Documento/Tablas/ResumenTecnologiasWL}


La Tabla~\ref{tab:redes_alcance_medio} complementa el análisis anterior al enfocarse en redes de alcance medio. Se comparan dos variantes del estándar IEEE 802.11: Wi-Fi tradicional y Wi-Fi HaLow (802.11ah). Aunque Wi-Fi convencional ofrece altas tasas de transferencia, su consumo energético es elevado y su cobertura limitada. En cambio, Wi-Fi HaLow opera en bandas sub-GHz, permitiendo mayor alcance, menor consumo y adecuada velocidad, lo que lo hace más apropiado para entornos remotos como cuerpos de agua lóticos.

\input{Documento/Tablas/ResumenTecnologiasWM}

Dadas las posibles soluciones, en la Tabla~\ref{tab:comparativa_transceptores} se comparan módulos transceptores comerciales representativos de cada tecnología evaluada. Se contrastan parámetros clave como tasa máxima de transmisión, potencia de transmisión y sensibilidad de recepción, alcance en línea de vista (LOS), y consumo de corriente en modos activo y en reposo. 

\begin{longtable}{|p{2.5cm}|C{2.7cm}|C{2.1cm}|C{2.1cm}|C{2.2cm}|C{2.5cm}|}
\caption{Comparativa de transceptores comerciales para WSN}
\label{tab:comparativa_transceptores} \\
\hline
\textbf{Parámetro} 
& \shortstack{\\ \textbf{Wi-Fi HaLow}\\ \textbf{(802.11ah)}} 
& \shortstack{\\ \textbf{Wi-Fi}\\ \textbf{(802.11n)}} 
& \shortstack{\\ \textbf{Zigbee}\\ \textbf{(802.15.4)}} 
& \shortstack{\\ \textbf{Bluetooth}\\ \textbf{(802.15.1)}}
& \shortstack{\\ \textbf{LoRaWAN}\\ \textbf{(P1451.5.5)}} \\ 
\hline
\endfirsthead

\multicolumn{6}{c}%
{{\bfseries \tablename\ \thetable{} -- continuación de la página anterior}} \\
\hline
\textbf{Parámetro} 
& \shortstack{\\ \textbf{Wi-Fi HaLow}\\ \textbf{(802.11ah)}} 
& \shortstack{\\ \textbf{Wi-Fi}\\ \textbf{(802.11n)}} 
& \shortstack{\\ \textbf{Zigbee}\\ \textbf{(802.15.4)}} 
& \shortstack{\\ \textbf{Bluetooth}\\ \textbf{(802.15.1)}}
& \shortstack{\\ \textbf{LoRaWAN}\\ \textbf{(P1451.5.5)}} \\ 
\hline
\endhead

\hline
\multicolumn{6}{r}{{Continúa en la siguiente página}} \\
\endfoot

\hline
\endlastfoot

\textbf{Módulo} & FGH100M \cite{QuectelFGH100M} & ESP32-S3 \cite{EspressifESP32S3} & XBee S2C \cite{DigiXBeeS2C} & HC-05 \cite{HC05Datasheet} & Wio E5 LE \cite{SeeedWioE5LE} \\ \hline
\textbf{Tasa Máx. (Mbps)} & 32.5 & 150 & 0.25 & 2.1 & 0.05 \\ \hline
\textbf{Potencia Tx / Sensibilidad Rx (dBm)} & 20 / -98 & 19 / -75 & 5 / -100 & 4 / -84 & 13.461 / -137 \\ \hline
\textbf{Alcance LOS\footnote{Line Of Sight} (m)} & 1000 & 120 & 1200 & 40 & 10000 \\ \hline
\textbf{Consumo Tx (mA)} & 54 & 286 & 33 & 30 & 26 \\ \hline
\textbf{Consumo Rx (mA)} & 43 & 95 & 28 & 30 & 13 \\ \hline
\textbf{Consumo en Sleep (µA)} & TBD\footnotemark{} & 240 & 1 & TBD\footnotemark[\value{footnote}] & 1.3 \\ \hline
\footnotetext{To Be Defined}
\end{longtable}

Esta comparativa refuerza la selección de Wi-Fi HaLow mediante el módulo FGH100M, al ofrecer un equilibrio favorable entre alcance, velocidad, consumo y viabilidad de integración en nodos autónomos. En la tabla \ref{tab:imagenes_transceptore} se presentan imágenes representativas de cada transceptor evaluado para facilitar su identificación visual.


\begin{table}[H]
\renewcommand{\arraystretch}{1.5}
\centering
\caption{Ilustraciones transceptores}
\label{tab:imagenes_transceptore}
\resizebox{\textwidth}{!}{
\begin{tabular}{|C{3.5cm}|C{3.5cm}|C{3.5cm}|C{3.5cm}|C{3.5cm}|}
\hline
 \textbf{FGH100M} & \textbf{ESP32-S3} & \textbf{XBee S2C} & \textbf{HC-05} & \textbf{Wio E5 LE} \\ \hline

\shortstack{\\ \includegraphics[width=1 \linewidth]{Documento/Imagenes/Análisis/Transceptores/Xiao Halow.jpg}}
&
\shortstack{\\ \includegraphics[width=1 \linewidth]{Documento/Imagenes/Análisis/Transceptores/Esp32.jpg}}
&
\shortstack{\\ \includegraphics[width=1 \linewidth]{Documento/Imagenes/Análisis/Transceptores/zigbee.jpg}}
&
\shortstack{\\ \includegraphics[width=1 \linewidth]{Documento/Imagenes/Análisis/Transceptores/HC-05.jpg}}
&
\shortstack{\\ \includegraphics[width=1 \linewidth]{Documento/Imagenes/Análisis/Transceptores/Lorawan.jpg}}
\\ \hline
\end{tabular}
}
\end{table}


Como solución óptima para la transmisión bidireccional de los datos, en este caso, datos de los sensores e imagenes captadas en entornos fluviales, se propone el módulo FGH100M (Wi-Fi HaLow) como transceptor preliminar. Esta selección se fundamenta por su  exelente equilibrio entre alcance efectivo (1 km en condiciones ideales) y consumo energético moderado (54 mA en TX/43 mA en RX), características críticas para la operación prolongada en topologías lineales con nodos distribuidos. Su operación en banda sub-1 GHz (902-928 MHz) proporciona mayor penetración en vegetación ribereña comparado con soluciones en 2.4 GHz, manteniendo compatibilidad con el estándar IEEE 802.11ah para gestión adaptativa de energía mediante mecanismos TIM/RAW. Adicionalmente, soporta tasas de transferencia suficientes (32.5 Mbps) para la transmisión eficiente de datos sensoriales e imágenes comprimidas, cumpliendo con los requisitos de bidireccionalidad para recepción de comandos desde el servidor central.
\end{comment}

%%%%%%%%%%%%%%%%%%%%%%%%%%%%%%%%%%%%%%%%%%%%%%%%%%%%
%             SECCIÓN: Cámaras                    %
%%%%%%%%%%%%%%%%%%%%%%%%%%%%%%%%%%%%%%%%%%%%%%%%%%%%
\section{Análisis de cámaras}

Para la selección de la cámara que formará parte del sistema de visión artificial, se evaluaron cuatro módulos de uso frecuente en plataformas embebidas: la OV2640, la OV5640, la Raspberry Pi Camera Module v2 y la Raspberry Pi Camera Module v3. La comparación se realizó con base en criterios como resolución, tipo de interfaz, compatibilidad con microcontroladores y computadoras de placa reducida, consumo energético, facilidad de integración y costo. Estos factores son determinantes para garantizar que el sistema sea eficiente en la captura de imágenes, considerando tanto la calidad visual necesaria para la detección de residuos sólidos flotantes en cuerpos de agua como las limitaciones energéticas y de procesamiento propias de una red de sensores. Además, la elección final también dependerá de la correspondencia entre las características de la cámara y la calidad de las imágenes disponibles en los conjuntos de datos utilizados para el entrenamiento del modelo de visión artificial, ya que la coherencia entre ambos elementos influye directamente en el desempeño del sistema.
%Para la selección de la cámara que formará parte del sistema de visión artificial, se evaluaron dos módulos ampliamente utilizados en plataformas embebidas: la OV2640 y la OV5640. La comparación se basó en criterios como resolución, tipo de interfaz, compatibilidad con microcontroladores, consumo energético, facilidad de integración y costo. Estos factores son determinantes para garantizar que el sistema sea eficiente en la captura de imágenes para la detección de residuos sólidos flotantes en cuerpos de agua.


\renewcommand{\arraystretch}{1.5}
\begin{longtable}{
    |p{2.9cm}
    |p{2.9cm}
    |p{2.9cm}
    |p{2.9cm}
    |p{2.9cm}|
}
\caption{Comparativa de cámaras para el sistema de visión artificial} 
\label{tab:cam_comparativa} \\
\hline
\textbf{Característica} & \textbf{OV2640} \cite{omnivision2006_OV2640} 
                        & \textbf{OV5640} \cite{omnivision2011_OV5640}
                        & \textbf{Rb Pi Camera Module v2} \cite{raspberrypi_camera2025}
                        & \textbf{Rb Pi Camera Module 3} \cite{raspberrypi_camera2025} \\
\hline
\endfirsthead

\hline
\textbf{Característica} & \textbf{OV2640} \cite{omnivision2006_OV2640} 
                        & \textbf{OV5640} \cite{omnivision2011_OV5640}
                        & \textbf{Rb Pi Camera Module v2} \cite{raspberrypi_camera2025}
                        & \textbf{Rb Pi Camera Module 3} \cite{raspberrypi_camera2025} \\

\hline
\endhead

\hline
\multicolumn{5}{r}{\textit{Continúa en la siguiente página}} \\
\endfoot

\hline
\endlastfoot

Resolución máxima 
    & 1600 × 1200 (UXGA) 
    & 2592 × 1944 (QSXGA) 
    & 3280 × 2464 (8 MP) 
    & 4608 × 2592 (12 MP) \\ \hline

Interfaz 
    & DVP / MIPI / I2C  
    & DVP / MIPI / I2C 
    & MIPI CSI-2 (2 lanes) 
    & MIPI CSI-2 (2 lanes) \\ \hline

Voltaje de operación 
    & 1.2 V (núcleo) / 2.5--3.0 V (E/S) 
    & 1.5 V (núcleo), 2.6 / 3.0 V (E/S) 
    & 3.3 V (a través del conector de cámara) 
    & 3.3 V (a través del conector de cámara) \\ \hline

Consumo activo 
    & 117 mA 
    & 140 mA
    & ~200 mA  
    & ~250 mA  \\ \hline

Consumo en reposo 
    & 620\,\unit{\micro\ampere} 
    & 20\,\unit{\micro\ampere} 
    & $<$ 1 mA 
    & $<$ 1 mA \\ \hline

Formato de salida 
    & 8-/10-bit RGB RAW 
    & 8-/10-bit RGB RAW 
    & 10-bit RAW 
    & 12-bit RAW, JPEG, YUV420, RGB \\ \hline

Velocidad máxima de transferencia de imágenes 
    & \begin{itemize}[leftmargin=*]
        \item UXGA: 15 fps 
        \item SXGA: 15 fps 
        \item SVGA: 30 fps 
        \item CIF: 60 fps
      \end{itemize}
    & \begin{itemize}[leftmargin=*]
        \item QSXGA (2592×1944): 15 fps 
        \item 1080p: 30 fps 
        \item 1280×960: 45 fps 
        \item 720p: 60 fps 
        \item VGA (640×480): 90 fps 
        \item QVGA (320×240): 120 fps
      \end{itemize} 
    & \begin{itemize}[leftmargin=*]
        \item 3280×2464: 15 fps
        \item 1080p: 30 fps 
        \item 720p: 60 fps 
        \item VGA: 90 fps
     \end{itemize}
    & \begin{itemize}[leftmargin=*]
        \item 4608×2592: 30 fps 
        \item 2304×1296: 60 fps 
        \item 1536×864: 120 fps
     \end{itemize}\\ \hline

Compatibilidad 
    & ESP32, STM32, Arduino, FPGA, Raspberry Pi 
    & ESP32, STM32, Arduino, FPGA, Raspberry Pi 
    & Raspberry Pi (todas las versiones con conector CSI) 
    & Raspberry Pi (todas las versiones con conector CSI) \\ \hline

Costo aproximado 
    & \$5--10 USD 
    & \$12--18 USD
    & \$60 USD 
    & \$64 USD \\ \hline

Ventajas destacadas 
    & Muy bajo consumo, compatibilidad con sistemas embebidos 
    & Alta resolución y mayor calidad de imagen 
    & Buena calidad, soporte oficial Raspberry Pi 
    & Autoenfoque, HDR, excelente calidad en baja luz \\ \hline

Imagen 
    & \shortstack{\\ \includegraphics[width=0.6\linewidth]{Documento/Imagenes/Análisis/Cámaras/Camara-OV2640.jpg}} 
    & \shortstack{\\ \includegraphics[width=0.6\linewidth]{Documento/Imagenes/Análisis/Cámaras/Camara-OV5640.jpg}} 
    & \shortstack{\\ \includegraphics[width=0.6\linewidth]{Documento/Imagenes/Análisis/Cámaras/module2.png}} 
    & \shortstack{\\ \includegraphics[width=0.6\linewidth]{Documento/Imagenes/Análisis/Cámaras/module3.jpg}} \\ \hline

\end{longtable}


%Ambas cámaras, OV2640 y OV5640, ofrecen características valiosas dependiendo de los requerimientos del sistema. La OV2640 es reconocida por su bajo consumo energético y facilidad de integración con microcontroladores de recursos limitados, lo que la hace adecuada para aplicaciones donde la eficiencia energética es prioritaria. La OV5640 mejora la resolución y la calidad de imagen, pero implica mayor consumo y complejidad. No obstante, debido a la necesidad de capturar imágenes de alta calidad para el análisis visual de residuos flotantes, se optó por la Raspberry Pi Camera Module 3. Esta cámara proporciona imágenes en alta resolución, mejor rendimiento en condiciones de baja iluminación y soporte para autoenfoque, lo que la convierte en la opción más adecuada para el sistema propuesto, a pesar de su mayor consumo energético.



%%%%%%%%%%%%%%%%%%%%%%%%%%%%%%%%%%%%%%%%%%%%%%%%%%%%
%             SECCIÓN: Microcontroladores          %
%%%%%%%%%%%%%%%%%%%%%%%%%%%%%%%%%%%%%%%%%%%%%%%%%%%%


\section{Análisis de microcontroladores} 
\label{sec:analisis_microcontroladores}   
La selección del microcontrolador constituye una decisión crítica en el diseño del sistema, determinando directamente las capacidades de integración sensorial, gestión energética, conectividad inalámbrica y procesamiento local de datos. Considerando los requerimientos específicos del proyecto, se llevó a cabo un análisis comparativo entre diversas placas de desarrollo ampliamente adoptadas en sistemas embebidos. Este estudio evaluó parámetros clave como arquitectura, capacidad de memoria, interfaces de comunicación, opciones de conectividad inalámbrica y costo. El objetivo fundamental es identificar la solución que optimice el equilibrio entre rendimiento computacional, eficiencia energética y viabilidad de implementación. A continuación, se presenta la comparativa técnica de las alternativas evaluadas.  

\renewcommand{\arraystretch}{1.5}
\small
\begin{longtable}{|p{3cm}|p{4cm}|p{4cm}|p{4cm}|}
\caption{Comparativa transpuesta de microcontroladores para el sistema de monitoreo}
\label{tab:comparativa_microcontroladores_transpuesta} \\
\hline
\multicolumn{4}{|c|}{Parte 1 Microcontroladores}\\
\hline
\textbf{Característica} 
& \textbf{LilyGO T-HaLow (ESP32-S3 N16R8)} 
& \textbf{ESP8266 NodeMCU} 
& \textbf{Raspberry Pi Pico} \\
\hline
\endfirsthead

\hline
\textbf{Característica} 
& \textbf{LilyGO T-HaLow (ESP32-S3 N16R8)} 
& \textbf{ESP8266 NodeMCU} 
& \textbf{Raspberry Pi Pico} \\
\hline
\endhead

\hline
\multicolumn{4}{r}{\textit{Continúa en la siguiente página}} \\
\endfoot

\hline
\endlastfoot

Arquitectura 
& Xtensa dual 32-bit LX7 
& 32-bit RISC 
& ARM Cortex \\ \hline

Procesamiento 
& 240 MHz 
& 160 MHz 
& 133 MHz \\ \hline

RAM 
& 520KB + 16MB PSRAM 
& 160KB SRAM 
& 264KB SRAM \\ \hline

Flash 
& 16MB 
& 4MB 
& 2MB \\ \hline

Consumo Activo 
& 91 mA& 80 mA & 50 mA \\
Consumo Sleep & 240 \unit{\uA} & 20 \unit{\uA} & 390 \unit{\uA} \\ \hline

Voltaje &3.3V & 3.3V & 3.3V \\ \hline

WiFi 
& 2.4GHz 
& 2.4GHz 
& No \\ \hline

Bluetooth 
& Classic + BLE 
& No 
& No \\ \hline

Sistema operativo (OS) 
& FreeRTOS 
& FreeRTOS 
& Bare-metal \\ \hline

Costo (USD) 
& \$20–30 
& \$3–6 
& \$10 \\ \hline

Periféricos 
& \shortstack[l]{\\• 36 GPIOs\\• ADC (20 canales)\\• UART\\• SPI}
& \shortstack[l]{\\• 17 GPIOs\\• ADC (1 canal)\\• UART\\• SPI}
& \shortstack[l]{\\• 23 GPIOs\\• ADC (3 canales)\\• 2×UART\\• SPI} \\ \hline

Ventajas 
& \shortstack[l]{\\• RAM extensa (16MB)\\• Alto rendimiento\\• Multiperiféricos\\• Soporte RTOS}
& \shortstack[l]{\\• Bajo costo\\• Suficiente para\\ aplicaciones básicas}
& \shortstack[l]{\\• Dual-core\\• Buen balance\\ precio/rendimiento} \\ \hline

Desventajas 
& \shortstack[l]{\\• Costo elevado\\• Complejidad de \\desarrollo\\• Solo 1 modo de \\operación (AP o STA)}
& \shortstack[l]{\\• RAM limitada\\• Un solo canal ADC}
& \shortstack[l]{\\• Sin PSRAM\\• Canales ADC\\ insuficientes} \\ \hline

Imagen 
& 
    \shortstack{\\ \includegraphics[width=1 \linewidth]{Documento/Imagenes/Análisis/Microcontroladores/lilygo-T-Halow.png}}
&
    \includegraphics[width=1 \linewidth]{Documento/Imagenes/Análisis/Microcontroladores/NodeMCU.png}
&  
    \includegraphics[width=0.9 \linewidth]{Documento/Imagenes/Análisis/Microcontroladores/RB Pico.jpg}   
\\ \hline

\end{longtable}




\begin{longtable}{|p{3cm}|p{4cm}|p{4cm}|p{4cm}|}
\hline
\multicolumn{4}{|c|}{Parte 2 Microcontroladores}\\
\hline
\textbf{Característica} 
& \textbf{Xiao ESP32-S3 R8} 
& \textbf{STM32F103} 
& \textbf{ATmega328P} \\
\hline
\endfirsthead

\hline
\textbf{Característica} 
& \textbf{Xiao ESP32-S3 R8} 
& \textbf{STM32F103} 
& \textbf{ATmega328P} \\
\hline
\endhead

\hline
\multicolumn{4}{r}{\textit{Continúa en la siguiente página}} \\
\endfoot

\hline
\endlastfoot

Arquitectura 
& Xtensa dual 32-bit LX7 
& ARM 32-bit M3 
& RISC 8-bit \\ \hline

Procesamiento 
& 240 MHz 
& 72 MHz 
& 16 MHz \\ \hline

RAM 
& 512KB + 8MB PSRAM 
& 64KB SRAM 
& 2KB SRAM \\ \hline

Flash 
& 8MB 
& 256–512KB 
& 32KB \\ \hline

Consumo Activo 
& 91 mA& 66 mA & 14 mA \\
Consumo Sleep & 240 \unit{\uA} & 8.5 mA & 60 \unit{\uA} \\ \hline

Voltaje & 3.3V & 5V & 5V \\ \hline

WiFi 
& 2.4GHz 
& No 
& No \\ \hline

Bluetooth 
& BLE 5 
& No 
& No \\ \hline

Sistema operativo (OS) 
& FreeRTOS 
& Bare-metal 
& Bare-metal \\ \hline

Costo (USD) 
& \$10–15 
& \$5 
& \$3 \\ \hline

Periféricos 
& \shortstack[l]{\\• 11 GPIOs\\• ADC (8 canales)\\• UART\\• SPI}
& \shortstack[l]{\\• 51 GPIOs\\• ADC (3 canales)\\• 5×UART\\• SPI}
& \shortstack[l]{\\• 23 GPIOs\\• ADC (8 canales)\\• USART\\• SPI} \\ \hline

Ventajas 
& \shortstack[l]{\\• RAM PSRAM (8MB)\\• Alto rendimiento\\• Bajo consumo \\• Compatible con \\tarjeta de expansión \\con cámara }
& \shortstack[l]{\\• Múltiples UARTs\\• Amplios GPIOs\\• Bajo costo}
& \shortstack[l]{\\• Amplia comunidad\\• Bajo consumo\\• Fácil prototipado} \\ \hline

Desventajas 
& \shortstack[l]{\\• GPIOs limitados\\• Costo moderado\\• Canales ADC\\ insuficientes}
& \shortstack[l]{\\• RAM limitada\\• Sin PSRAM\\• Procesamiento\\moderado}
& \shortstack[l]{\\• RAM mínima\\• Arquitectura 8-bit \\limitada} \\ \hline

Imagen 
& 
    \shortstack{\\ \includegraphics[width=1 \linewidth]{Documento/Imagenes/Análisis/Microcontroladores/xiao-esp32-s3-sense.jpg}}
&
    \includegraphics[width=1 \linewidth]{Documento/Imagenes/Análisis/Microcontroladores/STM32f103.jpg}
&  
    \includegraphics[width=0.9 \linewidth]{Documento/Imagenes/Análisis/Microcontroladores/Atmega328p.jpg}   
\\ \hline

\end{longtable}

\begin{longtable}{|p{3cm}|p{4cm}|p{4cm}|}
\hline
\multicolumn{3}{|c|}{Parte 3 Microcontroladores}\\
\hline
\textbf{Característica} 
& \textbf{Raspberry Pi Zero} 
& \textbf{Raspberry Pi Zero 2 W} 
 \\
\hline
\endfirsthead

\hline
\textbf{Característica} 
& \textbf{Raspberry Pi Zero} 
& \textbf{Raspberry Pi Zero 2 W} 
 \\
\hline
\endhead

\hline
\multicolumn{3}{r}{\textit{Continúa en la siguiente página}} \\
\endfoot

\hline
\endlastfoot

Arquitectura 
& ARM1176JZF-S 
& ARM Cortex-A53 (quad-core) 
 \\ \hline

Procesamiento 
& 1.0 GHz (single-core) 
& 1.0 GHz (quad-core) 
 \\ \hline

RAM 
& 512MB LPDDR2 
& 512MB LPDDR2 
 \\ \hline

Flash 
& microSD externa 
& microSD externa 
 \\ \hline

Consumo Activo 
&  2.5A
&  2.5A
 \\ \hline

Voltaje & 5V (microUSB) & 5V (microUSB)  \\ \hline

Broadcom chip & BCM2835 Single-core & RP3A0 Quad-core  \\ \hline

WiFi 
& Solo en modelo W: 2.4GHz (802.11n) 
& 2.4GHz (802.11n) 
 \\ \hline

Bluetooth 
& BLE 4.0 
& BLE 4.2 
 \\ \hline

Sistema operativo (OS) 
& Raspberry Pi OS (Linux) 
& Raspberry Pi OS (Linux) 
 \\ \hline

Costo (USD) 
& \$10 
& \$15 
 \\ \hline

Periféricos 
& \shortstack[l]{\\• 40 GPIOs\\• UART\\• SPI\\• I2C}
& \shortstack[l]{\\• 40 GPIOs\\• UART\\• SPI\\• I2C}
 \\ \hline

Ventajas 
& \shortstack[l]{\\• Bajo costo\\• Compatible con \\ecosistema Linux\\• Buena conectividad\\• Compacto}
& \shortstack[l]{\\• Multiprocesamiento \\(4 cores)\\• Más rápido que Zero\\• Ideal para visión\\ artificial}
 \\ \hline

Desventajas 
& \shortstack[l]{\\• Procesador antiguo\\• Capacidad de \\procesamiento limitada\\• Mayor consumo que \\MCUs}
& \shortstack[l]{\\• Sin GPIO analógicos\\• Requiere más energía \\que microcontroladores \\comunes}
 \\ \hline

Imagen 
& 
    \shortstack{\\ \includegraphics[width=0.9\linewidth]{Documento/Imagenes/Análisis/Microcontroladores/zerow.jpg}}
&
    \includegraphics[width=0.9\linewidth]{Documento/Imagenes/Análisis/Microcontroladores/zero2w.jpg}

\\ \hline

\end{longtable}
 

\begin{table}[H]
\centering
\caption{Comparativa de microcontroladores para nodos fluviales}
\resizebox{\textwidth}{!}{
\begin{tabular}{|l|c|c|c|c|c|}
\hline
\textbf{MCU} & \textbf{Periféricos} & \textbf{Memoria} & \textbf{Consumo Activo} & \textbf{Consumo Sleep} & \textbf{Conectividad} \\ \hline
\shortstack[l]{\textbf{Heltec Wireless}\\\textbf{Tracker}} & \textbf{ADC, SPI, I2C, UART} & \textbf{8MB Flash} & \textbf{\SI{\sim 110}{\milli\ampere}} & \textbf{\SI{\sim 20}{\micro\ampere}} & \shortstack[c]{\\ \textbf{LoRa + GNSS +}  \\\textbf{Wi-Fi + BLE}} \\ \hline
Xiao ESP32-S3 & 11 GPIOs, 8 ADC & 8MB Flash & 91 mA & 240 µA & WiFi 2.4GHz + BLE \\ \hline
LilyGO T-HaLow & 36 GPIOs, 20 ADC & 16MB Flash & 91 mA & 240 µA & WiFi HaLow (900MHz) \\ \hline
RPi Pico W & 23 GPIOs, 3 ADC & 2MB Flash & 50 mA & 390 µA & WiFi 2.4GHz \\ \hline
ESP8266 NodeMCU & 17 GPIOs, 1 ADC & 4MB Flash & 80 mA & 20 µA & WiFi 2.4GHz \\ \hline
RPi Zero W & 40 GPIOs (sin ADC) & Micro SD & 2.5 A & - & WiFi 2.4GHz + BLE 4.0 \\ \hline
RPi Zero 2 W & 40 GPIOs (sin ADC) & Micro SD & 2.5 A & - & WiFi 2.4GHz + BLE 4.2 \\ \hline
\end{tabular}
}
\end{table}

Para el nodo sensor, encargado de capturar datos de sensores (pH, turbidez, oxígeno disuelto), adquirir imágenes y gestionar la comunicación de largo alcance, se selecciona la placa de desarrollo \textbf{Heltec Wireless Tracker V1.1}. Debido a la superioridad en la integración de hardware que ofrece.

Al integrar en un solo PCB el microcontrolador \textbf{ESP32-S3}, el transceptor \textbf{LoRa SX1262} y el módulo \textbf{GNSS UC6580}, se simplifica drásticamente el diseño físico del nodo y se eliminan puntos de falla asociados al cableado de módulos externos, lo cual es crítico para la robustez de un dispositivo flotante. Además, su consumo en modo de reposo es extremadamente bajo (\SI{\sim 20}{\micro\ampere}), optimizando la autonomía de la batería. Su capacidad de procesamiento es suficiente para gestionar la lectura de sensores y la compresión de imágenes, mientras que su radio LoRa integrada garantiza la conectividad en el entorno fluvial.

\begin{comment}
\subsection{Comparativa de microcontroladores PIC para control de sensores}
\label{subsec:comparativa_pic}

Dado que el Xiao ESP32-S3 Sense presenta un número limitado de pines GPIO disponibles (especialmente al estar asignados a la cámara OV5640 y al módulo Wi-Fi HaLow), se decidió incorporar un microcontrolador auxiliar dedicado exclusivamente a la gestión de sensores. Este microcontrolador debe ser de bajo consumo, tamaño reducido, fácil de programar y con recursos suficientes para:

\begin{itemize}
    \item Controlar encendido y apagado de sensores vía demultiplexor (74HC238).
    \item Leer señales analógicas mediante entradas ADC.
    \item Comunicarse con la Xiao ESP32.
    \item Mantener una operación eficiente en modo \textit{sleep}.
\end{itemize}

Con estos criterios, se evaluaron tres microcontroladores PIC de arquitectura de 8 bits con tecnología \textit{nanoWatt XLP}, comparados en la Tabla~\ref{tab:comparativa_pic}.

\begin{table}[H]
\centering
\caption{Comparativa de microcontroladores PIC para nodo sensor}
\label{tab:comparativa_pic}
\renewcommand{\arraystretch}{1.4}
\begin{tabular}{@{}|l|c|c|c|@{}}
\hline
\textbf{Característica} & \textbf{PIC12LF1822} \cite{microchip2020_PIC12LF1822_16LF1823}
                        & \textbf{PIC16LF1823} \cite{microchip2020_PIC12LF1822_16LF1823}
                        & \textbf{PIC16LF1847} \cite{microchip2013_PIC16LF1847}\\
\hline
Arquitectura         & 8-bit & 8-bit & 8-bit \\ \hline
GPIO disponibles     & 6     & 12    & 16 \\ \hline
Entradas ADC         & 4     & 8     & 12 \\ \hline
Módulo UART (EUSART) & Sí    & Sí    & Sí \\ \hline
Módulo SPI & Sí    & Sí    & Sí \\ \hline
Módulo I2C & Sí    & Sí    & Sí \\ \hline
Frecuencia máx. (MHz)& 32    & 32    & 32 \\ \hline
Consumo \textit{Sleep} & 20 nA & 20 nA & 20 nA \\ \hline
Consumo activo (@1MHz) & 30~\unit{uA} & 30~\unit{uA} & 65~\unit{uA} \\ \hline
Memoria Flash (Bytes) & 2048  & 2048  & 8192 \\ \hline
RAM (Bytes)           & 128   & 128   & 1024 \\ \hline
Tamaño encapsulado    & 8 pines & 14 pines & 28 pines \\ \hline
Precio aproximado (USD) & \$3 & \$3 & \$3 \\ \hline
\end{tabular}
\end{table}


El modelo seleccionado para el diseño del nodo es el \textbf{PIC16LF1823}, ya que representa el punto óptimo entre capacidad, eficiencia y costo:

\begin{itemize}
    \item Dispone de 11 pines GPIO, suficientes para controlar el DEMUX (3 líneas), adquirir hasta 8 señales analógicas y mantener UART dedicado.
    \item Integra un módulo EUSART que permite comunicación estable con el Xiao ESP32.
    \item Su bajo consumo en modo \textit{sleep} (20nA) y modo activo (50\unit{uA}/MHz) lo hacen ideal para sistemas alimentados por batería.
    \item Su encapsulado de 14 pines facilita el montaje en placas compactas sin sacrificar funcionalidad.
    \item Presenta compatibilidad con herramientas estándar de Microchip (MPLAB X, XC8).
\end{itemize}

En contraste, el PIC12LF1822 resulta limitado en pines y entradas ADC, mientras que el PIC16LF1847 ofrece mayores recursos a costa de un tamaño físico mayor, mayor consumo y un costo menos competitivo, innecesario para esta aplicación específica.

\end{comment}


%%%%%%%%%%%%%%%%%%%%%%%%%%%%%%%%%%%%%%%%%%%%%%%%%%%%
%             SECCIÓN: Alimentación                %
%%%%%%%%%%%%%%%%%%%%%%%%%%%%%%%%%%%%%%%%%%%%%%%%%%%%


\section{Análisis de la Alimentación}
\label{sec:analisis_alimentacion}

El sistema de monitoreo propuesto está diseñado para operar de forma autónoma en entornos naturales remotos, donde no se dispone de infraestructura eléctrica. Por ello, el subsistema de alimentación eléctrica representa un componente crítico para garantizar la continuidad operativa. La solución adoptada se basa en el uso de baterías recargables de ion-litio, complementadas con paneles solares y un módulo de gestión de carga, asegurando la autosuficiencia energética de los nodos sensores distribuidos.

\subsection{Requerimientos de Alimentación de los Componentes}
\label{subsec:reqs_alimentacion}

La Tabla~\ref{tab:componentes_energia} presenta los principales componentes seleccionados para el nodo sensor (ver secciones \ref{sec:analisis_microcontroladores} y \ref{sec:analisis_sensores}), junto con sus requerimientos estimados de voltaje y corriente de operación, extraídos de sus respectivas hojas de datos. Estos valores son fundamentales para dimensionar el sistema de alimentación.

%\renewcommand{\arraystretch}{1.4}
\begin{longtable}{|p{5cm}|c|c|}
\caption{Consumo eléctrico de componentes}
\label{tab:componentes_energia} \\
\hline
\textbf{Componente} & \textbf{Voltaje} & \textbf{Corriente} \\
\hline
\endfirsthead

\hline
\textbf{Componente} & \textbf{Voltaje} & \textbf{Corriente} \\
\hline
\endhead

\hline
\multicolumn{3}{r}{\textit{Continúa en la siguiente página}} \\
\endfoot

\hline
\endlastfoot

\multicolumn{3}{|c|}{\textbf{Transceptor}} \\
\hline
\multirow{6}{*}{Quectel FGH100MHAAMD}& \multirow{6}{*}{3.3V} & \textbf{Modo Tx}\\
     &  & 54 \unit{\mA} \\ \cline{3-3}
     &  & \textbf{Modo RX} \\
     &  & 43 \unit{\mA} \\ \cline{3-3}
     &  & \textbf{Modo Sleep} \\
     &  & $<10$ \unit{\uA} \\
\hline

\multicolumn{3}{|c|}{\textbf{Microcontrolador}} \\
\hline
\multirow{4}{*}{ESP32-S3R8} & \multirow{4}{*}{3.3V} & \textbf{Modo activo} \\
   &  & 91 \unit{\mA} \\ \cline{3-3}
   &  & \textbf{Modo Sleep} \\
   &  & 240 \unit{\uA} \\
\hline

\multicolumn{3}{|c|}{\textbf{Sensores}} \\
\hline
Sensor de pH (Gravity V2)                & 5V         & 10 \unit{\mA} \\
Sensor de oxígeno disuelto               & 5V         & 10–50 \unit{\mA} \\
Sensor de turbidez                       & 5V         & 20–30 \unit{\mA} \\
Sensor de conductividad eléctrica        & 5V         & 30–40 \unit{\mA} \\
Sensor de temperatura (DS18B20)          & 3.3 - 5V   & 4 \unit{\mA} \\
\hline

\multicolumn{3}{|c|}{\textbf{Cámara}} \\
\hline
\multirow{4}{*}{Cámara OV2640}           & \multirow{4}{*}{3.3 - 5V} & \textbf{Active}  \\
& & 140 \unit{\mA}\\
                                         &                           & \textbf{Standby}  \\
& & 20 \unit{\uA} \\                                        
\hline
\end{longtable}


\begin{table}[H]
\centering
\caption{Consumo eléctrico estimado de componentes del nodo sensor (Plataforma Heltec).}
\label{tab:componentes_energia}
\renewcommand{\arraystretch}{1.2}
\begin{tabular}{l l S[table-format=1.1] S[table-format=3.3] l}
\toprule
\textbf{Componente} & \textbf{Modo} & {\textbf{Voltaje}} & {\textbf{Corriente}} & \textbf{Ref.} \\
 & & {(\unit{\volt})} & {(\unit{\milli\ampere})} & \\
\midrule
\multirow{3}{*}{Transceptor LoRa (SX1262)} & Tx (@ \SI{+22}{dBm}) & 3.3 & 118.0 & \cite{semtech_sx1261_2017} \\ 
 & Rx (Boosted) & 3.3 & 4.6 & \cite{semtech_sx1261_2017} \\
 & Sleep & 3.3 & <0.001 & \cite{semtech_sx1261_2017} \\
\midrule
\multirow{2}{*}{MCU ESP32-S3 (Heltec)} & Activo (Gestión) & 3.3 & 80.0 & \cite{heltec_wireless_tracker_2023} \\
 & Deep Sleep & 3.3 & 0.020 & \cite{heltec_wireless_tracker_2023} \\
\midrule
\multirow{3}{*}{GNSS (UC6580)} & Activo (Acq.) & 3.3 & 40.0 & \cite{unicore_uc6580_2023} \\ 
 & Activo (Track) & 3.3 & 40.0 & \cite{unicore_uc6580_2023} \\
 & Backup & 3.3 & 0.005 & \cite{unicore_uc6580_2023} \\
\midrule
Sensor pH (Industrial V2) & Activo & 5.0 & 10.0 & \cite{DFRobot_pH_Sensor} \\ 
Sensor OD (SEN0237) & Activo & 5.0 & 40.0 & \cite{DFRobot_DO_Sensor} \\
Sensor Turbidez (SEN0189) & Activo & 5.0 & 30.0 & \cite{DFRobot_Turbidity_Sensor} \\ 
Sensor EC (DFR0300) & Activo & 5.0 & 40.0 & \cite{DFRobot_EC_Sensor} \\
Sensor Temp (DS18B20) & Activo & 3.3 & 4.0 & \cite{DFRobot_DS18B20} \\ 
\midrule
\multirow{2}{*}{Cámara OV5640} & Activo & 3.3 & 140.0 & \cite{omnivision2011_OV5640} \\ 
 & Standby & 3.3 & 0.020 & \cite{omnivision2011_OV5640} \\
\bottomrule
\end{tabular}%
\end{table}

\begin{comment}
\begin{table}[H]
\centering
\caption{Consumo eléctrico estimado de componentes del nodo sensor.}
\label{tab:componentes_energia}
\renewcommand{\arraystretch}{1.2}
\begin{tabular}{l l S[table-format=1.1] S[table-format=3.3] l}
\toprule
\textbf{Componente} & \textbf{Modo} & {\textbf{Voltaje}} & {\textbf{Corriente}} & \textbf{Ref.} \\
 & & {(\unit{\volt})} & {(\unit{\milli\ampere})} & \\
\midrule
\multirow{3}{*}{Transceptor FGH100M} & Tx & 3.3 & 54 & \cite{QuectelFGH100M} \\ 
 & Rx & 3.3 & 43 & \cite{QuectelFGH100M} \\
 & Sleep & 3.3 & <0.010 & \cite{QuectelFGH100M} \\
\midrule
\multirow{2}{*}{MCU Xiao ESP32-S3} & Activo & 3.3 & 91 & \cite{espressif2023_ESP32-S3} \\ 
 & Sleep & 3.3 & 0.240 & \cite{espressif2023_ESP32-S3} \\
\midrule
MCU PIC16LF1823 & Activo (@ \SI{1}{\mega\hertz}) & 3.3 & 0.050 & \cite{microchip2020_PIC12LF1822_16LF1823} \\ 
 & Sleep & 3.3 & <0.001 & \cite{microchip2020_PIC12LF1822_16LF1823} \\
\midrule
Sensor pH (Industrial V2) & Activo & 5.0 & 10 & \cite{DFRobot_pH_Sensor} \\ 
Sensor OD (SEN0237) & Activo & 5.0 & 40 & \cite{DFRobot_DO_Sensor} \\
Sensor Turbidez (SEN0189) & Activo & 5.0 & 30 & \cite{DFRobot_Turbidity_Sensor} \\ 
Sensor EC (DFR0300) & Activo & 5.0 & 40 & \cite{DFRobot_EC_Sensor} \\
Sensor Temp (DS18B20) & Activo & 3.3 & 4 & \cite{DFRobot_DS18B20} \\ 
\midrule
\multirow{2}{*}{Cámara OV5640} & Activo & 3.3 & 140 & \cite{omnivision2011_OV5640} \\ 
 & Standby & 3.3 & 0.020 & \cite{omnivision2011_OV5640} \\
\bottomrule
\end{tabular}
\end{table}
\end{comment}



Para estimar de forma realista el consumo energético del sistema, es indispensable cuantificar la duración efectiva de las fases de transmisión. Este análisis utiliza los parámetros de rendimiento del transceptor seleccionados en la Sección \ref{sec:analisis_tecnologias}, diferenciando entre el envío rutinario de datos de sensores y la transmisión eventual de imágenes.
\subsection{Análisis de Tiempos de Transmisión y Carga de Red}
\label{subsec:analisis_tiempos}
El diseño de la red lineal P2P con modulación LoRa impone restricciones de tiempo que deben ser calculadas paso a paso para dimensionar la batería. A continuación, se detallan los cálculos realizados para determinar el volumen de datos a transmitir y el tiempo total de comunicación por nodo.

\subsubsection*{Volumen de Datos a Transmitir}
\label{ssubsec:volumen_datos}
Cada nodo sensor genera dos tipos de información con volúmenes drásticamente diferentes:

\begin{enumerate}
    \item \textbf{Muestra sensorial y Geolocalización (Paquete de Datos):} Compuesto por las lecturas de los sensores, la ubicación y metadatos básicos.
    \begin{itemize}
        \item 4 sensores analógicos (12 bits): $4 \times \SI{12}{\bit} = \SI{48}{\bit} = \SI{6}{\byte}$
        \item 1 sensor digital (DS18B20, 12 bits): \SI{2}{\byte}
        \item Datos GNSS (Latitud, Longitud): $2 \times \SI{32}{\bit} = \SI{8}{\byte}$
        \item Metadatos (ID, Timestamp, Batería): \SI{8}{\byte}
    \end{itemize}
    \[
    \text{Volumen}_{\text{sensores}} = 6 + 2 + 8 + 8 = \boxed{\SI{24}{\byte}}
    \]

    \item \textbf{Imagen Comprimida:} Capturada por la cámara OV5640. Se asume una resolución SVGA ($800 \times 600$) con compresión JPEG estándar, resultando en un tamaño típico de archivo de \SI{\approx 150}{\kilo\byte}.
    \[
    \text{Volumen}_{\text{imagen}} \approx \boxed{\SI{150}{\kilo\byte}}
    \]
\end{enumerate}


\subsubsection*{Tiempo Total de Comunicación por Nodo en Topología Lineal}
\label{ssubsec:tiempo_general_comm}

El protocolo de comunicación implementado utiliza la modulación LoRa en una topología de red lineal P2P (multisalto). Al operar en modo \textit{half-duplex} (el transceptor no puede transmitir y recibir simultáneamente), esto tiene un impacto acumulativo en el tiempo de actividad de los nodos que actúan como repetidores.

Consideremos una red lineal con $N$ nodos sensores (numerados del 1 al $N$), donde el Nodo 1 es el más alejado del concentrador y el Nodo $N$ es el más cercano. El concentrador se considera como el destino final después del Nodo $N$.

El tiempo base ($T_{base}$) para transmitir el volumen de datos ($V_n$) generado por un solo nodo en cada ciclo se calcula como:
\begin{equation} \label{eq:tiempo_base}
T_{base} = \frac{V_n \times \SI[per-mode = symbol]{8}{\bit\per\byte}}{R}
\end{equation}
Donde:
\begin{itemize}
    \item $T_{base}$ es el tiempo base de transmisión por nodo (en segundos).
    \item $V_n$ es el volumen de datos generado por un nodo por ciclo (en bytes).
    \item $R$ es la tasa de transmisión efectiva del enlace (en bits por segundo).
\end{itemize}

Un nodo genérico $n$ (donde $1 \le n \le N$) debe recibir los datos de los $n-1$ nodos anteriores y transmitir los $n$ paquetes acumulados (los $n-1$ recibidos más el suyo). Los tiempos de recepción ($T_{rx}$) y transmisión ($T_{tx}$) para el nodo $n$ son:
\begin{equation} \label{eq:tiempo_rx_n}
T_{rx}(n) = (n-1) \times T_{base}
\end{equation}
\begin{equation} \label{eq:tiempo_tx_n}
T_{tx}(n) = n \times T_{base}
\end{equation}
Donde:
\begin{itemize}
    \item $T_{rx}(n)$ es el tiempo total de recepción para el nodo $n$ (en segundos).
    \item $T_{tx}(n)$ es el tiempo total de transmisión para el nodo $n$ (en segundos).
    \item $n$ es el índice del nodo en la cadena lineal (1 a N).
    \item $T_{base}$ es el tiempo base de transmisión por nodo, ecuación \eqref{eq:tiempo_base}.
\end{itemize}

Por lo tanto, la fórmula general para el tiempo total de comunicación activa ($T_{comm}$) para un nodo $n$ en la cadena lineal es la suma de sus tiempos de recepción y transmisión, ecuaciones \eqref{eq:tiempo_rx_n} y \eqref{eq:tiempo_tx_n}:
\begin{equation} \label{eq:tiempo_comm_nodo_n}
T_{comm}(n) = T_{rx}(n) + T_{tx}(n) = ((n-1) + n) \times T_{base} = \boxed{(2n - 1) \times T_{base}}
\end{equation}
Donde:
\begin{itemize}
    \item $T_{comm}(n)$ es el tiempo total de comunicación activa para el nodo $n$ (en segundos).
    \item $n$ es el índice del nodo en la cadena lineal (1 a N).
    \item $T_{base}$ es el tiempo base de transmisión por nodo, ecuación \eqref{eq:tiempo_base}.
\end{itemize}


\subsubsection*{Tiempo Estimado de Transmisión (Time on Air)}
\label{ssubsec:tiempo_tx}

El tiempo de transmisión se calcula con base en la tasa de datos efectiva ($R_b$) del transceptor \textbf{SX1262}. Conforme al análisis realizado en la \textbf{Sección \ref{subsec:justificacion_lora}}, se utiliza la configuración de alta velocidad (SF7, BW \SI{500}{\kilo\hertz}, CR 4/5) para minimizar el tiempo de ocupación del canal, lo que proporciona una tasa de \SI[per-mode = symbol]{21.9}{\kilo\bit\per\second}.

Bajo estas condiciones, los tiempos de actividad del radio son:

\begin{itemize}
    \item \textbf{Tiempo para Datos de Sensores y GNSS:}
    \[
    T_{\text{base\_sen}} = \frac{\SI{24}{\byte}\times \SI[per-mode = symbol]{8}{\bit\per\byte}}{\SI[per-mode = symbol]{21900}{\bit\per\second}} \approx \textbf{\SI{8.76}{\milli\second}}
    \]
    \textit{Observación:} El impacto energético de enviar el paquete de datos estándar sigue siendo despreciable, ocupando el canal por menos de \SI{10}{\milli\second}.

    \item \textbf{Tiempo para una Imagen:}
    \[
    T_{\text{base\_img}} = \frac{ 150 \times \SI[per-mode = symbol]{1024}{\byte} \times \SI[per-mode = symbol]{8}{\bit\per\byte}}{\SI[per-mode = symbol]{21900}{\bit\per\second}} \approx \textbf{\SI{56.1}{\second}}
    \]
    \textit{Observación:} La transmisión de una sola imagen requiere mantener el transceptor activo durante casi un minuto (debido a la fragmentación y envío secuencial), lo cual representa el consumo energético dominante en los ciclos donde se requiera evidencia visual.
\end{itemize}

El tiempo base total de datos generado por un nodo en cada ciclo es la suma de ambos:
\begin{equation} \label{eq:tiempo_base_calculado}
\text{T}_{\text{base}} = \text{T}_{\text{base\_sen}} + \text{T}_{\text{base\_img}} \approx \SI{8.76}{\milli\second} + \SI{56.1}{\second} \approx \boxed{\SI{56.2}{\second}}
\end{equation}

\subsubsection*{Ejemplo de Cálculo para el Nodo 4 (N=4) - Ciclo Nominal:}
En este proyecto, se tienen $N=4$ nodos sensores. El nodo más cargado es el Nodo 4, ya que debe recibir y retransmitir los datos de los 3 nodos anteriores más los propios. Aplicando la Ecuación \eqref{eq:tiempo_comm_nodo_n} con $n=4$ y utilizando el tiempo base de transmisión calculado, ecuación \eqref{eq:tiempo_base_calculado}:

\begin{equation}\label{eq:tiempo_comm_nodo4}
    T_{comm}(4) = (2 \times 4 - 1) \times \SI{56.2}{\second} = 7 \times \SI{56.2}{\second} = \SI{393.4}{\second} = \boxed{\SI{6.5}{\minute}}
\end{equation}


Este valor representa el tiempo total que el Nodo 4 pasa activo comunicándose en cada ciclo de monitoreo. A diferencia de la transmisión de solo datos sensoriales, la inclusión de imágenes incrementa significativamente el tiempo de actividad del transceptor, lo cual representa una carga energética considerable que ha sido contemplada en el dimensionamiento del sistema de alimentación.


\begin{comment}
\paragraph{Ejemplo de Cálculo para el Nodo 4 (N=4) - Ciclo Nominal:}
En este proyecto, se tienen $N=4$ nodos sensores. El nodo más cargado es el Nodo 4. Aplicando la Ecuación \eqref{eq:tiempo_comm_nodo_n} con $n=4$ y utilizando el tiempo base de transmisión de sensores $T_{base} \approx \SI{8.8}{\milli\second}$ (calculado en la subsección \ref{ssubsec:tiempo_tx}):

\[
T_{comm}(4) = (2 \times 4 - 1) \times \SI{8.8}{\milli\second} = 7 \times \SI{8.8}{\milli\second} = \boxed{\SI{61.6}{\milli\second}}
\]

Este valor representa el tiempo total que el Nodo 4 pasa activo comunicándose en cada ciclo de monitoreo estándar. Es extremadamente bajo (\SI{< 0.1}{\second}), lo cual valida la eficiencia de la topología lineal para el tráfico de sensores.

\textit{Nota: En el caso eventual de transmitir una imagen ($T_{base} \approx \SI{56}{\second}$), el tiempo acumulado para el Nodo 4 sería de $7 \times 56 \approx \SI{392}{\second}$ (\SI{\sim 6.5}{\minute}), lo que confirma la necesidad de restringir el envío de imágenes a eventos bajo demanda.}
\end{comment}


\subsection{Escenario Operativo}
\label{subsec:escenario_operativo}

Para optimizar el consumo energético y garantizar una larga autonomía, el sistema opera en un ciclo de bajo ciclo de trabajo (\textit{low duty cycle}). Se establece un \textbf{ciclo operativo semanal} (\SI{168}{\hour}) como referencia para el cálculo de autonomía base, aunque la frecuencia de muestreo puede ajustarse según la disponibilidad de energía.

Dentro de cada semana (\SI{168}{\hour}), el nodo permanece en modo de ultra bajo consumo (\textit{Deep Sleep}) la gran mayor parte del tiempo, activándose una sola vez para ejecutar la secuencia de monitoreo. Considerando el escenario más exigente (Nodo 4 retransmitiendo datos e imágenes de toda la red), los tiempos estimados son (ver Figura \ref{fig:ciclo_mod_op}):

\begin{itemize}
    \item \textbf{Adquisición de Datos}:
        \begin{itemize}
            \item Lectura de Sensores: Se asume un tiempo de estabilización y lectura de \SI{5}{\second} por cada uno de los 5 sensores. Tiempo $= 5 \times \SI{5}{\second} = \SI{25}{\second}$.
            \item Captura de Imagen: Se estima un tiempo de \SI{10}{\second} para activar la cámara, capturar y comprimir la imagen.
            \item Obtención GNSS: Se estima un tiempo promedio de \SI{10}{\second} para actualizar las coordenadas de geolocalización.
        \end{itemize}
        Tiempo total de adquisición $\approx 25 + 10 + 10 = \textbf{\SI{45}{\second}}$.

    \item \textbf{Comunicación (Nodo 4)}: Tiempo total de Tx y Rx calculado previamente, considerando la transmisión de imágenes de los 4 nodos (ver Ecuación \ref{eq:tiempo_comm_nodo4}):
    \[ T_{comm}(4) \approx \SI{393.4}{\second} \approx \textbf{\SI{6.5}{\minute}} \]

    \item \textbf{Procesamiento Local}: Tiempo adicional para gestión del buffer, fragmentación de paquetes y lógica de control. Se estima conservadoramente en \SI{5}{\second}.

    \item \textbf{Tiempo Activo Total por Ciclo}: 
    \[ \SI{45}{\second} + \SI{393.4}{\second} + \SI{5}{\second} = \SI{443.4}{\second} \approx \textbf{\SI{7.4}{\minute}} \]

    \item \textbf{Tiempo en Modo Sleep por Semana}:
    \[ \SI{168}{\hour} - \SI{7.4}{\minute} \approx \textbf{\SI{167.88}{\hour}} \]
    Esto representa que el sistema permanece en reposo el \textbf{99.93\%} del tiempo, lo cual garantiza una eficiencia energética considerable.
\end{itemize}

\begin{figure}[H]
    \centering
    \includegraphics[width=0.5\linewidth]{Documento/Imagenes/Análisis/ciclo_modos_op.pdf}
    \caption{Ciclo simplificado de modos operativos del nodo sensor.}
    \label{fig:ciclo_mod_op}
\end{figure}


\subsection*{Cálculo del Consumo Energético Semanal}
\label{subsec:calculo_consumo}

Utilizando los tiempos de operación estimados para cada modo y las corrientes de consumo de la Tabla \ref{tab:componentes_energia}, se calcula el consumo energético total por ciclo semanal ($E_{\text{sem}}$) para el nodo más demandante (Nodo 4). La fórmula general es:

Dado que el ciclo ocurre una sola vez por semana, el cálculo se simplifica a la energía de un evento activo más la energía de reposo de toda la semana.

\begin{equation} \label{eq:energia_semanal}
E_{\text{sem}} = \sum_{\text{modo}} (I_{\text{modo}} \times t_{\text{modo}})
\end{equation}

Para la fase de comunicación del Nodo 4, se separan los tiempos y corrientes de recepción (Rx) y transmisión (Tx) calculados con la ecuación \eqref{eq:tiempo_rx_n} y \eqref{eq:tiempo_tx_n}:
\begin{itemize}
    \item Tiempo de Recepción ($T_{rx}(4)$) = \SI{168.6}{\second}
    \item Tiempo de Transmisión ($T_{tx}(4)$) = \SI{224.8}{\second}
\end{itemize}
Las corrientes totales durante estas fases son:
\begin{itemize}
    \item Corriente en Rx ($I_{Rx}$) = $I_{\text{ESP32\_activo}} + I_{\text{SX1262\_Rx}} = \SI{80}{\milli\ampere} + \SI{4.6}{\milli\ampere} = \SI{84.6}{\milli\ampere}$
    \item Corriente en Tx ($I_{Tx}$) = $I_{\text{ESP32\_activo}} + I_{\text{SX1262\_Tx}} = \SI{80}{\milli\ampere} + \SI{118}{\milli\ampere} = \SI{198}{\milli\ampere}$
\end{itemize}

La Tabla \ref{tab:consumo_detallado_semanal} detalla el cálculo completo.
\sisetup{exponent-mode=engineering}
\begin{table}[H]
\centering
\caption{Cálculo detallado del consumo energético semanal (Nodo 4, 1 ciclo/semana).}
\label{tab:consumo_detallado_semanal}
\renewcommand{\arraystretch}{1.3}
\resizebox{\textwidth}{!}{%
\begin{tabular}{l C{2cm} C{2.1cm} c c c}
\toprule
\textbf{Modo Operativo}&\multicolumn{2}{c}{\shortstack[c]{\textbf{Componentes}\\\textbf{activos}}}& \textbf{Corriente Total} & \textbf{Tiempo Semanal} & \textbf{Energía Semanal} \\
 &&& {(\unit{\milli\ampere})} & {(\unit{\hour})} & {(\unit{\milli\ampere\hour})} \\
\midrule
Lectura Sensores& \makecell[c]{ESP32 \\ 80 mA}& \makecell[c]{Sensor \\ 40 mA} & 120.0 & $25/3600 \approx \num{0.00694}$ & $\approx \num{0.8328}$ \\
Posición GNSS &\makecell[c]{ESP32 \\ 80 mA} & \makecell[c]{UC6580 \\ 40 mA} & 120.0 & $10/3600 \approx \num{0.00277}$ & $\approx \num{0.3324}$ \\
\midrule
Captura de Imagen &\makecell[c]{ESP32 \\ 80 mA} & \makecell[c]{Cam \\ 140 mA} & 220.0 & $10/3600 \approx \num{0.00277}$ & $\approx \num{0.6094}$ \\
Procesamiento Local&\makecell[c]{ESP32 \\ 80 mA}& -& 80.0 & $5/3600 \approx \num{0.00138}$ & $\approx \num{0.1104}$ \\
\midrule
Recepción &\makecell[c]{ESP32 \\ 80 mA} & \makecell[c]{SX1262\_Rx \\ 4.6 mA} & 84.6 & $168.6/3600 \approx \num{0.0468}$ & $\approx \num{3.959}$ \\
\midrule
Transmisión &\makecell[c]{ESP32 \\ 80 mA} & \makecell[c]{SX1262\_Tx \\ 118 mA} & 198.0 & $224.8/3600 \approx \num{0.0624}$ & $\approx \num{12.355}$ \\
\midrule
Modo Sleep &\multicolumn{2}{c}{Todo en sleep/standby}& 0.046 & $167.88$ & $\approx 7.722$ \\
\midrule
\textbf{Total Semanal}&& & & \textbf{168.0 h} & \textbf{$\approx \textbf{\SI{25.921}{\milli\ampere\hour}}$ }\\
\bottomrule
\end{tabular}
}
\end{table}

El análisis arroja un consumo energético semanal de \SI{25.921}{\milli\ampere\hour} para el nodo crítico. Este resultado demuestra que la adopción de un ciclo de trabajo de baja frecuencia es una medida eficaz para asegurar la sostenibilidad energética del sistema.

\subsection{Análisis y Selección de Componentes de Alimentación}
\label{subsec:seleccion_hardware_alimentacion}

Basándose en el consumo energético semanal estimado de \SI{25.921}{\milli\ampere\hour}, se procede a seleccionar los componentes del subsistema de alimentación: batería, panel solar y módulo de carga, buscando un balance entre capacidad, eficiencia, costo y seguridad.

\subsubsection{Análisis y Selección de la Batería}
La elección de la batería recargable es crucial para la autonomía del nodo. Se compararon opciones Li-ion y LiPo disponibles comercialmente (Tabla \ref{tab:baterias}), considerando capacidad, ciclos de vida, seguridad (presencia de BMS) y dimensiones.

% --- Tabla Comparativa Baterías ---
\renewcommand{\arraystretch}{1.5}
\begin{longtable}{|p{4.2cm}|c|c|c|}
\caption{Comparativa de baterías recargables para el nodo sensor}
\label{tab:baterias} \\
\hline
\textbf{Característica} & \textbf{104050} \cite{batteryLipo104050}
                        & \textbf{18650 2200mAh} \cite{batteryLiIon18650}
                        & \textbf{18650 3000mAh} \cite{batteryLiIon18650} \\
\hline
\endfirsthead

\hline
\textbf{Característica} & \textbf{104050} \cite{batteryLipo104050}
                        & \textbf{18650 2200mAh} \cite{batteryLiIon18650}
                        & \textbf{18650 3000mAh} \cite{batteryLiIon18650} \\
\hline
\endhead

\hline
\multicolumn{4}{r}{\textit{Continúa en la siguiente página}} \\
\endfoot

\hline
\endlastfoot

Tipo de batería & LiPo & Li-ion & Li-ion \\
\hline
Capacidad (mAh) & 2500 & 2200 & 3000 \\
\hline
Voltaje nominal (V) & 3.7 & 3.7 & 3.7 \\
\hline
Corriente máxima de descarga (A) & 2.5 & 2.2 & 3 \\
\hline
Dimensiones (mm) & 50x40x10 & 18x65 & 18x65 \\
\hline
Peso (g) & 44 & 43 & 46 \\
\hline
Número de ciclos de carga & $>800$ & $>1000$ & $>1000$ \\
\hline
Protección integrada (BMS) & Sí & No & No \\
\hline
Precio aproximado (MXN) & \$150 & \$80 & \$100 \\
\hline
Imagen 
& \shortstack{\\ \includegraphics[scale=0.1]{Documento/Imagenes/Análisis/Bat-Pan/Bateria-104050.jpg}}
& \shortstack{\\ \includegraphics[scale=0.11]{Documento/Imagenes/Análisis/Bat-Pan/Bateria-18650-2200mAh.jpg}}
& \shortstack{\\ \includegraphics[scale=0.1]{Documento/Imagenes/Análisis/Bat-Pan/Bateria-18650-3000mAh.jpg}} \\
\end{longtable}


 Se selecciona la batería \textbf{LiPo modelo 104050} con una capacidad nominal de \SI{2500}{\milli\ampere\hour}. Aunque las celdas Li-ion 18650 ofrecen un mayor número de ciclos de vida teóricos, la LiPo seleccionada presenta dos ventajas clave para este proyecto:
\begin{enumerate}
    \item \textbf{Seguridad Integrada:} Incluye un circuito de protección (BMS - Battery Management System) incorporado, que previene sobrecargas, sobredescargas y cortocircuitos, crucial para un dispositivo autónomo desatendido \cite{batteryLipo104050}.
    \item \textbf{Capacidad Suficiente:} Sus \SI{2500}{\milli\ampere\hour} exceden ampliamente la demanda energética calculada, proporcionando un gran margen de seguridad.
\end{enumerate}
La opción Li-ion 18650 requeriría un BMS externo, añadiendo complejidad al diseño.

\paragraph*{Cálculo de Autonomía Teórica:} 

Con la batería seleccionada, la autonomía estimada sin recarga solar es:
\begin{equation} \label{eq:autonomia}
\text{Autonomía} = \frac{\text{Capacidad Batería}}{\text{Consumo Semanal}} = \frac{\SI{2500}{\milli\ampere\hour}}{\SI{25.921}{\milli\ampere\hour / \text{sem}}} \approx \boxed{\num{96.4}~\text{semanas}} \approx \num{675}~\text{días}
\end{equation}
Este cálculo sugiere una autonomía teórica de \textbf{casi dos años} (aprox. 1.8 años), ofreciendo una excelente resiliencia ante periodos prolongados sin sol.

\subsubsection{Análisis y Selección del Panel Solar}
Para garantizar la operación continua, se necesita un panel solar que reponga el consumo energético. Se evaluaron paneles compactos (Tabla \ref{tab:paneles_solares}) en función de su potencia, voltaje/corriente de salida y dimensiones.

\begin{table}[H]
\centering
\renewcommand{\arraystretch}{1.5}
\caption{Comparativa de paneles solares para el nodo sensor}
\label{tab:paneles_solares}
\begin{tabular}{|p{4cm}|c|c|c|}
\hline
\textbf{Característica} & \textbf{Panel 5V} \cite{panelSolar5V}
                        & \textbf{Panel 6V} \cite{panelSolar6V}
                        & \textbf{Panel 12V} \cite{panelSolar12V} \\
\hline
Potencia nominal (W) & 1.9 & 1.92 & 3 \\\hline
Corriente de salida (mA) & 320 & 320 & 250 \\
\hline
Voltaje de salida (V) & 5 & 6 & 12 \\
\hline
Medición de intensidad de luz & \multicolumn{3}{|c|}{38000 LUX} \\
\hline
Dimensiones (mm) & 135x130 & 104x140 & 145x145 \\
\hline
Peso (g) & 65 & 54 & 34 \\
\hline
Precio aproximado (USD) & \$6 & \$5 & \$7 \\
\hline
Imagen 
&\shortstack{\\ \includegraphics[scale = 0.1]{Documento/Imagenes/Análisis/Bat-Pan/PanelSolar-5V.jpg}}
& \shortstack{\\ \includegraphics[scale = 0.1]{Documento/Imagenes/Análisis/Bat-Pan/Panel-Solar-6V.jpg}}
& \shortstack{\\ \includegraphics[scale = 0.1]{Documento/Imagenes/Análisis/Bat-Pan/Panel-Solar-12V.jpg}} \\
\hline
\end{tabular}
\end{table}

Se selecciona el \textbf{Panel Solar de \SI{6}{\volt} / \SI{1.92}{\watt}}. Aunque el panel de \SI{5}{\volt} ofrece características similares, el voltaje ligeramente superior del panel de \SI{6}{\volt} proporciona un mayor margen para la caída de tensión y es óptimo para módulos de carga como el CN3065, que requieren un voltaje de entrada superior al de la batería \cite{chipCN3065}. Además, este modelo es el más compacto y económico de las opciones evaluadas \cite{panelSolar6V}. El panel de \SI{12}{\volt} está sobredimensionado y es menos eficiente para cargar una batería de \SI{3.7}{\volt} con un cargador simple.

\paragraph*{Análisis de Generación vs. Consumo:}
\begin{itemize}
    \item Consumo Diario Promedio: $\SI{25.921}{\milli\ampere\hour} / 7\,\text{días} \approx \SI{3.7}{\milli\ampere\hour / \text{día}}$.
    \item Generación Diaria Estimada: Asumiendo un promedio conservador de \textbf{3 horas} de sol directo equivalente por día (HSP - Horas Solares Pico) para la Ciudad de México \cite{solarGIS}, la energía generada es:
    \begin{equation} \label{eq:generacion_diaria}
    \text{Generación}_{\text{diaria}} = I_{\text{mp}} \times \text{HSP} = \SI{320}{\milli\ampere} \times \SI{3}{\hour} = \SI{960}{\milli\ampere\hour}
    \end{equation}
\end{itemize}
El excedente energético diario ($\text{Generación}_{\text{diaria}} - \text{Consumo}_{\text{diario}} \approx \SI{956.3}{\milli\ampere\hour}$) es extremadamente amplio (más de 250 veces el consumo diario), lo que garantiza la recarga completa de la batería incluso bajo condiciones de irradiación solar subóptimas (días nublados, sombra parcial). El tiempo teórico necesario para reponer el consumo de una semana completa con sol directo sería mínimo: $\SI{25.92}{\milli\ampere\hour} / \SI{320}{\milli\ampere} \approx \SI{0.08}{\hour} \approx \SI{5}{\minute}$.


\subsubsection{Análisis de módulos de carga}

El módulo de carga es el componente encargado de gestionar la recarga segura de la batería desde el panel solar. Su función principal es proteger la batería contra sobrecargas, sobredescargas y controlar el flujo de corriente, lo cual es esencial para extender su vida útil. Esta subsección presenta una comparativa entre distintos módulos de carga compatibles con baterías Li-ion o LiPo, considerando su capacidad de carga, compatibilidad con paneles solares, protección integrada y facilidad de integración con el sistema(Tabla \ref{tab:modulos_carga}).

\renewcommand{\arraystretch}{1.5}
\begin{longtable}{|p{4.5cm}|c|c|c|}
\caption{Comparativa de módulos de carga para baterías recargables}
\label{tab:modulos_carga} \\
\hline
\textbf{Característica} & \textbf{CN3065} \cite{chipCN3065}
                        & \textbf{SD05CRMA} \cite{SD05CRMA}
                        & \textbf{CN3791} \cite{chipCN3791} \\
\hline
\endfirsthead

\hline
\textbf{Característica} & \textbf{CN3065} \cite{chipCN3065}
                        & \textbf{SD05CRMA} \cite{SD05CRMA}
                        & \textbf{CN3791} \cite{chipCN3791} \\
\hline
\endhead

\hline
\multicolumn{4}{r}{\textit{Continúa en la siguiente página}} \\
\endfoot

\hline
\endlastfoot

Tipo de batería compatible & Li-ion/Li-Po & Li-ion/Li-Po & Li-ion \\
\hline
Voltaje de entrada (V) & 4.4 - 6 & 4.4 - 6 & 4.5 - 28 \\
\hline
Corriente máxima de carga (mA) & 1000 & 1000 & 4000 \\
\hline
Protección integrada (sobre/descarga) & Sí & Sí & Sí \\
\hline
Compatibilidad con panel solar & Sí & Sí & Sí \\
\hline
Tamaño (mm) & 40x20x7 & 10.3x18.3 & 45x20 \\
\hline
Precio aproximado (USD) & \$2 & \$3 & \$4 \\
\hline
Imagen 
& \includegraphics[scale=0.1]{Documento/Imagenes/Análisis/Bat-Pan/Cargador-Solar.jpg}
& \includegraphics[scale=0.17]{Documento/Imagenes/Análisis/Bat-Pan/Cargador-SD05CRMA.jpg}
& \includegraphics[scale=0.1]{Documento/Imagenes/Análisis/Bat-Pan/Cargador-CN3791.jpg} \\
\end{longtable}

Se selecciona el módulo basado en el chip \textbf{CN3065}. Este circuito integrado está específicamente diseñado para la carga de baterías de litio de una celda directamente desde paneles solares de bajo voltaje (\SIrange{4.4}{6}{\volt}), coincidiendo perfectamente con el panel seleccionado \cite{chipCN3065}. Incluye las protecciones esenciales (sobrecarga, sobredescarga a través de pines dedicados si se usa un IC externo) y regula la corriente de carga (hasta \SI{500}{\milli\ampere}), lo cual es adecuado para la corriente generada por el panel (\SI{320}{\milli\ampere}).

Los módulos basados en el popular chip TP4056, aunque económicos, están optimizados para carga a través de USB (\SI{5}{\volt}) y no manejan de forma nativa la variabilidad de voltaje de un panel solar \cite{chipTP4056}. El módulo CN3791, si bien incluye MPPT (Maximum Power Point Tracking) para maximizar la extracción de energía del panel, representa una complejidad y costo innecesarios dado el enorme excedente energético ya calculado; el MPPT es más beneficioso en sistemas con paneles más grandes o consumos mucho más ajustados \cite{chipCN3791}. Por lo tanto, el CN3065 ofrece la solución más simple, económica y adecuada para este sistema.


%%%%%%%%%%%%%%%%%%%%%%%%%%%%%%%%%%%%%%%%%%%%%%%%%%%%
%             SECCIÓN: Cálculo de residuos         %
%%%%%%%%%%%%%%%%%%%%%%%%%%%%%%%%%%%%%%%%%%%%%%%%%%%%

\section{Análisis de la medición de residuos sólidos flotantes}
\label{sec:analisis_medicion_residuos}

La acumulación de residuos sólidos flotantes representa un problema crítico en cuerpos de agua lóticos, actuando como vectores de transporte de contaminantes. Para su evaluación, se toma como referencia la metodología de ``observación visual'' establecida en las guías operacionales de la UNEP/IOC para la basura marina \cite{cheshire2009directrices}, adaptándola al contexto fluvial y a las restricciones del sistema IoT.

La estrategia de monitoreo se fundamenta en tres pilares:
\begin{itemize}
    \item \textbf{Captura Programada:} Utilización de cámaras fijas para obtener una "instantánea" semanal del estado de la superficie del agua, permitiendo el seguimiento de tendencias a largo plazo sin saturar el ancho de banda de la red.
    \item \textbf{Área de Muestreo Definida:} Definición de un Campo de Visión (FoV) constante sobre la superficie del río, estableciendo un área de referencia ($A_{obs}$) para los cálculos.
    \item \textbf{Procesamiento Automatizado:} Aplicación de algoritmos de segmentación de instancias para cuantificar la superficie ocupada por los residuos.
\end{itemize}

\subsection{Cálculo de la Densidad de Cobertura}
\label{subsec:calculo_densidad}

Aprovechando la capacidad del modelo de IA seleccionado para delimitar el contorno preciso de los objetos (segmentación), se define como métrica principal la \textbf{Densidad de Cobertura de Residuos} ($D_{cob}$).

Este indicador cuantifica la proporción de la superficie del cuerpo de agua que se encuentra obstruida por materiales sólidos, proporcionando una medida directa del impacto visual y ambiental. Se calcula mediante la siguiente expresión:

\begin{equation} \label{eq:densidad_cobertura}
D_{cob} = \left( \frac{A_{residuos}}{A_{obs}} \right) \times 100
\end{equation}

Donde:
\begin{itemize}
    \item $D_{cob}$: Densidad de cobertura (expresada en porcentaje $\%$).
    \item $A_{residuos}$: Sumatoria del área de todos los píxeles clasificados como ``residuo'' por la máscara de segmentación. Representa la superficie física real ocupada por la basura.
    \item $A_{obs}$: Área superficial total del río observada por la cámara (constante determinada por la calibración geométrica).
\end{itemize}
El uso de esta métrica basada en segmentación ofrece una ventaja significativa sobre el simple conteo de objetos, ya que permite diferenciar, por ejemplo, entre un río con 50 botellas dispersas y uno con una gran isla de basura compacta. Esta distinción facilita una evaluación más precisa de la magnitud real de la contaminación y ayuda a identificar zonas críticas de acumulación de manera más efectiva que los métodos de conteo tradicionales. No obstante, es necesario considerar que factores externos, como la iluminación ambiental o la reflectancia de los materiales, pueden introducir incertidumbre en la estimación del parámetro $A_{residuos}$.



%%%%%%%%%%%%%%%%%%%%%%%%%%%%%%%%%%%%%%%%%%%%%%%%%%%%
%             SECCIÓN: Servidores                  %
%%%%%%%%%%%%%%%%%%%%%%%%%%%%%%%%%%%%%%%%%%%%%%%%%%%%

\section{Análisis de Servidor}

El servidor constituye el núcleo central de la arquitectura del sistema, operando bajo el modelo cliente-servidor, donde actúa como proveedor de recursos y servicios para los nodos de la red y la interfaz web \cite{supermicro2024}. La selección de una plataforma de servidor adecuada es crítica para garantizar la escalabilidad, fiabilidad y rendimiento del sistema, especialmente considerando el procesamiento de imágenes y la gestión de series temporales de datos.

El análisis se centra en dos modelos de despliegue principales: servidores locales (\textit{on-premise}) y servidores en la nube (\textit{cloud computing}), evaluando su idoneidad frente a los requerimientos del proyecto.

\subsection{Requerimientos Clave del Servidor}
\label{subsec:reqs_servidor}

Para cumplir con sus funciones, la plataforma servidora debe satisfacer los siguientes requerimientos técnicos:

\begin{enumerate}
    \item \textbf{Capacidad de Cómputo:} Debe disponer de recursos de CPU (y opcionalmente GPU) suficientes para ejecutar el algoritmo de visión artificial (YOLO) para la detección de residuos en las imágenes recibidas.
    \item \textbf{Almacenamiento Escalable:} Requiere una solución de almacenamiento capaz de gestionar eficientemente tanto los datos estructurados de los sensores (series temporales) como los datos no estructurados (imágenes en formato JPEG).
    \item \textbf{Alta Disponibilidad y Fiabilidad:} El sistema debe operar de forma continua para recibir y procesar los datos de la WSN sin interrupciones.
    \item \textbf{Conectividad de Red:} Debe contar con una conexión a internet estable y de ancho de banda suficiente para recibir los datos del nodo concentrador y servir la interfaz web.
    \item \textbf{Seguridad:} Es imperativo implementar medidas de seguridad para proteger la integridad y confidencialidad de los datos, así como para controlar el acceso a la API.
    \item \textbf{Viabilidad Económica:} La solución debe ser rentable, considerando los costos iniciales y operativos a largo plazo.
\end{enumerate}

\subsection{Comparativa de Modelos de Despliegue}
\label{subsec:modelos_despliegue}

Se evaluaron dos paradigmas fundamentales para la implementación del servidor:

\begin{itemize}
    \item \textbf{Servidor Local (On-Premise):} Implica la adquisición y mantenimiento de hardware físico en una ubicación controlada. El equipo es responsable de la instalación, configuración y seguridad \cite{nordlayer2024}.
        \begin{itemize}
            \item \textit{Ventajas:} Control total sobre el hardware y los datos, lo cual puede ser preferible para cumplir con estrictas normativas de cumplimiento regulatorio \cite{nordlayer2024}.
            \item \textit{Desventajas:} Alta inversión inicial en hardware y licencias, complejidad en la gestión, escalabilidad limitada y necesidad de asegurar la infraestructura de soporte (energía, red) \cite{softteco2024}.
        \end{itemize}
    \item \textbf{Servidor en la Nube (Cloud):} Utiliza un modelo para el acceso bajo demanda a un conjunto compartido de recursos de cómputo configurables que pueden ser rápidamente aprovisionados con un esfuerzo de gestión mínimo \cite{mell2011nist}.
        \begin{itemize}
            \item \textit{Ventajas:} Alta escalabilidad y elasticidad, fiabilidad garantizada por el proveedor (SLA), seguridad robusta y un modelo de costos basado en el pago por uso que elimina la inversión inicial en hardware \cite{softteco2024}.
            \item \textit{Desventajas:} Costos operativos recurrentes y dependencia de la conectividad a internet \cite{softteco2024}.
        \end{itemize}
\end{itemize}

\subsection{Análisis de Proveedores de Servicios en la Nube}
\label{subsec:proveedores_nube}

Dado que el proyecto requiere flexibilidad, escalabilidad y alta disponibilidad, el modelo de servidor en la nube se identifica como el más adecuado. A continuación, se comparan los tres principales proveedores de servicios en la nube, reconocidos como líderes en el sector por analistas de la industria como Gartner \cite{gartner2023magic}.

\renewcommand{\arraystretch}{1.5}
\begin{longtable}{
    |p{2.5cm}   % Columna 1: Característica
    |p{4.2cm}   % Columna 2: AWS
    |p{4.2cm}   % Columna 3: GCP
    |p{4.2cm}|  % Columna 4: Azure
}
\caption{Comparativa de Plataformas de Servidor en la Nube}
\label{tab:comparativa_cloud} \\
\hline
\textbf{Característica} 
    & \textbf{Amazon Web Services (AWS)} 
    & \textbf{Google Cloud Platform (GCP)} 
    & \textbf{Microsoft Azure} \\
\hline
\endfirsthead

\hline
\textbf{Característica} 
    & \textbf{Amazon Web Services (AWS)} 
    & \textbf{Google Cloud Platform (GCP)} 
    & \textbf{Microsoft Azure} \\
\hline
\endhead

\hline
\multicolumn{4}{r}{\textit{Continúa en la siguiente página}} \\
\endfoot

\hline
\endlastfoot

Cómputo 
    & \textbf{EC2} (Elastic Compute Cloud). Máquinas virtuales escalables con una amplia variedad de instancias, incluyendo las optimizadas para GPU \cite{qainsights2024}. 
    & \textbf{Compute Engine}. Máquinas virtuales de alto rendimiento. Fuerte enfoque en cargas de trabajo de IA y Machine Learning \cite{googlecloud2024products}. 
    & \textbf{Virtual Machines}. Servidores virtuales con una profunda integración con el ecosistema de Microsoft y Windows Server \cite{azure2024products}. \\ \hline

Almacenamiento 
    & \textbf{S3} (Simple Storage Service). Almacenamiento de objetos líder en la industria, diseñado para una durabilidad del 99.999999999\% \cite{qainsights2024}. 
    & \textbf{Cloud Storage}. Almacenamiento de objetos seguro y escalable, con diferentes clases de almacenamiento para optimizar costos \cite{googlecloud2024products}. 
    & \textbf{Blob Storage}. Solución de almacenamiento masivo para datos no estructurados como imágenes y videos \cite{azure2024storage}. \\ \hline

Base de Datos 
    & \textbf{RDS} (Relational Database Service) para SQL (ej. PostgreSQL) y \textbf{DynamoDB} para NoSQL de alto rendimiento \cite{qainsights2024}. 
    & \textbf{Cloud SQL} para bases de datos relacionales gestionadas y \textbf{Firestore/Bigtable} para soluciones NoSQL \cite{googlecloud2024sql}. 
    & \textbf{Azure SQL Database} para bases de datos SQL gestionadas y \textbf{Cosmos DB} para necesidades NoSQL multimodelo y distribuidas globalmente \cite{azure2024products}. \\ \hline

Servicios IoT 
    & \textbf{AWS IoT Core}. Plataforma completa para la conexión segura y gestión de millones de dispositivos, con enrutamiento de mensajes \cite{varonis2024}. 
    & \textbf{Cloud IoT Core}. Servicio gestionado para la ingesta segura de datos de dispositivos IoT (actualmente en transición a servicios de partners). 
    & \textbf{Azure IoT Hub}. Servicio escalable para la comunicación bidireccional fiable y segura entre la aplicación y los dispositivos \cite{varonis2024}. \\ \hline

Costo 
    & Modelo de pago por uso. Ofrece una capa gratuita (\textit{Free Tier}) por 12 meses, ideal para prototipos \cite{coursera2024}. 
    & Precios competitivos con descuentos por uso sostenido. Ofrece créditos iniciales y una capa gratuita competitiva \cite{datacamp2024}. 
    & Modelo de pago por uso con ventajas como el \textit{Hybrid Benefit} para clientes con licencias de Microsoft existentes \cite{varonis2024}. \\ \hline

\end{longtable}



Considerando los requerimientos del proyecto, se selecciona una arquitectura basada en la nube como la solución óptima. Entre los proveedores, \textbf{Amazon Web Services (AWS)} se elige como la plataforma para el desarrollo de este prototipo. La justificación se basa en los siguientes puntos:

\begin{enumerate}
    \item \textbf{Madurez y Ecosistema:} AWS es la plataforma más madura y con la mayor cuota de mercado, ofreciendo el ecosistema de servicios más amplio y una vasta documentación \cite{coursera2024}.
    \item \textbf{Capa Gratuita Robusta:} El \textit{Free Tier} de AWS permite desplegar los recursos necesarios para un prototipo funcional (ej. EC2, S3, RDS) sin costo durante el primer año, lo cual es ideal para un proyecto de tesis \cite{aws2024freetier}.
    \item \textbf{Servicios IoT Integrados:} AWS IoT Core proporciona una solución robusta y bien documentada para gestionar la comunicación desde el nodo concentrador, simplificando la autenticación de dispositivos y la ingesta de datos.
\end{enumerate}

La arquitectura de servicios propuesta en AWS consistiría en:
\begin{itemize}
    \item Un servidor de cómputo virtual (\textbf{Amazon EC2}) para ejecutar el \textit{backend} de la aplicación y el algoritmo de visión artificial.
    \item Un servicio de almacenamiento de objetos (\textbf{Amazon S3}) para el almacenamiento durable de las imágenes capturadas.
    \item Un \textbf{servicio de base de datos gestionada} para la persistencia, integridad y consulta de los datos de los sensores y los metadatos de las imágenes.
    \item Un servicio de conexión de dispositivos (\textbf{AWS IoT Core}) para gestionar la comunicación segura desde el nodo concentrador.
\end{itemize}

La selección específica del motor y el tipo de servicio de base de datos se justifica en detalle en la siguiente sección, basándose en un análisis de los modelos de datos del proyecto.


%%%%%%%%%%%%%%%%%%%%%%%%%%%%%%%%%%%%%%%%%%%%%%%%%%%%
%             SECCIÓN: Bases de datos              %
%%%%%%%%%%%%%%%%%%%%%%%%%%%%%%%%%%%%%%%%%%%%%%%%%%%%

\section{Análisis de Base de Datos}
\label{sec:analisis_db}

La base de datos es un componente crítico del servidor, responsable de la persistencia, integridad y disponibilidad de los datos generados por la WSN. La elección de una tecnología de base de datos adecuada impacta directamente en el rendimiento de las consultas, la escalabilidad del sistema y la facilidad con la que se pueden realizar análisis históricos.

\subsection{Requerimientos de la Base de Datos}
\label{subsec:reqs_db}

Con base en la arquitectura del sistema, la base de datos debe cumplir con los siguientes requerimientos específicos:

\begin{enumerate}
    \item \textbf{Manejo de Series Temporales:} El sistema debe almacenar eficientemente las mediciones de los sensores (pH, turbidez, etc.), que son datos de series temporales, caracterizados por tener una marca de tiempo (\textit{timestamp}) y un valor \cite{influxdata2024timeseries}. Esto requiere optimización para inserciones rápidas y consultas complejas basadas en rangos de tiempo.
    \item \textbf{Almacenamiento de Metadatos de Imágenes:} Debe gestionar los metadatos asociados a cada imagen capturada, incluyendo la URL de almacenamiento, la marca de tiempo, la identificación del nodo sensor y los resultados del análisis de visión artificial.
    \item \textbf{Escalabilidad:} La base de datos debe ser capaz de escalar horizontalmente para manejar el volumen creciente de datos a medida que la red de sensores opera a lo largo del tiempo o se expande con nuevos nodos \cite{mongodb2024scalability}.
    \item \textbf{Integridad y Consistencia:} Debe garantizar la integridad de los datos, asegurando que las mediciones se almacenen de forma fiable y consistente, siguiendo principios como los definidos por el modelo ACID en sistemas transaccionales \cite{oracle2024acid}.
    \item \textbf{Rendimiento de Consultas:} El sistema debe permitir consultas de baja latencia para alimentar la interfaz web con datos históricos y en tiempo real.
\end{enumerate}

\subsection{Análisis de Modelos de Base de Datos: SQL vs. NoSQL}
%\label{subsec:modelos_db}

Para satisfacer estos requerimientos, se analizaron los dos modelos de bases de datos predominantes: relacional (SQL) y no relacional (NoSQL).

\begin{itemize}
    \item \textbf{Bases de Datos Relacionales (SQL):} Organizan los datos en tablas estructuradas con un esquema predefinido y son conocidas por su robustez y consistencia (propiedades ACID) \cite{oracle2024sql}. Son muy adecuadas para almacenar los metadatos de las imágenes y las relaciones entre nodos y mediciones \cite{mongodb2024sqlvsnosql}.
    \item \textbf{Bases de Datos No Relacionales (NoSQL):} Utilizan modelos de datos flexibles y priorizan la escalabilidad y la disponibilidad, a menudo siguiendo el modelo BASE (Basically Available, Soft state, Eventually consistent) \cite{mongodb2024sqlvsnosql}. Son ideales para el almacenamiento de datos de series temporales, donde el volumen de inserciones es alto \cite{aws2024sqlvsnosql}.
\end{itemize}

\subsection{Comparativa de Servicios de Base de Datos en AWS}
\label{subsec:comparativa_db_aws}

Dada la selección de AWS como proveedor de nube, se comparan sus servicios de bases de datos gestionadas más relevantes, incluyendo opciones relacionales y una opción especializada en series temporales.

\renewcommand{\arraystretch}{1.3}
\begin{center}
\begin{longtable}{|p{3cm}|p{3cm}|p{3cm}|p{3cm}|p{3cm}|}
\caption{Comparativa de Servicios de Base de Datos Gestionadas en AWS}
\label{tab:comparativa_db} \\

\hline
\textbf{Característica} & \textbf{Amazon RDS para PostgreSQL} & \textbf{Amazon RDS para MySQL} & \textbf{Amazon Aurora} & \textbf{Amazon Timestream} \\
\hline
\endfirsthead

\multicolumn{5}{c}%
{{\bfseries \tablename\ \thetable{} -- continuación}} \\
\hline
\textbf{Característica} & \textbf{Amazon RDS para PostgreSQL} & \textbf{Amazon RDS para MySQL} & \textbf{Amazon Aurora} & \textbf{Amazon Timestream} \\
\hline
\endhead

\hline
\multicolumn{5}{r}{{\textit{Continúa en la siguiente página}}} \\
\hline
\endfoot

\hline
\endlastfoot

\textbf{Modelo} & Relacional (SQL) & Relacional (SQL) & Relacional (SQL), compatible con PostgreSQL y MySQL & NoSQL (Series Temporales) \\
\hline
\textbf{Caso de Uso Principal} & Aplicaciones que requieren consultas complejas, manejo de datos geoespaciales y extensibilidad \cite{aws2024rds}. & Aplicaciones web de propósito general, CMS y comercio electrónico. Muy popular en el ecosistema LAMP \cite{aws2024rds}. & Cargas de trabajo de alto rendimiento y alta disponibilidad a escala de nube. Aplicaciones empresariales críticas \cite{aws2024aurora}. & Monitoreo de IoT, análisis de datos de telemetría y series temporales a gran escala \cite{aws2024timestream}. \\
\hline
\textbf{Fortalezas Clave} & Soporte avanzado de tipos de datos y extensiones (ej. PostGIS). Cumplimiento estricto del estándar SQL. & Simplicidad, facilidad de uso y un ecosistema muy maduro. Buen rendimiento para lecturas. & Rendimiento hasta 3 veces superior a PostgreSQL estándar. Almacenamiento auto-reparable y distribuido. & Optimizado para ingesta masiva de datos temporales. Arquitectura de almacenamiento de dos niveles (memoria y magnético). \\
\hline
\textbf{Escalabilidad} & Escala verticalmente y horizontalmente mediante réplicas de lectura. & Escala verticalmente y horizontalmente mediante réplicas de lectura. & Escalabilidad superior con hasta 15 réplicas de lectura. Almacenamiento que escala automáticamente. & Escala horizontalmente de forma automática y sin servidor (\textit{serverless}). \\
\hline
\textbf{Costo} & Basado en horas de instancia y almacenamiento. Parte del \textit{Free Tier} \cite{aws2024rdsprice}. & Basado en horas de instancia y almacenamiento. Parte del \textit{Free Tier} \cite{aws2024rdsprice}. & Modelo de precios basado en el consumo, más costoso que RDS estándar. No forma parte del \textit{Free Tier} \cite{aws2024auroraprice}. & Basado en volumen de datos escritos, almacenados y consultados. Muy rentable para grandes volúmenes \cite{aws2024timestreamprice}. \\
\hline

\end{longtable}
\end{center}


Tras analizar los requerimientos y las plataformas disponibles, se ha decidido optar por una \textbf{arquitectura de base de datos unificada} para simplificar el diseño del sistema, reducir la complejidad operativa y centralizar la gestión de los datos.

La elección se inclina hacia un sistema de gestión de bases de datos relacional (RDBMS), ya que ofrece la versatilidad necesaria para manejar tanto los datos estructurados de los metadatos como las series temporales de los sensores, garantizando al mismo tiempo la máxima integridad de los datos.

Entre las opciones relacionales evaluadas, se selecciona \textbf{Amazon RDS para PostgreSQL} como la plataforma de base de datos única para el proyecto. Esta decisión se fundamenta en los siguientes puntos clave:

\begin{enumerate}
    \item \textbf{Versatilidad y Potencia:} PostgreSQL es conocido por su robustez y su estricto cumplimiento del estándar SQL. Es capaz de gestionar eficientemente los datos relacionales (configuración de nodos, metadatos de imágenes) y, al mismo tiempo, puede ser optimizado para manejar cargas de trabajo de series temporales mediante técnicas avanzadas como la partición de tablas por rangos de fecha y el uso de índices especializados, asegurando un buen rendimiento en las consultas temporales.
    
    \item \textbf{Superioridad para Datos Complejos:} En comparación con MySQL, PostgreSQL ofrece un manejo más avanzado de tipos de datos complejos y un ecosistema de extensiones más rico, como \textbf{PostGIS}, que podría ser crucial en futuras fases del proyecto para realizar análisis geoespaciales sobre la ubicación de los nodos \cite{postgresql2024about}. Esta capacidad de extensión lo convierte en una opción más sólida y preparada para el futuro en aplicaciones científicas y de monitoreo.

    \item \textbf{Viabilidad para el Prototipo:} Amazon Aurora, aunque técnicamente superior en rendimiento, se descarta para esta fase debido a su mayor costo y a que sus capacidades exceden los requerimientos del prototipo. Por su parte, Amazon Timestream se descarta como solución única, ya que su modelo NoSQL no es el adecuado para gestionar las relaciones y la integridad de los metadatos del sistema.
\end{enumerate}

Por lo tanto, Amazon RDS para PostgreSQL representa el balance óptimo entre rendimiento, costo, integridad de los datos y flexibilidad, proporcionando una base sólida y unificada para todas las necesidades de almacenamiento del proyecto.


%%%%%%%%%%%%%%%%%%%%%%%%%%%%%%%%%%%%%%%%%%%%%%%%%%%%
%       SECCIÓN: Arquitectura de Cómputo           %
%%%%%%%%%%%%%%%%%%%%%%%%%%%%%%%%%%%%%%%%%%%%%%%%%%%%

\section{Análisis de Arquitectura de Cómputo para Visión Artificial}
\label{sec:analisis_edge_vs_cloud}

Una decisión de diseño fundamental en cualquier sistema de IoT que integra IA es determinar \textit{dónde} se realizará el procesamiento de los datos. Para la tarea de visión artificial de este proyecto, existen dos paradigmas principales: el cómputo en la nube y el cómputo en el borde \cite{shi2016edge}. A continuación, se analizan ambos enfoques.

\begin{itemize}
    \item \textbf{Cómputo en la Nube (\textit{Cloud Computing}):} En esta arquitectura, el dispositivo en el campo (el nodo sensor) actúa como un simple capturador de datos. Captura la imagen en bruto y la transmite a través de la red a un servidor centralizado y potente en la nube (ej. AWS EC2). Todo el procesamiento intensivo, como la ejecución del modelo YOLO, se realiza en el servidor.
    \begin{itemize}
        \item \textbf{Ventajas:} Permite el uso de modelos de IA muy grandes y precisos, ya que no hay restricciones de hardware. El nodo sensor puede ser más simple, económico y de menor consumo (excepto en la transmisión). La actualización y el mantenimiento del modelo de IA se centralizan en un solo lugar.
        
        \item \textbf{Desventajas:} Requiere un ancho de banda de red significativo y confiable para transmitir datos pesados (imágenes). La transmisión inalámbrica de grandes volúmenes de datos es la operación de mayor consumo energético en un nodo sensor, impactando severamente la autonomía de la batería \cite{Saradha2025comprehensive}.
    \end{itemize}

    \item \textbf{Cómputo en el Borde (\textit{Edge Computing}):} En este enfoque, el procesamiento de los datos se realiza localmente, en o cerca del dispositivo que los captura. El nodo sensor estaría equipado con un microprocesador o un acelerador de IA (ej. un Coral Edge TPU) capaz de ejecutar una versión optimizada y ligera del modelo YOLO (ej. un modelo "cuantizado"), una técnica conocida como \textit{on-device learning} \cite{khouas2024training}.
    \begin{itemize}
        \item \textbf{Ventajas:} Reduce drásticamente el tráfico de red. En lugar de enviar una imagen de \SI{150}{\kilo\byte}, el nodo solo transmite el resultado del análisis (unos pocos bytes de texto, ej. `{"objeto": "botella", "confianza": 0.92}`). Esto se traduce en un ahorro masivo de energía en la comunicación inalámbrica y una mayor vida útil de la batería. También reduce la latencia y permite que el sistema funcione de forma autónoma si se pierde la conexión con el servidor central.
        \item \textbf{Desventajas:} Requiere hardware más potente y costoso en el nodo sensor. Limita el tamaño y la complejidad del modelo de IA que se puede ejecutar, lo que podría reducir la precisión en comparación con un modelo en la nube, ya que el entrenamiento y la inferencia en el borde enfrentan desafíos significativos de recursos.
    \end{itemize}
\end{itemize}

Para la presente fase del proyecto, \textbf{se ha seleccionado la arquitectura de cómputo en la nube}. La justificación se basa en los siguientes criterios:

\begin{enumerate}
    \item \textbf{Flexibilidad y Precisión del Modelo:} El enfoque en la nube permite experimentar con diferentes versiones del modelo YOLO sin estar limitados por el hardware del nodo, priorizando la máxima precisión en la detección de residuos.
    \item \textbf{Simplicidad del Nodo Sensor:} Mantiene el diseño del hardware del nodo más simple y enfocado en la eficiencia energética para la captura y comunicación, lo cual es un desafío de por sí.
    \item \textbf{Viabilidad del Protocolo de Comunicación:} La elección del protocolo Wi-Fi HaLow, con su ancho de banda relativamente alto (comparado con LPWAN como LoRa), hace que la transmisión de imágenes comprimidas sea técnicamente viable, aunque energéticamente costosa.
\end{enumerate}

No obstante, se reconoce que una arquitectura de cómputo en el borde es una optimización altamente deseable. En el capítulo de ``Conclusiones y Trabajo Futuro", se propondrá la exploración de un modelo híbrido o de borde como una evolución natural de este sistema.




\section{Análisis de Modelos para Detección y Segmentación de Objetos}
\label{sec:analisis_vision}

La selección de un algoritmo adecuado para identificar y cuantificar los residuos sólidos flotantes es un componente central de la funcionalidad de visión artificial del sistema. Dada la necesidad de estimar la densidad de basura (posiblemente a través del área cubierta en píxeles), se requiere un modelo capaz no solo de detectar objetos (\textit{object detection}) sino también de delinear sus contornos precisos (\textit{instance segmentation}). La familia de modelos \textbf{YOLO (\textit{You Only Look Once})} ofrece arquitecturas que cubren ambas tareas con un buen balance entre velocidad y precisión \cite{sapkota2025yolo}. Se realizó un análisis comparativo para identificar la versión más apropiada.

\subsection{Criterios de Selección}
\label{subsec:criterios_vision}

La elección del modelo se basa en los siguientes criterios prioritarios:

\begin{enumerate}
    \item \textbf{Precisión (mAP / Mask mAP):} El modelo debe ofrecer alta precisión tanto en la detección (localización con cuadros delimitadores) como, preferiblemente, en la segmentación (delimitación a nivel de píxel). Se toma como referencia el rendimiento en el dataset COCO.
    \item \textbf{Capacidad de Segmentación Integrada:} Se valora positivamente que el modelo soporte la segmentación de instancias de forma nativa dentro de su framework, simplificando el entrenamiento y la inferencia.
    \item \textbf{Tamaño del Modelo y Requerimientos de Hardware:} Aunque el procesamiento se realizará en la nube, se prefiere un modelo con un tamaño razonable para facilitar su despliegue y gestión.
    \item \textbf{Facilidad de Uso y Ecosistema:} Disponibilidad de implementaciones robustas, buena documentación y una comunidad activa.
    \item \textbf{Flexibilidad para Re-entrenamiento:} Capacidad del modelo para ser afinado (\textit{fine-tuning}) con un dataset específico de residuos flotantes.
\end{enumerate}

\subsection{Comparativa de Modelos YOLO Relevantes}
\label{subsec:comparativa_yolo}

La Tabla \ref{tab:comparativa_yolo_condensada} presenta una comparativa de las versiones de YOLO consideradas, destacando su capacidad para la segmentación. Los valores de mAP corresponden al dataset COCO y dependen del hardware \cite{terven2023yolo, sapkota2025yolo}.

\renewcommand{\arraystretch}{1.3}
\begin{longtable}{
    |p{1.6cm}  % Modelo
    |p{1.7cm}         % Precisión
    |p{2.4cm}         % Segmentación
    |p{2.9cm}  % Requerimientos
    |p{2.1cm}         % Facilidad de uso
    |p{3.5cm}| % Ideal para
}
\caption{Comparativa de Modelos YOLO Relevantes (Dataset COCO)} \label{tab:comparativa_yolo_condensada} \\
\hline
\textbf{Modelo} & 
\textbf{Precisión (mAP Det.)} & 
\textbf{Soporte Segmentación Nativo?} & 
\textbf{Requerimientos Hardware (GPU)} & 
\textbf{Facilidad de Uso} & 
\textbf{Ideal Para} \\ 
\hline
\endfirsthead

\hline
\textbf{Modelo} & 
\textbf{Precisión (mAP Det.)} & 
\textbf{Soporte Segmentación Nativo?} & 
\textbf{Requerimientos Hardware (GPU)} & 
\textbf{Facilidad de Uso} & 
\textbf{Ideal Para} \\ 
\hline
\endhead

\hline
\multicolumn{6}{r}{\textit{Continúa en la siguiente página}} \\
\endfoot

\hline
\endlastfoot

YOLOv5 
& $\sim$ 55.8\% (v5x)
& No (Requiere forks/mods)
& Baja/Moderada (Básica a Media)
& Muy Alta (PyTorch Nativo)
& Detección rápida, buen balance, hardware moderado \\ \hline

YOLOv7 
& $\sim$ 56.8\% (v7x)
& No (Principalmente detección)
& Alta (Gama Alta)
& Moderada
& Detección de alta precisión y velocidad \\ \hline

YOLOv8 
& $\sim$ 53.9\% (v8x)
& \textbf{Sí} (Framework Unificado)
& Alta (Gama Alta)
& Muy Alta (Framework Ultralytics)
& \textbf{Detección y Segmentación}, versatilidad, estado del arte \\ \hline

YOLOv11 
& $\sim$ 55.4\% (Est.)
& \textbf{Sí} (Framework Unificado)
& Alta (Gama Alta)
& Alta (Reciente)
& Detección/Segmentación, mejoras arquitectónicas recientes \\ \hline

\multicolumn{6}{p{16cm}}{\footnotesize \textit{Nota: Los tamaños de modelo varían según la versión específica (n, s, m, l, x). El mAP corresponde a COCO val2017 para detección. La facilidad de uso de YOLOv11 puede variar por ser más reciente.}} \\

\end{longtable}


\subsection{Justificación y Selección del Modelo}
\label{subsec:justificacion_yolo}

La necesidad explícita de realizar segmentación de instancias para estimar la densidad de residuos mediante el área cubierta en píxeles inclina la balanza hacia modelos que ofrezcan esta funcionalidad de manera integrada y eficiente.

\begin{itemize}
    \item \textbf{YOLOv5:} Aunque es un modelo excelente y fácil de usar para la detección de objetos \cite{jocher2020yolov5}, no incluye la segmentación de instancias en su rama principal. Lograr segmentación requiere utilizar variantes o modificaciones específicas, lo que añade complejidad al desarrollo y mantenimiento.

    \item \textbf{YOLOv7:} Similar a YOLOv5, su enfoque principal ha sido optimizar la detección en términos de velocidad y precisión, sin integrar la segmentación como una tarea central.

    \item \textbf{YOLOv8:} Desarrollado por Ultralytics, se presenta como un framework unificado que maneja de forma nativa y eficiente tanto la detección como la segmentación de instancias, clasificación y estimación de pose \cite{ultralyticsYOLOv8}. Ofrece un rendimiento de vanguardia, es fácil de usar y cuenta con un excelente soporte y documentación.

    \item \textbf{YOLOv11:} Representa una de las evoluciones más recientes, prometiendo mejoras arquitectónicas \cite{sapkota2025yolo}. Aunque también soporta segmentación, al ser más nuevo, podría tener menos documentación comunitaria o requerir ajustes en las herramientas de entrenamiento en comparación con YOLOv8.
\end{itemize}

Considerando la necesidad de segmentación, la madurez relativa dentro de las arquitecturas modernas y la robustez del framework, se selecciona \textbf{YOLOv8-YOLOv11} como el modelo para este proyecto. Específicamente, se optará por una de sus variantes pre-entrenadas (ej. `yolov8s-seg.pt`) que ya incluye la capacidad de segmentación.

Esta elección ofrece el mejor compromiso entre:
\begin{itemize}
    \item \textbf{Funcionalidad Requerida:} Soporte nativo y eficiente para segmentación de instancias.
    \item \textbf{Precisión:} Rendimiento de vanguardia demostrado en benchmarks estándar.
    \item \textbf{Facilidad de Implementación:} Un framework unificado y bien documentado que simplifica el entrenamiento (fine-tuning) y la inferencia.
\end{itemize}

Si bien YOLOv5 sigue siendo una excelente opción si la detección simple fuese suficiente, la necesidad de cuantificar el área cubierta por los residuos hace que YOLOv8 sea la opción técnicamente superior para los objetivos planteados.






%\newpage
%\begin{comment}
\chapter{Diseño}
\section{Diseño de boya para nodos sensores y nodo concentrador}
\section{Diagrama de circuito fuente de alimentacion}
\section{Diagrama de circuitos de nodo sensor}
\section{Diagrama de circuitos de nodo concentrado}
\section{Diseño de boya para nodo sensor}
\section{Diagrama de comunicacion del nodo concentrador con el servidor}
\section{Diseño de Base de datos}
\section{Modelo Relacional}
\section{Arquitectura}
\section{Diagramas UML}
\section{Diagramas de caso de uso}
\section{Diagramas de secuencia}
\section{Diccionario de Datos}
\section{Diseño de Mock-up para la pagina web}
\section{Pagina principal}
\section{Diseño de la interfaz para la página Web}
\section{Integracion Final del Sistema}
\section{Escenario de pruebas}    
\end{comment}

%opcion 2

\chapter{Diseño}
\begin{comment}
\section{Diseño del Sistema}
\subsection{Arquitectura del Sistema}

\section{Diseño de Hardware}

\subsection{Diagrama eléctrico de los nodos}
\label{sec:diagrama_electrico}

La red está conformada por nodos sensores y un nodo concentrador. Cada nodo sensor se encarga de medir parámetros físico-químicos del agua y transmitir la información recopilada hacia el nodo final, el cual actúa como concentrador. Con base en los componentes analizados y seleccionados en capítulos anteriores, se presenta el diseño esquemático del circuito electrónico del nodo sensor (Figura~\ref{fig:esquema_nodo_sensor}).

\subsection*{Descripción general del circuito}

El circuito del nodo sensor se compone de dos microcontroladores principales: un \textbf{PIC16F1823} y un \textbf{Xiao ESP32-S3}. El primero se encarga del sensado y control energético, mientras que el segundo maneja la comunicación inalámbrica con el siguiente nodo (o concentrador) a través de un módulo \textbf{Wi-Fi HaLow FGH100M}.

\begin{itemize}
    \item \textbf{PIC16F1823}: gestiona la activación secuencial de los sensores mediante un demultiplexor \textbf{74HC238}, activando solo un sensor a la vez para minimizar el consumo energético. Los sensores analógicos (pH, conductividad, OD, turbidez) son activados por MOSFETs \textbf{AO3400A} y su señal de salida se dirige a entradas ADC del PIC.
    
    \item \textbf{Sensor de temperatura DS18B20}: se conecta directamente al PIC mediante una línea digital dedicada para adquisición por protocolo 1-Wire.

    \item \textbf{Demultiplexor 74HC238}: permite seleccionar cuál de los sensores analógicos se activa en cada momento, controlado por tres líneas digitales del PIC.

    \item \textbf{Xiao ESP32-S3}: se encarga de la comunicación inalámbrica. Recibe los datos procesados por el PIC a través de un canal \textbf{UART}, y los transmite al siguiente nodo o al concentrador mediante el módulo HaLow conectado en sus pines digitales.

    \item \textbf{Módulo HaLow FGH100M}: es operado por el Xiao ESP32 y sirve como interfaz de comunicación bajo el estándar IEEE 802.11ah.

    \item \textbf{Fuente de alimentación}: el sistema se alimenta desde una batería Li-ion de 3.7 V regulada por un cargador solar \textbf{CN3065}. La salida alimenta un convertidor \textbf{step-up} que genera 5V, y un LDO que provee 3.3V para los componentes lógicos.
\end{itemize}

\begin{figure}[H]
\centering
\includegraphics[width=0.8\textwidth]{Documento/Imagenes/Diseño/Diagramas electricos/Controller Sensor_fit.pdf}
\caption{Diagrama eléctrico del control de sensores.}
\label{fig:esquema_nodo_sensor}
\end{figure}

\begin{figure}[H]
\centering
\includegraphics[width=0.8\textwidth]{Documento/Imagenes/Diseño/Diagramas electricos/microcontroller_fit.pdf}
\caption{Diagrama eléctrico comunicación inalámbrica y fuente de alimentación.}
\label{fig:esquema_nodo_sensor2}
\end{figure}


\subsection*{Funcionamiento del sistema}

Durante cada ciclo de adquisición:

\begin{enumerate}
    \item El PIC activa uno por uno los sensores mediante el demultiplexor y sus respectivas compuertas MOSFET.
    \item Cada sensor permanece encendido aproximadamente 5 segundos para estabilizar la medición.
    \item El PIC adquiere los valores analógicos o digitales, almacena las lecturas y las envía por UART al Xiao ESP32.
    \item El Xiao ESP32 activa la cámara, captura una imagen (JPG), y empaqueta los datos sensados junto con la imagen.
    \item Finalmente, activa el módulo HaLow para transmitir toda la información al nodo siguiente o al nodo concentrador.
    \item Al finalizar, el sistema entra en modo \textit{sleep} hasta el próximo ciclo.
\end{enumerate}
\end{comment}

\section{Diseño del Circuito Electrónico}
\label{sec:diseno_circuito}

El diseño electrónico del nodo sensor implementa una arquitectura de \textbf{procesamiento distribuido} utilizando dos microcontroladores para separar las tareas críticas de comunicación y sensado de las tareas intensivas de adquisición de imágenes. El diagrama esquemático general se presenta en la Figura \ref{fig:diagrama_circuito}.

\begin{figure}[h!]
    \centering
    \includegraphics[width=0.95\linewidth]{Documento/Imagenes/Diseño/circuitos/circuitopt_1.pdf}
    \caption{Diagrama esquemático del nodo sensor, mostrando la interconexión entre microcontroladores y el bloque de sensores.}
    \label{fig:diagrama_circuito}
\end{figure}

El sistema se estructura en tres bloques funcionales interconectados:

\subsection{Unidad Central de Control y Telemetría (MCU1)}
El núcleo del nodo es la plataforma \textbf{Heltec Wireless Tracker V1.1} (basada en ESP32-S3 + SX1262). Este microcontrolador actúa como el coordinador principal del sistema y es responsable de:

\begin{itemize}
    \item \textbf{Gestión de Sensores:} Controla directamente el encendido y apagado de los sensores mediante un multiplexor y realiza la lectura de las señales analógicas y digitales.
    \item \textbf{Comunicación LoRa:} Administra el transceptor SX1262 integrado para el envío de paquetes de datos y fragmentos de imagen a la red.
    \item \textbf{Geolocalización:} Obtiene las coordenadas del módulo GNSS integrado.
    \item \textbf{Coordinación de Imagen:} Envía la orden de captura al módulo de cámara (MCU2) y recibe el flujo de bytes de la imagen procesada vía UART.
\end{itemize}

\subsection{Módulo de Adquisición de Imágenes (MCU2)}
Para aislar el manejo de la cámara y asegurar la estabilidad del sistema principal, se utiliza un módulo \textbf{ESP32-CAM} dedicado exclusivamente a la visión. Sus funciones son:
\begin{itemize}
    \item \textbf{Control de la Cámara:} Inicializa y configura el sensor OV5640.
    \item \textbf{Captura y Compresión:} Toma la fotografía en alta resolución y realiza la compresión JPEG en hardware.
    \item \textbf{Transmisión Serial:} Envía la imagen comprimida al MCU1 a través de un puerto serie de alta velocidad, actuando como un periférico esclavo.
\end{itemize}

\subsection{Subsistema de Alimentación y Regulación}
\label{subsec:diseno_alimentacion}

Para garantizar la estabilidad operativa de los microcontroladores y la precisión de las mediciones analógicas, el diseño implementa una arquitectura de alimentación distribuida que gestiona la conversión de energía y los niveles de voltaje.

\begin{enumerate}
    \item \textbf{Gestión de Carga Solar:} Se integra el módulo basado en el controlador CN3065, encargado de administrar el flujo de energía desde el panel solar hacia la batería LiPo (\SI{3.7}{\volt}). Este módulo protege la batería contra sobrecargas y optimiza la corriente de carga disponible.

    \item \textbf{Regulación Principal (Bus de \SI{5}{\volt}):} Dado que el voltaje de la batería varía durante la descarga (\SIrange{3.2}{4.2}{\volt}), se utiliza un convertidor DC-DC tipo \textit{Step-Up} (Boost) para elevar y estabilizar la tensión a \textbf{\SI{5}{\volt}}. Este bus alimenta las entradas de potencia de las placas de desarrollo principales:
    \begin{itemize}
        \item \textbf{Heltec Wireless Tracker (MCU1):} Se alimenta a través de su pin de \SI{5}{\volt}, permitiendo que su regulador LDO interno genere el voltaje lógico de \SI{3.3}{\volt} limpio para el ESP32 y la radio LoRa.
        \item \textbf{ESP32-CAM (MCU2):} Recibe \SI{5}{\volt} estables para garantizar el correcto funcionamiento de la cámara, evitando reinicios por caídas de tensión (\textit{Brown-out}).
    \end{itemize}
\end{enumerate}

\subsection{Subsistema de Sensores y Eficiencia Energética}
\label{subsec:diseno_sensores_control}

Los sensores analógicos y digitales se alimentan a \textbf{\SI{3.3}{\volt}} para igualar los niveles de referencia del Conversor Analógico-Digital (ADC) del ESP32-S3. Para optimizar el consumo de corriente (que puede ser elevado en sensores como OD o Turbidez), se implementa una estrategia de energía bajo demanda (\textit{Power Gating}) controlada por el MCU1.

\begin{enumerate}
    \item \textbf{Multiplexación de Potencia:} El circuito incorpora un decodificador/demultiplexor 74HC238 y un arreglo de transistores MOSFET que actúan como interruptores de potencia.
    
    \item \textbf{Lógica de Control:}
    \begin{itemize}
        \item El MCU1 envía la dirección binaria del sensor requerido al multiplexor.
        \item La salida correspondiente del 74HC238 activa la compuerta (\textit{Gate}) del transistor MOSFET asociado a ese sensor específico.
        \item Esto permite el paso de los \SI{3.3}{\volt} únicamente al sensor seleccionado, manteniendo los demás físicamente desconectados y con consumo cero.
    \end{itemize}

    \item \textbf{Adquisición:} Una vez estabilizado el sensor activo, el MCU1 realiza la lectura a través de sus entradas analógicas (ADC) o digitales (OneWire) y procede inmediatamente a apagar el sensor antes de activar el siguiente o entrar en reposo.
\end{enumerate}

Esta arquitectura garantiza que el consumo pico del sistema se mantenga dentro de los límites soportados por la batería y evita interferencias cruzadas entre las lecturas de los sensores.
%%%%%%%%%%%%%%%%%%%%%%%%%%%%%%%%%%%%%%%%%%%%%%%%%%%%
%             SECCIÓN: base de datos         %
%%%%%%%%%%%%%%%%%%%%%%%%%%%%%%%%%%%%%%%%%%%%%%%%%%%%


\section{Diseño de la Base de Datos}
\label{sec:diseno_bd}

Con base en el análisis previo, se ha diseñado un esquema de base de datos relacional implementado sobre el motor \textbf{PostgreSQL} (alojado en AWS RDS). Este diseño tiene como objetivo garantizar la persistencia, integridad y disponibilidad de los datos generados por la Red Inalámbrica de Sensores y los resultados del procesamiento de visión artificial.

\subsection{Modelo Conceptual (Entidad-Relación)}
\label{subsec:modelo_er}

El modelo de datos se estructura en torno a tres entidades principales que reflejan la arquitectura del sistema:

\begin{enumerate}
    \item \textbf{Nodos (nodes):} Representa los dispositivos físicos (nodos sensores) desplegados en el río. Almacena su configuración estática y ubicación.
    \item \textbf{Lecturas de Sensores (sensor\_readings):} Almacena las series temporales de los parámetros físico-químicos y el estado de la batería, vinculadas a un nodo específico.
    \item \textbf{Detecciones de Residuos (waste\_detections):} Almacena los metadatos resultantes del análisis de imágenes (densidad de cobertura), la referencia a la imagen almacenada (URL) y la trazabilidad del modelo de IA.
\end{enumerate}

La Figura \ref{fig:diagrama_er} ilustra el Diagrama Entidad-Relación (ER) del sistema, mostrando las relaciones de ``uno a muchos'' ($1:N$) entre los nodos y sus mediciones/detecciones.

% \begin{figure}[h!]
%     \centering
%     % Reemplaza con tu archivo de imagen. Si no tienes uno, avísame para ayudarte a describirlo.
%     \includegraphics[width=0.7\linewidth]{Documento/Imagenes/Diseño/BD/entidad-relacion.pdf}
%     \caption{Diagrama Entidad-Relación (ER) de la base de datos del sistema de monitoreo.}
%     \label{fig:diagrama_er}
% \end{figure}
\begin{figure}[h]
    \centering
    \begin{tikzpicture}[
    node distance=2cm and 2cm,
    % Estilo para las tablas
    entity/.style={
        rectangle split,
        rectangle split parts=2,
        draw=black,
        very thick,
        rounded corners,
        fill=white,
        drop shadow,
        text width=5.5cm,
        align=left,
        font=\scriptsize
    },
    relation/.style={
        draw=black,
        thick,
        -{Circle[open] Crow's Foot},
    }
]

    % --- TABLA: NODES (Centro) ---
    \node[entity] (nodes) {
        \nodepart{text} \textbf{NODES}
        \nodepart{second}
        \underline{VARCHAR(20)} \textbf{node\_id} (PK)\\
        VARCHAR(100) description\\
        DECIMAL(10,8) latitude\\
        DECIMAL(11,8) longitude\\
        TIMESTAMP last\_seen
    };

    % --- TABLA: USERS (Izquierda) ---
    \node[entity, left=2cm of nodes] (users) {
        \nodepart{text} \textbf{USERS}
        \nodepart{second}
        \underline{BIGSERIAL} \textbf{user\_id} (PK)\\
        VARCHAR(100) full\_name\\
        VARCHAR(100) email (UK)\\
        VARCHAR(255) password\_hash\\
        VARCHAR(20) role\\
        BOOLEAN is\_active
    };

    % --- TABLA: SENSOR_READINGS (Abajo Izquierda) ---
    \node[entity, below left=1.5cm and -2cm of nodes] (readings) {
        \nodepart{text} \textbf{SENSOR\_READINGS}
        \nodepart{second}
        \underline{BIGSERIAL} \textbf{reading\_id} (PK)\\
        \dashuline{VARCHAR(20)} \textbf{node\_id} (FK)\\
        TIMESTAMP timestamp\\
        DECIMAL(4,2) ph\\
        DECIMAL(5,2) dissolved\_oxygen\\
        DECIMAL(6,2) turbidity\\
        DECIMAL(8,2) conductivity\\
        DECIMAL(5,2) temperature\\
        DECIMAL(4,2) battery\_level
    };

    % --- TABLA: WASTE_DETECTIONS (Abajo Derecha) ---
    \node[entity, below right=1.5cm and -2cm of nodes] (detections) {
        \nodepart{text} \textbf{WASTE\_DETECTIONS}
        \nodepart{second}
        \underline{BIGSERIAL} \textbf{detection\_id} (PK)\\
        \dashuline{VARCHAR(20)} \textbf{node\_id} (FK)\\
        TIMESTAMP timestamp\\
        DECIMAL(5,2) coverage\_percent\\
        BYTEA image\_data\\
        VARCHAR(50) model\_version\\
        DECIMAL(3,2) confidence
    };

    % --- RELACIONES ---
    
    % 1. Users -> Nodes (1 a N: Un usuario gestiona N nodos, o N usuarios gestionan N nodos)
    % Para simplificar visualmente, mostramos que Users tienen acceso a Nodes
    \draw[thick] ([xshift=-0.5cm]nodes.west) |- (users.east);
    \draw[thick] (nodes.west) |- (users.east);
    \draw[thick, {Classical TikZ Rightarrow[length=2.5mm] }-{Classical TikZ Rightarrow[length=2.5mm] }] ([xshift=0.25cm]users.east)-- ++(-0.1,0) -- ([xshift=-0.15cm]nodes.west)-- ++(-0.1,0);
    \node[anchor=south] at (barycentric cs:users=0.5,nodes=0.5) {\tiny Gestiona};

    % 2. Nodes -> Readings
    \draw[thick] ([xshift=-0.5cm]nodes.south) |- (readings.east);
    \draw[thick, {Circle[open,length=3mm]}-] ([xshift=-0.5cm]nodes.south) -- ++(0,-0.5);
    \draw[thick, -{Classical TikZ Rightarrow[length=2.5mm]}] ([xshift=0.15cm]readings.east) -- ++(0.1,0);

    % 3. Nodes -> Detections
    \draw[thick] ([xshift=0.5cm]nodes.south) |- (detections.west);
    \draw[thick, {Circle[open,length=3mm]}-] ([xshift=0.5cm]nodes.south) -- ++(0,-0.5);
    \draw[thick, -{Classical TikZ Rightarrow[length=2.5mm]}] ([xshift=-0.15cm]detections.west) -- ++(-0.1,0);

\end{tikzpicture}
    \caption{Diagrama Entidad-Relación (ER) de la base de datos del sistema de monitoreo.}
    \label{fig:diagrama_er}
\end{figure}


\subsection{Diseño Lógico y Diccionario de Datos}
\label{subsec:diccionario_datos}

A continuación, se detalla la estructura física de las tablas diseñadas, especificando tipos de datos y restricciones para asegurar la integridad referencial.

\subsubsection*{Tabla: Nodes (Nodos)}
Esta tabla actúa como catálogo maestro de los dispositivos autorizados en la red.
\renewcommand{\arraystretch}{1.2}

\begin{longtable}{l l l p{6cm}}
\caption{Diccionario de datos: Tabla \texttt{nodes}.}
\label{tab:dd_nodes} \\

\toprule
\textbf{Columna} & \textbf{Tipo de Dato} & \textbf{Restricción} & \textbf{Descripción} \\
\midrule
\endfirsthead

\toprule
\textbf{Columna} & \textbf{Tipo de Dato} & \textbf{Restricción} & \textbf{Descripción} \\
\midrule
\endhead

\midrule
\multicolumn{4}{r}{\textit{Continúa en la siguiente página}} \\
\endfoot

\bottomrule
\endlastfoot

\texttt{node\_id}     & VARCHAR(20)   & PK                 & Identificador único del nodo (ej. "NODO\_01"). \\
\texttt{description}  & VARCHAR(100)  & NULL               & Descripción o ubicación descriptiva (ej. "Puente Norte"). \\
\texttt{latitude}     & DECIMAL(10,8) & NOT NULL           & Coordenada de latitud fija del nodo. \\
\texttt{longitude}    & DECIMAL(11,8) & NOT NULL           & Coordenada de longitud fija del nodo. \\
\texttt{status}       & VARCHAR(20)   & DEFAULT 'ACTIVE'   & Estado operativo (ACTIVE, INACTIVE, MAINT). \\
\texttt{last\_seen}   & TIMESTAMP     & NULL               & Fecha/hora de la última comunicación recibida. \\

\end{longtable}


\subsubsection*{Tabla: Sensor\_Readings (Lecturas de Sensores)}
Almacena el histórico de mediciones. Se crean índices en \texttt{timestamp} y \texttt{node\_id} para optimizar las consultas de series de tiempo (RF25).

\renewcommand{\arraystretch}{1.2}

\begin{longtable}{l l l p{6cm}}
\caption{Diccionario de datos: Tabla \texttt{sensor\_readings}.}
\label{tab:dd_readings} \\

\toprule
\textbf{Columna} & \textbf{Tipo de Dato} & \textbf{Restricción} & \textbf{Descripción} \\
\midrule
\endfirsthead

\toprule
\textbf{Columna} & \textbf{Tipo de Dato} & \textbf{Restricción} & \textbf{Descripción} \\
\midrule
\endhead

\midrule
\multicolumn{4}{r}{\textit{Continúa en la siguiente página}} \\
\endfoot

\bottomrule
\endlastfoot

\texttt{reading\_id}       & BIGSERIAL      & PK         & Identificador autoincremental único de la lectura. \\
\texttt{node\_id}          & VARCHAR(20)    & FK         & Referencia al nodo que generó el dato. \\
\texttt{timestamp}         & TIMESTAMP      & NOT NULL   & Fecha y hora exacta de la toma de muestra. \\
\texttt{battery\_level}    & DECIMAL(4,2)   & NOT NULL   & Nivel de voltaje de la batería (V). \\
\texttt{ph}                & DECIMAL(4,2)   & NULL       & Valor de pH (0.00 -- 14.00). \\
\texttt{temperature}       & DECIMAL(5,2)   & NULL       & Temperatura del agua (\si{\celsius}). \\
\texttt{turbidity}         & DECIMAL(6,2)   & NULL       & Turbidez (NTU). \\
\texttt{dissolved\_oxygen} & DECIMAL(5,2)   & NULL       & Oxígeno disuelto (mg/L). \\
\texttt{conductivity}      & DECIMAL(8,2)   & NULL       & Conductividad eléctrica (\si{\micro\siemens\per\centi\meter}). \\

\end{longtable}


\subsubsection*{Tabla: Waste\_Detections (Detecciones de Residuos)}
Almacena los resultados del análisis y la evidencia visual. Se opta por almacenar la imagen directamente en la base de datos (BLOB) para garantizar la integridad referencial y simplificar la arquitectura del prototipo.

\renewcommand{\arraystretch}{1.2}

\begin{longtable}{l l l p{6cm}}
\caption{Diccionario de datos: Tabla \texttt{waste\_detections}.}
\label{tab:dd_detections} \\

\toprule
\textbf{Columna} & \textbf{Tipo de Dato} & \textbf{Restricción} & \textbf{Descripción} \\
\midrule
\endfirsthead

\toprule
\textbf{Columna} & \textbf{Tipo de Dato} & \textbf{Restricción} & \textbf{Descripción} \\
\midrule
\endhead

\bottomrule
\endfoot

\bottomrule
\endlastfoot

\texttt{detection\_id} & BIGSERIAL & PK & Identificador único del evento de detección. \\
\texttt{node\_id} & VARCHAR(20) & FK & Referencia al nodo (cámara origen). \\
\texttt{timestamp} & TIMESTAMP & NOT NULL & Fecha y hora de captura de la imagen. \\
\texttt{coverage\_percent} & DECIMAL(5,2) & NOT NULL & \% del área de agua cubierta por residuos ($D_{cob}$). \\
\texttt{image\_data} & BYTEA & NOT NULL & Binario de la imagen JPEG (aprox. 90 kB). \\
\texttt{model\_version} & VARCHAR(50) & NOT NULL & Versión del modelo YOLO utilizado. \\
\texttt{confidence} & DECIMAL(3,2) & NULL & Nivel de confianza promedio de la detección. \\

\end{longtable}



\subsubsection*{Tabla: Users (Usuarios del Sistema)}
Almacena las credenciales de acceso para los administradores y el dueño del sistema. El control de privilegios se gestiona mediante el campo de rol.

\renewcommand{\arraystretch}{1.2}

\begin{longtable}{l l l p{6cm}}
\caption{Diccionario de datos: Tabla \texttt{users}.}
\label{tab:dd_users} \\

\toprule
\textbf{Columna} & \textbf{Tipo de Dato} & \textbf{Restricción} & \textbf{Descripción} \\
\midrule
\endfirsthead

\toprule
\textbf{Columna} & \textbf{Tipo de Dato} & \textbf{Restricción} & \textbf{Descripción} \\
\midrule
\endhead

\bottomrule
\endfoot

\bottomrule
\endlastfoot

\texttt{user\_id} & BIGSERIAL & PK & Identificador único del usuario. \\
\texttt{full\_name} & VARCHAR(100) & NOT NULL & Nombre completo del personal. \\
\texttt{email} & VARCHAR(100) & UNIQUE & Correo electrónico (usado para Login). \\
\texttt{password\_hash} & VARCHAR(255) & NOT NULL & Contraseña cifrada (Hash). \\
\texttt{role} & VARCHAR(20) & NOT NULL & Nivel de acceso: 'OWNER' o 'ADMIN'. \\
\texttt{created\_at} & TIMESTAMP & DEFAULT NOW() & Fecha de creación de la cuenta. \\
\texttt{is\_active} & BOOLEAN & DEFAULT TRUE & Indica si el usuario tiene permiso de acceso. \\

\end{longtable}



%%%%%%%%%%%%%%%%%%%%%%%%%%%%%%%%%%%%%%%%%%%%%%%%%%%%
%             SECCIÓN: casos de uso         %
%%%%%%%%%%%%%%%%%%%%%%%%%%%%%%%%%%%%%%%%%%%%%%%%%%%%
\section{Diagramas de Casos de Uso}
\label{sec:diagramas_casos_uso}

Para modelar el comportamiento funcional del sistema y las interacciones entre los actores (usuarios, hardware y entorno) con los distintos módulos, se han elaborado diagramas de casos de uso basados en los requerimientos definidos en la Sección \ref{sec:analisis_requerimientos}. Debido a la naturaleza distribuida del sistema, se presentan tres vistas correspondientes a los subsistemas principales.

\subsection{Casos de Uso del Subsistema de Adquisición (WSN)}
La Figura \ref{fig:uc_wsn} ilustra las operaciones autónomas realizadas por los nodos sensores y el nodo concentrador (Gateway). Se destaca el rol del \textbf{Entorno Fluvial} como un actor externo que provee los estímulos físicos (parámetros y luz para la imagen) que activan el proceso de monitoreo. El diagrama muestra el flujo de datos en la topología lineal multisalto, donde cada nodo intermedio no solo genera su información, sino que también recibe y retransmite datos de sus vecinos (\textit{store-and-forward}).

\begin{figure}[H]
    \centering
    \includegraphics[width=0.9\linewidth]{Documento/Imagenes/Diseño/UML/UMLNodos.pdf}
    \caption{Diagrama de casos de uso para la Red Inalámbrica de Sensores y el Gateway, mostrando la interacción con el entorno y el relevo de datos.}
    \label{fig:uc_wsn}
\end{figure}

\subsection{Casos de Uso del Servidor y Procesamiento}
La Figura \ref{fig:uc_backend} detalla el flujo de información en el servidor (Backend). Este subsistema actúa como el núcleo de procesamiento, iniciando su operación con la recepción de datos desde el Gateway. Se ilustra la secuencia de procesamiento que incluye el reensamblaje de imágenes fragmentadas, la ejecución de los algoritmos de Inteligencia Artificial (segmentación de residuos) y el almacenamiento persistente de los resultados para su posterior consulta vía API.

\begin{figure}[H]
    \centering
    \includegraphics[width=0.8\linewidth]{Documento/Imagenes/Diseño/UML/servidor.pdf}
    \caption{Diagrama de casos de uso para el Backend, abarcando la ingesta, procesamiento de IA y almacenamiento de datos.}
    \label{fig:uc_backend}
\end{figure}

\subsection{Casos de Uso de la Interfaz Web}
La Figura \ref{fig:uc_webapp} muestra las interacciones disponibles para los usuarios humanos a través de la plataforma web. Se modelan tres niveles de actores:
\begin{itemize}
    \item \textbf{Usuario Público:} Con acceso libre a la visualización de datos actuales, históricos y evidencia visual.
    \item \textbf{Administrador:} Hereda los permisos de visualización y añade capacidades de gestión sobre los nodos sensores (altas, bajas, configuración) previo inicio de sesión.
    \item \textbf{Dueño (Owner):} Actor con el máximo nivel de privilegios, capaz de gestionar las cuentas de los administradores del sistema.
\end{itemize}

\begin{figure}[H]
    \centering
    \includegraphics[width=0.6\linewidth]{Documento/Imagenes/Diseño/UML/aplicacionWeb.pdf}
    \caption{Diagrama de casos de uso para la Aplicación Web, detallando los roles de usuario y sus privilegios.}
    \label{fig:uc_webapp}
\end{figure}


%%%%%%%%%%%%%%%%%%%%%%%%%%%%%%%%%%%%%%%%%%%%%%%%%%%%
%             SECCIÓN: Mockups         %
%%%%%%%%%%%%%%%%%%%%%%%%%%%%%%%%%%%%%%%%%%%%%%%%%%%%

\section{Diseño de Interfaz de Usuario}
\label{sec:diseno_interfaz}

La interfaz de usuario se ha diseñado bajo un enfoque web responsivo, dividida en dos entornos claramente diferenciados según el perfil del actor: un \textbf{Portal Público} de acceso libre para la consulta ciudadana y un \textbf{Portal Administrativo} para la gestión técnica del sistema. A continuación, se describen las pantallas principales diseñadas.

\subsection{Portal Público}
\label{subsec:portal_publico}

Este módulo tiene como objetivo la divulgación y transparencia de la información. No requiere autenticación y está optimizado para una navegación intuitiva.

\subsubsection*{Página de Inicio (Landing Page)}
Es el punto de entrada al sistema. Presenta el proyecto ``MONICA'', su visión y el impacto esperado. Incluye un menú de navegación superior y llamadas a la acción claras para dirigir al usuario a los datos sensados.

\begin{figure}[H]
    \centering
    % Ajusta el nombre del archivo si es necesario. 'page=1' selecciona la primera página del PDF.
    \includegraphics[width=0.9\linewidth, page=1]{Documento/Imagenes/Diseño/mockups/mockups.pdf} 
    \caption{Página de inicio del portal público con información general del proyecto.}
    \label{fig:mockup_home}
\end{figure}

\subsubsection*{Mapa Interactivo}
Permite la geolocalización de los nodos sensores sobre el río, facilitando la identificación de los puntos de monitoreo.

\begin{figure}[H]
    \centering
    % Asegúrate de que la ruta sea correcta (Diseno sin ñ)
    \includegraphics[width=0.85\linewidth, page=2]{Documento/Imagenes/Diseño/mockups/mockups.pdf}
    \caption{Mapa interactivo de nodos para la geolocalización de puntos de monitoreo.}
    \label{fig:mockup_mapa}
\end{figure}

\subsubsection*{Visualización de Estadísticas}
Al seleccionar un nodo, el usuario visualiza un tablero detallado con los valores actuales de los sensores y gráficos de tendencias.

\begin{figure}[H]
    \centering
    \includegraphics[width=0.85\linewidth, page=3]{Documento/Imagenes/Diseño/mockups/mockups.pdf}
    \caption{Tablero de visualización de estadísticas y datos sensoriales.}
    \label{fig:mockup_stats}
\end{figure}

\begin{figure}[H]
    \centering
    \includegraphics[width=0.85\linewidth, page=4]{Documento/Imagenes/Diseño/mockups/mockups.pdf}
    \caption{Grafica de visualización de estadísticas y datos sensoriales.}
    \label{fig:mockup_stats_graph}
\end{figure}

\subsubsection*{Galería de Evidencias Visuales}
Esta sección muestra las imágenes capturadas por los nodos sensores. Permite al usuario visualizar la evidencia física del estado del río (ej. presencia de residuos) filtrando por fecha y ubicación, lo cual complementa los datos numéricos.

\begin{figure}[H]
    \centering
    \includegraphics[width=0.8\linewidth, page=5]{Documento/Imagenes/Diseño/mockups/mockups.pdf}
    \caption{Galería de imágenes capturadas por la red de sensores.}
    \label{fig:mockup_galeria}
\end{figure}

\subsection{Portal Administrativo}
\label{subsec:portal_admin}

Este módulo es de acceso restringido y permite la configuración y mantenimiento del sistema.

\subsubsection*{Acceso y Tablero de Control (Dashboard Admin)}
El ingreso se realiza mediante credenciales seguras. Al acceder, el administrador visualiza un resumen operativo del sistema: estado de los nodos (activos/inactivos), alertas recientes (ej. batería baja, parámetros fuera de norma) y métricas de usuarios registrados.

\begin{figure}[H]
    \centering
    \includegraphics[width=0.8\linewidth, page=7]{Documento/Imagenes/Diseño/mockups/mockups.pdf}
    \caption{Portal de acceso.}
    \label{fig:mockup_login}
\end{figure}

\begin{figure}[H]
    \centering
    \includegraphics[width=0.8\linewidth, page=8]{Documento/Imagenes/Diseño/mockups/mockups.pdf}
    \caption{Tablero de control administrativo con resumen del estado del sistema.}
    \label{fig:mockup_dashboard_admin}
\end{figure}

\subsubsection*{Gestión de Nodos y Datos}
Permite a los administradores registrar nuevos nodos sensores, editar su ubicación geográfica o dar de baja dispositivos. También se incluye una vista tabular para la administración cruda de los datos recibidos, útil para auditoría y corrección de errores.

\begin{figure}[H]
    \centering
    \includegraphics[width=0.8\linewidth, page=11]{Documento/Imagenes/Diseño/mockups/mockups.pdf}
    \caption{Interfaz para la gestión del inventario de nodos sensores.}
    \label{fig:mockup_nodos}
\end{figure}

\subsubsection*{Gestión de Usuarios (Rol: Dueño/Superusuario)}
Esta pantalla es crítica para la seguridad del sistema y \textbf{solo es accesible para el usuario con rol de Dueño (Owner)}. Permite registrar nuevos administradores, modificar roles y gestionar el acceso al sistema. Los usuarios con rol de ``Administrador'' estándar no tienen acceso a esta vista, garantizando un control jerárquico de la plataforma.

\begin{figure}[H]
    \centering
    \includegraphics[width=0.8\linewidth, page=10]{Documento/Imagenes/Diseño/mockups/mockups.pdf}
    \caption{Panel de gestión de usuarios, exclusivo para el rol de Dueño.}
    \label{fig:mockup_usuarios}
\end{figure}

\subsubsection*{Administración de Datos}
Este módulo permite la gestión integral de la información recolectada por el sistema. Los administradores pueden visualizar, filtrar y gestionar tanto los registros de los \textbf{datos de los sensores} (parámetros físico-químicos) como el historial de \textbf{imágenes y detecciones}.

\begin{figure}[H]
    \centering
    \includegraphics[width=0.8\linewidth, page=12]{Documento/Imagenes/Diseño/mockups/mockups.pdf}
    \caption{Interfaz para la administración y consulta de datos de nodos sensores.}
    \label{fig:mockup_reportes_datos}
\end{figure}


\begin{figure}[H]
    \centering
    \includegraphics[width=0.8\linewidth, page=13]{Documento/Imagenes/Diseño/mockups/mockups.pdf}
    \caption{Interfaz para la administración y consulta de imagenes de nodos sensores.}
    \label{fig:mockup_reportes_img}
\end{figure}



\begin{comment}
\section{Diseño de Software}
\begin{table}[H]
\centering
\caption{Tabla Nodo.}
\label{tab:usuario}
\begin{tabular}{|l|l|l|}
\hline
\textbf{Campo}              & \textbf{Tipo de dato} & \textbf{Llave}   \\ \hline
idNodo                  & int                   & primaria         \\ \hline
Nombre                    & varchar               & N/A              \\ \hline
apellido                  & varchar               & N/A              \\ \hline
seg\_apellido             & varchar               & N/A              \\ \hline
correo                    & varchar               & N/A              \\ \hline
fecha\_nacimiento         & date                  & N/A              \\ \hline
genero                    & varchar               & N/A              \\ \hline
bandera\_administrador    & tinyint               & N/A              \\ \hline
nodo                      & int                   & N/A              \\ \hline
\end{tabular}
\end{table}

\subsection{Conexión entre nodos, concentrador y servidor}.

En el sistema propuesto, los nodos sensores distribuidos a lo largo del entorno fluvial establecen comunicación con un nodo concentrador central mediante una red inalámbrica local basada en el estándar Wi-Fi HaLow (IEEE 802.11ah). Esta tecnología opera en la banda sub-1 GHz, lo cual le otorga ventajas significativas en términos de alcance (hasta 1 km en condiciones ideales) y penetración en entornos con vegetación densa, además de un consumo energético moderado que favorece la operación autónoma con baterías o paneles solares.

El nodo concentrador cumple una función crítica dentro de la arquitectura, ya que actúa como gateway inalámbrico. Es decir, se encarga de recibir la información sensorial proveniente de los nodos HaLow y reenviarla al servidor central utilizando una interfaz de red externa. Esta interfaz puede implementarse mediante un módulo de red celular (4G/5G), ideal para zonas con baja infraestructura de conectividad, o a través de una red WLAN convencional (2.4/5 GHz) disponible en estaciones fijas cercanas al sitio de monitoreo.

Para establecer la comunicación entre el nodo concentrador y el servidor, se asume que el concentrador cuenta con dicha interfaz de red externa, permitiendo el envío de datos a través del protocolo HTTP. Esta suposición garantiza la interoperabilidad con servicios web modernos y facilita la integración con plataformas de almacenamiento, análisis o visualización de datos.

Una vez recibidos, los datos son procesados por el servidor y almacenados en una base de datos estructurada, permitiendo su consulta posterior, análisis automatizado o incluso la generación de comandos que puedan enviarse de regreso a los nodos sensores, habilitando así una comunicación bidireccional dentro del sistema.
\end{comment}

%\subsection{Base de datos}
%\subsubsection{Diseño de base de datos}
%\subsubsection{Modelo Relacional}
%\subsubsection{Diccionario de datos}

%\section{Diseño de Comunicación}
%\subsection{Diagrama de comunicación del nodo concentrador con el servidor}

%\section{Diseño de Interfaces de Usuario}
%\subsection{Diseño de Mock-up para la página web}
%\subsection{Página principal}

%\section{Modelado del Sistema}
%\subsection{Diagramas UML}
%\subsubsection{Diagramas de caso de uso}
%\subsubsection{Diagramas de secuencia}

%\section{Integración y Pruebas}
%\subsection{Integración final del sistema}
%\subsection{Escenario de pruebas}



%\newpage
%\input{Documento/7-Escenario de Pruebas3}
%\newpage
%\section{Cronograma}
A continuación, se presenta el cronograma propuesto para las actividades correspondientes a la fase de Proyecto Terminal 1. En este cronograma se detallan las actividades clave, los responsables de su ejecución y los objetivos específicos de cada una.
\begin{table}[H]
    \centering
    \renewcommand{\arraystretch}{1.5}
    \begin{tabular}{ |p{3cm}|p{8cm}| }
        \hline
        \textbf{Alumno 1} & Dompablo Celaya Oscar Alfredo \\\hline
        \textbf{Alumno 2} & López Ramírez Itzel \\\hline
    \end{tabular}
    \caption{Nombres de los responsables}
    \label{tab:Nombre de los responsables}
\end{table}

\begin{table}[H]
    \centering
    \renewcommand{\arraystretch}{1.5}
    \begin{tabular}{ |p{1.5cm}|p{5cm}|p{5.5cm}|p{2.5cm}| }

        \hline
        \textbf{N° Actividad} & \textbf{Nombre de la Actividad} & \textbf{Objetivo} & \textbf{Responsable} \\
        \hline
        1 & Investigación sobre tecnologías de monitoreo y visión artificial & Identificar soluciones actuales y evaluar su aplicabilidad para medir la calidad del agua y detectar residuos flotantes & Alumno 1, Alumno 2 \\\hline
        2 & Análisis de redes inalámbricas de sensores & Comprensión detallada del funcionamiento y configuración de redes inalámbricas de sensores para su aplicación en monitoreo ambiental & Alumno 1, Alumno 2 \\\hline
        3 & Estudio sobre el procesamiento de imágenes & Adquisición de conocimientos sobre técnicas de procesamiento de imágenes, con énfasis en su aplicación para la detección de residuos sólidos mediante visión artificial & Alumno 1, Alumno 2 \\\hline
        4 & Análisis de algoritmos de visión artificial & Evaluar algoritmos para clasificar residuos flotantes & Alumno 2 \\\hline
        5 & Definición de objetivos y alcance del proyecto & Establecer Requerimientos funcionales y específicos del sistema & Alumno 1, Alumno 2 \\\hline
        6 & Estudio de variables de calidad del agua & Investigar parámetros clave como pH, oxigenación, turbidez, temperatura y conductividad & Alumno 1 \\\hline
        7 & Selección preliminar de sensores & Determinar sensores más adecuados para las variables de calidad del agua & Alumno 1, Alumno 2 \\\hline
        8 & Investigación de protocolos de comunicación & Analizar protocolos como LoRa y ZigBee para la red inalámbrica & Alumno 1 \\\hline
        9 & Estudio de microcontroladores y cámaras & Evaluar opciones técnicas para seleccionar componentes adecuados & Alumno 2 \\\hline
        10 & Diseño preliminar de arquitectura del sistema & Definir la topología de red y la interacción de módulos & Alumno 1, Alumno 2 \\\hline

    \end{tabular}
    \caption{Cronograma de actividades del proyecto - Parte 1}
    \label{tab:cronograma_proyecto_parte1}
\end{table}
\begin{table}[H]
    \centering
    \renewcommand{\arraystretch}{1.5}
    \begin{tabular}{ |p{1.5cm}|p{5cm}|p{5.5cm}|p{2.5cm}| }

        \hline
        \textbf{N° Actividad} & \textbf{Nombre de la Actividad} & \textbf{Objetivo} & \textbf{Responsable} \\
        \hline

       11 & Caracterización de sensores & Probar las especificaciones técnicas de los sensores seleccionados & Alumno 1 \\\hline
        12 & Diseño conceptual del sistema de visión artificial & Establecer el enfoque inicial para la detección de residuos flotantes & Alumno 2 \\\hline
        13 & Selección de infraestructura de almacenamiento & Determinar servidores y bases de datos adecuados para el proyecto & Alumno 1 \\\hline
        14 & Diseño de una base de datos  & Diseñar una base de datos estructurada para el almacenamiento de los datos & Alumno 1, Alumno 2\\\hline
        15 & Diseño preliminar de la aplicación web & Prototipar una interfaz web para visualización de datos & Alumno 1, Alumno 2 \\\hline
        16 & Desarrollo de la lógica de transmisión de datos & Planificar la integración de datos de sensores en el servidor & Alumno 2 \\\hline
        17 & Simulación de la red inalámbrica & Validar la comunicación de los nodos en un entorno controlado & Alumno 1 \\\hline
        18 & Desarrollo de algoritmos de visión artificial & Implementar algoritmos iniciales para detección y clasificación de residuos & Alumno 2 \\\hline
        19 & Diseño del nodo sensor & Crear el diseño conceptual del nodo con sensores, comunicación y alimentación & Alumno 1 \\\hline
        20 & Diseño del módulo de visión artificial & Definir la integración de cámaras y algoritmos para procesar imágenes & Alumno 2 \\\hline
        21 & Diseño del servidor y base de datos & Configurar el almacenamiento y procesamiento centralizado de datos & Alumno 1 \\\hline

    \end{tabular}
    \caption{Cronograma de actividades del proyecto - Parte 2}
    \label{tab:cronograma_proyecto_parte2}
\end{table}
\begin{table}[H]
    \centering
    \renewcommand{\arraystretch}{1.5}
    \begin{tabular}{ |p{1.5cm}|p{5cm}|p{5.5cm}|p{2.5cm}| }

        \hline
        \textbf{N° Actividad} & \textbf{Nombre de la Actividad} & \textbf{Objetivo} & \textbf{Responsable} \\
        \hline
        22 & Redacción del documento técnico intermedio & Documentar avances y ajustar planificación para la siguiente fase & Alumno 1, Alumno 2 \\\hline
        23 & Validación inicial de componentes & Probar individualmente sensores y cámaras para asegurar su funcionalidad & Alumno 1, Alumno 2 \\\hline
        24 & Desarrollo del prototipo de red inalámbrica & Implementar la comunicación básica entre nodos y servidor & Alumno 1 \\\hline
        25 & Evaluación de precisión en sensores & Realizar pruebas para ajustar parámetros según resultados obtenidos & Alumno 1 \\\hline
        26 & Pruebas de visión artificial & Verificar la capacidad de clasificación de residuos en imágenes controladas & Alumno 2 \\\hline
        27 & Integración parcial de módulos & Unificar nodos sensores, visión artificial y servidor para pruebas iniciales & Alumno 1, Alumno 2 \\\hline

    \end{tabular}
    \caption{Cronograma de actividades del proyecto - Parte 3}
    \label{tab:cronograma_proyecto_parte3}
\end{table}

%=================== Bibliografia ===================
%\newpage

\cleardoublepage
\phantomsection
\addcontentsline{toc}{section}{Referencias}
\printbibliography[title={Referencias}] % Referencias
\end{document}
